%% Generated by Sphinx.
\def\sphinxdocclass{report}
\documentclass[letterpaper,10pt,english,openany,oneside]{sphinxmanual}
\ifdefined\pdfpxdimen
   \let\sphinxpxdimen\pdfpxdimen\else\newdimen\sphinxpxdimen
\fi \sphinxpxdimen=.75bp\relax
\ifdefined\pdfimageresolution
    \pdfimageresolution= \numexpr \dimexpr1in\relax/\sphinxpxdimen\relax
\fi
%% let collapsible pdf bookmarks panel have high depth per default
\PassOptionsToPackage{bookmarksdepth=5}{hyperref}

\PassOptionsToPackage{warn}{textcomp}
\usepackage[utf8]{inputenc}
\ifdefined\DeclareUnicodeCharacter
% support both utf8 and utf8x syntaxes
  \ifdefined\DeclareUnicodeCharacterAsOptional
    \def\sphinxDUC#1{\DeclareUnicodeCharacter{"#1}}
  \else
    \let\sphinxDUC\DeclareUnicodeCharacter
  \fi
  \sphinxDUC{00A0}{\nobreakspace}
  \sphinxDUC{2500}{\sphinxunichar{2500}}
  \sphinxDUC{2502}{\sphinxunichar{2502}}
  \sphinxDUC{2514}{\sphinxunichar{2514}}
  \sphinxDUC{251C}{\sphinxunichar{251C}}
  \sphinxDUC{2572}{\textbackslash}
\fi
\usepackage{cmap}
\usepackage[T1]{fontenc}
\usepackage{amsmath,amssymb,amstext}
\usepackage{babel}



\usepackage{tgtermes}
\usepackage{tgheros}
\renewcommand{\ttdefault}{txtt}



\usepackage[Bjarne]{fncychap}
\usepackage{sphinx}

\fvset{fontsize=auto}
\usepackage{geometry}


% Include hyperref last.
\usepackage{hyperref}
% Fix anchor placement for figures with captions.
\usepackage{hypcap}% it must be loaded after hyperref.
% Set up styles of URL: it should be placed after hyperref.
\urlstyle{same}

\addto\captionsenglish{\renewcommand{\contentsname}{Contents:}}

\usepackage{sphinxmessages}
\setcounter{tocdepth}{1}



\title{TissUUmaps}
\date{Apr 18, 2022}
\release{3.0}
\author{Nicolas Pielawski\and Axel Andersson\and Christophe Avenel\and Andrea Behanova\and Eduard Chelebian\and Anna Klemm\and Fredrik Nysjö\and Leslie Solorzano\and Carolina Wählby}
\newcommand{\sphinxlogo}{\vbox{}}
\renewcommand{\releasename}{Release}
\makeindex
\begin{document}

\pagestyle{empty}
\sphinxmaketitle
\pagestyle{plain}
\sphinxtableofcontents
\pagestyle{normal}
\phantomsection\label{\detokenize{index::doc}}


\sphinxAtStartPar
This page hosts the documentation for TissUUmaps 3.0. You can find a pdf version of thie documentation \sphinxhref{https://tissuumaps.github.io/TissUUmaps-docs/index.pdf}{here}.

\sphinxAtStartPar
For more information on the TissUUmaps project, including video tutorials and demos, visit our website: \sphinxurl{https://tissuumaps.github.io}.

\begin{sphinxShadowBox}
\sphinxstylesidebartitle{Work in progress!}

\sphinxAtStartPar
This page is mostly empty for now. We are working actively on writing this documentation, more content will be available soon!
\end{sphinxShadowBox}

\sphinxstepscope


\chapter{Introduction}
\label{\detokenize{docs/intro/index:introduction}}\label{\detokenize{docs/intro/index::doc}}
\sphinxstepscope


\section{About TissUUmaps}
\label{\detokenize{docs/intro/about:about-tissuumaps}}\label{\detokenize{docs/intro/about::doc}}
\sphinxAtStartPar
\sphinxhref{https://tissuumaps.github.io/}{TissUUmaps} is a free and open source browser\sphinxhyphen{}based tool for GPU\sphinxhyphen{}accelerated visualization and interactive exploration of tens of millions of datapoints overlaying tissue samples. Users can visualize markers and regions, explore spatial statistics and quantitative analyses of tissue morphology, and assess the quality of decoding in situ transcriptomics data. TissUUmaps provides instant multi\sphinxhyphen{}resolution image viewing, can be customized, shared, and also integrated in Jupyter Notebooks. We envision TissUUmaps to contribute to broader dissemination and flexible sharing of large\sphinxhyphen{}scale spatial omics data.

\sphinxAtStartPar
Currently, microscopy data can be cumbersome to share: physically transferring the images is often necessary and dedicated software must be installed. Instead, researchers can now share their findings with a simple link to a website running TissUUmaps. The images are loaded in real time, together with annotations, markers, and masks that may also be modified by the user. We also provide tools for quality control and image processing. The software is designed to display and interact with images at multiple resolutions and large numbers of markers, especially data from spatially resolved omics techniques and tissue atlases. TissUUmaps is compatible with many different bioimage informatics tools, and provides new ways to develop insights when exploring and sharing data.

\sphinxAtStartPar
You can access the \sphinxhref{https://tissuumaps.github.io/gallery/}{TissUUmaps project gallery} with interactive examples to explore data from in situ sequencing and spatial transcriptomics experiments and view localized quantification of cell and tissue morphology, including links to publications. For seeing examples of TissUUmaps compatibility with other platforms you can access the \sphinxhref{https://tissuumaps.github.io/tutorials/}{tutorials page}.

\sphinxstepscope


\section{Installation}
\label{\detokenize{docs/intro/installation:installation}}\label{\detokenize{docs/intro/installation::doc}}
\sphinxAtStartPar
\sphinxhref{https://tissuumaps.github.io/}{TissUUmaps} is a browser\sphinxhyphen{}based tool for fast visualization and exploration of millions of data points overlaying a tissue sample. TissUUmaps can be used as a web service or locally in your computer, and allows users to share regions of interest and local statistics.


\subsection{Windows installation}
\label{\detokenize{docs/intro/installation:windows-installation}}\begin{enumerate}
\sphinxsetlistlabels{\arabic}{enumi}{enumii}{}{.}%
\item {} 
\sphinxAtStartPar
Download the Windows Installer from \sphinxhref{https://github.com/TissUUmaps/TissUUmaps/releases/latest}{the last release} and install it. Note that the installer is not signed yet and may trigger warnings from the browser and from the firewall. You can safely pass these warnings.

\end{enumerate}


\subsection{PIP installation (for Linux and Mac)}
\label{\detokenize{docs/intro/installation:pip-installation-for-linux-and-mac}}\begin{enumerate}
\sphinxsetlistlabels{\arabic}{enumi}{enumii}{}{.}%
\item {} 
\sphinxAtStartPar
Install \sphinxcode{\sphinxupquote{libvips}} for your system: \sphinxurl{https://www.libvips.org/install.html}

\sphinxAtStartPar
An easy way to install \sphinxcode{\sphinxupquote{libvips}} is to use an \sphinxhref{https://docs.anaconda.com/anaconda/install/index.html}{Anaconda} environment with \sphinxcode{\sphinxupquote{libvips}}:

\begin{sphinxVerbatim}[commandchars=\\\{\}]
conda create \PYGZhy{}y \PYGZhy{}n tissuumaps\PYGZus{}env \PYGZhy{}c conda\PYGZhy{}forge \PYG{n+nv}{python}\PYG{o}{=}\PYG{l+m}{3}.9 libvips
conda activate tissuumaps\PYGZus{}env
\end{sphinxVerbatim}

\item {} 
\sphinxAtStartPar
Install the TissUUmaps library using \sphinxcode{\sphinxupquote{pip}}:

\begin{sphinxVerbatim}[commandchars=\\\{\}]
pip install \PYG{l+s+s2}{\PYGZdq{}TissUUmaps[full]\PYGZdq{}}
\end{sphinxVerbatim}

\item {} 
\sphinxAtStartPar
Start the TissUUmaps user interface:

\begin{sphinxVerbatim}[commandchars=\\\{\}]
tissuumaps
\end{sphinxVerbatim}

\item {} 
\sphinxAtStartPar
Or start TissUUmaps as a local server:

\begin{sphinxVerbatim}[commandchars=\\\{\}]
tissuumaps\PYGZus{}server path\PYGZus{}to\PYGZus{}your\PYGZus{}images
\end{sphinxVerbatim}

\sphinxAtStartPar
And open \sphinxurl{http://127.0.0.1:5000/} in your favorite browser.

\end{enumerate}

\sphinxstepscope


\section{Citing TissUUmaps}
\label{\detokenize{docs/intro/citing:citing-tissuumaps}}\label{\detokenize{docs/intro/citing::doc}}
\sphinxAtStartPar
Please cite our \sphinxhref{https://www.biorxiv.org/content/10.1101/2022.01.28.478131v1}{preprint} on bioRxiv if using TissUUmaps in your work:

\sphinxAtStartPar
\sphinxstylestrong{TissUUmaps 3: Interactive visualization and quality assessment of large\sphinxhyphen{}scale spatial omics data.} \sphinxstyleemphasis{Nicolas Pielawski, Axel Andersson, Christophe Avenel, Andrea Behanova, Eduard Chelebian, Anna Klemm, Fredrik Nysjö, Leslie Solorzano, Carolina Wählby,} bioRxiv 2022.01.28.478131; doi: \sphinxurl{https://doi.org/10.1101/2022.01.28.478131}.

\sphinxstepscope


\section{Changelog}
\label{\detokenize{docs/intro/versions:changelog}}\label{\detokenize{docs/intro/versions::doc}}

\subsection{3.0.8.5}
\label{\detokenize{docs/intro/versions:id1}}\begin{itemize}
\item {} 
\sphinxAtStartPar
Minor fixes.

\end{itemize}


\subsection{3.0.8.4}
\label{\detokenize{docs/intro/versions:id2}}\begin{itemize}
\item {} 
\sphinxAtStartPar
Add tiling to viewport capture for higher resolution output

\item {} 
\sphinxAtStartPar
Increase resolution of markers on high resolution devices

\item {} 
\sphinxAtStartPar
Fix jumps on pan with mouse gesture (mobile)

\item {} 
\sphinxAtStartPar
Add fix for bright image canvas on Safari

\item {} 
\sphinxAtStartPar
Add an option to remove markers’ outlines.

\end{itemize}


\subsection{3.0.8.3}
\label{\detokenize{docs/intro/versions:id3}}\begin{itemize}
\item {} 
\sphinxAtStartPar
Fix png artifact in Firefox, by generating jpg tiles.

\end{itemize}


\subsection{3.0.8.2}
\label{\detokenize{docs/intro/versions:id4}}\begin{itemize}
\item {} 
\sphinxAtStartPar
Add high resolution capture of viewport, up to 4096x4096 pixels.

\end{itemize}


\subsection{3.0.8.1}
\label{\detokenize{docs/intro/versions:id5}}\begin{itemize}
\item {} 
\sphinxAtStartPar
Fix multiple dataset alignment when no background image

\end{itemize}


\subsection{3.0.8}
\label{\detokenize{docs/intro/versions:id6}}\begin{itemize}
\item {} 
\sphinxAtStartPar
Fix black images generated by VIPS

\item {} 
\sphinxAtStartPar
Fix Linux and Mac open of captures

\item {} 
\sphinxAtStartPar
Auto save datasets as buttons when saving tmap projects

\item {} 
\sphinxAtStartPar
Add \sphinxcode{\sphinxupquote{mpp}} (microns per pixel) option in tmap files, to add scale bar to viewer

\item {} 
\sphinxAtStartPar
Make region line thickness depend on zoom level

\item {} 
\sphinxAtStartPar
Add compatibility with JupyterLab

\item {} 
\sphinxAtStartPar
Add opacity per marker option

\end{itemize}


\subsection{3.0.7}
\label{\detokenize{docs/intro/versions:id7}}\begin{itemize}
\item {} 
\sphinxAtStartPar
Add menu to load plugins through an update\sphinxhyphen{}site

\end{itemize}


\subsection{3.0.6}
\label{\detokenize{docs/intro/versions:id8}}\begin{itemize}
\item {} 
\sphinxAtStartPar
Fix multiple plugins opening always last plugin

\item {} 
\sphinxAtStartPar
Move to OpenSeadragon 3.0.0

\item {} 
\sphinxAtStartPar
Add tooltip format in Advanced Settings

\item {} 
\sphinxAtStartPar
Add drag and drop to open CSV files and images

\item {} 
\sphinxAtStartPar
Add “Add layer” button for flask version

\item {} 
\sphinxAtStartPar
Add viewport capture

\end{itemize}


\subsection{3.0.5}
\label{\detokenize{docs/intro/versions:id9}}\begin{itemize}
\item {} 
\sphinxAtStartPar
Move csv loading to Papa Parse streaming, to allow better memory management

\end{itemize}


\subsection{3.0.4}
\label{\detokenize{docs/intro/versions:id10}}\begin{itemize}
\item {} 
\sphinxAtStartPar
Add filtering of markers

\end{itemize}


\subsection{3.0}
\label{\detokenize{docs/intro/versions:id11}}\begin{itemize}
\item {} 
\sphinxAtStartPar
Add tissuumaps.jupyter module

\end{itemize}

\sphinxstepscope


\chapter{Getting started}
\label{\detokenize{docs/starting/index:getting-started}}\label{\detokenize{docs/starting/index::doc}}
\sphinxstepscope


\section{Images}
\label{\detokenize{docs/starting/images:images}}\label{\detokenize{docs/starting/images::doc}}

\subsection{Supported image formats}
\label{\detokenize{docs/starting/images:supported-image-formats}}
\sphinxAtStartPar
TissUUmaps can read whole slide images in any format recognized by the OpenSlide library:
\begin{itemize}
\item {} 
\sphinxAtStartPar
Aperio (.svs, .tif)

\item {} 
\sphinxAtStartPar
Hamamatsu (.ndpi, .vms, .vmu)

\item {} 
\sphinxAtStartPar
Leica (.scn)

\item {} 
\sphinxAtStartPar
MIRAX (.mrxs)

\item {} 
\sphinxAtStartPar
Philips (.tiff)

\item {} 
\sphinxAtStartPar
Sakura (.svslide)

\item {} 
\sphinxAtStartPar
Trestle (.tif)

\item {} 
\sphinxAtStartPar
Ventana (.bif, .tif)

\item {} 
\sphinxAtStartPar
Generic tiled TIFF (.tif)

\end{itemize}

\sphinxAtStartPar
TissUUmaps will automatically convert any other format into a pyramidal tiff (in a temporary \sphinxcode{\sphinxupquote{.tissuumaps}} folder created in the original image folder) using vips.

\sphinxAtStartPar
If your image fails to open, try converting it to \sphinxcode{\sphinxupquote{tif}} format using an external tool.


\subsection{Load images}
\label{\detokenize{docs/starting/images:load-images}}



\subsection{Apply filters}
\label{\detokenize{docs/starting/images:apply-filters}}
\sphinxstepscope


\section{Markers}
\label{\detokenize{docs/starting/markers:markers}}\label{\detokenize{docs/starting/markers::doc}}

\subsection{Supported marker format}
\label{\detokenize{docs/starting/markers:supported-marker-format}}
\sphinxAtStartPar
TissUUmaps can read CSV (Comma Separated Values) files with a header row, and at least spatial coordinate columns (X and Y). CSV files are not limited in the number of columns or number of rows. Other columns can contain information for displaying markers (key to group markers, color, size, shape, piecharts, etc.)

\sphinxAtStartPar
CSV files can be exported from any spreadsheet program, or any programming language (Python, R, etc.)


\subsection{Load markers}
\label{\detokenize{docs/starting/markers:load-markers}}

\subsection{Markers settings}
\label{\detokenize{docs/starting/markers:markers-settings}}

\subsubsection{File and coordinates}
\label{\detokenize{docs/starting/markers:file-and-coordinates}}

\subsubsection{Render options}
\label{\detokenize{docs/starting/markers:render-options}}

\subsubsection{Advanced options}
\label{\detokenize{docs/starting/markers:advanced-options}}

\subsubsection{Table of markers}
\label{\detokenize{docs/starting/markers:table-of-markers}}
\sphinxstepscope


\section{Regions}
\label{\detokenize{docs/starting/regions:regions}}\label{\detokenize{docs/starting/regions::doc}}

\subsection{Supported region formats}
\label{\detokenize{docs/starting/regions:supported-region-formats}}
\sphinxAtStartPar
TissUUmaps can read and write region files in the \sphinxhref{https://geojson.org/}{GeoJSON} format.

\sphinxAtStartPar
Only a subset of the GeoJSON format is supported, as TissUUmaps uses only polygonal regions:

\sphinxAtStartPar
\sphinxstylestrong{Main types}:
\begin{itemize}
\item {} 
\sphinxAtStartPar
Feature

\item {} 
\sphinxAtStartPar
FeatureCollection

\item {} 
\sphinxAtStartPar
GeometryCollection

\end{itemize}

\sphinxAtStartPar
\sphinxstylestrong{Geometries}:
\begin{itemize}
\item {} 
\sphinxAtStartPar
Polygon

\item {} 
\sphinxAtStartPar
Multipolygon

\end{itemize}

\sphinxAtStartPar
The coordinate system must be the same as the image and marker coordinate systems.


\subsection{Draw Regions}
\label{\detokenize{docs/starting/regions:draw-regions}}

\subsection{Analyze Regions}
\label{\detokenize{docs/starting/regions:analyze-regions}}

\subsection{Load Regions}
\label{\detokenize{docs/starting/regions:load-regions}}

\subsection{Export Regions}
\label{\detokenize{docs/starting/regions:export-regions}}
\sphinxstepscope


\section{Projects}
\label{\detokenize{docs/starting/projects:projects}}\label{\detokenize{docs/starting/projects::doc}}

\subsection{Saving and loading projects}
\label{\detokenize{docs/starting/projects:saving-and-loading-projects}}

\subsection{The TMAP file format}
\label{\detokenize{docs/starting/projects:the-tmap-file-format}}
\sphinxAtStartPar
The TMAP format contains a description of image layers, markers, regions, and settings. It is highly recommended to create .tmap files by saving projects from TissUUmaps, but you can also edit the files manually to add or change projects’ settings, or generate them as exported data from other software for import in TissUUmaps.

\sphinxAtStartPar
The TMAP format uses JSON, with the following specifications:


\subsubsection{TMAP project specifications}
\label{\detokenize{docs/starting/projects:tmap-project-specifications}}\label{\detokenize{docs/starting/projects:tmap-project-specifications}}

\begin{savenotes}\sphinxatlongtablestart\begin{longtable}[c]{|*{4}{\X{1}{4}|}}
\hline

\endfirsthead

\multicolumn{4}{c}%
{\makebox[0pt]{\sphinxtablecontinued{\tablename\ \thetable{} \textendash{} continued from previous page}}}\\
\hline

\endhead

\hline
\multicolumn{4}{r}{\makebox[0pt][r]{\sphinxtablecontinued{continues on next page}}}\\
\endfoot

\endlastfoot
\sphinxstartmulticolumn{4}%
\begin{varwidth}[t]{\sphinxcolwidth{4}{4}}
\sphinxAtStartPar
Description of image layers, markers, regions, and settings of a project. Required properties are shown in \sphinxstylestrong{bold} text
\par
\vskip-\baselineskip\vbox{\hbox{\strut}}\end{varwidth}%
\sphinxstopmulticolumn
\\
\hline
\sphinxAtStartPar
type
&\sphinxstartmulticolumn{3}%
\begin{varwidth}[t]{\sphinxcolwidth{3}{4}}
\sphinxAtStartPar
\sphinxstyleemphasis{object}
\par
\vskip-\baselineskip\vbox{\hbox{\strut}}\end{varwidth}%
\sphinxstopmulticolumn
\\
\hline\sphinxstartmulticolumn{4}%
\begin{varwidth}[t]{\sphinxcolwidth{4}{4}}
\sphinxAtStartPar
properties
\par
\vskip-\baselineskip\vbox{\hbox{\strut}}\end{varwidth}%
\sphinxstopmulticolumn
\\
\hline\sphinxmultirow{2}{5}{%
\begin{varwidth}[t]{\sphinxcolwidth{1}{4}}
\begin{itemize}
\item {} 
\sphinxAtStartPar
\sphinxstylestrong{filename}

\end{itemize}
\par
\vskip-\baselineskip\vbox{\hbox{\strut}}\end{varwidth}%
}%
&\sphinxstartmulticolumn{3}%
\begin{varwidth}[t]{\sphinxcolwidth{3}{4}}
\sphinxAtStartPar
Name of the project
\par
\vskip-\baselineskip\vbox{\hbox{\strut}}\end{varwidth}%
\sphinxstopmulticolumn
\\
\cline{2-4}\sphinxtablestrut{5}&
\sphinxAtStartPar
type
&\sphinxstartmulticolumn{2}%
\begin{varwidth}[t]{\sphinxcolwidth{2}{4}}
\sphinxAtStartPar
\sphinxstyleemphasis{string}
\par
\vskip-\baselineskip\vbox{\hbox{\strut}}\end{varwidth}%
\sphinxstopmulticolumn
\\
\hline\sphinxmultirow{4}{9}{%
\begin{varwidth}[t]{\sphinxcolwidth{1}{4}}
\begin{itemize}
\item {} 
\sphinxAtStartPar
layers

\end{itemize}
\par
\vskip-\baselineskip\vbox{\hbox{\strut}}\end{varwidth}%
}%
&
\sphinxAtStartPar
type
&\sphinxstartmulticolumn{2}%
\begin{varwidth}[t]{\sphinxcolwidth{2}{4}}
\sphinxAtStartPar
\sphinxstyleemphasis{array}
\par
\vskip-\baselineskip\vbox{\hbox{\strut}}\end{varwidth}%
\sphinxstopmulticolumn
\\
\cline{2-4}\sphinxtablestrut{9}&
\sphinxAtStartPar
default
&\sphinxstartmulticolumn{2}%
\begin{varwidth}[t]{\sphinxcolwidth{2}{4}}
\sphinxAtStartPar
{[}{]}
\par
\vskip-\baselineskip\vbox{\hbox{\strut}}\end{varwidth}%
\sphinxstopmulticolumn
\\
\cline{2-4}\sphinxtablestrut{9}&\sphinxstartmulticolumn{3}%
\begin{varwidth}[t]{\sphinxcolwidth{3}{4}}
\sphinxAtStartPar
items
\par
\vskip-\baselineskip\vbox{\hbox{\strut}}\end{varwidth}%
\sphinxstopmulticolumn
\\
\cline{2-4}\sphinxtablestrut{9}&\begin{itemize}
\item {} 
\end{itemize}
&\sphinxstartmulticolumn{2}%
\begin{varwidth}[t]{\sphinxcolwidth{2}{4}}
\sphinxAtStartPar
{\hyperref[\detokenize{docs/starting/projects:layer}]{\sphinxcrossref{Layer}}}
\par
\vskip-\baselineskip\vbox{\hbox{\strut}}\end{varwidth}%
\sphinxstopmulticolumn
\\
\hline\sphinxmultirow{3}{17}{%
\begin{varwidth}[t]{\sphinxcolwidth{1}{4}}
\begin{itemize}
\item {} 
\sphinxAtStartPar
layerOpacities

\end{itemize}
\par
\vskip-\baselineskip\vbox{\hbox{\strut}}\end{varwidth}%
}%
&
\sphinxAtStartPar
type
&\sphinxstartmulticolumn{2}%
\begin{varwidth}[t]{\sphinxcolwidth{2}{4}}
\sphinxAtStartPar
\sphinxstyleemphasis{object}
\par
\vskip-\baselineskip\vbox{\hbox{\strut}}\end{varwidth}%
\sphinxstopmulticolumn
\\
\cline{2-4}\sphinxtablestrut{17}&\sphinxstartmulticolumn{3}%
\begin{varwidth}[t]{\sphinxcolwidth{3}{4}}
\sphinxAtStartPar
patternProperties
\par
\vskip-\baselineskip\vbox{\hbox{\strut}}\end{varwidth}%
\sphinxstopmulticolumn
\\
\cline{2-4}\sphinxtablestrut{17}&\begin{itemize}
\item {} 
\sphinxAtStartPar
\textasciicircum{}{[}0\sphinxhyphen{}9{]}+\$

\end{itemize}
&
\sphinxAtStartPar
type
&
\sphinxAtStartPar
\sphinxstyleemphasis{integer}
\\
\hline\sphinxmultirow{3}{24}{%
\begin{varwidth}[t]{\sphinxcolwidth{1}{4}}
\begin{itemize}
\item {} 
\sphinxAtStartPar
layerVisibilities

\end{itemize}
\par
\vskip-\baselineskip\vbox{\hbox{\strut}}\end{varwidth}%
}%
&
\sphinxAtStartPar
type
&\sphinxstartmulticolumn{2}%
\begin{varwidth}[t]{\sphinxcolwidth{2}{4}}
\sphinxAtStartPar
\sphinxstyleemphasis{object}
\par
\vskip-\baselineskip\vbox{\hbox{\strut}}\end{varwidth}%
\sphinxstopmulticolumn
\\
\cline{2-4}\sphinxtablestrut{24}&\sphinxstartmulticolumn{3}%
\begin{varwidth}[t]{\sphinxcolwidth{3}{4}}
\sphinxAtStartPar
patternProperties
\par
\vskip-\baselineskip\vbox{\hbox{\strut}}\end{varwidth}%
\sphinxstopmulticolumn
\\
\cline{2-4}\sphinxtablestrut{24}&\begin{itemize}
\item {} 
\sphinxAtStartPar
\textasciicircum{}{[}0\sphinxhyphen{}9{]}+\$

\end{itemize}
&
\sphinxAtStartPar
type
&
\sphinxAtStartPar
\sphinxstyleemphasis{boolean}
\\
\hline\sphinxmultirow{3}{31}{%
\begin{varwidth}[t]{\sphinxcolwidth{1}{4}}
\begin{itemize}
\item {} 
\sphinxAtStartPar
layerFilters

\end{itemize}
\par
\vskip-\baselineskip\vbox{\hbox{\strut}}\end{varwidth}%
}%
&
\sphinxAtStartPar
type
&\sphinxstartmulticolumn{2}%
\begin{varwidth}[t]{\sphinxcolwidth{2}{4}}
\sphinxAtStartPar
\sphinxstyleemphasis{object}
\par
\vskip-\baselineskip\vbox{\hbox{\strut}}\end{varwidth}%
\sphinxstopmulticolumn
\\
\cline{2-4}\sphinxtablestrut{31}&\sphinxstartmulticolumn{3}%
\begin{varwidth}[t]{\sphinxcolwidth{3}{4}}
\sphinxAtStartPar
patternProperties
\par
\vskip-\baselineskip\vbox{\hbox{\strut}}\end{varwidth}%
\sphinxstopmulticolumn
\\
\cline{2-4}\sphinxtablestrut{31}&\begin{itemize}
\item {} 
\sphinxAtStartPar
\textasciicircum{}{[}0\sphinxhyphen{}9{]}+\$

\end{itemize}
&\sphinxstartmulticolumn{2}%
\begin{varwidth}[t]{\sphinxcolwidth{2}{4}}
\sphinxAtStartPar
{\hyperref[\detokenize{docs/starting/projects:layerfilter}]{\sphinxcrossref{LayerFilter}}}
\par
\vskip-\baselineskip\vbox{\hbox{\strut}}\end{varwidth}%
\sphinxstopmulticolumn
\\
\hline\sphinxmultirow{5}{37}{%
\begin{varwidth}[t]{\sphinxcolwidth{1}{4}}
\begin{itemize}
\item {} 
\sphinxAtStartPar
filters

\end{itemize}
\par
\vskip-\baselineskip\vbox{\hbox{\strut}}\end{varwidth}%
}%
&\sphinxstartmulticolumn{3}%
\begin{varwidth}[t]{\sphinxcolwidth{3}{4}}
\sphinxAtStartPar
List of filters shown as active filters in the GUI under the Image layers tab
\par
\vskip-\baselineskip\vbox{\hbox{\strut}}\end{varwidth}%
\sphinxstopmulticolumn
\\
\cline{2-4}\sphinxtablestrut{37}&
\sphinxAtStartPar
type
&\sphinxstartmulticolumn{2}%
\begin{varwidth}[t]{\sphinxcolwidth{2}{4}}
\sphinxAtStartPar
\sphinxstyleemphasis{array}
\par
\vskip-\baselineskip\vbox{\hbox{\strut}}\end{varwidth}%
\sphinxstopmulticolumn
\\
\cline{2-4}\sphinxtablestrut{37}&
\sphinxAtStartPar
default
&\sphinxstartmulticolumn{2}%
\begin{varwidth}[t]{\sphinxcolwidth{2}{4}}
\sphinxAtStartPar
{[}“Saturation”, “Brightness”, “Contrast”{]}
\par
\vskip-\baselineskip\vbox{\hbox{\strut}}\end{varwidth}%
\sphinxstopmulticolumn
\\
\cline{2-4}\sphinxtablestrut{37}&\sphinxstartmulticolumn{3}%
\begin{varwidth}[t]{\sphinxcolwidth{3}{4}}
\sphinxAtStartPar
items
\par
\vskip-\baselineskip\vbox{\hbox{\strut}}\end{varwidth}%
\sphinxstopmulticolumn
\\
\cline{2-4}\sphinxtablestrut{37}&\begin{itemize}
\item {} 
\end{itemize}
&\sphinxstartmulticolumn{2}%
\begin{varwidth}[t]{\sphinxcolwidth{2}{4}}
\sphinxAtStartPar
{\hyperref[\detokenize{docs/starting/projects:filter}]{\sphinxcrossref{\DUrole{std,std-ref}{Filter}}}}
\par
\vskip-\baselineskip\vbox{\hbox{\strut}}\end{varwidth}%
\sphinxstopmulticolumn
\\
\hline\sphinxmultirow{3}{46}{%
\begin{varwidth}[t]{\sphinxcolwidth{1}{4}}
\begin{itemize}
\item {} 
\sphinxAtStartPar
compositeMode

\end{itemize}
\par
\vskip-\baselineskip\vbox{\hbox{\strut}}\end{varwidth}%
}%
&\sphinxstartmulticolumn{3}%
\begin{varwidth}[t]{\sphinxcolwidth{3}{4}}
\sphinxAtStartPar
Mode defining how image layers will be merged (composited) with each other. Valid string values are “source\sphinxhyphen{}over” and “lighter”, which correspond to ‘Channels’ and ‘Composite’ in the GUI.
\par
\vskip-\baselineskip\vbox{\hbox{\strut}}\end{varwidth}%
\sphinxstopmulticolumn
\\
\cline{2-4}\sphinxtablestrut{46}&
\sphinxAtStartPar
type
&\sphinxstartmulticolumn{2}%
\begin{varwidth}[t]{\sphinxcolwidth{2}{4}}
\sphinxAtStartPar
\sphinxstyleemphasis{string}
\par
\vskip-\baselineskip\vbox{\hbox{\strut}}\end{varwidth}%
\sphinxstopmulticolumn
\\
\cline{2-4}\sphinxtablestrut{46}&
\sphinxAtStartPar
default
&\sphinxstartmulticolumn{2}%
\begin{varwidth}[t]{\sphinxcolwidth{2}{4}}
\sphinxAtStartPar
source\sphinxhyphen{}over
\par
\vskip-\baselineskip\vbox{\hbox{\strut}}\end{varwidth}%
\sphinxstopmulticolumn
\\
\hline\sphinxmultirow{4}{52}{%
\begin{varwidth}[t]{\sphinxcolwidth{1}{4}}
\begin{itemize}
\item {} 
\sphinxAtStartPar
markerFiles

\end{itemize}
\par
\vskip-\baselineskip\vbox{\hbox{\strut}}\end{varwidth}%
}%
&
\sphinxAtStartPar
type
&\sphinxstartmulticolumn{2}%
\begin{varwidth}[t]{\sphinxcolwidth{2}{4}}
\sphinxAtStartPar
\sphinxstyleemphasis{array}
\par
\vskip-\baselineskip\vbox{\hbox{\strut}}\end{varwidth}%
\sphinxstopmulticolumn
\\
\cline{2-4}\sphinxtablestrut{52}&
\sphinxAtStartPar
default
&\sphinxstartmulticolumn{2}%
\begin{varwidth}[t]{\sphinxcolwidth{2}{4}}
\sphinxAtStartPar
{[}{]}
\par
\vskip-\baselineskip\vbox{\hbox{\strut}}\end{varwidth}%
\sphinxstopmulticolumn
\\
\cline{2-4}\sphinxtablestrut{52}&\sphinxstartmulticolumn{3}%
\begin{varwidth}[t]{\sphinxcolwidth{3}{4}}
\sphinxAtStartPar
items
\par
\vskip-\baselineskip\vbox{\hbox{\strut}}\end{varwidth}%
\sphinxstopmulticolumn
\\
\cline{2-4}\sphinxtablestrut{52}&\begin{itemize}
\item {} 
\end{itemize}
&\sphinxstartmulticolumn{2}%
\begin{varwidth}[t]{\sphinxcolwidth{2}{4}}
\sphinxAtStartPar
{\hyperref[\detokenize{docs/starting/projects:markerfile}]{\sphinxcrossref{MarkerFile}}}
\par
\vskip-\baselineskip\vbox{\hbox{\strut}}\end{varwidth}%
\sphinxstopmulticolumn
\\
\hline\sphinxmultirow{3}{60}{%
\begin{varwidth}[t]{\sphinxcolwidth{1}{4}}
\begin{itemize}
\item {} 
\sphinxAtStartPar
regions

\end{itemize}
\par
\vskip-\baselineskip\vbox{\hbox{\strut}}\end{varwidth}%
}%
&\sphinxstartmulticolumn{3}%
\begin{varwidth}[t]{\sphinxcolwidth{3}{4}}
\sphinxAtStartPar
GeoJSON object, see {\hyperref[\detokenize{docs/starting/regions:regions}]{\sphinxcrossref{\DUrole{std,std-ref}{Regions section}}}}.
\par
\vskip-\baselineskip\vbox{\hbox{\strut}}\end{varwidth}%
\sphinxstopmulticolumn
\\
\cline{2-4}\sphinxtablestrut{60}&
\sphinxAtStartPar
type
&\sphinxstartmulticolumn{2}%
\begin{varwidth}[t]{\sphinxcolwidth{2}{4}}
\sphinxAtStartPar
\sphinxstyleemphasis{object}
\par
\vskip-\baselineskip\vbox{\hbox{\strut}}\end{varwidth}%
\sphinxstopmulticolumn
\\
\cline{2-4}\sphinxtablestrut{60}&
\sphinxAtStartPar
default
&\sphinxstartmulticolumn{2}%
\begin{varwidth}[t]{\sphinxcolwidth{2}{4}}
\sphinxAtStartPar
\{\}
\par
\vskip-\baselineskip\vbox{\hbox{\strut}}\end{varwidth}%
\sphinxstopmulticolumn
\\
\hline\sphinxmultirow{2}{66}{%
\begin{varwidth}[t]{\sphinxcolwidth{1}{4}}
\begin{itemize}
\item {} 
\sphinxAtStartPar
regionFile

\end{itemize}
\par
\vskip-\baselineskip\vbox{\hbox{\strut}}\end{varwidth}%
}%
&
\sphinxAtStartPar
type
&\sphinxstartmulticolumn{2}%
\begin{varwidth}[t]{\sphinxcolwidth{2}{4}}
\sphinxAtStartPar
\sphinxstyleemphasis{string}
\par
\vskip-\baselineskip\vbox{\hbox{\strut}}\end{varwidth}%
\sphinxstopmulticolumn
\\
\cline{2-4}\sphinxtablestrut{66}&
\sphinxAtStartPar
default
&\sphinxstartmulticolumn{2}%
\begin{varwidth}[t]{\sphinxcolwidth{2}{4}}
\par
\vskip-\baselineskip\vbox{\hbox{\strut}}\end{varwidth}%
\sphinxstopmulticolumn
\\
\hline\sphinxmultirow{3}{71}{%
\begin{varwidth}[t]{\sphinxcolwidth{1}{4}}
\begin{itemize}
\item {} 
\sphinxAtStartPar
regionFiles

\end{itemize}
\par
\vskip-\baselineskip\vbox{\hbox{\strut}}\end{varwidth}%
}%
&
\sphinxAtStartPar
type
&\sphinxstartmulticolumn{2}%
\begin{varwidth}[t]{\sphinxcolwidth{2}{4}}
\sphinxAtStartPar
\sphinxstyleemphasis{array}
\par
\vskip-\baselineskip\vbox{\hbox{\strut}}\end{varwidth}%
\sphinxstopmulticolumn
\\
\cline{2-4}\sphinxtablestrut{71}&
\sphinxAtStartPar
default
&\sphinxstartmulticolumn{2}%
\begin{varwidth}[t]{\sphinxcolwidth{2}{4}}
\sphinxAtStartPar
{[}{]}
\par
\vskip-\baselineskip\vbox{\hbox{\strut}}\end{varwidth}%
\sphinxstopmulticolumn
\\
\cline{2-4}\sphinxtablestrut{71}&\sphinxstartmulticolumn{3}%
\begin{varwidth}[t]{\sphinxcolwidth{3}{4}}
\sphinxAtStartPar
items
\par
\vskip-\baselineskip\vbox{\hbox{\strut}}\end{varwidth}%
\sphinxstopmulticolumn
\\
\hline\sphinxmultirow{5}{77}{%
\begin{varwidth}[t]{\sphinxcolwidth{1}{4}}
\begin{itemize}
\item {} 
\sphinxAtStartPar
plugins

\end{itemize}
\par
\vskip-\baselineskip\vbox{\hbox{\strut}}\end{varwidth}%
}%
&\sphinxstartmulticolumn{3}%
\begin{varwidth}[t]{\sphinxcolwidth{3}{4}}
\sphinxAtStartPar
List of plugins to load with the project. See also the {\hyperref[\detokenize{docs/starting/plugins:plugins}]{\sphinxcrossref{\DUrole{std,std-ref}{Plugins section}}}}.
\par
\vskip-\baselineskip\vbox{\hbox{\strut}}\end{varwidth}%
\sphinxstopmulticolumn
\\
\cline{2-4}\sphinxtablestrut{77}&
\sphinxAtStartPar
type
&\sphinxstartmulticolumn{2}%
\begin{varwidth}[t]{\sphinxcolwidth{2}{4}}
\sphinxAtStartPar
\sphinxstyleemphasis{array}
\par
\vskip-\baselineskip\vbox{\hbox{\strut}}\end{varwidth}%
\sphinxstopmulticolumn
\\
\cline{2-4}\sphinxtablestrut{77}&
\sphinxAtStartPar
default
&\sphinxstartmulticolumn{2}%
\begin{varwidth}[t]{\sphinxcolwidth{2}{4}}
\sphinxAtStartPar
{[}{]}
\par
\vskip-\baselineskip\vbox{\hbox{\strut}}\end{varwidth}%
\sphinxstopmulticolumn
\\
\cline{2-4}\sphinxtablestrut{77}&\sphinxstartmulticolumn{3}%
\begin{varwidth}[t]{\sphinxcolwidth{3}{4}}
\sphinxAtStartPar
items
\par
\vskip-\baselineskip\vbox{\hbox{\strut}}\end{varwidth}%
\sphinxstopmulticolumn
\\
\cline{2-4}\sphinxtablestrut{77}&\begin{itemize}
\item {} 
\end{itemize}
&
\sphinxAtStartPar
type
&
\sphinxAtStartPar
\sphinxstyleemphasis{string}
\\
\hline\sphinxmultirow{3}{87}{%
\begin{varwidth}[t]{\sphinxcolwidth{1}{4}}
\begin{itemize}
\item {} 
\sphinxAtStartPar
hideTabs

\end{itemize}
\par
\vskip-\baselineskip\vbox{\hbox{\strut}}\end{varwidth}%
}%
&\sphinxstartmulticolumn{3}%
\begin{varwidth}[t]{\sphinxcolwidth{3}{4}}
\sphinxAtStartPar
Hide tabs of markers dataset. Only use when you have a unique marker tab.
\par
\vskip-\baselineskip\vbox{\hbox{\strut}}\end{varwidth}%
\sphinxstopmulticolumn
\\
\cline{2-4}\sphinxtablestrut{87}&
\sphinxAtStartPar
type
&\sphinxstartmulticolumn{2}%
\begin{varwidth}[t]{\sphinxcolwidth{2}{4}}
\sphinxAtStartPar
\sphinxstyleemphasis{boolean}
\par
\vskip-\baselineskip\vbox{\hbox{\strut}}\end{varwidth}%
\sphinxstopmulticolumn
\\
\cline{2-4}\sphinxtablestrut{87}&
\sphinxAtStartPar
default
&\sphinxstartmulticolumn{2}%
\begin{varwidth}[t]{\sphinxcolwidth{2}{4}}
\sphinxAtStartPar
false
\par
\vskip-\baselineskip\vbox{\hbox{\strut}}\end{varwidth}%
\sphinxstopmulticolumn
\\
\hline\sphinxmultirow{4}{93}{%
\begin{varwidth}[t]{\sphinxcolwidth{1}{4}}
\begin{itemize}
\item {} 
\sphinxAtStartPar
settings

\end{itemize}
\par
\vskip-\baselineskip\vbox{\hbox{\strut}}\end{varwidth}%
}%
&
\sphinxAtStartPar
type
&\sphinxstartmulticolumn{2}%
\begin{varwidth}[t]{\sphinxcolwidth{2}{4}}
\sphinxAtStartPar
\sphinxstyleemphasis{array}
\par
\vskip-\baselineskip\vbox{\hbox{\strut}}\end{varwidth}%
\sphinxstopmulticolumn
\\
\cline{2-4}\sphinxtablestrut{93}&
\sphinxAtStartPar
default
&\sphinxstartmulticolumn{2}%
\begin{varwidth}[t]{\sphinxcolwidth{2}{4}}
\sphinxAtStartPar
{[}{]}
\par
\vskip-\baselineskip\vbox{\hbox{\strut}}\end{varwidth}%
\sphinxstopmulticolumn
\\
\cline{2-4}\sphinxtablestrut{93}&\sphinxstartmulticolumn{3}%
\begin{varwidth}[t]{\sphinxcolwidth{3}{4}}
\sphinxAtStartPar
items
\par
\vskip-\baselineskip\vbox{\hbox{\strut}}\end{varwidth}%
\sphinxstopmulticolumn
\\
\cline{2-4}\sphinxtablestrut{93}&\begin{itemize}
\item {} 
\end{itemize}
&\sphinxstartmulticolumn{2}%
\begin{varwidth}[t]{\sphinxcolwidth{2}{4}}
\sphinxAtStartPar
{\hyperref[\detokenize{docs/starting/projects:setting}]{\sphinxcrossref{Setting}}}
\par
\vskip-\baselineskip\vbox{\hbox{\strut}}\end{varwidth}%
\sphinxstopmulticolumn
\\
\hline
\end{longtable}\sphinxatlongtableend\end{savenotes}


\paragraph{Layer}
\label{\detokenize{docs/starting/projects:layer}}\label{\detokenize{docs/starting/projects:layer}}

\begin{savenotes}\sphinxattablestart
\centering
\begin{tabular}[t]{|*{3}{\X{1}{3}|}}
\hline
\sphinxstartmulticolumn{3}%
\begin{varwidth}[t]{\sphinxcolwidth{3}{3}}
\sphinxAtStartPar
Description of an image layer. Required properties are shown in \sphinxstylestrong{bold} text
\par
\vskip-\baselineskip\vbox{\hbox{\strut}}\end{varwidth}%
\sphinxstopmulticolumn
\\
\hline
\sphinxAtStartPar
type
&\sphinxstartmulticolumn{2}%
\begin{varwidth}[t]{\sphinxcolwidth{2}{3}}
\sphinxAtStartPar
\sphinxstyleemphasis{object}
\par
\vskip-\baselineskip\vbox{\hbox{\strut}}\end{varwidth}%
\sphinxstopmulticolumn
\\
\hline\sphinxstartmulticolumn{3}%
\begin{varwidth}[t]{\sphinxcolwidth{3}{3}}
\sphinxAtStartPar
properties
\par
\vskip-\baselineskip\vbox{\hbox{\strut}}\end{varwidth}%
\sphinxstopmulticolumn
\\
\hline\sphinxmultirow{2}{5}{%
\begin{varwidth}[t]{\sphinxcolwidth{1}{3}}
\begin{itemize}
\item {} 
\sphinxAtStartPar
\sphinxstylestrong{name}

\end{itemize}
\par
\vskip-\baselineskip\vbox{\hbox{\strut}}\end{varwidth}%
}%
&\sphinxstartmulticolumn{2}%
\begin{varwidth}[t]{\sphinxcolwidth{2}{3}}
\sphinxAtStartPar
Name of the image layer
\par
\vskip-\baselineskip\vbox{\hbox{\strut}}\end{varwidth}%
\sphinxstopmulticolumn
\\
\cline{2-3}\sphinxtablestrut{5}&
\sphinxAtStartPar
type
&
\sphinxAtStartPar
\sphinxstyleemphasis{string}
\\
\hline\sphinxmultirow{2}{9}{%
\begin{varwidth}[t]{\sphinxcolwidth{1}{3}}
\begin{itemize}
\item {} 
\sphinxAtStartPar
\sphinxstylestrong{tileSource}

\end{itemize}
\par
\vskip-\baselineskip\vbox{\hbox{\strut}}\end{varwidth}%
}%
&\sphinxstartmulticolumn{2}%
\begin{varwidth}[t]{\sphinxcolwidth{2}{3}}
\sphinxAtStartPar
Relative path to an image file in a supported format. See also the {\hyperref[\detokenize{docs/starting/images:images}]{\sphinxcrossref{\DUrole{std,std-ref}{Images section}}}}.
\par
\vskip-\baselineskip\vbox{\hbox{\strut}}\end{varwidth}%
\sphinxstopmulticolumn
\\
\cline{2-3}\sphinxtablestrut{9}&
\sphinxAtStartPar
type
&
\sphinxAtStartPar
\sphinxstyleemphasis{string}
\\
\hline
\end{tabular}
\par
\sphinxattableend\end{savenotes}


\paragraph{LayerFilter}
\label{\detokenize{docs/starting/projects:layerfilter}}\label{\detokenize{docs/starting/projects:layerfilter}}

\begin{savenotes}\sphinxattablestart
\centering
\begin{tabular}[t]{|*{4}{\X{1}{4}|}}
\hline
\sphinxstartmulticolumn{4}%
\begin{varwidth}[t]{\sphinxcolwidth{4}{4}}
\sphinxAtStartPar
Description of an image filter to be applied to the pixels in an image layer. Required properties are shown in \sphinxstylestrong{bold} text
\par
\vskip-\baselineskip\vbox{\hbox{\strut}}\end{varwidth}%
\sphinxstopmulticolumn
\\
\hline
\sphinxAtStartPar
type
&\sphinxstartmulticolumn{3}%
\begin{varwidth}[t]{\sphinxcolwidth{3}{4}}
\sphinxAtStartPar
\sphinxstyleemphasis{array}
\par
\vskip-\baselineskip\vbox{\hbox{\strut}}\end{varwidth}%
\sphinxstopmulticolumn
\\
\hline\sphinxstartmulticolumn{4}%
\begin{varwidth}[t]{\sphinxcolwidth{4}{4}}
\sphinxAtStartPar
items
\par
\vskip-\baselineskip\vbox{\hbox{\strut}}\end{varwidth}%
\sphinxstopmulticolumn
\\
\hline\sphinxmultirow{6}{5}{%
\begin{varwidth}[t]{\sphinxcolwidth{1}{4}}
\begin{itemize}
\item {} 
\end{itemize}
\par
\vskip-\baselineskip\vbox{\hbox{\strut}}\end{varwidth}%
}%
&
\sphinxAtStartPar
type
&\sphinxstartmulticolumn{2}%
\begin{varwidth}[t]{\sphinxcolwidth{2}{4}}
\sphinxAtStartPar
\sphinxstyleemphasis{object}
\par
\vskip-\baselineskip\vbox{\hbox{\strut}}\end{varwidth}%
\sphinxstopmulticolumn
\\
\cline{2-4}\sphinxtablestrut{5}&\sphinxstartmulticolumn{3}%
\begin{varwidth}[t]{\sphinxcolwidth{3}{4}}
\sphinxAtStartPar
properties
\par
\vskip-\baselineskip\vbox{\hbox{\strut}}\end{varwidth}%
\sphinxstopmulticolumn
\\
\cline{2-4}\sphinxtablestrut{5}&\sphinxmultirow{2}{9}{%
\begin{varwidth}[t]{\sphinxcolwidth{1}{4}}
\begin{itemize}
\item {} 
\sphinxAtStartPar
\sphinxstylestrong{name}

\end{itemize}
\par
\vskip-\baselineskip\vbox{\hbox{\strut}}\end{varwidth}%
}%
&\sphinxstartmulticolumn{2}%
\begin{varwidth}[t]{\sphinxcolwidth{2}{4}}
\sphinxAtStartPar
Filter name. See {\hyperref[\detokenize{docs/starting/projects:filter}]{\sphinxcrossref{\DUrole{std,std-ref}{Filter}}}} for more details.
\par
\vskip-\baselineskip\vbox{\hbox{\strut}}\end{varwidth}%
\sphinxstopmulticolumn
\\
\cline{3-4}\sphinxtablestrut{5}&\sphinxtablestrut{9}&
\sphinxAtStartPar
type
&
\sphinxAtStartPar
\sphinxstyleemphasis{string}
\\
\cline{2-4}\sphinxtablestrut{5}&\sphinxmultirow{2}{13}{%
\begin{varwidth}[t]{\sphinxcolwidth{1}{4}}
\begin{itemize}
\item {} 
\sphinxAtStartPar
\sphinxstylestrong{value}

\end{itemize}
\par
\vskip-\baselineskip\vbox{\hbox{\strut}}\end{varwidth}%
}%
&\sphinxstartmulticolumn{2}%
\begin{varwidth}[t]{\sphinxcolwidth{2}{4}}
\sphinxAtStartPar
Filter parameter. See {\hyperref[\detokenize{docs/starting/projects:filter}]{\sphinxcrossref{\DUrole{std,std-ref}{Filter}}}} for more details.
\par
\vskip-\baselineskip\vbox{\hbox{\strut}}\end{varwidth}%
\sphinxstopmulticolumn
\\
\cline{3-4}\sphinxtablestrut{5}&\sphinxtablestrut{13}&
\sphinxAtStartPar
type
&
\sphinxAtStartPar
\sphinxstyleemphasis{string}
\\
\hline
\end{tabular}
\par
\sphinxattableend\end{savenotes}


\paragraph{Filter}
\label{\detokenize{docs/starting/projects:filter}}\label{\detokenize{docs/starting/projects:filter}}

\begin{savenotes}\sphinxattablestart
\centering
\begin{tabulary}{\linewidth}[t]{|T|T|}
\hline
\sphinxstartmulticolumn{2}%
\begin{varwidth}[t]{\sphinxcolwidth{2}{2}}
\sphinxAtStartPar
TissUUmaps supports most filters available in OpenSeadragon via the \sphinxurl{https://github.com/usnistgov/OpenSeadragonFiltering} plugin.
\par
\vskip-\baselineskip\vbox{\hbox{\strut}}\end{varwidth}%
\sphinxstopmulticolumn
\\
\hline
\sphinxAtStartPar
enum
&
\sphinxAtStartPar
Color, Brightness, Exposure, Hue, Contrast, Vibrance, Noise, Saturation, Gamma, Invert, Greyscale, Threshold, Erosion, Dilation
\\
\hline
\end{tabulary}
\par
\sphinxattableend\end{savenotes}


\paragraph{ColorScale}
\label{\detokenize{docs/starting/projects:colorscale}}\label{\detokenize{docs/starting/projects:colorscale}}

\begin{savenotes}\sphinxattablestart
\centering
\begin{tabulary}{\linewidth}[t]{|T|T|}
\hline
\sphinxstartmulticolumn{2}%
\begin{varwidth}[t]{\sphinxcolwidth{2}{2}}
\sphinxAtStartPar
TissUUmaps supports most of the color scales available in the D3.js library. See \sphinxurl{https://github.com/d3/d3-scale-chromatic} for reference. Note: the colors for ‘interpolateRainbow’ are currently overridden by a custom Turbo\sphinxhyphen{}like color scale in version 3.0.x of TissUUmaps.
\par
\vskip-\baselineskip\vbox{\hbox{\strut}}\end{varwidth}%
\sphinxstopmulticolumn
\\
\hline
\sphinxAtStartPar
enum
&
\sphinxAtStartPar
interpolateCubehelixDefault, interpolateRainbow, interpolateWarm, interpolateCool, interpolateViridis, interpolateMagma, interpolateInferno, interpolatePlasma, interpolateBlues, interpolateBrBG, interpolateBuGn, interpolateBuPu, interpolateCividis, interpolateGnBu, interpolateGreens, interpolateGreys, interpolateOrRd, interpolateOranges, interpolatePRGn, interpolatePiYG, interpolatePuBu, interpolatePuBuGn, interpolatePuOr, interpolatePuRd, interpolatePurples, interpolateRdBu, interpolateRdGy, interpolateRdPu, interpolateRdYlBu, interpolateRdYlGn, interpolateReds, interpolateSinebow, interpolateSpectral, interpolateTurbo, interpolateYlGn, interpolateYlGnBu, interpolateYlOrBr, interpolateYlOrRd
\\
\hline
\end{tabulary}
\par
\sphinxattableend\end{savenotes}


\paragraph{Shape}
\label{\detokenize{docs/starting/projects:shape}}\label{\detokenize{docs/starting/projects:shape}}

\begin{savenotes}\sphinxattablestart
\centering
\begin{tabulary}{\linewidth}[t]{|T|T|}
\hline
\sphinxstartmulticolumn{2}%
\begin{varwidth}[t]{\sphinxcolwidth{2}{2}}
\sphinxAtStartPar
TissUUmaps supports most of the marker shapes that are also used by the Napari software, \sphinxurl{https://napari.org}. In addition to the name strings listed below, shape can also be specified by a corresponding index in range 0\sphinxhyphen{}13.
\par
\vskip-\baselineskip\vbox{\hbox{\strut}}\end{varwidth}%
\sphinxstopmulticolumn
\\
\hline
\sphinxAtStartPar
enum
&
\sphinxAtStartPar
cross, diamond, square, triangle up, star, clobber, disc, hbar, vbar, tailed arrow, triangle down, ring, x, arrow
\\
\hline
\end{tabulary}
\par
\sphinxattableend\end{savenotes}


\paragraph{MarkerFile}
\label{\detokenize{docs/starting/projects:markerfile}}\label{\detokenize{docs/starting/projects:markerfile}}

\begin{savenotes}\sphinxattablestart
\centering
\begin{tabular}[t]{|*{3}{\X{1}{3}|}}
\hline
\sphinxstartmulticolumn{3}%
\begin{varwidth}[t]{\sphinxcolwidth{3}{3}}
\sphinxAtStartPar
Description of settings and GUI objects for a marker dataset loaded from CSV file. Required properties are shown in \sphinxstylestrong{bold} text.
\par
\vskip-\baselineskip\vbox{\hbox{\strut}}\end{varwidth}%
\sphinxstopmulticolumn
\\
\hline
\sphinxAtStartPar
type
&\sphinxstartmulticolumn{2}%
\begin{varwidth}[t]{\sphinxcolwidth{2}{3}}
\sphinxAtStartPar
\sphinxstyleemphasis{object}
\par
\vskip-\baselineskip\vbox{\hbox{\strut}}\end{varwidth}%
\sphinxstopmulticolumn
\\
\hline\sphinxstartmulticolumn{3}%
\begin{varwidth}[t]{\sphinxcolwidth{3}{3}}
\sphinxAtStartPar
properties
\par
\vskip-\baselineskip\vbox{\hbox{\strut}}\end{varwidth}%
\sphinxstopmulticolumn
\\
\hline\sphinxmultirow{2}{5}{%
\begin{varwidth}[t]{\sphinxcolwidth{1}{3}}
\begin{itemize}
\item {} 
\sphinxAtStartPar
\sphinxstylestrong{title}

\end{itemize}
\par
\vskip-\baselineskip\vbox{\hbox{\strut}}\end{varwidth}%
}%
&\sphinxstartmulticolumn{2}%
\begin{varwidth}[t]{\sphinxcolwidth{2}{3}}
\sphinxAtStartPar
Name of marker button
\par
\vskip-\baselineskip\vbox{\hbox{\strut}}\end{varwidth}%
\sphinxstopmulticolumn
\\
\cline{2-3}\sphinxtablestrut{5}&
\sphinxAtStartPar
type
&
\sphinxAtStartPar
\sphinxstyleemphasis{string}
\\
\hline\sphinxmultirow{3}{9}{%
\begin{varwidth}[t]{\sphinxcolwidth{1}{3}}
\begin{itemize}
\item {} 
\sphinxAtStartPar
comment

\end{itemize}
\par
\vskip-\baselineskip\vbox{\hbox{\strut}}\end{varwidth}%
}%
&\sphinxstartmulticolumn{2}%
\begin{varwidth}[t]{\sphinxcolwidth{2}{3}}
\sphinxAtStartPar
Optional description text shown next to marker button
\par
\vskip-\baselineskip\vbox{\hbox{\strut}}\end{varwidth}%
\sphinxstopmulticolumn
\\
\cline{2-3}\sphinxtablestrut{9}&
\sphinxAtStartPar
type
&
\sphinxAtStartPar
\sphinxstyleemphasis{string}
\\
\cline{2-3}\sphinxtablestrut{9}&
\sphinxAtStartPar
default
&\\
\hline\sphinxmultirow{2}{15}{%
\begin{varwidth}[t]{\sphinxcolwidth{1}{3}}
\begin{itemize}
\item {} 
\sphinxAtStartPar
\sphinxstylestrong{name}

\end{itemize}
\par
\vskip-\baselineskip\vbox{\hbox{\strut}}\end{varwidth}%
}%
&\sphinxstartmulticolumn{2}%
\begin{varwidth}[t]{\sphinxcolwidth{2}{3}}
\sphinxAtStartPar
Name of marker tab
\par
\vskip-\baselineskip\vbox{\hbox{\strut}}\end{varwidth}%
\sphinxstopmulticolumn
\\
\cline{2-3}\sphinxtablestrut{15}&
\sphinxAtStartPar
type
&
\sphinxAtStartPar
\sphinxstyleemphasis{string}
\\
\hline\sphinxmultirow{3}{19}{%
\begin{varwidth}[t]{\sphinxcolwidth{1}{3}}
\begin{itemize}
\item {} 
\sphinxAtStartPar
autoLoad

\end{itemize}
\par
\vskip-\baselineskip\vbox{\hbox{\strut}}\end{varwidth}%
}%
&\sphinxstartmulticolumn{2}%
\begin{varwidth}[t]{\sphinxcolwidth{2}{3}}
\sphinxAtStartPar
If the CSV file for the marker dataset should be automatically loaded when the TMAP project is opened. If this is false, the user instead has to click on the marker button in the GUI to load the dataset.
\par
\vskip-\baselineskip\vbox{\hbox{\strut}}\end{varwidth}%
\sphinxstopmulticolumn
\\
\cline{2-3}\sphinxtablestrut{19}&
\sphinxAtStartPar
type
&
\sphinxAtStartPar
\sphinxstyleemphasis{boolean}
\\
\cline{2-3}\sphinxtablestrut{19}&
\sphinxAtStartPar
default
&
\sphinxAtStartPar
false
\\
\hline\sphinxmultirow{2}{25}{%
\begin{varwidth}[t]{\sphinxcolwidth{1}{3}}
\begin{itemize}
\item {} 
\sphinxAtStartPar
\sphinxstylestrong{uid}

\end{itemize}
\par
\vskip-\baselineskip\vbox{\hbox{\strut}}\end{varwidth}%
}%
&\sphinxstartmulticolumn{2}%
\begin{varwidth}[t]{\sphinxcolwidth{2}{3}}
\sphinxAtStartPar
A unique identifier used internally by TissUUmaps to reference the marker dataset
\par
\vskip-\baselineskip\vbox{\hbox{\strut}}\end{varwidth}%
\sphinxstopmulticolumn
\\
\cline{2-3}\sphinxtablestrut{25}&
\sphinxAtStartPar
type
&
\sphinxAtStartPar
\sphinxstyleemphasis{string}
\\
\hline\begin{itemize}
\item {} 
\sphinxAtStartPar
\sphinxstylestrong{expectedHeader}

\end{itemize}
&\sphinxstartmulticolumn{2}%
\begin{varwidth}[t]{\sphinxcolwidth{2}{3}}
\sphinxAtStartPar
{\hyperref[\detokenize{docs/starting/projects:expectedheader}]{\sphinxcrossref{\DUrole{std,std-ref}{ExpectedHeader}}}}
\par
\vskip-\baselineskip\vbox{\hbox{\strut}}\end{varwidth}%
\sphinxstopmulticolumn
\\
\hline\begin{itemize}
\item {} 
\sphinxAtStartPar
\sphinxstylestrong{expectedRadios}

\end{itemize}
&\sphinxstartmulticolumn{2}%
\begin{varwidth}[t]{\sphinxcolwidth{2}{3}}
\sphinxAtStartPar
{\hyperref[\detokenize{docs/starting/projects:expectedradios}]{\sphinxcrossref{\DUrole{std,std-ref}{ExpectedRadios}}}}
\par
\vskip-\baselineskip\vbox{\hbox{\strut}}\end{varwidth}%
\sphinxstopmulticolumn
\\
\hline\sphinxmultirow{2}{33}{%
\begin{varwidth}[t]{\sphinxcolwidth{1}{3}}
\begin{itemize}
\item {} 
\sphinxAtStartPar
\sphinxstylestrong{path}

\end{itemize}
\par
\vskip-\baselineskip\vbox{\hbox{\strut}}\end{varwidth}%
}%
&\sphinxstartmulticolumn{2}%
\begin{varwidth}[t]{\sphinxcolwidth{2}{3}}
\sphinxAtStartPar
Relative file path to CSV file in which marker data is stored
\par
\vskip-\baselineskip\vbox{\hbox{\strut}}\end{varwidth}%
\sphinxstopmulticolumn
\\
\cline{2-3}\sphinxtablestrut{33}&
\sphinxAtStartPar
type
&
\sphinxAtStartPar
\sphinxstyleemphasis{string}
\\
\hline\sphinxmultirow{4}{37}{%
\begin{varwidth}[t]{\sphinxcolwidth{1}{3}}
\begin{itemize}
\item {} 
\sphinxAtStartPar
settings

\end{itemize}
\par
\vskip-\baselineskip\vbox{\hbox{\strut}}\end{varwidth}%
}%
&
\sphinxAtStartPar
type
&
\sphinxAtStartPar
\sphinxstyleemphasis{array}
\\
\cline{2-3}\sphinxtablestrut{37}&
\sphinxAtStartPar
default
&
\sphinxAtStartPar
{[}{]}
\\
\cline{2-3}\sphinxtablestrut{37}&\sphinxstartmulticolumn{2}%
\begin{varwidth}[t]{\sphinxcolwidth{2}{3}}
\sphinxAtStartPar
items
\par
\vskip-\baselineskip\vbox{\hbox{\strut}}\end{varwidth}%
\sphinxstopmulticolumn
\\
\cline{2-3}\sphinxtablestrut{37}&\begin{itemize}
\item {} 
\end{itemize}
&
\sphinxAtStartPar
{\hyperref[\detokenize{docs/starting/projects:setting}]{\sphinxcrossref{Setting}}}
\\
\hline
\end{tabular}
\par
\sphinxattableend\end{savenotes}


\paragraph{ExpectedHeader}
\label{\detokenize{docs/starting/projects:expectedheader}}\label{\detokenize{docs/starting/projects:expectedheader}}

\begin{savenotes}\sphinxatlongtablestart\begin{longtable}[c]{|*{3}{\X{1}{3}|}}
\hline

\endfirsthead

\multicolumn{3}{c}%
{\makebox[0pt]{\sphinxtablecontinued{\tablename\ \thetable{} \textendash{} continued from previous page}}}\\
\hline

\endhead

\hline
\multicolumn{3}{r}{\makebox[0pt][r]{\sphinxtablecontinued{continues on next page}}}\\
\endfoot

\endlastfoot
\sphinxstartmulticolumn{3}%
\begin{varwidth}[t]{\sphinxcolwidth{3}{3}}
\sphinxAtStartPar
Input field values for settings in a marker tab. Required properties are shown in \sphinxstylestrong{bold} text.
\par
\vskip-\baselineskip\vbox{\hbox{\strut}}\end{varwidth}%
\sphinxstopmulticolumn
\\
\hline
\sphinxAtStartPar
type
&\sphinxstartmulticolumn{2}%
\begin{varwidth}[t]{\sphinxcolwidth{2}{3}}
\sphinxAtStartPar
\sphinxstyleemphasis{object}
\par
\vskip-\baselineskip\vbox{\hbox{\strut}}\end{varwidth}%
\sphinxstopmulticolumn
\\
\hline\sphinxstartmulticolumn{3}%
\begin{varwidth}[t]{\sphinxcolwidth{3}{3}}
\sphinxAtStartPar
properties
\par
\vskip-\baselineskip\vbox{\hbox{\strut}}\end{varwidth}%
\sphinxstopmulticolumn
\\
\hline\sphinxmultirow{2}{5}{%
\begin{varwidth}[t]{\sphinxcolwidth{1}{3}}
\begin{itemize}
\item {} 
\sphinxAtStartPar
\sphinxstylestrong{X}

\end{itemize}
\par
\vskip-\baselineskip\vbox{\hbox{\strut}}\end{varwidth}%
}%
&\sphinxstartmulticolumn{2}%
\begin{varwidth}[t]{\sphinxcolwidth{2}{3}}
\sphinxAtStartPar
Name of CSV column to use as X\sphinxhyphen{}coordinate
\par
\vskip-\baselineskip\vbox{\hbox{\strut}}\end{varwidth}%
\sphinxstopmulticolumn
\\
\cline{2-3}\sphinxtablestrut{5}&
\sphinxAtStartPar
type
&
\sphinxAtStartPar
\sphinxstyleemphasis{string}
\\
\hline\sphinxmultirow{2}{9}{%
\begin{varwidth}[t]{\sphinxcolwidth{1}{3}}
\begin{itemize}
\item {} 
\sphinxAtStartPar
\sphinxstylestrong{Y}

\end{itemize}
\par
\vskip-\baselineskip\vbox{\hbox{\strut}}\end{varwidth}%
}%
&\sphinxstartmulticolumn{2}%
\begin{varwidth}[t]{\sphinxcolwidth{2}{3}}
\sphinxAtStartPar
Name of CSV column to use as Y\sphinxhyphen{}coordinate
\par
\vskip-\baselineskip\vbox{\hbox{\strut}}\end{varwidth}%
\sphinxstopmulticolumn
\\
\cline{2-3}\sphinxtablestrut{9}&
\sphinxAtStartPar
type
&
\sphinxAtStartPar
\sphinxstyleemphasis{string}
\\
\hline\sphinxmultirow{3}{13}{%
\begin{varwidth}[t]{\sphinxcolwidth{1}{3}}
\begin{itemize}
\item {} 
\sphinxAtStartPar
gb\_col

\end{itemize}
\par
\vskip-\baselineskip\vbox{\hbox{\strut}}\end{varwidth}%
}%
&\sphinxstartmulticolumn{2}%
\begin{varwidth}[t]{\sphinxcolwidth{2}{3}}
\sphinxAtStartPar
Name of CSV column to use as key to group markers by
\par
\vskip-\baselineskip\vbox{\hbox{\strut}}\end{varwidth}%
\sphinxstopmulticolumn
\\
\cline{2-3}\sphinxtablestrut{13}&
\sphinxAtStartPar
type
&
\sphinxAtStartPar
\sphinxstyleemphasis{string}
\\
\cline{2-3}\sphinxtablestrut{13}&
\sphinxAtStartPar
default
&
\sphinxAtStartPar
null
\\
\hline\sphinxmultirow{3}{19}{%
\begin{varwidth}[t]{\sphinxcolwidth{1}{3}}
\begin{itemize}
\item {} 
\sphinxAtStartPar
gb\_name

\end{itemize}
\par
\vskip-\baselineskip\vbox{\hbox{\strut}}\end{varwidth}%
}%
&\sphinxstartmulticolumn{2}%
\begin{varwidth}[t]{\sphinxcolwidth{2}{3}}
\sphinxAtStartPar
Name of CSV column to display for groups instead of group key value
\par
\vskip-\baselineskip\vbox{\hbox{\strut}}\end{varwidth}%
\sphinxstopmulticolumn
\\
\cline{2-3}\sphinxtablestrut{19}&
\sphinxAtStartPar
type
&
\sphinxAtStartPar
\sphinxstyleemphasis{string}
\\
\cline{2-3}\sphinxtablestrut{19}&
\sphinxAtStartPar
default
&
\sphinxAtStartPar
null
\\
\hline\sphinxmultirow{3}{25}{%
\begin{varwidth}[t]{\sphinxcolwidth{1}{3}}
\begin{itemize}
\item {} 
\sphinxAtStartPar
cb\_cmap

\end{itemize}
\par
\vskip-\baselineskip\vbox{\hbox{\strut}}\end{varwidth}%
}%
&\sphinxstartmulticolumn{2}%
\begin{varwidth}[t]{\sphinxcolwidth{2}{3}}
\sphinxAtStartPar
Name of D3 color scale to be used for color mapping. See {\hyperref[\detokenize{docs/starting/projects:colorscale}]{\sphinxcrossref{\DUrole{std,std-ref}{ColorScale}}}} for valid string values.
\par
\vskip-\baselineskip\vbox{\hbox{\strut}}\end{varwidth}%
\sphinxstopmulticolumn
\\
\cline{2-3}\sphinxtablestrut{25}&
\sphinxAtStartPar
type
&
\sphinxAtStartPar
\sphinxstyleemphasis{string}
\\
\cline{2-3}\sphinxtablestrut{25}&
\sphinxAtStartPar
default
&\\
\hline\sphinxmultirow{3}{31}{%
\begin{varwidth}[t]{\sphinxcolwidth{1}{3}}
\begin{itemize}
\item {} 
\sphinxAtStartPar
cb\_col

\end{itemize}
\par
\vskip-\baselineskip\vbox{\hbox{\strut}}\end{varwidth}%
}%
&\sphinxstartmulticolumn{2}%
\begin{varwidth}[t]{\sphinxcolwidth{2}{3}}
\sphinxAtStartPar
Name of CSV column containing scalar values for color mapping or hexadecimal RGB colors in format ‘\#ff0000’
\par
\vskip-\baselineskip\vbox{\hbox{\strut}}\end{varwidth}%
\sphinxstopmulticolumn
\\
\cline{2-3}\sphinxtablestrut{31}&
\sphinxAtStartPar
type
&
\sphinxAtStartPar
\sphinxstyleemphasis{string}
\\
\cline{2-3}\sphinxtablestrut{31}&
\sphinxAtStartPar
default
&
\sphinxAtStartPar
null
\\
\hline\sphinxmultirow{3}{37}{%
\begin{varwidth}[t]{\sphinxcolwidth{1}{3}}
\begin{itemize}
\item {} 
\sphinxAtStartPar
cb\_gr\_dict

\end{itemize}
\par
\vskip-\baselineskip\vbox{\hbox{\strut}}\end{varwidth}%
}%
&\sphinxstartmulticolumn{2}%
\begin{varwidth}[t]{\sphinxcolwidth{2}{3}}
\sphinxAtStartPar
JSON string specifying a custom dictionary for mapping group keys to group colors. Example: ‘\{“key1”: “\#ff0000”, “key2”: “\#00ff00”, “key3”: “\#0000ff”\}’
\par
\vskip-\baselineskip\vbox{\hbox{\strut}}\end{varwidth}%
\sphinxstopmulticolumn
\\
\cline{2-3}\sphinxtablestrut{37}&
\sphinxAtStartPar
type
&
\sphinxAtStartPar
\sphinxstyleemphasis{string}
\\
\cline{2-3}\sphinxtablestrut{37}&
\sphinxAtStartPar
default
&\\
\hline\sphinxmultirow{3}{43}{%
\begin{varwidth}[t]{\sphinxcolwidth{1}{3}}
\begin{itemize}
\item {} 
\sphinxAtStartPar
scale\_col

\end{itemize}
\par
\vskip-\baselineskip\vbox{\hbox{\strut}}\end{varwidth}%
}%
&\sphinxstartmulticolumn{2}%
\begin{varwidth}[t]{\sphinxcolwidth{2}{3}}
\sphinxAtStartPar
Name of CSV column containing scalar values for changing the size of markers
\par
\vskip-\baselineskip\vbox{\hbox{\strut}}\end{varwidth}%
\sphinxstopmulticolumn
\\
\cline{2-3}\sphinxtablestrut{43}&
\sphinxAtStartPar
type
&
\sphinxAtStartPar
\sphinxstyleemphasis{string}
\\
\cline{2-3}\sphinxtablestrut{43}&
\sphinxAtStartPar
default
&
\sphinxAtStartPar
null
\\
\hline\sphinxmultirow{3}{49}{%
\begin{varwidth}[t]{\sphinxcolwidth{1}{3}}
\begin{itemize}
\item {} 
\sphinxAtStartPar
scale\_factor

\end{itemize}
\par
\vskip-\baselineskip\vbox{\hbox{\strut}}\end{varwidth}%
}%
&\sphinxstartmulticolumn{2}%
\begin{varwidth}[t]{\sphinxcolwidth{2}{3}}
\sphinxAtStartPar
Numerical value for a fixed scale factor to be applied to markers
\par
\vskip-\baselineskip\vbox{\hbox{\strut}}\end{varwidth}%
\sphinxstopmulticolumn
\\
\cline{2-3}\sphinxtablestrut{49}&
\sphinxAtStartPar
type
&
\sphinxAtStartPar
\sphinxstyleemphasis{string}
\\
\cline{2-3}\sphinxtablestrut{49}&
\sphinxAtStartPar
default
&
\sphinxAtStartPar
1
\\
\hline\sphinxmultirow{3}{55}{%
\begin{varwidth}[t]{\sphinxcolwidth{1}{3}}
\begin{itemize}
\item {} 
\sphinxAtStartPar
pie\_col

\end{itemize}
\par
\vskip-\baselineskip\vbox{\hbox{\strut}}\end{varwidth}%
}%
&\sphinxstartmulticolumn{2}%
\begin{varwidth}[t]{\sphinxcolwidth{2}{3}}
\sphinxAtStartPar
Name of CSV column containing data for pie chart sectors. TissUUmaps expects labels and numerical values for sectors to be separated by ‘:’ characters in the CSV column data.
\par
\vskip-\baselineskip\vbox{\hbox{\strut}}\end{varwidth}%
\sphinxstopmulticolumn
\\
\cline{2-3}\sphinxtablestrut{55}&
\sphinxAtStartPar
type
&
\sphinxAtStartPar
\sphinxstyleemphasis{string}
\\
\cline{2-3}\sphinxtablestrut{55}&
\sphinxAtStartPar
default
&
\sphinxAtStartPar
null
\\
\hline\sphinxmultirow{3}{61}{%
\begin{varwidth}[t]{\sphinxcolwidth{1}{3}}
\begin{itemize}
\item {} 
\sphinxAtStartPar
pie\_dict

\end{itemize}
\par
\vskip-\baselineskip\vbox{\hbox{\strut}}\end{varwidth}%
}%
&\sphinxstartmulticolumn{2}%
\begin{varwidth}[t]{\sphinxcolwidth{2}{3}}
\sphinxAtStartPar
JSON string specifying a custom dictionary for mapping pie chart sector indices to colors. Example: ‘\{0: “\#ff0000”, 1: “\#00ff00”, 2: “\#0000ff”\}’. If no dictionary is specified, TissUUmaps will use a default color palette instead.
\par
\vskip-\baselineskip\vbox{\hbox{\strut}}\end{varwidth}%
\sphinxstopmulticolumn
\\
\cline{2-3}\sphinxtablestrut{61}&
\sphinxAtStartPar
type
&
\sphinxAtStartPar
\sphinxstyleemphasis{string}
\\
\cline{2-3}\sphinxtablestrut{61}&
\sphinxAtStartPar
default
&\\
\hline\sphinxmultirow{3}{67}{%
\begin{varwidth}[t]{\sphinxcolwidth{1}{3}}
\begin{itemize}
\item {} 
\sphinxAtStartPar
shape\_col

\end{itemize}
\par
\vskip-\baselineskip\vbox{\hbox{\strut}}\end{varwidth}%
}%
&\sphinxstartmulticolumn{2}%
\begin{varwidth}[t]{\sphinxcolwidth{2}{3}}
\sphinxAtStartPar
Name of CSV column containing a name or an index for marker shape. See also {\hyperref[\detokenize{docs/starting/projects:shape}]{\sphinxcrossref{\DUrole{std,std-ref}{Shape}}}}.
\par
\vskip-\baselineskip\vbox{\hbox{\strut}}\end{varwidth}%
\sphinxstopmulticolumn
\\
\cline{2-3}\sphinxtablestrut{67}&
\sphinxAtStartPar
type
&
\sphinxAtStartPar
\sphinxstyleemphasis{string}
\\
\cline{2-3}\sphinxtablestrut{67}&
\sphinxAtStartPar
default
&
\sphinxAtStartPar
null
\\
\hline\sphinxmultirow{3}{73}{%
\begin{varwidth}[t]{\sphinxcolwidth{1}{3}}
\begin{itemize}
\item {} 
\sphinxAtStartPar
shape\_fixed

\end{itemize}
\par
\vskip-\baselineskip\vbox{\hbox{\strut}}\end{varwidth}%
}%
&\sphinxstartmulticolumn{2}%
\begin{varwidth}[t]{\sphinxcolwidth{2}{3}}
\sphinxAtStartPar
Name or index of a single fixed shape to be used for all markers. See {\hyperref[\detokenize{docs/starting/projects:shape}]{\sphinxcrossref{\DUrole{std,std-ref}{Shape}}}} for valid string values.
\par
\vskip-\baselineskip\vbox{\hbox{\strut}}\end{varwidth}%
\sphinxstopmulticolumn
\\
\cline{2-3}\sphinxtablestrut{73}&
\sphinxAtStartPar
type
&
\sphinxAtStartPar
\sphinxstyleemphasis{string}
\\
\cline{2-3}\sphinxtablestrut{73}&
\sphinxAtStartPar
default
&
\sphinxAtStartPar
cross
\\
\hline\sphinxmultirow{3}{79}{%
\begin{varwidth}[t]{\sphinxcolwidth{1}{3}}
\begin{itemize}
\item {} 
\sphinxAtStartPar
shape\_gr\_dict

\end{itemize}
\par
\vskip-\baselineskip\vbox{\hbox{\strut}}\end{varwidth}%
}%
&\sphinxstartmulticolumn{2}%
\begin{varwidth}[t]{\sphinxcolwidth{2}{3}}
\sphinxAtStartPar
JSON string specifying a custom dictionary for mapping group keys to group shapes. Example: ‘\{“key1”: “square”, “key2”: “diamond”, “key3”: “triangle up”\}’. See also {\hyperref[\detokenize{docs/starting/projects:shape}]{\sphinxcrossref{\DUrole{std,std-ref}{Shape}}}}.
\par
\vskip-\baselineskip\vbox{\hbox{\strut}}\end{varwidth}%
\sphinxstopmulticolumn
\\
\cline{2-3}\sphinxtablestrut{79}&
\sphinxAtStartPar
type
&
\sphinxAtStartPar
\sphinxstyleemphasis{string}
\\
\cline{2-3}\sphinxtablestrut{79}&
\sphinxAtStartPar
default
&\\
\hline\sphinxmultirow{3}{85}{%
\begin{varwidth}[t]{\sphinxcolwidth{1}{3}}
\begin{itemize}
\item {} 
\sphinxAtStartPar
opacity\_col

\end{itemize}
\par
\vskip-\baselineskip\vbox{\hbox{\strut}}\end{varwidth}%
}%
&\sphinxstartmulticolumn{2}%
\begin{varwidth}[t]{\sphinxcolwidth{2}{3}}
\sphinxAtStartPar
Name of CSV column containing scalar values for opacities
\par
\vskip-\baselineskip\vbox{\hbox{\strut}}\end{varwidth}%
\sphinxstopmulticolumn
\\
\cline{2-3}\sphinxtablestrut{85}&
\sphinxAtStartPar
type
&
\sphinxAtStartPar
\sphinxstyleemphasis{string}
\\
\cline{2-3}\sphinxtablestrut{85}&
\sphinxAtStartPar
default
&
\sphinxAtStartPar
null
\\
\hline\sphinxmultirow{3}{91}{%
\begin{varwidth}[t]{\sphinxcolwidth{1}{3}}
\begin{itemize}
\item {} 
\sphinxAtStartPar
opacity

\end{itemize}
\par
\vskip-\baselineskip\vbox{\hbox{\strut}}\end{varwidth}%
}%
&\sphinxstartmulticolumn{2}%
\begin{varwidth}[t]{\sphinxcolwidth{2}{3}}
\sphinxAtStartPar
Numerical value for a fixed opacity factor to be applied to markers
\par
\vskip-\baselineskip\vbox{\hbox{\strut}}\end{varwidth}%
\sphinxstopmulticolumn
\\
\cline{2-3}\sphinxtablestrut{91}&
\sphinxAtStartPar
type
&
\sphinxAtStartPar
\sphinxstyleemphasis{string}
\\
\cline{2-3}\sphinxtablestrut{91}&
\sphinxAtStartPar
default
&
\sphinxAtStartPar
1
\\
\hline\sphinxmultirow{3}{97}{%
\begin{varwidth}[t]{\sphinxcolwidth{1}{3}}
\begin{itemize}
\item {} 
\sphinxAtStartPar
tooltip\_fmt

\end{itemize}
\par
\vskip-\baselineskip\vbox{\hbox{\strut}}\end{varwidth}%
}%
&\sphinxstartmulticolumn{2}%
\begin{varwidth}[t]{\sphinxcolwidth{2}{3}}
\sphinxAtStartPar
Custom formatting string used for displaying metadata about a selected marker. See \sphinxurl{https://github.com/TissUUmaps/TissUUmaps/issues/2} for an overview of the grammer and keywords. If no string is specified, TissUUmaps will show default metadata depending on the context.
\par
\vskip-\baselineskip\vbox{\hbox{\strut}}\end{varwidth}%
\sphinxstopmulticolumn
\\
\cline{2-3}\sphinxtablestrut{97}&
\sphinxAtStartPar
type
&
\sphinxAtStartPar
\sphinxstyleemphasis{string}
\\
\cline{2-3}\sphinxtablestrut{97}&
\sphinxAtStartPar
default
&\\
\hline
\end{longtable}\sphinxatlongtableend\end{savenotes}


\paragraph{ExpectedRadios}
\label{\detokenize{docs/starting/projects:expectedradios}}\label{\detokenize{docs/starting/projects:expectedradios}}

\begin{savenotes}\sphinxatlongtablestart\begin{longtable}[c]{|*{3}{\X{1}{3}|}}
\hline

\endfirsthead

\multicolumn{3}{c}%
{\makebox[0pt]{\sphinxtablecontinued{\tablename\ \thetable{} \textendash{} continued from previous page}}}\\
\hline

\endhead

\hline
\multicolumn{3}{r}{\makebox[0pt][r]{\sphinxtablecontinued{continues on next page}}}\\
\endfoot

\endlastfoot
\sphinxstartmulticolumn{3}%
\begin{varwidth}[t]{\sphinxcolwidth{3}{3}}
\sphinxAtStartPar
Radio button state and checkbox state for settings in a marker tab. Required properties are shown in \sphinxstylestrong{bold} text.
\par
\vskip-\baselineskip\vbox{\hbox{\strut}}\end{varwidth}%
\sphinxstopmulticolumn
\\
\hline
\sphinxAtStartPar
type
&\sphinxstartmulticolumn{2}%
\begin{varwidth}[t]{\sphinxcolwidth{2}{3}}
\sphinxAtStartPar
\sphinxstyleemphasis{object}
\par
\vskip-\baselineskip\vbox{\hbox{\strut}}\end{varwidth}%
\sphinxstopmulticolumn
\\
\hline\sphinxstartmulticolumn{3}%
\begin{varwidth}[t]{\sphinxcolwidth{3}{3}}
\sphinxAtStartPar
properties
\par
\vskip-\baselineskip\vbox{\hbox{\strut}}\end{varwidth}%
\sphinxstopmulticolumn
\\
\hline\sphinxmultirow{3}{5}{%
\begin{varwidth}[t]{\sphinxcolwidth{1}{3}}
\begin{itemize}
\item {} 
\sphinxAtStartPar
cb\_col

\end{itemize}
\par
\vskip-\baselineskip\vbox{\hbox{\strut}}\end{varwidth}%
}%
&\sphinxstartmulticolumn{2}%
\begin{varwidth}[t]{\sphinxcolwidth{2}{3}}
\sphinxAtStartPar
If markers should be colored by data in CSV column
\par
\vskip-\baselineskip\vbox{\hbox{\strut}}\end{varwidth}%
\sphinxstopmulticolumn
\\
\cline{2-3}\sphinxtablestrut{5}&
\sphinxAtStartPar
type
&
\sphinxAtStartPar
\sphinxstyleemphasis{boolean}
\\
\cline{2-3}\sphinxtablestrut{5}&
\sphinxAtStartPar
default
&
\sphinxAtStartPar
false
\\
\hline\sphinxmultirow{3}{11}{%
\begin{varwidth}[t]{\sphinxcolwidth{1}{3}}
\begin{itemize}
\item {} 
\sphinxAtStartPar
cb\_gr

\end{itemize}
\par
\vskip-\baselineskip\vbox{\hbox{\strut}}\end{varwidth}%
}%
&\sphinxstartmulticolumn{2}%
\begin{varwidth}[t]{\sphinxcolwidth{2}{3}}
\sphinxAtStartPar
If markers should be colored by group
\par
\vskip-\baselineskip\vbox{\hbox{\strut}}\end{varwidth}%
\sphinxstopmulticolumn
\\
\cline{2-3}\sphinxtablestrut{11}&
\sphinxAtStartPar
type
&
\sphinxAtStartPar
\sphinxstyleemphasis{boolean}
\\
\cline{2-3}\sphinxtablestrut{11}&
\sphinxAtStartPar
default
&
\sphinxAtStartPar
true
\\
\hline\sphinxmultirow{3}{17}{%
\begin{varwidth}[t]{\sphinxcolwidth{1}{3}}
\begin{itemize}
\item {} 
\sphinxAtStartPar
cb\_gr\_rand

\end{itemize}
\par
\vskip-\baselineskip\vbox{\hbox{\strut}}\end{varwidth}%
}%
&\sphinxstartmulticolumn{2}%
\begin{varwidth}[t]{\sphinxcolwidth{2}{3}}
\sphinxAtStartPar
If group color should be generated randomly
\par
\vskip-\baselineskip\vbox{\hbox{\strut}}\end{varwidth}%
\sphinxstopmulticolumn
\\
\cline{2-3}\sphinxtablestrut{17}&
\sphinxAtStartPar
type
&
\sphinxAtStartPar
\sphinxstyleemphasis{boolean}
\\
\cline{2-3}\sphinxtablestrut{17}&
\sphinxAtStartPar
default
&
\sphinxAtStartPar
false
\\
\hline\sphinxmultirow{3}{23}{%
\begin{varwidth}[t]{\sphinxcolwidth{1}{3}}
\begin{itemize}
\item {} 
\sphinxAtStartPar
cb\_gr\_dict

\end{itemize}
\par
\vskip-\baselineskip\vbox{\hbox{\strut}}\end{varwidth}%
}%
&\sphinxstartmulticolumn{2}%
\begin{varwidth}[t]{\sphinxcolwidth{2}{3}}
\sphinxAtStartPar
If group color should be read from custom dictionary
\par
\vskip-\baselineskip\vbox{\hbox{\strut}}\end{varwidth}%
\sphinxstopmulticolumn
\\
\cline{2-3}\sphinxtablestrut{23}&
\sphinxAtStartPar
type
&
\sphinxAtStartPar
\sphinxstyleemphasis{boolean}
\\
\cline{2-3}\sphinxtablestrut{23}&
\sphinxAtStartPar
default
&
\sphinxAtStartPar
false
\\
\hline\sphinxmultirow{3}{29}{%
\begin{varwidth}[t]{\sphinxcolwidth{1}{3}}
\begin{itemize}
\item {} 
\sphinxAtStartPar
cb\_gr\_key

\end{itemize}
\par
\vskip-\baselineskip\vbox{\hbox{\strut}}\end{varwidth}%
}%
&\sphinxstartmulticolumn{2}%
\begin{varwidth}[t]{\sphinxcolwidth{2}{3}}
\sphinxAtStartPar
If group color should be generated from group key
\par
\vskip-\baselineskip\vbox{\hbox{\strut}}\end{varwidth}%
\sphinxstopmulticolumn
\\
\cline{2-3}\sphinxtablestrut{29}&
\sphinxAtStartPar
type
&
\sphinxAtStartPar
\sphinxstyleemphasis{boolean}
\\
\cline{2-3}\sphinxtablestrut{29}&
\sphinxAtStartPar
default
&
\sphinxAtStartPar
true
\\
\hline\sphinxmultirow{3}{35}{%
\begin{varwidth}[t]{\sphinxcolwidth{1}{3}}
\begin{itemize}
\item {} 
\sphinxAtStartPar
pie\_check

\end{itemize}
\par
\vskip-\baselineskip\vbox{\hbox{\strut}}\end{varwidth}%
}%
&\sphinxstartmulticolumn{2}%
\begin{varwidth}[t]{\sphinxcolwidth{2}{3}}
\sphinxAtStartPar
If markers should be rendered as pie charts
\par
\vskip-\baselineskip\vbox{\hbox{\strut}}\end{varwidth}%
\sphinxstopmulticolumn
\\
\cline{2-3}\sphinxtablestrut{35}&
\sphinxAtStartPar
type
&
\sphinxAtStartPar
\sphinxstyleemphasis{boolean}
\\
\cline{2-3}\sphinxtablestrut{35}&
\sphinxAtStartPar
default
&
\sphinxAtStartPar
false
\\
\hline\sphinxmultirow{3}{41}{%
\begin{varwidth}[t]{\sphinxcolwidth{1}{3}}
\begin{itemize}
\item {} 
\sphinxAtStartPar
scale\_check

\end{itemize}
\par
\vskip-\baselineskip\vbox{\hbox{\strut}}\end{varwidth}%
}%
&\sphinxstartmulticolumn{2}%
\begin{varwidth}[t]{\sphinxcolwidth{2}{3}}
\sphinxAtStartPar
If markers should be scaled by data in CSV column
\par
\vskip-\baselineskip\vbox{\hbox{\strut}}\end{varwidth}%
\sphinxstopmulticolumn
\\
\cline{2-3}\sphinxtablestrut{41}&
\sphinxAtStartPar
type
&
\sphinxAtStartPar
\sphinxstyleemphasis{boolean}
\\
\cline{2-3}\sphinxtablestrut{41}&
\sphinxAtStartPar
default
&
\sphinxAtStartPar
false
\\
\hline\sphinxmultirow{3}{47}{%
\begin{varwidth}[t]{\sphinxcolwidth{1}{3}}
\begin{itemize}
\item {} 
\sphinxAtStartPar
shape\_col

\end{itemize}
\par
\vskip-\baselineskip\vbox{\hbox{\strut}}\end{varwidth}%
}%
&\sphinxstartmulticolumn{2}%
\begin{varwidth}[t]{\sphinxcolwidth{2}{3}}
\sphinxAtStartPar
If markers should get their shape from data in CSV column
\par
\vskip-\baselineskip\vbox{\hbox{\strut}}\end{varwidth}%
\sphinxstopmulticolumn
\\
\cline{2-3}\sphinxtablestrut{47}&
\sphinxAtStartPar
type
&
\sphinxAtStartPar
\sphinxstyleemphasis{boolean}
\\
\cline{2-3}\sphinxtablestrut{47}&
\sphinxAtStartPar
default
&
\sphinxAtStartPar
false
\\
\hline\sphinxmultirow{3}{53}{%
\begin{varwidth}[t]{\sphinxcolwidth{1}{3}}
\begin{itemize}
\item {} 
\sphinxAtStartPar
shape\_gr

\end{itemize}
\par
\vskip-\baselineskip\vbox{\hbox{\strut}}\end{varwidth}%
}%
&\sphinxstartmulticolumn{2}%
\begin{varwidth}[t]{\sphinxcolwidth{2}{3}}
\sphinxAtStartPar
If markers should get their shape from group
\par
\vskip-\baselineskip\vbox{\hbox{\strut}}\end{varwidth}%
\sphinxstopmulticolumn
\\
\cline{2-3}\sphinxtablestrut{53}&
\sphinxAtStartPar
type
&
\sphinxAtStartPar
\sphinxstyleemphasis{boolean}
\\
\cline{2-3}\sphinxtablestrut{53}&
\sphinxAtStartPar
default
&
\sphinxAtStartPar
true
\\
\hline\sphinxmultirow{3}{59}{%
\begin{varwidth}[t]{\sphinxcolwidth{1}{3}}
\begin{itemize}
\item {} 
\sphinxAtStartPar
shape\_gr\_rand

\end{itemize}
\par
\vskip-\baselineskip\vbox{\hbox{\strut}}\end{varwidth}%
}%
&\sphinxstartmulticolumn{2}%
\begin{varwidth}[t]{\sphinxcolwidth{2}{3}}
\sphinxAtStartPar
If group shape should be generated randomly
\par
\vskip-\baselineskip\vbox{\hbox{\strut}}\end{varwidth}%
\sphinxstopmulticolumn
\\
\cline{2-3}\sphinxtablestrut{59}&
\sphinxAtStartPar
type
&
\sphinxAtStartPar
\sphinxstyleemphasis{boolean}
\\
\cline{2-3}\sphinxtablestrut{59}&
\sphinxAtStartPar
default
&
\sphinxAtStartPar
true
\\
\hline\sphinxmultirow{3}{65}{%
\begin{varwidth}[t]{\sphinxcolwidth{1}{3}}
\begin{itemize}
\item {} 
\sphinxAtStartPar
shape\_gr\_dict

\end{itemize}
\par
\vskip-\baselineskip\vbox{\hbox{\strut}}\end{varwidth}%
}%
&\sphinxstartmulticolumn{2}%
\begin{varwidth}[t]{\sphinxcolwidth{2}{3}}
\sphinxAtStartPar
If group shape should be read from custom dictionary
\par
\vskip-\baselineskip\vbox{\hbox{\strut}}\end{varwidth}%
\sphinxstopmulticolumn
\\
\cline{2-3}\sphinxtablestrut{65}&
\sphinxAtStartPar
type
&
\sphinxAtStartPar
\sphinxstyleemphasis{boolean}
\\
\cline{2-3}\sphinxtablestrut{65}&
\sphinxAtStartPar
default
&
\sphinxAtStartPar
false
\\
\hline\sphinxmultirow{3}{71}{%
\begin{varwidth}[t]{\sphinxcolwidth{1}{3}}
\begin{itemize}
\item {} 
\sphinxAtStartPar
shape\_fixed

\end{itemize}
\par
\vskip-\baselineskip\vbox{\hbox{\strut}}\end{varwidth}%
}%
&\sphinxstartmulticolumn{2}%
\begin{varwidth}[t]{\sphinxcolwidth{2}{3}}
\sphinxAtStartPar
If a single fixed shape should be used for all markers
\par
\vskip-\baselineskip\vbox{\hbox{\strut}}\end{varwidth}%
\sphinxstopmulticolumn
\\
\cline{2-3}\sphinxtablestrut{71}&
\sphinxAtStartPar
type
&
\sphinxAtStartPar
\sphinxstyleemphasis{boolean}
\\
\cline{2-3}\sphinxtablestrut{71}&
\sphinxAtStartPar
default
&
\sphinxAtStartPar
false
\\
\hline\sphinxmultirow{3}{77}{%
\begin{varwidth}[t]{\sphinxcolwidth{1}{3}}
\begin{itemize}
\item {} 
\sphinxAtStartPar
opacity\_check

\end{itemize}
\par
\vskip-\baselineskip\vbox{\hbox{\strut}}\end{varwidth}%
}%
&\sphinxstartmulticolumn{2}%
\begin{varwidth}[t]{\sphinxcolwidth{2}{3}}
\sphinxAtStartPar
If markers should get their opacities from data in CSV column
\par
\vskip-\baselineskip\vbox{\hbox{\strut}}\end{varwidth}%
\sphinxstopmulticolumn
\\
\cline{2-3}\sphinxtablestrut{77}&
\sphinxAtStartPar
type
&
\sphinxAtStartPar
\sphinxstyleemphasis{boolean}
\\
\cline{2-3}\sphinxtablestrut{77}&
\sphinxAtStartPar
default
&
\sphinxAtStartPar
false
\\
\hline\sphinxmultirow{3}{83}{%
\begin{varwidth}[t]{\sphinxcolwidth{1}{3}}
\begin{itemize}
\item {} 
\sphinxAtStartPar
\_no\_outline

\end{itemize}
\par
\vskip-\baselineskip\vbox{\hbox{\strut}}\end{varwidth}%
}%
&\sphinxstartmulticolumn{2}%
\begin{varwidth}[t]{\sphinxcolwidth{2}{3}}
\sphinxAtStartPar
If marker shapes should be rendered without outline
\par
\vskip-\baselineskip\vbox{\hbox{\strut}}\end{varwidth}%
\sphinxstopmulticolumn
\\
\cline{2-3}\sphinxtablestrut{83}&
\sphinxAtStartPar
type
&
\sphinxAtStartPar
\sphinxstyleemphasis{boolean}
\\
\cline{2-3}\sphinxtablestrut{83}&
\sphinxAtStartPar
default
&
\sphinxAtStartPar
false
\\
\hline
\end{longtable}\sphinxatlongtableend\end{savenotes}


\paragraph{Setting}
\label{\detokenize{docs/starting/projects:setting}}\label{\detokenize{docs/starting/projects:setting}}

\begin{savenotes}\sphinxattablestart
\centering
\begin{tabular}[t]{|*{3}{\X{1}{3}|}}
\hline
\sphinxstartmulticolumn{3}%
\begin{varwidth}[t]{\sphinxcolwidth{3}{3}}
\sphinxAtStartPar
{[}Add description{]}. Required properties are shown in \sphinxstylestrong{bold} text.
\par
\vskip-\baselineskip\vbox{\hbox{\strut}}\end{varwidth}%
\sphinxstopmulticolumn
\\
\hline
\sphinxAtStartPar
type
&\sphinxstartmulticolumn{2}%
\begin{varwidth}[t]{\sphinxcolwidth{2}{3}}
\sphinxAtStartPar
\sphinxstyleemphasis{object}
\par
\vskip-\baselineskip\vbox{\hbox{\strut}}\end{varwidth}%
\sphinxstopmulticolumn
\\
\hline\sphinxstartmulticolumn{3}%
\begin{varwidth}[t]{\sphinxcolwidth{3}{3}}
\sphinxAtStartPar
properties
\par
\vskip-\baselineskip\vbox{\hbox{\strut}}\end{varwidth}%
\sphinxstopmulticolumn
\\
\hline\begin{itemize}
\item {} 
\sphinxAtStartPar
\sphinxstylestrong{function}

\end{itemize}
&
\sphinxAtStartPar
type
&
\sphinxAtStartPar
\sphinxstyleemphasis{string}
\\
\hline\begin{itemize}
\item {} 
\sphinxAtStartPar
\sphinxstylestrong{module}

\end{itemize}
&
\sphinxAtStartPar
type
&
\sphinxAtStartPar
\sphinxstyleemphasis{string}
\\
\hline\begin{itemize}
\item {} 
\sphinxAtStartPar
\sphinxstylestrong{value}

\end{itemize}
&
\sphinxAtStartPar
type
&
\sphinxAtStartPar
\sphinxstyleemphasis{number}
\\
\hline
\end{tabular}
\par
\sphinxattableend\end{savenotes}


\subsubsection{Example of a .tmap file}
\label{\detokenize{docs/starting/projects:example-of-a-tmap-file}}
\begin{sphinxVerbatim}[commandchars=\\\{\}]
\PYG{p}{\PYGZob{}}
\PYG{+w}{    }\PYG{n+nt}{\PYGZdq{}filename\PYGZdq{}}\PYG{p}{:}\PYG{+w}{ }\PYG{l+s+s2}{\PYGZdq{}TissUUmaps\PYGZus{}Example.tmap\PYGZdq{}}\PYG{p}{,}
\PYG{+w}{    }\PYG{n+nt}{\PYGZdq{}layers\PYGZdq{}}\PYG{p}{:}\PYG{+w}{ }\PYG{p}{[}
\PYG{+w}{        }\PYG{p}{\PYGZob{}}
\PYG{+w}{            }\PYG{n+nt}{\PYGZdq{}name\PYGZdq{}}\PYG{p}{:}\PYG{+w}{ }\PYG{l+s+s2}{\PYGZdq{}Round1\PYGZus{}A.tif\PYGZdq{}}\PYG{p}{,}
\PYG{+w}{            }\PYG{n+nt}{\PYGZdq{}tileSource\PYGZdq{}}\PYG{p}{:}\PYG{+w}{ }\PYG{l+s+s2}{\PYGZdq{}images/Round1\PYGZus{}A.tif.dzi\PYGZdq{}}
\PYG{+w}{        }\PYG{p}{\PYGZcb{},}
\PYG{+w}{        }\PYG{p}{\PYGZob{}}
\PYG{+w}{            }\PYG{n+nt}{\PYGZdq{}name\PYGZdq{}}\PYG{p}{:}\PYG{+w}{ }\PYG{l+s+s2}{\PYGZdq{}Round1\PYGZus{}C.tif\PYGZdq{}}\PYG{p}{,}
\PYG{+w}{            }\PYG{n+nt}{\PYGZdq{}tileSource\PYGZdq{}}\PYG{p}{:}\PYG{+w}{ }\PYG{l+s+s2}{\PYGZdq{}images/Round1\PYGZus{}C.tif.dzi\PYGZdq{}}
\PYG{+w}{        }\PYG{p}{\PYGZcb{}}
\PYG{+w}{    }\PYG{p}{],}
\PYG{+w}{    }\PYG{n+nt}{\PYGZdq{}layerOpacities\PYGZdq{}}\PYG{p}{:}\PYG{+w}{ }\PYG{p}{\PYGZob{}}
\PYG{+w}{        }\PYG{n+nt}{\PYGZdq{}0\PYGZdq{}}\PYG{p}{:}\PYG{+w}{ }\PYG{l+s+s2}{\PYGZdq{}1\PYGZdq{}}\PYG{p}{,}
\PYG{+w}{        }\PYG{n+nt}{\PYGZdq{}1\PYGZdq{}}\PYG{p}{:}\PYG{+w}{ }\PYG{l+s+s2}{\PYGZdq{}1\PYGZdq{}}
\PYG{+w}{    }\PYG{p}{\PYGZcb{},}
\PYG{+w}{    }\PYG{n+nt}{\PYGZdq{}layerVisibilities\PYGZdq{}}\PYG{p}{:}\PYG{+w}{ }\PYG{p}{\PYGZob{}}
\PYG{+w}{        }\PYG{n+nt}{\PYGZdq{}0\PYGZdq{}}\PYG{p}{:}\PYG{+w}{ }\PYG{k+kc}{true}\PYG{p}{,}
\PYG{+w}{        }\PYG{n+nt}{\PYGZdq{}1\PYGZdq{}}\PYG{p}{:}\PYG{+w}{ }\PYG{k+kc}{false}\PYG{p}{,}
\PYG{+w}{    }\PYG{p}{\PYGZcb{},}
\PYG{+w}{    }\PYG{n+nt}{\PYGZdq{}layerFilters\PYGZdq{}}\PYG{p}{:}\PYG{+w}{ }\PYG{p}{\PYGZob{}}
\PYG{+w}{        }\PYG{n+nt}{\PYGZdq{}0\PYGZdq{}}\PYG{p}{:}\PYG{+w}{ }\PYG{p}{[}
\PYG{+w}{            }\PYG{p}{\PYGZob{}}
\PYG{+w}{                }\PYG{n+nt}{\PYGZdq{}name\PYGZdq{}}\PYG{p}{:}\PYG{+w}{ }\PYG{l+s+s2}{\PYGZdq{}Color\PYGZdq{}}\PYG{p}{,}
\PYG{+w}{                }\PYG{n+nt}{\PYGZdq{}value\PYGZdq{}}\PYG{p}{:}\PYG{+w}{ }\PYG{l+s+s2}{\PYGZdq{}0,100,0\PYGZdq{}}
\PYG{+w}{            }\PYG{p}{\PYGZcb{}}
\PYG{+w}{        }\PYG{p}{],}
\PYG{+w}{        }\PYG{n+nt}{\PYGZdq{}1\PYGZdq{}}\PYG{p}{:}\PYG{+w}{ }\PYG{p}{[}
\PYG{+w}{            }\PYG{p}{\PYGZob{}}
\PYG{+w}{                }\PYG{n+nt}{\PYGZdq{}name\PYGZdq{}}\PYG{p}{:}\PYG{+w}{ }\PYG{l+s+s2}{\PYGZdq{}Color\PYGZdq{}}\PYG{p}{,}
\PYG{+w}{                }\PYG{n+nt}{\PYGZdq{}value\PYGZdq{}}\PYG{p}{:}\PYG{+w}{ }\PYG{l+s+s2}{\PYGZdq{}0,100,0\PYGZdq{}}
\PYG{+w}{            }\PYG{p}{\PYGZcb{}}
\PYG{+w}{        }\PYG{p}{]}
\PYG{+w}{    }\PYG{p}{\PYGZcb{},}
\PYG{+w}{    }\PYG{n+nt}{\PYGZdq{}filters\PYGZdq{}}\PYG{p}{:}\PYG{+w}{ }\PYG{p}{[}
\PYG{+w}{        }\PYG{l+s+s2}{\PYGZdq{}Color\PYGZdq{}}
\PYG{+w}{    }\PYG{p}{],}
\PYG{+w}{    }\PYG{n+nt}{\PYGZdq{}compositeMode\PYGZdq{}}\PYG{p}{:}\PYG{+w}{ }\PYG{l+s+s2}{\PYGZdq{}lighter\PYGZdq{}}\PYG{p}{,}
\PYG{+w}{    }\PYG{n+nt}{\PYGZdq{}markerFiles\PYGZdq{}}\PYG{p}{:}\PYG{+w}{ }\PYG{p}{[}
\PYG{+w}{        }\PYG{p}{\PYGZob{}}
\PYG{+w}{            }\PYG{n+nt}{\PYGZdq{}autoLoad\PYGZdq{}}\PYG{p}{:}\PYG{+w}{ }\PYG{k+kc}{false}\PYG{p}{,}
\PYG{+w}{            }\PYG{n+nt}{\PYGZdq{}comment\PYGZdq{}}\PYG{p}{:}\PYG{+w}{ }\PYG{l+s+s2}{\PYGZdq{}\PYGZdq{}}\PYG{p}{,}
\PYG{+w}{            }\PYG{n+nt}{\PYGZdq{}expectedHeader\PYGZdq{}}\PYG{p}{:}\PYG{+w}{ }\PYG{p}{\PYGZob{}}
\PYG{+w}{                }\PYG{n+nt}{\PYGZdq{}X\PYGZdq{}}\PYG{p}{:}\PYG{+w}{ }\PYG{l+s+s2}{\PYGZdq{}global\PYGZus{}x\PYGZdq{}}\PYG{p}{,}
\PYG{+w}{                }\PYG{n+nt}{\PYGZdq{}Y\PYGZdq{}}\PYG{p}{:}\PYG{+w}{ }\PYG{l+s+s2}{\PYGZdq{}global\PYGZus{}y\PYGZdq{}}\PYG{p}{,}
\PYG{+w}{                }\PYG{n+nt}{\PYGZdq{}cb\PYGZus{}cmap\PYGZdq{}}\PYG{p}{:}\PYG{+w}{ }\PYG{l+s+s2}{\PYGZdq{}\PYGZdq{}}\PYG{p}{,}
\PYG{+w}{                }\PYG{n+nt}{\PYGZdq{}cb\PYGZus{}col\PYGZdq{}}\PYG{p}{:}\PYG{+w}{ }\PYG{l+s+s2}{\PYGZdq{}null\PYGZdq{}}\PYG{p}{,}
\PYG{+w}{                }\PYG{n+nt}{\PYGZdq{}cb\PYGZus{}gr\PYGZus{}dict\PYGZdq{}}\PYG{p}{:}\PYG{+w}{ }\PYG{l+s+s2}{\PYGZdq{}\PYGZdq{}}\PYG{p}{,}
\PYG{+w}{                }\PYG{n+nt}{\PYGZdq{}gb\PYGZus{}col\PYGZdq{}}\PYG{p}{:}\PYG{+w}{ }\PYG{l+s+s2}{\PYGZdq{}Gene\PYGZdq{}}\PYG{p}{,}
\PYG{+w}{                }\PYG{n+nt}{\PYGZdq{}gb\PYGZus{}name\PYGZdq{}}\PYG{p}{:}\PYG{+w}{ }\PYG{l+s+s2}{\PYGZdq{}\PYGZdq{}}\PYG{p}{,}
\PYG{+w}{                }\PYG{n+nt}{\PYGZdq{}opacity\PYGZdq{}}\PYG{p}{:}\PYG{+w}{ }\PYG{l+s+s2}{\PYGZdq{}1\PYGZdq{}}\PYG{p}{,}
\PYG{+w}{                }\PYG{n+nt}{\PYGZdq{}opacity\PYGZus{}col\PYGZdq{}}\PYG{p}{:}\PYG{+w}{ }\PYG{l+s+s2}{\PYGZdq{}null\PYGZdq{}}\PYG{p}{,}
\PYG{+w}{                }\PYG{n+nt}{\PYGZdq{}pie\PYGZus{}col\PYGZdq{}}\PYG{p}{:}\PYG{+w}{ }\PYG{l+s+s2}{\PYGZdq{}null\PYGZdq{}}\PYG{p}{,}
\PYG{+w}{                }\PYG{n+nt}{\PYGZdq{}pie\PYGZus{}dict\PYGZdq{}}\PYG{p}{:}\PYG{+w}{ }\PYG{l+s+s2}{\PYGZdq{}\PYGZdq{}}\PYG{p}{,}
\PYG{+w}{                }\PYG{n+nt}{\PYGZdq{}scale\PYGZus{}col\PYGZdq{}}\PYG{p}{:}\PYG{+w}{ }\PYG{l+s+s2}{\PYGZdq{}null\PYGZdq{}}\PYG{p}{,}
\PYG{+w}{                }\PYG{n+nt}{\PYGZdq{}scale\PYGZus{}factor\PYGZdq{}}\PYG{p}{:}\PYG{+w}{ }\PYG{l+s+s2}{\PYGZdq{}0.5\PYGZdq{}}\PYG{p}{,}
\PYG{+w}{                }\PYG{n+nt}{\PYGZdq{}shape\PYGZus{}col\PYGZdq{}}\PYG{p}{:}\PYG{+w}{ }\PYG{l+s+s2}{\PYGZdq{}null\PYGZdq{}}\PYG{p}{,}
\PYG{+w}{                }\PYG{n+nt}{\PYGZdq{}shape\PYGZus{}fixed\PYGZdq{}}\PYG{p}{:}\PYG{+w}{ }\PYG{l+s+s2}{\PYGZdq{}cross\PYGZdq{}}\PYG{p}{,}
\PYG{+w}{                }\PYG{n+nt}{\PYGZdq{}shape\PYGZus{}gr\PYGZus{}dict\PYGZdq{}}\PYG{p}{:}\PYG{+w}{ }\PYG{l+s+s2}{\PYGZdq{}\PYGZdq{}}\PYG{p}{,}
\PYG{+w}{                }\PYG{n+nt}{\PYGZdq{}tooltip\PYGZus{}fmt\PYGZdq{}}\PYG{p}{:}\PYG{+w}{ }\PYG{l+s+s2}{\PYGZdq{}\PYGZdq{}}
\PYG{+w}{            }\PYG{p}{\PYGZcb{},}
\PYG{+w}{            }\PYG{n+nt}{\PYGZdq{}expectedRadios\PYGZdq{}}\PYG{p}{:}\PYG{+w}{ }\PYG{p}{\PYGZob{}}
\PYG{+w}{                }\PYG{n+nt}{\PYGZdq{}cb\PYGZus{}col\PYGZdq{}}\PYG{p}{:}\PYG{+w}{ }\PYG{k+kc}{false}\PYG{p}{,}
\PYG{+w}{                }\PYG{n+nt}{\PYGZdq{}cb\PYGZus{}gr\PYGZdq{}}\PYG{p}{:}\PYG{+w}{ }\PYG{k+kc}{true}\PYG{p}{,}
\PYG{+w}{                }\PYG{n+nt}{\PYGZdq{}cb\PYGZus{}gr\PYGZus{}dict\PYGZdq{}}\PYG{p}{:}\PYG{+w}{ }\PYG{k+kc}{false}\PYG{p}{,}
\PYG{+w}{                }\PYG{n+nt}{\PYGZdq{}cb\PYGZus{}gr\PYGZus{}key\PYGZdq{}}\PYG{p}{:}\PYG{+w}{ }\PYG{k+kc}{true}\PYG{p}{,}
\PYG{+w}{                }\PYG{n+nt}{\PYGZdq{}cb\PYGZus{}gr\PYGZus{}rand\PYGZdq{}}\PYG{p}{:}\PYG{+w}{ }\PYG{k+kc}{false}\PYG{p}{,}
\PYG{+w}{                }\PYG{n+nt}{\PYGZdq{}pie\PYGZus{}check\PYGZdq{}}\PYG{p}{:}\PYG{+w}{ }\PYG{k+kc}{false}\PYG{p}{,}
\PYG{+w}{                }\PYG{n+nt}{\PYGZdq{}scale\PYGZus{}check\PYGZdq{}}\PYG{p}{:}\PYG{+w}{ }\PYG{k+kc}{false}\PYG{p}{,}
\PYG{+w}{                }\PYG{n+nt}{\PYGZdq{}shape\PYGZus{}col\PYGZdq{}}\PYG{p}{:}\PYG{+w}{ }\PYG{k+kc}{false}\PYG{p}{,}
\PYG{+w}{                }\PYG{n+nt}{\PYGZdq{}shape\PYGZus{}fixed\PYGZdq{}}\PYG{p}{:}\PYG{+w}{ }\PYG{k+kc}{false}\PYG{p}{,}
\PYG{+w}{                }\PYG{n+nt}{\PYGZdq{}shape\PYGZus{}gr\PYGZdq{}}\PYG{p}{:}\PYG{+w}{ }\PYG{k+kc}{true}\PYG{p}{,}
\PYG{+w}{                }\PYG{n+nt}{\PYGZdq{}shape\PYGZus{}gr\PYGZus{}dict\PYGZdq{}}\PYG{p}{:}\PYG{+w}{ }\PYG{k+kc}{false}\PYG{p}{,}
\PYG{+w}{                }\PYG{n+nt}{\PYGZdq{}shape\PYGZus{}gr\PYGZus{}rand\PYGZdq{}}\PYG{p}{:}\PYG{+w}{ }\PYG{k+kc}{true}\PYG{p}{,}
\PYG{+w}{                }\PYG{n+nt}{\PYGZdq{}opacity\PYGZus{}check\PYGZdq{}}\PYG{p}{:}\PYG{+w}{ }\PYG{k+kc}{false}
\PYG{+w}{            }\PYG{p}{\PYGZcb{},}
\PYG{+w}{            }\PYG{n+nt}{\PYGZdq{}name\PYGZdq{}}\PYG{p}{:}\PYG{+w}{ }\PYG{l+s+s2}{\PYGZdq{} markers\PYGZdq{}}\PYG{p}{,}
\PYG{+w}{            }\PYG{n+nt}{\PYGZdq{}path\PYGZdq{}}\PYG{p}{:}\PYG{+w}{ }\PYG{l+s+s2}{\PYGZdq{}./istdeco\PYGZus{}codes\PYGZus{}n.csv\PYGZdq{}}\PYG{p}{,}
\PYG{+w}{            }\PYG{n+nt}{\PYGZdq{}title\PYGZdq{}}\PYG{p}{:}\PYG{+w}{ }\PYG{l+s+s2}{\PYGZdq{}Download markers\PYGZdq{}}\PYG{p}{,}
\PYG{+w}{            }\PYG{n+nt}{\PYGZdq{}uid\PYGZdq{}}\PYG{p}{:}\PYG{+w}{ }\PYG{l+s+s2}{\PYGZdq{}uniquetab\PYGZdq{}}
\PYG{+w}{        }\PYG{p}{\PYGZcb{}}
\PYG{+w}{    }\PYG{p}{],}
\PYG{+w}{    }\PYG{n+nt}{\PYGZdq{}regions\PYGZdq{}}\PYG{p}{:}\PYG{+w}{ }\PYG{p}{\PYGZob{}\PYGZcb{},}
\PYG{+w}{    }\PYG{n+nt}{\PYGZdq{}plugins\PYGZdq{}}\PYG{p}{:}\PYG{+w}{ }\PYG{p}{[}
\PYG{+w}{        }\PYG{l+s+s2}{\PYGZdq{}Spot\PYGZus{}Inspector\PYGZdq{}}
\PYG{+w}{    }\PYG{p}{],}
\PYG{+w}{    }\PYG{n+nt}{\PYGZdq{}hideTabs\PYGZdq{}}\PYG{p}{:}\PYG{+w}{ }\PYG{k+kc}{true}\PYG{p}{,}
\PYG{+w}{    }\PYG{n+nt}{\PYGZdq{}settings\PYGZdq{}}\PYG{p}{:}\PYG{+w}{ }\PYG{p}{[]}
\PYG{p}{\PYGZcb{}}
\end{sphinxVerbatim}

\sphinxstepscope


\section{Exporting screenshots}
\label{\detokenize{docs/starting/capture:exporting-screenshots}}\label{\detokenize{docs/starting/capture::doc}}
\sphinxAtStartPar
TissUUmaps allows high resolution capture of the image viewport. Go to \sphinxcode{\sphinxupquote{Menu \textgreater{} File \textgreater{} Capture viewport}} and chose a zoom factor for export (1 = screen resolution).

\sphinxAtStartPar
The screen capture will contain all filtered layers, markers, and regions. Note that legends will not be part of the export and must be added manually.

\sphinxstepscope


\section{Plugins}
\label{\detokenize{docs/starting/plugins:plugins}}\label{\detokenize{docs/starting/plugins::doc}}

\subsection{Load plugins}
\label{\detokenize{docs/starting/plugins:load-plugins}}

\subsection{Make your own plugin}
\label{\detokenize{docs/starting/plugins:make-your-own-plugin}}
\sphinxAtStartPar
Download the Plugin Template python and javascript files from the \sphinxhref{https://tissuumaps.github.io/TissUUmaps/plugins/}{Plugin Update Site} and put both files in your local folder \sphinxcode{\sphinxupquote{\$USER\_PATH/.tissuumaps/plugins/}}. You can then change the plugin name and add your own options and functions.


\subsubsection{Javascript file}
\label{\detokenize{docs/starting/plugins:javascript-file}}
\sphinxAtStartPar
When loading a plugin, the function \sphinxcode{\sphinxupquote{PluginName.init(container)}} will be called. The \sphinxcode{\sphinxupquote{container}} is an html Element that will be added to the plugin menu. Use this element to add options and texts related to your plugin.

\sphinxAtStartPar
\sphinxincludegraphics{{plugin_container}.png}

\sphinxAtStartPar
Here is a minimal example of plugin:

\begin{sphinxVerbatim}[commandchars=\\\{\}]
\PYG{k+kd}{var} \PYG{n+nx}{Plugin\PYGZus{}template}\PYG{p}{;}
\PYG{n+nx}{Plugin\PYGZus{}template} \PYG{o}{=} \PYG{p}{\PYGZob{}}
    \PYG{n+nx}{name}\PYG{o}{:}\PYG{l+s+s2}{\PYGZdq{}Template Plugin\PYGZdq{}}
\PYG{p}{\PYGZcb{}}
 
\PYG{c+cm}{/**}
\PYG{c+cm}{ * This method is called when the document is loaded.}
\PYG{c+cm}{ * The container element is a div where the plugin options will be displayed. */}
\PYG{n+nx}{Plugin\PYGZus{}template}\PYG{p}{.}\PYG{n+nx}{init} \PYG{o}{=} \PYG{k+kd}{function} \PYG{p}{(}\PYG{n+nx}{container}\PYG{p}{)} \PYG{p}{\PYGZob{}}
    \PYG{n+nx}{container}\PYG{p}{.}\PYG{n+nx}{innerHTML} \PYG{o}{=} \PYG{l+s+s2}{\PYGZdq{}Hello world\PYGZdq{}}\PYG{p}{;}
\PYG{p}{\PYGZcb{}}
\end{sphinxVerbatim}

\sphinxAtStartPar
You can access the TissUUmaps javascript API \sphinxhref{https://tissuumaps.github.io/TissUUmapsCore/}{here}.


\subsubsection{Python file}
\label{\detokenize{docs/starting/plugins:python-file}}
\sphinxAtStartPar
You only need to use the Python file if your plugin needs to do processing on the server side. For pure javascript plugins, you can leave this file empty.

\sphinxAtStartPar
The python file should implement the class \sphinxcode{\sphinxupquote{Plugin}}:

\begin{sphinxVerbatim}[commandchars=\\\{\}]
\PYG{k}{class} \PYG{n+nc}{Plugin} \PYG{p}{(}\PYG{p}{)}\PYG{p}{:}
    \PYG{k}{def} \PYG{n+nf+fm}{\PYGZus{}\PYGZus{}init\PYGZus{}\PYGZus{}}\PYG{p}{(}\PYG{n+nb+bp}{self}\PYG{p}{,} \PYG{n}{app}\PYG{p}{)}\PYG{p}{:}
        \PYG{n+nb+bp}{self}\PYG{o}{.}\PYG{n}{app} \PYG{o}{=} \PYG{n}{app}
\end{sphinxVerbatim}

\sphinxAtStartPar
The \sphinxcode{\sphinxupquote{app}} object being the flask application running the TissUUmaps server.

\sphinxAtStartPar
You can call a Python method inside the \sphinxcode{\sphinxupquote{Plugin}} class from Javascript using Ajax and the Python API. The endpoint for a method \sphinxcode{\sphinxupquote{methodName}} of the plugin \sphinxcode{\sphinxupquote{PluginName}} will be: \sphinxcode{\sphinxupquote{/plugins/methodName/functionName}}. Data can be transmitted through Ajax as stringified JSON, and will be available as a parameter inside the method.

\sphinxAtStartPar
See the Plugin Template for a working example of Javascript / Python communication.

\sphinxstepscope


\chapter{Sharing projects}
\label{\detokenize{docs/sharing/index:sharing-projects}}\label{\detokenize{docs/sharing/index::doc}}
\sphinxstepscope


\section{Apache server}
\label{\detokenize{docs/sharing/apache:apache-server}}\label{\detokenize{docs/sharing/apache::doc}}
\sphinxAtStartPar
TissUUmaps projects can be exported into static webpages, that can be uploaded to any Apache server.
\begin{enumerate}
\sphinxsetlistlabels{\arabic}{enumi}{enumii}{}{.}%
\item {} 
\sphinxAtStartPar
Save your project from TissUUmaps (\sphinxcode{\sphinxupquote{menu \textgreater{} File \textgreater{} Save project}})

\item {} 
\sphinxAtStartPar
Export to static page (\sphinxcode{\sphinxupquote{menu \textgreater{} File \textgreater{} Export to static webpage}})

\item {} 
\sphinxAtStartPar
Copy the exported folder on your Apache server

\end{enumerate}

\sphinxstepscope


\section{Docker container}
\label{\detokenize{docs/sharing/docker:docker-container}}\label{\detokenize{docs/sharing/docker::doc}}\begin{enumerate}
\sphinxsetlistlabels{\arabic}{enumi}{enumii}{}{.}%
\item {} 
\sphinxAtStartPar
Start the docker container \sphinxcode{\sphinxupquote{cavenel/tissuumaps:latest}} from Docker Hub:

\end{enumerate}

\begin{sphinxVerbatim}[commandchars=\\\{\}]
docker run \PYGZhy{}it \PYGZhy{}p \PYG{l+m}{56733}:80 \PYGZhy{}\PYGZhy{}name\PYG{o}{=}tissuumaps \PYGZhy{}v /path/to/local/images:/mnt/data cavenel/tissuumaps:latest
\end{sphinxVerbatim}
\begin{enumerate}
\sphinxsetlistlabels{\arabic}{enumi}{enumii}{}{.}%
\item {} 
\sphinxAtStartPar
Place your images in the local folder \sphinxcode{\sphinxupquote{/path/to/local/images/share}}.

\item {} 
\sphinxAtStartPar
Open \sphinxurl{http://127.0.0.1:56733/} in your favorite browser.

\end{enumerate}

\sphinxstepscope


\chapter{Advanced usage}
\label{\detokenize{docs/advanced/index:advanced-usage}}\label{\detokenize{docs/advanced/index::doc}}
\sphinxstepscope


\section{Jupyter notebooks}
\label{\detokenize{docs/advanced/jupyter:jupyter-notebooks}}\label{\detokenize{docs/advanced/jupyter::doc}}
\sphinxAtStartPar
TissUUmaps can easily be used inside a Jupyter Notebook or Jupyter Lab.

\sphinxAtStartPar
Simple example to load an image in TissUUmaps:

\begin{sphinxVerbatim}[commandchars=\\\{\}]
\PYG{k+kn}{import} \PYG{n+nn}{tissuumaps}\PYG{n+nn}{.}\PYG{n+nn}{jupyter} \PYG{k}{as} \PYG{n+nn}{tj}
\PYG{n}{viewer} \PYG{o}{=} \PYG{n}{tj}\PYG{o}{.}\PYG{n}{loaddata}\PYG{p}{(}\PYG{p}{[}\PYG{l+s+s2}{\PYGZdq{}}\PYG{l+s+s2}{image.png}\PYG{l+s+s2}{\PYGZdq{}}\PYG{p}{]}\PYG{p}{)}

\PYG{n}{viewer}\PYG{o}{.}\PYG{n}{screenshot}\PYG{p}{(}\PYG{p}{)}
\end{sphinxVerbatim}
\phantomsection\label{\detokenize{docs/advanced/jupyter:module-tissuumaps.jupyter}}\index{module@\spxentry{module}!tissuumaps.jupyter@\spxentry{tissuumaps.jupyter}}\index{tissuumaps.jupyter@\spxentry{tissuumaps.jupyter}!module@\spxentry{module}}

\subsection{tissuumaps.jupyter}
\label{\detokenize{docs/advanced/jupyter:tissuumaps-jupyter}}
\sphinxAtStartPar
Module used to run TissUUmaps from a Jupyter Notebook or from Jupyter Lab.
\index{opentmap() (in module tissuumaps.jupyter)@\spxentry{opentmap()}\spxextra{in module tissuumaps.jupyter}}

\begin{fulllineitems}
\phantomsection\label{\detokenize{docs/advanced/jupyter:tissuumaps.jupyter.opentmap}}
\pysigstartsignatures
\pysiglinewithargsret{\sphinxcode{\sphinxupquote{tissuumaps.jupyter.}}\sphinxbfcode{\sphinxupquote{opentmap}}}{\emph{\DUrole{n}{path}}, \emph{\DUrole{n}{port}\DUrole{o}{=}\DUrole{default_value}{5100}}, \emph{\DUrole{n}{host}\DUrole{o}{=}\DUrole{default_value}{\textquotesingle{}localhost\textquotesingle{}}}, \emph{\DUrole{n}{height}\DUrole{o}{=}\DUrole{default_value}{700}}}{}
\pysigstopsignatures
\sphinxAtStartPar
Open a tmap project
\begin{quote}\begin{description}
\item[{Parameters}] \leavevmode\begin{itemize}
\item {} 
\sphinxAtStartPar
\sphinxstyleliteralstrong{\sphinxupquote{path}} (\sphinxstyleliteralemphasis{\sphinxupquote{str}}) \textendash{} The path to a tmap file

\item {} 
\sphinxAtStartPar
\sphinxstyleliteralstrong{\sphinxupquote{port}} (\sphinxstyleliteralemphasis{\sphinxupquote{int}}) \textendash{} The port to run the TissUUmaps server

\item {} 
\sphinxAtStartPar
\sphinxstyleliteralstrong{\sphinxupquote{host}} (\sphinxstyleliteralemphasis{\sphinxupquote{str}}) \textendash{} The host to run the TissUUmaps server

\item {} 
\sphinxAtStartPar
\sphinxstyleliteralstrong{\sphinxupquote{height}} (\sphinxstyleliteralemphasis{\sphinxupquote{int}}) \textendash{} The height of the jupyter iframe

\end{itemize}

\item[{Returns}] \leavevmode
\sphinxAtStartPar
The TissUUmaps viewer

\item[{Return type}] \leavevmode
\sphinxAtStartPar
{\hyperref[\detokenize{docs/advanced/jupyter:tissuumaps.jupyter.TissUUmapsViewer}]{\sphinxcrossref{TissUUmapsViewer}}}

\end{description}\end{quote}

\end{fulllineitems}

\index{loaddata() (in module tissuumaps.jupyter)@\spxentry{loaddata()}\spxextra{in module tissuumaps.jupyter}}

\begin{fulllineitems}
\phantomsection\label{\detokenize{docs/advanced/jupyter:tissuumaps.jupyter.loaddata}}
\pysigstartsignatures
\pysiglinewithargsret{\sphinxcode{\sphinxupquote{tissuumaps.jupyter.}}\sphinxbfcode{\sphinxupquote{loaddata}}}{\emph{\DUrole{n}{images}\DUrole{o}{=}\DUrole{default_value}{{[}{]}}}, \emph{\DUrole{n}{csvFiles}\DUrole{o}{=}\DUrole{default_value}{{[}{]}}}, \emph{\DUrole{n}{xSelector}\DUrole{o}{=}\DUrole{default_value}{\textquotesingle{}x\textquotesingle{}}}, \emph{\DUrole{n}{ySelector}\DUrole{o}{=}\DUrole{default_value}{\textquotesingle{}y\textquotesingle{}}}, \emph{\DUrole{n}{keySelector}\DUrole{o}{=}\DUrole{default_value}{None}}, \emph{\DUrole{n}{nameSelector}\DUrole{o}{=}\DUrole{default_value}{None}}, \emph{\DUrole{n}{colorSelector}\DUrole{o}{=}\DUrole{default_value}{None}}, \emph{\DUrole{n}{piechartSelector}\DUrole{o}{=}\DUrole{default_value}{None}}, \emph{\DUrole{n}{shapeSelector}\DUrole{o}{=}\DUrole{default_value}{None}}, \emph{\DUrole{n}{scaleSelector}\DUrole{o}{=}\DUrole{default_value}{None}}, \emph{\DUrole{n}{fixedShape}\DUrole{o}{=}\DUrole{default_value}{None}}, \emph{\DUrole{n}{scaleFactor}\DUrole{o}{=}\DUrole{default_value}{1}}, \emph{\DUrole{n}{colormap}\DUrole{o}{=}\DUrole{default_value}{None}}, \emph{\DUrole{n}{compositeMode}\DUrole{o}{=}\DUrole{default_value}{\textquotesingle{}source\sphinxhyphen{}over\textquotesingle{}}}, \emph{\DUrole{n}{boundingBox}\DUrole{o}{=}\DUrole{default_value}{None}}, \emph{\DUrole{n}{port}\DUrole{o}{=}\DUrole{default_value}{5100}}, \emph{\DUrole{n}{host}\DUrole{o}{=}\DUrole{default_value}{\textquotesingle{}localhost\textquotesingle{}}}, \emph{\DUrole{n}{height}\DUrole{o}{=}\DUrole{default_value}{700}}, \emph{\DUrole{n}{tmapFilename}\DUrole{o}{=}\DUrole{default_value}{\textquotesingle{}\_project\textquotesingle{}}}, \emph{\DUrole{n}{plugins}\DUrole{o}{=}\DUrole{default_value}{{[}{]}}}}{}
\pysigstopsignatures
\sphinxAtStartPar
Load data in TissUUmaps
\begin{quote}\begin{description}
\item[{Parameters}] \leavevmode\begin{itemize}
\item {} 
\sphinxAtStartPar
\sphinxstyleliteralstrong{\sphinxupquote{images}} (\sphinxstyleliteralemphasis{\sphinxupquote{list}}\sphinxstyleliteralemphasis{\sphinxupquote{ | }}\sphinxstyleliteralemphasis{\sphinxupquote{str}}) \textendash{} List of images or single image to display

\item {} 
\sphinxAtStartPar
\sphinxstyleliteralstrong{\sphinxupquote{csvFiles}} (list {\color{red}\bfseries{}|}str) \textendash{} List of csv files or single csv file to display

\item {} 
\sphinxAtStartPar
\sphinxstyleliteralstrong{\sphinxupquote{xSelector}} (\sphinxstyleliteralemphasis{\sphinxupquote{str}}) \textendash{} Name of the csv column defining the X coordinates

\item {} 
\sphinxAtStartPar
\sphinxstyleliteralstrong{\sphinxupquote{ySelector}} (\sphinxstyleliteralemphasis{\sphinxupquote{str}}) \textendash{} Name of the csv column defining the Y coordinates

\item {} 
\sphinxAtStartPar
\sphinxstyleliteralstrong{\sphinxupquote{keySelector}} (\sphinxstyleliteralemphasis{\sphinxupquote{str}}) \textendash{} Name of the csv column defining the grouping key

\item {} 
\sphinxAtStartPar
\sphinxstyleliteralstrong{\sphinxupquote{nameSelector}} (\sphinxstyleliteralemphasis{\sphinxupquote{str}}) \textendash{} Name of the csv column defining the group name

\item {} 
\sphinxAtStartPar
\sphinxstyleliteralstrong{\sphinxupquote{colorSelector}} (\sphinxstyleliteralemphasis{\sphinxupquote{str}}) \textendash{} Name of the csv column defining the group color

\item {} 
\sphinxAtStartPar
\sphinxstyleliteralstrong{\sphinxupquote{piechartSelector}} (\sphinxstyleliteralemphasis{\sphinxupquote{str}}) \textendash{} Name of the csv column defining pie\sphinxhyphen{}charts

\item {} 
\sphinxAtStartPar
\sphinxstyleliteralstrong{\sphinxupquote{shapeSelector}} (\sphinxstyleliteralemphasis{\sphinxupquote{str}}) \textendash{} Name of the csv column defining markers’ shape

\item {} 
\sphinxAtStartPar
\sphinxstyleliteralstrong{\sphinxupquote{scaleSelector}} (\sphinxstyleliteralemphasis{\sphinxupquote{str}}) \textendash{} Name of the csv column defining markers’ scale

\item {} 
\sphinxAtStartPar
\sphinxstyleliteralstrong{\sphinxupquote{fixedShape}} (\sphinxstyleliteralemphasis{\sphinxupquote{int}}) \textendash{} Name of the markers’ shape

\item {} 
\sphinxAtStartPar
\sphinxstyleliteralstrong{\sphinxupquote{scaleFactor}} (\sphinxstyleliteralemphasis{\sphinxupquote{int}}) \textendash{} Global scale of markers

\item {} 
\sphinxAtStartPar
\sphinxstyleliteralstrong{\sphinxupquote{colormap}} (\sphinxstyleliteralemphasis{\sphinxupquote{str}}) \textendash{} Name of the colormap used if colorSelector is set

\item {} 
\sphinxAtStartPar
\sphinxstyleliteralstrong{\sphinxupquote{compositeMode}} \textendash{} (str): Composite mode used for images

\item {} 
\sphinxAtStartPar
\sphinxstyleliteralstrong{\sphinxupquote{boundingBox}} (\sphinxstyleliteralemphasis{\sphinxupquote{list}}) \textendash{} {[}X,Y,W,H{]} of the bounding box to display

\item {} 
\sphinxAtStartPar
\sphinxstyleliteralstrong{\sphinxupquote{port}} (\sphinxstyleliteralemphasis{\sphinxupquote{int}}) \textendash{} The port to run the TissUUmaps server

\item {} 
\sphinxAtStartPar
\sphinxstyleliteralstrong{\sphinxupquote{host}} (\sphinxstyleliteralemphasis{\sphinxupquote{str}}) \textendash{} The host to run the TissUUmaps server

\item {} 
\sphinxAtStartPar
\sphinxstyleliteralstrong{\sphinxupquote{height}} (\sphinxstyleliteralemphasis{\sphinxupquote{int}}) \textendash{} The height of the jupyter iframe

\item {} 
\sphinxAtStartPar
\sphinxstyleliteralstrong{\sphinxupquote{tmapFilename}} (\sphinxstyleliteralemphasis{\sphinxupquote{str}}) \textendash{} Name of the project file that will be created

\item {} 
\sphinxAtStartPar
\sphinxstyleliteralstrong{\sphinxupquote{plugins}} (\sphinxstyleliteralemphasis{\sphinxupquote{list}}) \textendash{} List of plugins to add to the tmap project

\end{itemize}

\item[{Returns}] \leavevmode
\sphinxAtStartPar
The TissUUmaps viewer

\item[{Return type}] \leavevmode
\sphinxAtStartPar
{\hyperref[\detokenize{docs/advanced/jupyter:tissuumaps.jupyter.TissUUmapsViewer}]{\sphinxcrossref{TissUUmapsViewer}}}

\end{description}\end{quote}

\end{fulllineitems}

\index{TissUUmapsViewer (class in tissuumaps.jupyter)@\spxentry{TissUUmapsViewer}\spxextra{class in tissuumaps.jupyter}}

\begin{fulllineitems}
\phantomsection\label{\detokenize{docs/advanced/jupyter:tissuumaps.jupyter.TissUUmapsViewer}}
\pysigstartsignatures
\pysiglinewithargsret{\sphinxbfcode{\sphinxupquote{class\DUrole{w}{  }}}\sphinxcode{\sphinxupquote{tissuumaps.jupyter.}}\sphinxbfcode{\sphinxupquote{TissUUmapsViewer}}}{\emph{\DUrole{n}{server}}, \emph{\DUrole{n}{image}}, \emph{\DUrole{n}{height}\DUrole{o}{=}\DUrole{default_value}{700}}}{}
\pysigstopsignatures
\sphinxAtStartPar
Class representing a TissUUmaps viewer instance
\index{screenshot() (tissuumaps.jupyter.TissUUmapsViewer method)@\spxentry{screenshot()}\spxextra{tissuumaps.jupyter.TissUUmapsViewer method}}

\begin{fulllineitems}
\phantomsection\label{\detokenize{docs/advanced/jupyter:tissuumaps.jupyter.TissUUmapsViewer.screenshot}}
\pysigstartsignatures
\pysiglinewithargsret{\sphinxbfcode{\sphinxupquote{screenshot}}}{}{}
\pysigstopsignatures
\sphinxAtStartPar
Capture the TissUUmaps viewport and display image in the Notebook.

\end{fulllineitems}


\end{fulllineitems}

\index{TissUUmapsServer (class in tissuumaps.jupyter)@\spxentry{TissUUmapsServer}\spxextra{class in tissuumaps.jupyter}}

\begin{fulllineitems}
\phantomsection\label{\detokenize{docs/advanced/jupyter:tissuumaps.jupyter.TissUUmapsServer}}
\pysigstartsignatures
\pysiglinewithargsret{\sphinxbfcode{\sphinxupquote{class\DUrole{w}{  }}}\sphinxcode{\sphinxupquote{tissuumaps.jupyter.}}\sphinxbfcode{\sphinxupquote{TissUUmapsServer}}}{\emph{\DUrole{n}{slideDir}}, \emph{\DUrole{n}{port}\DUrole{o}{=}\DUrole{default_value}{5000}}, \emph{\DUrole{n}{host}\DUrole{o}{=}\DUrole{default_value}{\textquotesingle{}0.0.0.0\textquotesingle{}}}}{}
\pysigstopsignatures
\sphinxAtStartPar
Class representing a TissUUmaps server instance

\end{fulllineitems}


\sphinxstepscope


\section{Napari}
\label{\detokenize{docs/advanced/napari:napari}}\label{\detokenize{docs/advanced/napari::doc}}
\sphinxAtStartPar
Napari features an important hub containing 118 plugins at the time of writing, many of them expanding further the capabilities of Napari when dealing with biomedical imaging. We thus created our own plugin to allow users to work in Napari, benefit from the tools, scripting and existing plugins, and easily visualize and share the output of their research through TissUUmaps.

\sphinxAtStartPar
The \sphinxhref{https://github.com/TissUUmaps/napari-tissuumaps}{Napari\sphinxhyphen{}TissUUmaps plugin} is available on Napari Hub which makes the installation trivial: from the Napari install/uninstall plugins menu, the \sphinxcode{\sphinxupquote{napari\sphinxhyphen{}tissuumaps}} appears in the list and can be installed with a single click. Alternatively, the plugin can be installed with the Python package manager: \sphinxcode{\sphinxupquote{pip install napari\sphinxhyphen{}tissuumaps}}.

\sphinxAtStartPar
The plugin can export all standard Napari layers, such as images, labels, points, and shapes and preserves the metadata (opacity, visibility), but also the objects parameters (e.g.: label colors, marker colors and symbols, etc…). To export a TissUUmaps project, care must be taken to save all layers of interest and type in a name with the extension \sphinxcode{\sphinxupquote{.tmap}}, e.g.: \sphinxcode{\sphinxupquote{myProject.tmap}}. This is important for Napari to delegate the saving of the files to the plugin. A folder is created and contains all the necessary files and can be loaded in the TissUUmaps server, software, Jupyter Notebook, or shared with the community.

\sphinxAtStartPar
The project folders generated by the plugin contain the metadata in a \sphinxcode{\sphinxupquote{main.tmap}} file, along with folders for each Napari layer types: images, labels, points and regions. Images and labels are saved as plain tif images, points are saved as CSV files, and shapes are saved as GeoJSON. We hope that the use of a simple structure and widespread file formats can simplify the modifying and updating of the TissUUmaps project when prototyping with e.g. Jupyter Notebooks.
The source code is available at \sphinxurl{https://github.com/TissUUmaps/napari-tissuumaps} under the permissive MIT license.
A demonstration of the Cellpose plugin of Napari being exported to the TissUUmaps web viewer is available at: \sphinxurl{https://tissuumaps.github.io/tutorials/\#napari}.

\sphinxstepscope


\section{AnnData}
\label{\detokenize{docs/advanced/anndata:anndata}}\label{\detokenize{docs/advanced/anndata::doc}}
\sphinxAtStartPar
Work in progress



\renewcommand{\indexname}{Index}
\printindex
\end{document}