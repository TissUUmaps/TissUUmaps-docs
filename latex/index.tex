%% Generated by Sphinx.
\def\sphinxdocclass{report}
\documentclass[letterpaper,10pt,english,openany,oneside]{sphinxmanual}
\ifdefined\pdfpxdimen
   \let\sphinxpxdimen\pdfpxdimen\else\newdimen\sphinxpxdimen
\fi \sphinxpxdimen=.75bp\relax
\ifdefined\pdfimageresolution
    \pdfimageresolution= \numexpr \dimexpr1in\relax/\sphinxpxdimen\relax
\fi
%% let collapsible pdf bookmarks panel have high depth per default
\PassOptionsToPackage{bookmarksdepth=5}{hyperref}

\PassOptionsToPackage{warn}{textcomp}
\usepackage[utf8]{inputenc}
\ifdefined\DeclareUnicodeCharacter
% support both utf8 and utf8x syntaxes
  \ifdefined\DeclareUnicodeCharacterAsOptional
    \def\sphinxDUC#1{\DeclareUnicodeCharacter{"#1}}
  \else
    \let\sphinxDUC\DeclareUnicodeCharacter
  \fi
  \sphinxDUC{00A0}{\nobreakspace}
  \sphinxDUC{2500}{\sphinxunichar{2500}}
  \sphinxDUC{2502}{\sphinxunichar{2502}}
  \sphinxDUC{2514}{\sphinxunichar{2514}}
  \sphinxDUC{251C}{\sphinxunichar{251C}}
  \sphinxDUC{2572}{\textbackslash}
\fi
\usepackage{cmap}
\usepackage[T1]{fontenc}
\usepackage{amsmath,amssymb,amstext}
\usepackage{babel}



\usepackage{tgtermes}
\usepackage{tgheros}
\renewcommand{\ttdefault}{txtt}



\usepackage[Bjarne]{fncychap}
\usepackage{sphinx}

\fvset{fontsize=auto}
\usepackage{geometry}


% Include hyperref last.
\usepackage{hyperref}
% Fix anchor placement for figures with captions.
\usepackage{hypcap}% it must be loaded after hyperref.
% Set up styles of URL: it should be placed after hyperref.
\urlstyle{same}

\addto\captionsenglish{\renewcommand{\contentsname}{Contents:}}

\usepackage{sphinxmessages}
\setcounter{tocdepth}{1}



\title{TissUUmaps}
\date{Nov 28, 2023}
\release{3.0}
\author{Nicolas Pielawski\and Axel Andersson\and Christophe Avenel\and Andrea Behanova\and Eduard Chelebian\and Anna Klemm\and Fredrik Nysjö\and Leslie Solorzano\and Carolina Wählby}
\newcommand{\sphinxlogo}{\vbox{}}
\renewcommand{\releasename}{Release}
\makeindex
\begin{document}

\pagestyle{empty}
\sphinxmaketitle
\pagestyle{plain}
\sphinxtableofcontents
\pagestyle{normal}
\phantomsection\label{\detokenize{index::doc}}


\sphinxAtStartPar
This page hosts the documentation for TissUUmaps 3.2. You can find a pdf version of this documentation \sphinxhref{https://tissuumaps.github.io/TissUUmaps-docs/index.pdf}{here}.

\sphinxAtStartPar
For more information on the TissUUmaps project, including \sphinxhref{https://tissuumaps.github.io/tutorials/}{video tutorials} and \sphinxhref{https://tissuumaps.github.io/gallery/}{demos}, visit our website: \sphinxurl{https://tissuumaps.github.io}.

\begin{sphinxShadowBox}
\sphinxstylesidebartitle{Work in progress!}

\sphinxAtStartPar
We are working actively on writing this documentation, more content will be available soon!
\end{sphinxShadowBox}

\sphinxstepscope


\chapter{Introduction}
\label{\detokenize{docs/intro/index:introduction}}\label{\detokenize{docs/intro/index::doc}}
\sphinxstepscope


\section{About TissUUmaps}
\label{\detokenize{docs/intro/about:about-tissuumaps}}\label{\detokenize{docs/intro/about::doc}}
\sphinxAtStartPar
\sphinxhref{https://tissuumaps.github.io/}{TissUUmaps} is a free and open source browser\sphinxhyphen{}based tool for GPU\sphinxhyphen{}accelerated visualization and interactive exploration of tens of millions of datapoints overlaying tissue samples. Users can visualize markers and regions, explore spatial statistics and quantitative analyses of tissue morphology, and assess the quality of decoding in situ transcriptomics data. TissUUmaps provides instant multi\sphinxhyphen{}resolution image viewing, can be customized, shared, and also integrated in Jupyter Notebooks. We envision TissUUmaps to contribute to broader dissemination and flexible sharing of large\sphinxhyphen{}scale spatial omics data.

\sphinxAtStartPar
Currently, microscopy data can be cumbersome to share: physically transferring the images is often necessary and dedicated software must be installed. Instead, researchers can now share their findings with a simple link to a website running TissUUmaps. The images are loaded in real time, together with annotations, markers, and masks that may also be modified by the user. We also provide tools for quality control and image processing. The software is designed to display and interact with images at multiple resolutions and large numbers of markers, especially data from spatially resolved omics techniques and tissue atlases. TissUUmaps is compatible with many different bioimage informatics tools, and provides new ways to develop insights when exploring and sharing data.

\sphinxAtStartPar
You can access the \sphinxhref{https://tissuumaps.github.io/gallery/}{TissUUmaps project gallery} with interactive examples to explore data from in situ sequencing and spatial transcriptomics experiments and view localized quantification of cell and tissue morphology, including links to publications. For seeing examples of TissUUmaps compatibility with other platforms you can access the \sphinxhref{https://tissuumaps.github.io/tutorials/}{tutorials page}.

\sphinxstepscope


\section{Installation}
\label{\detokenize{docs/intro/installation:installation}}\label{\detokenize{docs/intro/installation::doc}}
\sphinxAtStartPar
\sphinxhref{https://tissuumaps.github.io/}{TissUUmaps} is a browser\sphinxhyphen{}based tool for fast visualization and exploration of millions of data points overlaying a tissue sample. TissUUmaps can be used as a web service or locally in your computer, and allows users to share regions of interest and local statistics.


\subsection{Windows installation}
\label{\detokenize{docs/intro/installation:windows-installation}}\begin{enumerate}
\sphinxsetlistlabels{\arabic}{enumi}{enumii}{}{.}%
\item {} 
\sphinxAtStartPar
Download the Windows Installer (\sphinxcode{\sphinxupquote{.exe}}) from the last release and install it. Note that the installer is not signed yet and may trigger warnings from the browser and from the firewall. You can safely pass these warnings.

\end{enumerate}


\subsection{MacOS installation}
\label{\detokenize{docs/intro/installation:macos-installation}}\begin{enumerate}
\sphinxsetlistlabels{\arabic}{enumi}{enumii}{}{.}%
\item {} 
\sphinxAtStartPar
Install vips with \sphinxcode{\sphinxupquote{brew install vips}} (needs homebrew installed https://brew.sh/).

\item {} 
\sphinxAtStartPar
Download the macOS installer (\sphinxcode{\sphinxupquote{.dmg}}) from the last release and install it. Download the \sphinxcode{\sphinxupquote{x86\_64}} file if you have an Intel CPU, or download the \sphinxcode{\sphinxupquote{arm64}} file if you have an Apple Silicon (M1/M2) CPU.

\item {} 
\sphinxAtStartPar
In the installer, drag\sphinxhyphen{}and\sphinxhyphen{}drop the TissUUmaps bundle to the Applications directory.

\item {} 
\sphinxAtStartPar
When the copy is finished, double\sphinxhyphen{}click the Applications icon in the installer and right\sphinxhyphen{}click + open TissUUmaps from the Applications menu.
\sphinxhyphen{} A warning should be prompted “macOS cannot verify the developer of TissUUmaps (…)”, click open and the program should launch.

\end{enumerate}


\subsection{Debian / Ubuntu installation}
\label{\detokenize{docs/intro/installation:debian-ubuntu-installation}}\begin{enumerate}
\sphinxsetlistlabels{\arabic}{enumi}{enumii}{}{.}%
\item {} 
\sphinxAtStartPar
Download the Ubuntu installer (\sphinxcode{\sphinxupquote{.deb}}) from the last release .deb file (20.04 or 22.04 depending on your Ubuntu version)

\end{enumerate}


\subsection{PIP installation}
\label{\detokenize{docs/intro/installation:pip-installation}}
\sphinxAtStartPar
If you want specific Python packages to be installed with TissUUmaps, or if your no installer is available for your operating system, you will need to install TissUUmaps using \sphinxcode{\sphinxupquote{pip}}:
\begin{enumerate}
\sphinxsetlistlabels{\arabic}{enumi}{enumii}{}{.}%
\item {} 
\sphinxAtStartPar
Install \sphinxcode{\sphinxupquote{libvips}} for your system: https://www.libvips.org/install.html:

\sphinxAtStartPar
An easy way to install \sphinxcode{\sphinxupquote{libvips}} is to use an Anaconda environment with \sphinxcode{\sphinxupquote{libvips}}:

\begin{sphinxVerbatim}[commandchars=\\\{\}]
conda\PYG{+w}{ }create\PYG{+w}{ }\PYGZhy{}y\PYG{+w}{ }\PYGZhy{}n\PYG{+w}{ }tissuumaps\PYGZus{}env\PYG{+w}{ }\PYGZhy{}c\PYG{+w}{ }conda\PYGZhy{}forge\PYG{+w}{ }\PYG{n+nv}{python}\PYG{o}{=}\PYG{l+m}{3}.9
conda\PYG{+w}{ }activate\PYG{+w}{ }tissuumaps\PYGZus{}env
\end{sphinxVerbatim}

\item {} 
\sphinxAtStartPar
Install dependencies using \sphinxcode{\sphinxupquote{conda}}:

\begin{sphinxVerbatim}[commandchars=\\\{\}]
conda\PYG{+w}{ }install\PYG{+w}{ }\PYGZhy{}c\PYG{+w}{ }conda\PYGZhy{}forge\PYG{+w}{ }libvips\PYG{+w}{ }pyvips\PYG{+w}{ }openslide\PYGZhy{}python
\end{sphinxVerbatim}

\item {} 
\sphinxAtStartPar
Install the TissUUmaps library using \sphinxcode{\sphinxupquote{pip}}:

\begin{sphinxVerbatim}[commandchars=\\\{\}]
pip\PYG{+w}{ }install\PYG{+w}{ }\PYG{l+s+s2}{\PYGZdq{}TissUUmaps[full]\PYGZdq{}}
\end{sphinxVerbatim}

\begin{sphinxadmonition}{note}{Note:}
\sphinxAtStartPar
If the installation fails with PyQt6, you can remove \sphinxcode{\sphinxupquote{{[}full{]}}} from the previous command and run step 5 to start TissUUmaps server.
\end{sphinxadmonition}

\item {} 
\sphinxAtStartPar
Start the TissUUmaps user interface:

\begin{sphinxVerbatim}[commandchars=\\\{\}]
tissuumaps
\end{sphinxVerbatim}

\item {} 
\sphinxAtStartPar
Or start TissUUmaps as a local server:

\begin{sphinxVerbatim}[commandchars=\\\{\}]
tissuumaps\PYGZus{}server\PYG{+w}{ }path\PYGZus{}to\PYGZus{}your\PYGZus{}images
\end{sphinxVerbatim}

\sphinxAtStartPar
And open \sphinxurl{http://127.0.0.1:5000/} in your favorite browser.

\end{enumerate}

\sphinxstepscope


\section{Citing TissUUmaps}
\label{\detokenize{docs/intro/citing:citing-tissuumaps}}\label{\detokenize{docs/intro/citing::doc}}
\sphinxAtStartPar
Please cite our \sphinxhref{https://www.biorxiv.org/content/10.1101/2022.01.28.478131v1}{preprint} on bioRxiv if using TissUUmaps in your work:

\sphinxAtStartPar
\sphinxstylestrong{TissUUmaps 3: Interactive visualization and quality assessment of large\sphinxhyphen{}scale spatial omics data.} \sphinxstyleemphasis{Nicolas Pielawski, Axel Andersson, Christophe Avenel, Andrea Behanova, Eduard Chelebian, Anna Klemm, Fredrik Nysjö, Leslie Solorzano, Carolina Wählby,} bioRxiv 2022.01.28.478131; doi: \sphinxurl{https://doi.org/10.1101/2022.01.28.478131}.

\sphinxstepscope


\section{Changelog}
\label{\detokenize{docs/intro/versions:changelog}}\label{\detokenize{docs/intro/versions::doc}}

\subsection{3.2}
\label{\detokenize{docs/intro/versions:id1}}\begin{itemize}
\item {} 
\sphinxAtStartPar
Add annotation tools (in collaboration with Sanofi Digital R\&D)

\item {} 
\sphinxAtStartPar
Move to WebGL rendering for regions

\item {} 
\sphinxAtStartPar
Refactor completely the region tab to better handle large number of items

\item {} 
\sphinxAtStartPar
Allow import of regions from multiple GeoJSON files

\item {} 
\sphinxAtStartPar
Add support for Geobuf (.pbf) region files

\item {} 
\sphinxAtStartPar
Add drag\sphinxhyphen{}and\sphinxhyphen{}drop opening of GeoJSON files

\item {} 
\sphinxAtStartPar
Add histogram button for analyzing individual regions

\item {} 
\sphinxAtStartPar
Make regions work in collection mode and with multiple image layers

\item {} 
\sphinxAtStartPar
Add marker outline and fill options in Advanced options

\item {} 
\sphinxAtStartPar
Increase number of barcodes that can be displayed for markers

\item {} 
\sphinxAtStartPar
Add validation of .tmap files (via tissuumaps\sphinxhyphen{}schema)

\item {} 
\sphinxAtStartPar
Move standalone installer from Python 3.8 to 3.11

\item {} 
\sphinxAtStartPar
Move to PySide6 instead of PyQt6

\item {} 
\sphinxAtStartPar
Add installers for MacOS X (.dmg) and Linux (.deb)

\item {} 
\sphinxAtStartPar
Update to OpenSeadragon 4.1.0

\item {} 
\sphinxAtStartPar
Add dzi support (opening as images directly)

\item {} 
\sphinxAtStartPar
Lots of minor fixes

\end{itemize}


\subsection{3.1.1.6}
\label{\detokenize{docs/intro/versions:id2}}\begin{itemize}
\item {} 
\sphinxAtStartPar
Add a min\sphinxhyphen{}max filter to image layers for better contrast (@glbarlow)

\item {} 
\sphinxAtStartPar
Fix marker filtering

\item {} 
\sphinxAtStartPar
Update license from BSD\sphinxhyphen{}3 to MIT

\item {} 
\sphinxAtStartPar
Add collapsible sections in plugins

\item {} 
\sphinxAtStartPar
Minor fixes

\end{itemize}


\subsection{3.1.1.5}
\label{\detokenize{docs/intro/versions:id3}}\begin{itemize}
\item {} 
\sphinxAtStartPar
Fix multiple images dropped on empty project

\item {} 
\sphinxAtStartPar
Change threading to allow higher priority to GUI than to server side

\item {} 
\sphinxAtStartPar
Add .tmap extension to all saved projects

\item {} 
\sphinxAtStartPar
Fix pie charts display update and colors

\item {} 
\sphinxAtStartPar
Add error message if update view fails

\end{itemize}


\subsection{3.1.1.4}
\label{\detokenize{docs/intro/versions:id4}}\begin{itemize}
\item {} 
\sphinxAtStartPar
Fix bug of modal not hiding properly

\item {} 
\sphinxAtStartPar
Fix capture with scale of 1

\item {} 
\sphinxAtStartPar
Fix loading of AnnData files without X matrix

\item {} 
\sphinxAtStartPar
Add DEFAULT\_PROJECT parameter for docker and server

\end{itemize}


\subsection{3.1.1.3}
\label{\detokenize{docs/intro/versions:id5}}\begin{itemize}
\item {} 
\sphinxAtStartPar
Minor fixes

\end{itemize}


\subsection{3.1.1.2}
\label{\detokenize{docs/intro/versions:id6}}\begin{itemize}
\item {} 
\sphinxAtStartPar
Make load of HDF5 data parallel

\item {} 
\sphinxAtStartPar
Add Colormaps and Channel splitting filters

\end{itemize}


\subsection{3.1.1}
\label{\detokenize{docs/intro/versions:id7}}\begin{itemize}
\item {} 
\sphinxAtStartPar
Move to OpenSeadragon 4.0.0

\item {} 
\sphinxAtStartPar
Move docker to python\sphinxhyphen{}alpine for security reasons

\item {} 
\sphinxAtStartPar
Add sorting options of markers (applied automatically on AnnData observations)

\item {} 
\sphinxAtStartPar
Add zOrder parameter for draw order between datasets

\item {} 
\sphinxAtStartPar
Make Update View button always visible

\item {} 
\sphinxAtStartPar
Add colorbar canvas to png captures

\item {} 
\sphinxAtStartPar
Add flip and rotation of markers for each collection item

\item {} 
\sphinxAtStartPar
Add transformation inputs in Image Layers menu

\item {} 
\sphinxAtStartPar
Add background color input

\item {} 
\sphinxAtStartPar
Add gaussian marker shape

\item {} 
\sphinxAtStartPar
Fix flask deprecated function

\item {} 
\sphinxAtStartPar
Minor fixes

\end{itemize}


\subsection{3.1.0.8}
\label{\detokenize{docs/intro/versions:id8}}\begin{itemize}
\item {} 
\sphinxAtStartPar
Fix chunk size for h5 files

\item {} 
\sphinxAtStartPar
Fix iPhone not loading markers with \textgreater{} 192 collection items

\end{itemize}


\subsection{3.1.0.7}
\label{\detokenize{docs/intro/versions:id9}}\begin{itemize}
\item {} 
\sphinxAtStartPar
Critical fix for Mac OS (see row\_major in WebGL2)

\end{itemize}


\subsection{3.1.0.6}
\label{\detokenize{docs/intro/versions:id10}}\begin{itemize}
\item {} 
\sphinxAtStartPar
Minor fixes

\end{itemize}


\subsection{3.1.0.5}
\label{\detokenize{docs/intro/versions:id11}}\begin{itemize}
\item {} 
\sphinxAtStartPar
Add flip and rotation of markers for each collection item

\item {} 
\sphinxAtStartPar
Add transformation inputs in Image Layers menu

\item {} 
\sphinxAtStartPar
Fix colormap with NaN values

\end{itemize}


\subsection{3.1.0.4}
\label{\detokenize{docs/intro/versions:id12}}\begin{itemize}
\item {} 
\sphinxAtStartPar
Fix issue with slow rendering of web\sphinxhyphen{}gl using instancing

\item {} 
\sphinxAtStartPar
Hide colorbar on png captures when needed

\item {} 
\sphinxAtStartPar
Other minor fixes

\end{itemize}


\subsection{3.1.0.3}
\label{\detokenize{docs/intro/versions:id13}}\begin{itemize}
\item {} 
\sphinxAtStartPar
Fix h5 autocomplete on iframes

\item {} 
\sphinxAtStartPar
Fix wrong marker index for color column when more than 100k markers

\item {} 
\sphinxAtStartPar
Add colorbar canvas to png captures

\item {} 
\sphinxAtStartPar
Small fixes

\end{itemize}


\subsection{3.1.0.2}
\label{\detokenize{docs/intro/versions:id14}}\begin{itemize}
\item {} 
\sphinxAtStartPar
Fix crash on layer from a parent layer.

\item {} 
\sphinxAtStartPar
Change dropdown selection from Chosen to Select2 for faster loading.

\item {} 
\sphinxAtStartPar
Update docker to Alpine for security reasons.

\item {} 
\sphinxAtStartPar
Small fixes

\end{itemize}


\subsection{3.1.0.1}
\label{\detokenize{docs/intro/versions:id15}}\begin{itemize}
\item {} 
\sphinxAtStartPar
Move docker to python\sphinxhyphen{}alpine for security reasons

\item {} 
\sphinxAtStartPar
Add sorting options of markers (applied automatically on AnnData observations)

\item {} 
\sphinxAtStartPar
Make Update View button always visible

\item {} 
\sphinxAtStartPar
Minor fixes

\end{itemize}


\subsection{3.1}
\label{\detokenize{docs/intro/versions:id16}}\begin{itemize}
\item {} 
\sphinxAtStartPar
Adding HDF5 support on the client side

\item {} 
\sphinxAtStartPar
Adding AnnData support on the server side / GUI

\item {} 
\sphinxAtStartPar
Adding Network diagram visualization

\item {} 
\sphinxAtStartPar
Tabs now saved automatically even without buttons

\item {} 
\sphinxAtStartPar
Adding Plugin helpers in javascript

\item {} 
\sphinxAtStartPar
Many fixes on the interface

\item {} 
\sphinxAtStartPar
Move to PyQt6

\end{itemize}


\subsection{3.0.10.4}
\label{\detokenize{docs/intro/versions:id17}}\begin{itemize}
\item {} 
\sphinxAtStartPar
Fix path issue on json loading from server

\end{itemize}


\subsection{3.0.10.3}
\label{\detokenize{docs/intro/versions:id18}}\begin{itemize}
\item {} 
\sphinxAtStartPar
Reset all input dropdowns when new data selected

\end{itemize}


\subsection{3.0.10.2}
\label{\detokenize{docs/intro/versions:id19}}\begin{itemize}
\item {} 
\sphinxAtStartPar
Add scale factor for coordinates of markers

\end{itemize}


\subsection{3.0.10.1}
\label{\detokenize{docs/intro/versions:id20}}\begin{itemize}
\item {} 
\sphinxAtStartPar
Add optional offset (x, y) and scale properties to tmap.layers

\end{itemize}


\subsection{3.0.10}
\label{\detokenize{docs/intro/versions:id21}}\begin{itemize}
\item {} 
\sphinxAtStartPar
Add collection mode (to display images next to each other with markers correctly placed)

\item {} 
\sphinxAtStartPar
IFrame mode (to hide navbar and make menu smaller when TissUUmaps is run inside an iFrame)

\end{itemize}


\subsection{3.0.9.6}
\label{\detokenize{docs/intro/versions:id22}}\begin{itemize}
\item {} 
\sphinxAtStartPar
Add debug menu when running in debug mode, with debug access in javascript

\item {} 
\sphinxAtStartPar
Fix linux bugs with Qt displaying all blank

\item {} 
\sphinxAtStartPar
Fix empty columns in marker csv file

\end{itemize}


\subsection{3.0.9.5}
\label{\detokenize{docs/intro/versions:id23}}\begin{itemize}
\item {} 
\sphinxAtStartPar
Add / fix key shortcuts (https://tissuumaps.github.io/TissUUmaps\sphinxhyphen{}docs/docs/starting/shortcuts.html)

\item {} 
\sphinxAtStartPar
Change default GUI port to avoid collisions with server

\item {} 
\sphinxAtStartPar
Add plugin support to docker server

\end{itemize}


\subsection{3.0.9.3}
\label{\detokenize{docs/intro/versions:id24}}\begin{itemize}
\item {} 
\sphinxAtStartPar
Go back to webGL 1 for compatibility issue with Safari 14

\item {} 
\sphinxAtStartPar
Fix missing .tissuumaps folder for recent files

\end{itemize}


\subsection{3.0.9.1}
\label{\detokenize{docs/intro/versions:id25}}\begin{itemize}
\item {} 
\sphinxAtStartPar
Enable larger markers at high resolutiion (up to 1024x1024px)

\item {} 
\sphinxAtStartPar
Fix pinch to zoom center

\item {} 
\sphinxAtStartPar
Add code of conduct

\item {} 
\sphinxAtStartPar
Clean code and use ci (pre\sphinxhyphen{}commit)

\end{itemize}


\subsection{3.0.9}
\label{\detokenize{docs/intro/versions:id26}}\begin{itemize}
\item {} 
\sphinxAtStartPar
Move to webgl2

\item {} 
\sphinxAtStartPar
Add Open Recent sub menu in File menu

\item {} 
\sphinxAtStartPar
Fix path for linux and mac in server mode

\item {} 
\sphinxAtStartPar
Minor fixes

\end{itemize}


\subsection{3.0.8.9}
\label{\detokenize{docs/intro/versions:id27}}\begin{itemize}
\item {} 
\sphinxAtStartPar
Make it possible to update to newer version of plugins

\item {} 
\sphinxAtStartPar
Fully support \textendash{}debug option in command line

\item {} 
\sphinxAtStartPar
Add tooltip title for piecharts

\item {} 
\sphinxAtStartPar
Add documentation https://tissuumaps.github.io/TissUUmaps\sphinxhyphen{}docs/

\item {} 
\sphinxAtStartPar
Fix marker picking when pixel ratio != 1

\item {} 
\sphinxAtStartPar
Other minor fixes and cleaning

\end{itemize}


\subsection{3.0.8.5}
\label{\detokenize{docs/intro/versions:id28}}\begin{itemize}
\item {} 
\sphinxAtStartPar
Minor fixes.

\end{itemize}


\subsection{3.0.8.4}
\label{\detokenize{docs/intro/versions:id29}}\begin{itemize}
\item {} 
\sphinxAtStartPar
Add tiling to viewport capture for higher resolution output

\item {} 
\sphinxAtStartPar
Increase resolution of markers on high resolution devices

\item {} 
\sphinxAtStartPar
Fix jumps on pan with mouse gesture (mobile)

\item {} 
\sphinxAtStartPar
Add fix for bright image canvas on Safari

\item {} 
\sphinxAtStartPar
Add an option to remove markers’ outlines.

\end{itemize}


\subsection{3.0.8.3}
\label{\detokenize{docs/intro/versions:id30}}\begin{itemize}
\item {} 
\sphinxAtStartPar
Fix png artifact in Firefox, by generating jpg tiles.

\end{itemize}


\subsection{3.0.8.2}
\label{\detokenize{docs/intro/versions:id31}}\begin{itemize}
\item {} 
\sphinxAtStartPar
Add high resolution capture of viewport, up to 4096x4096 pixels.

\end{itemize}


\subsection{3.0.8.1}
\label{\detokenize{docs/intro/versions:id32}}\begin{itemize}
\item {} 
\sphinxAtStartPar
Fix multiple dataset alignment when no background image

\end{itemize}


\subsection{3.0.8}
\label{\detokenize{docs/intro/versions:id33}}\begin{itemize}
\item {} 
\sphinxAtStartPar
Fix black images generated by VIPS

\item {} 
\sphinxAtStartPar
Fix Linux and Mac open of captures

\item {} 
\sphinxAtStartPar
Auto save datasets as buttons when saving tmap projects

\item {} 
\sphinxAtStartPar
Add \sphinxcode{\sphinxupquote{mpp}} (microns per pixel) option in tmap files, to add scale bar to viewer

\item {} 
\sphinxAtStartPar
Make region line thickness depend on zoom level

\item {} 
\sphinxAtStartPar
Add compatibility with JupyterLab

\item {} 
\sphinxAtStartPar
Add opacity per marker option

\end{itemize}


\subsection{3.0.7}
\label{\detokenize{docs/intro/versions:id34}}\begin{itemize}
\item {} 
\sphinxAtStartPar
Add menu to load plugins through an update\sphinxhyphen{}site

\end{itemize}


\subsection{3.0.6}
\label{\detokenize{docs/intro/versions:id35}}\begin{itemize}
\item {} 
\sphinxAtStartPar
Fix multiple plugins opening always last plugin

\item {} 
\sphinxAtStartPar
Move to OpenSeadragon 3.0.0

\item {} 
\sphinxAtStartPar
Add tooltip format in Advanced Settings

\item {} 
\sphinxAtStartPar
Add drag and drop to open CSV files and images

\item {} 
\sphinxAtStartPar
Add “Add layer” button for flask version

\item {} 
\sphinxAtStartPar
Add viewport capture

\end{itemize}


\subsection{3.0.5}
\label{\detokenize{docs/intro/versions:id36}}\begin{itemize}
\item {} 
\sphinxAtStartPar
Move csv loading to Papa Parse streaming, to allow better memory management

\end{itemize}


\subsection{3.0.4}
\label{\detokenize{docs/intro/versions:id37}}\begin{itemize}
\item {} 
\sphinxAtStartPar
Add filtering of markers

\end{itemize}


\subsection{3.0}
\label{\detokenize{docs/intro/versions:id38}}\begin{itemize}
\item {} 
\sphinxAtStartPar
Add tissuumaps.jupyter module

\end{itemize}

\sphinxstepscope


\chapter{Getting started}
\label{\detokenize{docs/starting/index:getting-started}}\label{\detokenize{docs/starting/index::doc}}
\sphinxstepscope


\section{Images}
\label{\detokenize{docs/starting/images:images}}\label{\detokenize{docs/starting/images::doc}}

\subsection{Supported image formats}
\label{\detokenize{docs/starting/images:supported-image-formats}}
\sphinxAtStartPar
TissUUmaps can read whole slide images in any format recognized by the OpenSlide library:
\begin{itemize}
\item {} 
\sphinxAtStartPar
Aperio (.svs, .tif)

\item {} 
\sphinxAtStartPar
Hamamatsu (.ndpi, .vms, .vmu)

\item {} 
\sphinxAtStartPar
Leica (.scn)

\item {} 
\sphinxAtStartPar
MIRAX (.mrxs)

\item {} 
\sphinxAtStartPar
Philips (.tiff)

\item {} 
\sphinxAtStartPar
Sakura (.svslide)

\item {} 
\sphinxAtStartPar
Trestle (.tif)

\item {} 
\sphinxAtStartPar
Ventana (.bif, .tif)

\item {} 
\sphinxAtStartPar
Generic tiled TIFF (.tif)

\end{itemize}

\sphinxAtStartPar
TissUUmaps will automatically convert any other format into a pyramidal tiff (in a temporary \sphinxcode{\sphinxupquote{.tissuumaps}} folder created in the original image folder) using vips.

\sphinxAtStartPar
If your image fails to open, try converting it to \sphinxcode{\sphinxupquote{tif}} format using an external tool.


\subsection{Load images}
\label{\detokenize{docs/starting/images:load-images}}
\sphinxAtStartPar
You can load the images when you select the Image layer tab as you can see in the figure below:
\sphinxincludegraphics{{image_layers}.png}

\sphinxAtStartPar
Then click the button Add image layer and select the desired image from your computer. Subsequently, the image is listed in the Image layer tab. You can load several images into TissUUmaps.
\sphinxincludegraphics{{image_layers_many}.png}

\sphinxAtStartPar
You can also drag and drop the image from file explorer into TissUUmaps.
\sphinxincludegraphics{{drag_drop_image}.png}


\subsection{Load images using TissUUmaps server}
\label{\detokenize{docs/starting/images:load-images-using-tissuumaps-server}}
\sphinxAtStartPar
If you are running TissUUmaps in server mode and not through the GUI, you must specify an image folder in the command line:

\begin{sphinxVerbatim}[commandchars=\\\{\}]
python\PYG{+w}{ }\PYGZhy{}m\PYG{+w}{ }tissuumaps\PYG{+w}{ }\PYG{l+s+s2}{\PYGZdq{}/home/username/Documents/myImages/\PYGZdq{}}\PYG{+w}{ }\PYGZhy{}p\PYG{+w}{ }\PYG{l+m}{5005}
\end{sphinxVerbatim}

\sphinxAtStartPar
You can then access your images from your web browser by accessing the url \sphinxurl{http://localhost:5005}, and using the \sphinxcode{\sphinxupquote{File \textgreater{} Open}} menu.

\sphinxAtStartPar
\sphinxincludegraphics{{server_open_menu1}.png}

\sphinxAtStartPar
\sphinxincludegraphics{{server_open_menu2}.png}


\subsection{Apply filters}
\label{\detokenize{docs/starting/images:apply-filters}}
\sphinxAtStartPar
You can apply several filters to the images. The ones we can be adjusted by default are saturation, brightness, and contrast. Additionally, when opening the Filter settings menu, there are various other filters, such as exposure, noise, erosion, etc. When you check their box, they are automatically added to the filter panel above. The filter’s sliders can be adjusted so that the filter is applied at the desired intensity. Another option in filter settings is merging mode (bottom part), where you can merge the channels as a composite.

\sphinxAtStartPar
\sphinxincludegraphics{{Filters}.png}


\subsection{Apply transformations}
\label{\detokenize{docs/starting/images:apply-transformations}}
\sphinxAtStartPar
From the image list in the right menu, you can apply several transformations to each image individually: translation, scale, rotation and flip. Markers and regions linked to the image will be transformed accordingly.

\sphinxAtStartPar
\sphinxincludegraphics{{image_tranformations}.png}

\sphinxstepscope


\section{Markers}
\label{\detokenize{docs/starting/markers:markers}}\label{\detokenize{docs/starting/markers::doc}}

\subsection{Supported marker format}
\label{\detokenize{docs/starting/markers:supported-marker-format}}
\sphinxAtStartPar
TissUUmaps can read CSV (Comma Separated Values) files with a header row, and at least spatial coordinate columns (X and Y). CSV files are not limited in the number of columns or number of rows. Other columns can contain information for displaying markers (key to group markers, color, size, shape, piecharts, etc.)

\sphinxAtStartPar
CSV files can be exported from any spreadsheet program, or any programming language (Python, R, etc.)

\sphinxAtStartPar
TissUUmaps can also read hdf5 files, which is  a hierarchical file format commonly used in scientific computing. In the case of TissUUmaps, it is used to store the markers in a more efficient way than CSV files. TissUUmaps will give access to the markers stored in the hdf5 file in a similar way as for CSV files.


\subsection{Load markers}
\label{\detokenize{docs/starting/markers:load-markers}}
\sphinxAtStartPar
You can load the markers when you select the \sphinxstyleemphasis{Markers} tab and click the button + as you can see in the figure below. You can click the plus several times to load various marker files.
\sphinxincludegraphics{{markers}.png}

\sphinxAtStartPar
You can also load markers directly using drag and drop from a File Explorer if you are using the TissUUmaps GUI.


\subsection{Markers settings}
\label{\detokenize{docs/starting/markers:markers-settings}}
\sphinxAtStartPar
Before the markers are displayed you have to set up the markers settings.


\subsubsection{File and coordinates}
\label{\detokenize{docs/starting/markers:file-and-coordinates}}
\sphinxAtStartPar
The first step is to select the desired file from your computer under the tab \sphinxstyleemphasis{File and coordinates \sphinxhyphen{} Choose file}.

\sphinxAtStartPar
\sphinxincludegraphics{{load_markers}.png}

\sphinxAtStartPar
You can change the \sphinxstyleemphasis{Tab name} to the desired name, so it is easier to navigate between them when there are more tabs.

\sphinxAtStartPar
\sphinxincludegraphics{{Tab_name}.png}

\sphinxAtStartPar
The next step is to select the column names from the .csv file corresponding to the X and Y coordinates.

\sphinxAtStartPar
\sphinxincludegraphics{{XY_coordinates}.png}


\subsubsection{Render options}
\label{\detokenize{docs/starting/markers:render-options}}
\sphinxAtStartPar
Here, you can define a \sphinxstyleemphasis{key to group by}, what is a column from the .csv file which will be used to display the dataset grouped by different colors and shapes of the markers. In this example, we use the column \sphinxstyleemphasis{genes}, where different colors and shapes of markers represent different genes.

\sphinxAtStartPar
\sphinxincludegraphics{{group_by}.png}

\sphinxAtStartPar
There is an option to display an extra column, for example when the data are clustered but you want to see the original genes and also the cluster names.

\sphinxAtStartPar
\sphinxincludegraphics{{group_by2}.png}

\sphinxAtStartPar
In \sphinxstyleemphasis{Color options}, you can select to color by groups where each group has a different color. Then on the right side, you can select the color palette:
\begin{itemize}
\item {} 
\sphinxAtStartPar
Key value \sphinxhyphen{} Colors are generated from the name of the group (first 4 letters). Groups starting with the same letter have similar colors.

\item {} 
\sphinxAtStartPar
Randomly \sphinxhyphen{} Colors are generated randomly.

\item {} 
\sphinxAtStartPar
Dictionary \sphinxhyphen{} you can insert a specific dictionary in the text area which will be used for generating the colors.

\end{itemize}

\sphinxAtStartPar
\sphinxincludegraphics{{Color_options}.png}

\sphinxAtStartPar
If you want to \sphinxstyleemphasis{color by markers}, you have to select the column from the .csv file which will be used to create the colors, and the colormap, but only if the color column is numeral.

\sphinxAtStartPar
\sphinxincludegraphics{{Color_options2}.png}


\subsubsection{Advanced options}
\label{\detokenize{docs/starting/markers:advanced-options}}
\sphinxAtStartPar
TissUUmaps tool contains also advanced options when working with the data. The first one is adjustable marker size. This is usually done in the right upper corner of the visualization panel. However, in the advanced setting, the user can change the size factor of the slider to any value.

\sphinxAtStartPar
\sphinxincludegraphics{{Advanced_size}.png}

\sphinxAtStartPar
Additionally, there can be used a different size per marker based on a selected column. In the example below, I  chose column \sphinxstyleemphasis{counts} which represents the number of counts in that cell (marker). This means that a larger marker represents a cell that contains more counts in it.

\sphinxAtStartPar
\sphinxincludegraphics{{Advanced_size_ex}.png}

\sphinxAtStartPar
Another advanced option is the choice to display markers as pie\sphinxhyphen{}charts, it can represent the probability of that marker belonging to different groups. The user needs to select the \sphinxstyleemphasis{pie\sphinxhyphen{}chart column}, which contains mentioned probabilities for all the markers. All the probabilities for that specific marker need to be in a row divided by a semicolon.

\sphinxAtStartPar
\sphinxincludegraphics{{Advanced_pie}.png}

\sphinxAtStartPar
In the example below the pie\sphinxhyphen{}charts represent the probability of the marker being of each cell type. In the left upper corner can be seen the legend of the cell types. By default, there are only 10 colors so the colors are used in the loop. This can be changed by using pie\sphinxhyphen{}chart colors from a dictionary.

\sphinxAtStartPar
\sphinxincludegraphics{{Advanced_pie_ex}.png}

\sphinxAtStartPar
The shape of the markers can also be changed. By default, it is set up to be selected by the group, which means that each group has a different marker shape chosen from the list of shapes iteratively. The user can also pre\sphinxhyphen{}define the shapes from the dictionary to ensure visualization robustness in different sessions.

\sphinxAtStartPar
\sphinxincludegraphics{{Advanced_shape}.png}

\sphinxAtStartPar
In the example below each group has specific color as well as a specific marker shape.

\sphinxAtStartPar
\sphinxincludegraphics{{Advanced_shape_ex}.png}

\sphinxAtStartPar
Another option for the marker shape is \sphinxstyleemphasis{shape by marker}. Here, the user needs to select a column with category values, and each category is used for a different shape.

\sphinxAtStartPar
\sphinxincludegraphics{{Advanced_shape2}.png}

\sphinxAtStartPar
In the example below, the selected column is ARIH1, which contains 10 categories, so you can see that there are 10 shapes in the visualization.

\sphinxAtStartPar
\sphinxincludegraphics{{Advanced_shape2_ex}.png}

\sphinxAtStartPar
The third option in the marker shape is to \sphinxstyleemphasis{use a fixed shape}. This can be used if the user is not happy with all the different marker shapes and wants to make it homogeneous.

\sphinxAtStartPar
\sphinxincludegraphics{{Advanced_shape3}.png}

\sphinxAtStartPar
In the example below, the selected shape is a clobber and you can see that all the markers are in the shape of a clobber.

\sphinxAtStartPar
\sphinxincludegraphics{{Advanced_shape3_ex}.png}

\sphinxAtStartPar
The last option in the marker shape is to \sphinxstyleemphasis{remove outline}. This can be used to remove the dark outline of markers when the check box is checked.

\sphinxAtStartPar
\sphinxincludegraphics{{Advanced_shape4}.png}

\sphinxAtStartPar
In the example below, the outline is included on the left side and the outline is removed on the right side.

\sphinxAtStartPar
\sphinxincludegraphics{{Advanced_shape4_ex}.png}

\sphinxAtStartPar
The next advanced option is \sphinxstyleemphasis{marker opacity} which is adjustable. The user can change the opacity in order to display things underneath.

\sphinxAtStartPar
\sphinxincludegraphics{{Advanced_opacity}.png}

\sphinxAtStartPar
In the example below, the opacity value was set to 0.6 which made the markers a bit transparent.

\sphinxAtStartPar
\sphinxincludegraphics{{Advanced_opacity_ex}.png}

\sphinxAtStartPar
In the example below, we checked \sphinxstyleemphasis{use different opacity per marker}. The user needs to select opacity column which will be used for displaying different opacities in markers.

\sphinxAtStartPar
\sphinxincludegraphics{{Advanced_opacity_ex2}.png}

\sphinxAtStartPar
The next option is the \sphinxstyleemphasis{marker tooltip}, which is the text which is displayed when the user clicks on the marker. By default, it displayed the key group the marker belongs to.

\sphinxAtStartPar
\sphinxincludegraphics{{Advanced_tip}.png}

\sphinxAtStartPar
As you can see in the example below, the green marker we clicked on belongs to the cell type Proliferating epithelial 2.

\sphinxAtStartPar
\sphinxincludegraphics{{Advanced_tip_ex}.png}

\sphinxAtStartPar
However, this can be modified by writing text into the text area. In this example, we wrote \{tab\} \sphinxhyphen{} \{key\} \sphinxhyphen{} \{col\_counts\}. \sphinxstyleemphasis{\{tab\}} represents the tab name that we set when we loaded the markers, \sphinxstyleemphasis{\{key\}} represents the key group to which the marker belongs and \sphinxstyleemphasis{\{col\_counts\}} represents the value of that marker in the column called counts. The word counts can be replaced by any column name in order to display it on the tooltip.

\sphinxAtStartPar
\sphinxincludegraphics{{Advanced_tip2}.png}

\sphinxAtStartPar
In the example below, the green dot which we clicked on is from the tab Cell types, belongs to the cell type Proliferating epithelial 2 and it has 127 counts in it.

\sphinxAtStartPar
\sphinxincludegraphics{{Advanced_tip2_ex}.png}

\sphinxAtStartPar
The last advanced option is the button \sphinxstyleemphasis{Generate button from tab}. This button incorporates all the display settings the user set up into a single button.

\sphinxAtStartPar
\sphinxincludegraphics{{Advanced_gen_button}.png}

\sphinxAtStartPar
The user can choose the relative path to the csv file, button inner text and comment which will be displayed next to the button.

\sphinxAtStartPar
\sphinxincludegraphics{{Advanced_gen_win}.png}

\sphinxAtStartPar
In the example below, you can see the generated button \sphinxstyleemphasis{Download data} placed on the top of the tabs panel. On the right of the button is placed text \sphinxstyleemphasis{My settings}.

\sphinxAtStartPar
\sphinxincludegraphics{{Advanced_gen_ex}.png}


\subsubsection{Table of markers}
\label{\detokenize{docs/starting/markers:table-of-markers}}
\sphinxAtStartPar
When the markers are loaded, a table of markers will appear in order to interact with the marker. Each row represents a group of markers with a specific color and shape. In the figure below, column \sphinxstyleemphasis{A)} represents if a specific row of markers is displayed or not, the second column \sphinxstyleemphasis{B)} represents the list of groups, the third column \sphinxstyleemphasis{C)} represents group counts, the fourth column \sphinxstyleemphasis{D)} represents the shape of the group markers, the fifth column \sphinxstyleemphasis{E)} represents the color of the group markers and the sixth column \sphinxstyleemphasis{F)} can display specific group when the cursor is on the eye icon.

\sphinxAtStartPar
\sphinxincludegraphics{{Table_general}.png}

\sphinxAtStartPar
If the check box is checked \sphinxhyphen{} the group is displayed, if the check box is unchecked \sphinxhyphen{} the group is not displayed. In the example below, we checked two groups of cell types: Airway Fibroblast and Airway smooth muscle, and only these two groups are displayed on the left visualization panel. The first checkbox \sphinxstyleemphasis{All} ensures displaying of all the markers.

\sphinxAtStartPar
\sphinxincludegraphics{{Table_check}.png}

\sphinxAtStartPar
In the fourth column \sphinxstyleemphasis{Shape}, the user can select which shape is preferred for each marker group. In the figure below, there is a list of 14 different shapes which can be used.

\sphinxAtStartPar
\sphinxincludegraphics{{Table_Shape}.png}

\sphinxAtStartPar
In the fifth column \sphinxstyleemphasis{Color}, the user can select which color is preferred for each marker group. In the figure below, it is possible to choose from some list of basic colors, select a specific color by the cursor from the palette and also use numbers to generate color, either RGB, HSV, or HTML.

\sphinxAtStartPar
\sphinxincludegraphics{{Table_Color}.png}

\sphinxAtStartPar
In the example below can be seen that if the cursor is placed on the eye icon in the row Airway fibroblast, only markers of this group are displayed on the visualization panel.

\sphinxAtStartPar
\sphinxincludegraphics{{Table_eye}.png}

\sphinxstepscope


\section{Regions}
\label{\detokenize{docs/starting/regions:regions}}\label{\detokenize{docs/starting/regions::doc}}

\subsection{Supported region formats}
\label{\detokenize{docs/starting/regions:supported-region-formats}}
\sphinxAtStartPar
TissUUmaps can read and write region files in the GeoJSON format.

\sphinxAtStartPar
Only a subset of the GeoJSON format is supported, as TissUUmaps uses only polygonal regions:

\sphinxAtStartPar
\sphinxstylestrong{Main types}:
\begin{itemize}
\item {} 
\sphinxAtStartPar
Feature

\item {} 
\sphinxAtStartPar
FeatureCollection

\item {} 
\sphinxAtStartPar
GeometryCollection

\end{itemize}

\sphinxAtStartPar
\sphinxstylestrong{Geometries}:
\begin{itemize}
\item {} 
\sphinxAtStartPar
Polygon

\item {} 
\sphinxAtStartPar
Multipolygon

\end{itemize}

\sphinxAtStartPar
The coordinate system must be the same as the image and marker coordinate systems.

\sphinxAtStartPar
TissUUmaps is a powerful annotation tool designed to facilitate the annotation process for biological tissues.


\subsection{Toolbar Features}
\label{\detokenize{docs/starting/regions:toolbar-features}}
\sphinxAtStartPar
When entering the Regions tab, the user will see the following toolbar:
\sphinxincludegraphics{{Regions_toolbar}.png}


\subsubsection{Drawing Tools}
\label{\detokenize{docs/starting/regions:drawing-tools}}
\sphinxAtStartPar
When the user clicks on the drawing tool, a dropdown menu will appear with five different drawing tools:
\sphinxincludegraphics{{Regions_drawing_tools}.png}


\paragraph{1. Free Hand Drawing}
\label{\detokenize{docs/starting/regions:free-hand-drawing}}
\sphinxAtStartPar
The free hand drawing tool allows users to annotate regions with free\sphinxhyphen{}form shapes.


\paragraph{2. Point\sphinxhyphen{}Based Drawing}
\label{\detokenize{docs/starting/regions:point-based-drawing}}
\sphinxAtStartPar
Use the point\sphinxhyphen{}based drawing tool to create annotations by placing individual points.


\paragraph{3. Brush\sphinxhyphen{}Based Drawing}
\label{\detokenize{docs/starting/regions:brush-based-drawing}}
\sphinxAtStartPar
The brush\sphinxhyphen{}based drawing tool enables users to draw annotations using a brush\sphinxhyphen{}like tool. Press \sphinxcode{\sphinxupquote{Shift}} to erase and \sphinxcode{\sphinxupquote{Ctrl}} to add to selected region.


\paragraph{4. Rectangle Drawing}
\label{\detokenize{docs/starting/regions:rectangle-drawing}}
\sphinxAtStartPar
Create rectangular annotations by selecting the rectangle drawing tool. Pressing Shift while dragging makes it a square, and Ctrl centers it around the cursor.


\paragraph{5. Ellipse Drawing}
\label{\detokenize{docs/starting/regions:ellipse-drawing}}
\sphinxAtStartPar
Similar to the rectangle tool, the ellipse drawing tool allows users to create ellipses. Press Shift for a circle and Ctrl for centering.


\subsubsection{Other Tools}
\label{\detokenize{docs/starting/regions:other-tools}}\begin{itemize}
\item {} 
\sphinxAtStartPar
\sphinxstylestrong{Selection Tool}: Click on regions to select them. Press Shift to select multiple regions.

\item {} 
\sphinxAtStartPar
\sphinxstylestrong{Show Instance}: Color each region randomly to distinguish between polygons.

\item {} 
\sphinxAtStartPar
\sphinxstylestrong{Fill Opacity}: Control if regions are filled, and the opacity of the filling.

\item {} 
\sphinxAtStartPar
\sphinxstylestrong{Line Width}: Adjust line width and determine if it adapts when zooming.

\end{itemize}


\subsubsection{Selected Region Tools}
\label{\detokenize{docs/starting/regions:selected-region-tools}}
\sphinxAtStartPar
When a region is selected, additional tools become available:
\begin{itemize}
\item {} 
\sphinxAtStartPar
\sphinxstylestrong{Zoom to Selected Regions}

\item {} 
\sphinxAtStartPar
\sphinxstylestrong{Unselect All Regions (Shortcut: Escape)}

\item {} 
\sphinxAtStartPar
\sphinxstylestrong{Delete Selected Region}

\item {} 
\sphinxAtStartPar
\sphinxstylestrong{Duplicate Region}

\item {} 
\sphinxAtStartPar
\sphinxstylestrong{Scale Region}

\item {} 
\sphinxAtStartPar
\sphinxstylestrong{Erode/Dilate Regions}

\item {} 
\sphinxAtStartPar
\sphinxstylestrong{Split Multipolygons into Multiple Regions}

\item {} 
\sphinxAtStartPar
\sphinxstylestrong{Fill Holes in Regions}

\end{itemize}


\subsubsection{Multiple Selected Regions}
\label{\detokenize{docs/starting/regions:multiple-selected-regions}}
\sphinxAtStartPar
When multiple regions are selected, access the “Boolean Operation” dropdown with options like:
\begin{itemize}
\item {} 
\sphinxAtStartPar
\sphinxstylestrong{Merge Selected Regions}

\item {} 
\sphinxAtStartPar
\sphinxstylestrong{XOR Selected Regions}

\item {} 
\sphinxAtStartPar
\sphinxstylestrong{Intersect Selected Regions}

\end{itemize}


\subsection{List of regions in the right menu}
\label{\detokenize{docs/starting/regions:list-of-regions-in-the-right-menu}}
\sphinxAtStartPar
\sphinxincludegraphics{{Regions_List_menu}.png}

\sphinxAtStartPar
On the right side, there is a menu listing all regions ordered by class. For each region:
\begin{itemize}
\item {} 
\sphinxAtStartPar
\sphinxstylestrong{Select}: Click to select the region in the viewer.

\item {} 
\sphinxAtStartPar
\sphinxstylestrong{Change Class}: Modify the class of the region.

\item {} 
\sphinxAtStartPar
\sphinxstylestrong{Change Name}: Edit the name of the region.

\item {} 
\sphinxAtStartPar
\sphinxstylestrong{Statistics}: View region statistics.

\item {} 
\sphinxAtStartPar
\sphinxstylestrong{Hide}: Toggle the visibility of the region.

\item {} 
\sphinxAtStartPar
\sphinxstylestrong{Delete}: Remove the region.

\end{itemize}

\sphinxAtStartPar
For groups of regions of one class:
\begin{itemize}
\item {} 
\sphinxAtStartPar
\sphinxstylestrong{Rename Class}

\item {} 
\sphinxAtStartPar
\sphinxstylestrong{Change Class Color}

\item {} 
\sphinxAtStartPar
\sphinxstylestrong{Hide All Regions}

\item {} 
\sphinxAtStartPar
\sphinxstylestrong{Delete All Regions}

\end{itemize}

\sphinxAtStartPar
For all regions:
\begin{itemize}
\item {} 
\sphinxAtStartPar
\sphinxstylestrong{Hide All}

\item {} 
\sphinxAtStartPar
\sphinxstylestrong{Delete All}

\end{itemize}


\subsection{Region Statistics}
\label{\detokenize{docs/starting/regions:region-statistics}}
\sphinxAtStartPar
Clicking on the statistics button for a specific region reveals:
\begin{itemize}
\item {} 
\sphinxAtStartPar
\sphinxstylestrong{Area}

\item {} 
\sphinxAtStartPar
\sphinxstylestrong{Perimeter}

\item {} 
\sphinxAtStartPar
\sphinxstylestrong{Number of Sub\sphinxhyphen{}regions}

\item {} 
\sphinxAtStartPar
\sphinxstylestrong{Bounds (left, top, right, bottom in pixel coordinates)}

\item {} 
\sphinxAtStartPar
\sphinxstylestrong{Number of Each Type of Markers}

\end{itemize}

\sphinxAtStartPar
\sphinxincludegraphics{{Regions_statistics}.png}


\subsection{Import Regions}
\label{\detokenize{docs/starting/regions:import-regions}}
\sphinxAtStartPar
Regions can be imported from .json file, which could be achieved from an external software or also from TissUUmaps’ plugin \sphinxstyleemphasis{Points2Regions}. The user just click on the tab \sphinxstyleemphasis{Import} \sphinxhyphen{}\textgreater{} \sphinxstyleemphasis{Choose File} and press the button \sphinxstyleemphasis{Import}.

\sphinxAtStartPar
\sphinxincludegraphics{{Regions_Import}.png}

\sphinxAtStartPar
After that, the displayed regions appear in the left panel and the list of regions in the right panel as you can see in the example below. In this case, there are 10 different regions, called clusters. The user can change the color, the name, and the class of the regions if necessary. The user can as well draw some extra regions. These regions can be analyzed to observe the marker expression.

\sphinxAtStartPar
\sphinxincludegraphics{{Regions_Import_ex}.png}


\subsection{Export Regions}
\label{\detokenize{docs/starting/regions:export-regions}}
\sphinxAtStartPar
The regions can be exported by clicking the tab \sphinxstyleemphasis{Export}, there the user can export two types of files. The first one is the .json file and the name can be selected. The second file is the marker expression in the regions which can be exported as .csv file (this is exported only if the regions were analyzed).

\sphinxAtStartPar
\sphinxincludegraphics{{Regions_Export}.png}

\sphinxAtStartPar
In the figure below can be seen an example of the exported .cvs file.

\sphinxAtStartPar
\sphinxincludegraphics{{Regions_Export_ex}.png}

\sphinxstepscope


\section{Projects}
\label{\detokenize{docs/starting/projects:projects}}\label{\detokenize{docs/starting/projects::doc}}

\subsection{Saving projects}
\label{\detokenize{docs/starting/projects:saving-projects}}
\sphinxAtStartPar
When the user has finished the visualization adjustments, region drawings, etc., the project is ready to be saved in order to continue working on it later or just basically to save it as it is for further consistency. The user needs to press \sphinxstyleemphasis{File} in the menu and then \sphinxstyleemphasis{Save project} or Ctrl + S.

\sphinxAtStartPar
\sphinxincludegraphics{{Project_Save}.png}

\sphinxAtStartPar
In order to save the project together with the .csv file, it is necessary to generate a button first. The warning window below appears and the user needs to generate the button. The path to the .csv file needs to be relative to the path of the image. In this example, the image layer and the .csv file are in the exact same directory.

\sphinxAtStartPar
\sphinxincludegraphics{{Project_Save_GenButton}.png}

\sphinxAtStartPar
Then the user selects a suitable directory to save the project and writes the project file name, i.e. My\_project.tmap, and the project is saved.

\sphinxAtStartPar
\sphinxincludegraphics{{Project_Save_dic}.png}


\subsection{Loading projects}
\label{\detokenize{docs/starting/projects:loading-projects}}
\sphinxAtStartPar
The .tmap project can be loaded by two approaches. The first one is opening the TissUUmaps program, click \sphinxstyleemphasis{File} in the menu and then \sphinxstyleemphasis{Open} or Crtl + O. Then the user navigates in the directory and selects the .tmap file. By default, the directory navigates in the recent .tmap project.

\sphinxAtStartPar
\sphinxincludegraphics{{Project_load_open}.png}

\sphinxAtStartPar
The second option is directly double click on the .tmap file in file explorer in your computer.

\sphinxAtStartPar
\sphinxincludegraphics{{Project_load_directly}.png}

\sphinxAtStartPar
After clicking the button \sphinxstyleemphasis{Download data}, both these approaches will lead to loading the project as can be seen in the example below.

\sphinxAtStartPar
\sphinxincludegraphics{{Project_load_result}.png}

\sphinxAtStartPar
For more information on the tmap file format and specifications, see {\hyperref[\detokenize{docs/advanced/tmap:the-tmap-file-format}]{\sphinxcrossref{\DUrole{std,std-ref}{The TMAP file format}}}}.


\subsection{Editing .tmap file manually}
\label{\detokenize{docs/starting/projects:editing-tmap-file-manually}}
\sphinxAtStartPar
You can edit the \sphinxcode{\sphinxupquote{.tmap}} file manually in a text editor. The \sphinxcode{\sphinxupquote{.tmap}} file is a JSON file, which is a human\sphinxhyphen{}readable format. The \sphinxcode{\sphinxupquote{.tmap}} file contains all the information about the project, including the image layers, regions, and annotations. You can read the specification of the \sphinxcode{\sphinxupquote{.tmap}} file in {\hyperref[\detokenize{docs/advanced/tmap:the-tmap-file-format}]{\sphinxcrossref{\DUrole{std,std-ref}{The TMAP file format}}}}.


\subsection{Existing projects}
\label{\detokenize{docs/starting/projects:existing-projects}}

\subsubsection{Human Developmental Lung Cell Atlas (pcw 5\sphinxhyphen{} pcw 14)}
\label{\detokenize{docs/starting/projects:human-developmental-lung-cell-atlas-pcw-5-pcw-14}}
\sphinxAtStartPar
The human lung is a highly complex tubular organ, whose main function is the gas exchange between blood and breathed air. In contains a large number of specialized cell\sphinxhyphen{}types of epithelial, endothelial, neuronal, stromal and immune cells that are necessary for normal organ function and structural integrity. To understand how this cell heterogeneity develops to create a healthy mature lung, we focused on the 1st trimester of gestation and applied state of art technologies to capture the gene expression profiles of all the cells in the developing organ, in time and space.

\sphinxAtStartPar
\sphinxincludegraphics{{Lung_Cell_Atlas}.png}

\sphinxAtStartPar
\sphinxstylestrong{TissUUmaps interactive viewer}: Single\sphinxhyphen{}cell RNA\sphinxhyphen{}sequencing UMAP representation of single\sphinxhyphen{}cell clusters and sub\sphinxhyphen{}clusters, gene expression and metadata.

\sphinxAtStartPar
\sphinxstylestrong{In situ sequencing data (ISS) \sphinxhyphen{} TissUUmaps interactive viewer}: pcw 5  pcw 6  pcw 13In situ sequencing data. Spot location + identity, per bin pie chart view of cell type probabilities and imputed genes.

\sphinxAtStartPar
\sphinxstylestrong{SCRINSHOT data \sphinxhyphen{} TissUUmaps interactive viewer}: pcw 6  pcw 8  pcw 11  pcw 14SCRINSHOT data. Spot location + identity.

\sphinxAtStartPar
\sphinxstylestrong{Spatial Transcriptomics data \sphinxhyphen{} TissUUmaps interactive viewer}: pcw 6  pcw 8  pcw 10  pcw 11Per gene or pie chart view of gene expression.

\sphinxAtStartPar
More information is available in the original \sphinxhref{https://doi.org/10.1101/2022.01.11.475631}{publication}: A. Sountoulidis, S.M. Salas, E. Braun, C. Avenel, J. Bergenstråhle, M. Vicari, P. Czarnewski, J. Theelke, A. Liontos, X. Abalo, Ž. Andrusivová, M. Asp, X. Li, L. Hu, S. Sariyar, A.M. Casals, B. Ayoglu, A. Firsova, J. Michaëlsson, E. Lundberg, C. Wählby, E. Sundström, S. Linnarsson, J. Lundeberg, M. Nilsson, C. Samakovlis. Developmental origins of cell heterogeneity in the human lung. BioRxiv doi: \sphinxurl{https://doi.org/10.1101/2022.01.11.475631}


\subsubsection{Modelling of cell\sphinxhyphen{}type signatures in the developmental human heart}
\label{\detokenize{docs/starting/projects:modelling-of-cell-type-signatures-in-the-developmental-human-heart}}
\sphinxAtStartPar
With the emergence of high throughput single cell techniques, the understanding of cellular diversity in biologically complex processes has rapidly increased. The next step towards comprehension of e.g. key organs in the mammal development is to obtain spatiotemporal atlases of the cellular diversity. However, targeted cell typing approaches relying on existing single cell data achieve incomplete and biased maps that could mask the molecular and cellular heterogeneity present in a tissue slide. Here we applied spage2vec, a de novo approach to spatially resolve and characterize cellular diversity during human heart development. Data from the original in situ sequencing experiment as well as identified cell types can be viewed in TissUUmaps.

\sphinxAtStartPar
\sphinxincludegraphics{{de_novo}.png}

\sphinxAtStartPar
\sphinxstylestrong{TissUUmaps interactive viewer}:
Human heart

\sphinxAtStartPar
More information is available in the original \sphinxhref{https://doi.org/10.1101/2021.07.10.451822}{publication}: RS. Marco Salas, X. Yuan,  C. Sylven,  M. Nilsson,  C. Wählby and  G.Partel. De novo spatiotemporal modelling of cell\sphinxhyphen{}type signatures identifies novel cell populations in the developmental human heart. BioRxiv doi: \sphinxurl{https://doi.org/10.1101/2021.07.10.451822}


\subsubsection{Automated identification of the mouse brain’s spatial compartments from in situ sequencing data}
\label{\detokenize{docs/starting/projects:automated-identification-of-the-mouse-brain-s-spatial-compartments-from-in-situ-sequencing-data}}
\sphinxAtStartPar
Neuroanatomical compartments of the mouse brain are identified and outlined mainly based on manual annotations of samples using features related to tissue and cellular morphology, taking advantage of publicly available reference atlases. However, this task is challenging since sliced tissue sections are rarely perfectly parallel or angled with respect to sections in the reference atlas and organs from different individuals may vary in size and shape and requires manual annotation. Here, we show how in situ sequencing data combined with dimensionality reduction and unsupervised clustering can be used to identify spatial compartments that correspond to known anatomical compartments of the brain. Here we show results on four different sections of mouse brains.

\sphinxAtStartPar
\sphinxincludegraphics{{automated}.png}

\sphinxAtStartPar
\sphinxstylestrong{TissUUmaps interactive viewer}:
Mouse brain

\sphinxAtStartPar
More information is available in this \sphinxhref{https://doi.org/10.1186/s12915-020-00874-5}{publication}: G. Partel, M.M. Hilscher, G. Milli, L. Solorzano, A.H. Klemm, M. Nilsson, and C. Wählby.  Automated identification of the mouse brain’s spatial compartments from in situ sequencing data.  BMC Biology, \sphinxurl{https://doi.org/10.1186/s12915-020-00874-5}, Oct 2020.

\sphinxAtStartPar
The original raw ISS data was published in Qian, X., Harris, K. D., Hauling, T., Nicoloutsopoulos, D., Muñoz\sphinxhyphen{}Manchado, A. B., Skene, N., … \& Nilsson, M. (2020). Probabilistic cell typing enables fine mapping of closely related cell types in situ. \sphinxhref{https://doi.org/10.1038/s41592-019-0631-4}{Nature methods}, 17(1), 101\sphinxhyphen{}106.

\sphinxAtStartPar
Data and code availability: All software was developed in Python 3 using open source libraries, and data processing of pipeline workflows was carried out using \sphinxhref{https://doi.org/10.1093/bioinformatics/btz133}{Anduril2} analysis framework. The processing pipelines, data, and the software version used to generate the analysis results and figures presented in this paper are available at \sphinxurl{https://doi.org/10.5281/zenodo.3928219} or from our github repository \sphinxurl{https://github.com/wahlby-lab/graph-iss}.


\subsubsection{Spage2vec: Unsupervised representation of localized spatial gene expression signatures}
\label{\detokenize{docs/starting/projects:spage2vec-unsupervised-representation-of-localized-spatial-gene-expression-signatures}}
\sphinxAtStartPar
Spage2vec is an unsupervised segmentation free approach for decrypting the spatial transcriptomic heterogeneity of complex tissues at subcellular resolution. Spage2vec represents the spatial transcriptomic landscape of tissue samples as a graph and leverage powerful machine learning graph representation technique to create a lower dimensional representation of local spatial gene expression. Here we visualize spage2vec localized gene expression signatures of different spatial transcriptomic datasets. We thank Mats Nilsson, Sten Linnarsson and Xiaowei Zhuang for making their datasets publicly available.

\sphinxAtStartPar
\sphinxincludegraphics{{spage2vec}.png}

\sphinxAtStartPar
\sphinxstylestrong{TissUUmaps interactive viewer 1}: In situ sequencing mouse brain hippocampal area CA1 Qian, X., Harris, K. D., Hauling, T., Nicoloutsopoulos, D., Muñoz\sphinxhyphen{}Manchado, A. B., Skene, N., … \& Nilsson, M. (2020). Probabilistic cell typing enables fine mapping of closely related cell types in situ. \sphinxhref{https://doi.org/10.1038/s41592-019-0631-4}{Nature methods}, 17(1), 101\sphinxhyphen{}106.

\sphinxAtStartPar
\sphinxstylestrong{TissUUmaps interactive viewer 2}: osmFISH mouse brain somatosensory cortex Codeluppi, S., Borm, L. E., Zeisel, A., La Manno, G., van Lunteren, J. A., Svensson, C. I., \& Linnarsson, S. (2018). Spatial organization of the somatosensory cortex revealed by osmFISH. \sphinxhref{https://doi.org/10.1038/s41592-018-0175-z}{Nature methods}, 15(11), 932\sphinxhyphen{}935.

\sphinxAtStartPar
\sphinxstylestrong{TissUUmaps interactive viewer 3}: MERFISH mouse brain hypothalamic preoptic area Moffitt, J. R., Bambah\sphinxhyphen{}Mukku, D., Eichhorn, S. W., Vaughn, E., Shekhar, K., Perez, J. D., … \& Zhuang, X. (2018). Molecular, spatial, and functional single\sphinxhyphen{}cell profiling of the hypothalamic preoptic region. \sphinxhref{https://doi.org/10.1126/science.aau5324}{Science}, 362(6416), eaau5324.

\sphinxAtStartPar
\sphinxstylestrong{TissUUmaps interactive viewer 4}: MERFISH human fibroblast cells (IMR90) Chen, K. H., Boettiger, A. N., Moffitt, J. R., Wang, S., \& Zhuang, X. (2015). Spatially resolved, highly multiplexed RNA profiling in single cells. \sphinxhref{https://doi.org/10.1126/science.aaa6090}{Science}, 348(6233), aaa6090.

\sphinxAtStartPar
Data and code availability: Spatial gene expression data are available in Zenodo database at \sphinxurl{https://doi.org/10.5281/zenodo.3897401}.
Source code for reproducing analysis results and figures is available in Zenodo database at \sphinxurl{http://www.doi.org/10.5281/zenodo.4030404}.


\subsubsection{Artificial intelligence for diagnosis and grading of prostate cancer in biopsies: a population\sphinxhyphen{}based}
\label{\detokenize{docs/starting/projects:artificial-intelligence-for-diagnosis-and-grading-of-prostate-cancer-in-biopsies-a-population-based}}
\sphinxAtStartPar
An increasing volume of prostate biopsies and a worldwide shortage of urological pathologists puts a
strain on pathology departments. Additionally, the high intra\sphinxhyphen{}observer and inter\sphinxhyphen{}observer variability in grading can
result in overtreatment and undertreatment of prostate cancer. To alleviate these problems, we aimed to develop an
artificial intelligence (AI) system with clinically acceptable accuracy for prostate cancer detection, localisation, and
Gleason grading. Here we show examples of full\sphinxhyphen{}resolution digitized biopsies and corresponding AI\sphinxhyphen{}based grading.

\sphinxAtStartPar
\sphinxincludegraphics{{prostate}.png}

\sphinxAtStartPar
An overview of all sample \sphinxstylestrong{datasets} can be found here: Prostate cancer in biopsies 

\sphinxAtStartPar
More information is available in this \sphinxhref{https://www.sciencedirect.com/science/article/pii/S1470204519307387}{publication}: P. Ström, K. Kartasalo, H. Olsson, L. Solorzano et al. Artificial intelligence for diagnosis and grading of prostate cancer in biopsies: a population\sphinxhyphen{}based, diagnostic study. The Lancet Oncology, Volume 21, Issue 2, 2020, Pages 222\sphinxhyphen{}232, ISSN 1470\sphinxhyphen{}2045,  doi: 10.1016/S1470\sphinxhyphen{}2045(19)30738\sphinxhyphen{}7, url: \sphinxurl{https://www.sciencedirect.com/science/article/pii/S1470204519307387}

\sphinxstepscope


\section{Exporting screenshots}
\label{\detokenize{docs/starting/capture:exporting-screenshots}}\label{\detokenize{docs/starting/capture::doc}}
\sphinxAtStartPar
TissUUmaps allows high resolution capture of the image viewport. Go to \sphinxcode{\sphinxupquote{Menu \textgreater{} File \textgreater{} Capture viewport}} and chose a zoom factor for export (1 = screen resolution).

\sphinxAtStartPar
The screen capture will contain all filtered layers, markers, and regions. Note that legends will not be part of the export and must be added manually.

\sphinxstepscope


\section{Plugins}
\label{\detokenize{docs/starting/plugins:plugins}}\label{\detokenize{docs/starting/plugins::doc}}

\subsection{Load plugins}
\label{\detokenize{docs/starting/plugins:load-plugins}}
\sphinxAtStartPar
In order to load plugins, first, they need to be installed. This can be done in the menu \sphinxcode{\sphinxupquote{Plugins \textgreater{} Add plugin}} as can be seen in the example below.

\sphinxAtStartPar
\sphinxincludegraphics{{Add_plugin1}.png}

\sphinxAtStartPar
Consequently, the user can check any number of plugins they desire and press \sphinxstyleemphasis{OK}.

\sphinxAtStartPar
\sphinxincludegraphics{{Add_plugin2}.png}

\sphinxAtStartPar
Then the tool warns the user then the installed plugins will be available after restarting TissUUmaps.

\sphinxAtStartPar
\sphinxincludegraphics{{Add_plugin3}.png}

\sphinxAtStartPar
After restarting the TissUUmaps, all the installed plugins are listed in the menu \sphinxcode{\sphinxupquote{Plugins}} as you can see in the figure below.

\sphinxAtStartPar
\sphinxincludegraphics{{Add_plugin4}.png}

\sphinxAtStartPar
Once the user selects any of the installed plugins (in the example below I selected \sphinxstyleemphasis{Feature\_Space}), a new tab \sphinxstyleemphasis{Plugins} appears in the upper right part of the screen with all the required boxes for filling.

\sphinxAtStartPar
\sphinxincludegraphics{{Add_plugin5}.png}


\subsection{Make your own plugin}
\label{\detokenize{docs/starting/plugins:make-your-own-plugin}}
\sphinxAtStartPar
Download the Plugin Template python and javascript files from the \sphinxhref{https://tissuumaps.github.io/TissUUmaps/plugins/}{Plugin Update Site} and put both files in your local folder \sphinxcode{\sphinxupquote{\$USER\_PATH/.tissuumaps/plugins/}}. You can then change the plugin name and add your own options and functions.


\subsubsection{Javascript file}
\label{\detokenize{docs/starting/plugins:javascript-file}}
\sphinxAtStartPar
When loading a plugin, the function \sphinxcode{\sphinxupquote{PluginName.init(container)}} will be called. The \sphinxcode{\sphinxupquote{container}} is an html Element that will be added to the plugin menu. Use this element to add options and texts related to your plugin.

\sphinxAtStartPar
\sphinxincludegraphics{{plugin_container}.png}


\paragraph{Javascript example:}
\label{\detokenize{docs/starting/plugins:javascript-example}}
\sphinxAtStartPar
File \sphinxcode{\sphinxupquote{Plugin\_template.js}}:

\begin{sphinxVerbatim}[commandchars=\\\{\}]
\PYG{k+kd}{var}\PYG{+w}{ }\PYG{n+nx}{Plugin\PYGZus{}template}\PYG{p}{;}
\PYG{n+nx}{Plugin\PYGZus{}template}\PYG{+w}{ }\PYG{o}{=}\PYG{+w}{ }\PYG{p}{\PYGZob{}}
\PYG{+w}{  }\PYG{n+nx}{name}\PYG{o}{:}\PYG{+w}{ }\PYG{l+s+s2}{\PYGZdq{}Template Plugin\PYGZdq{}}\PYG{p}{,}
\PYG{+w}{  }\PYG{n+nx}{parameters}\PYG{o}{:}\PYG{+w}{ }\PYG{p}{\PYGZob{}}
\PYG{+w}{    }\PYG{n+nx}{\PYGZus{}section\PYGZus{}test}\PYG{o}{:}\PYG{+w}{ }\PYG{p}{\PYGZob{}}
\PYG{+w}{      }\PYG{l+s+s2}{\PYGZdq{}label\PYGZdq{}}\PYG{o}{:}\PYG{+w}{ }\PYG{l+s+s2}{\PYGZdq{}Test section\PYGZdq{}}\PYG{p}{,}
\PYG{+w}{      }\PYG{l+s+s2}{\PYGZdq{}title\PYGZdq{}}\PYG{o}{:}\PYG{+w}{ }\PYG{l+s+s2}{\PYGZdq{}Section 1\PYGZdq{}}\PYG{p}{,}
\PYG{+w}{      }\PYG{l+s+s2}{\PYGZdq{}type\PYGZdq{}}\PYG{o}{:}\PYG{+w}{ }\PYG{l+s+s2}{\PYGZdq{}section\PYGZdq{}}\PYG{p}{,}
\PYG{+w}{      }\PYG{l+s+s2}{\PYGZdq{}collapsed\PYGZdq{}}\PYG{o}{:}\PYG{+w}{ }\PYG{k+kc}{false}
\PYG{+w}{    }\PYG{p}{\PYGZcb{}}\PYG{p}{,}
\PYG{+w}{    }\PYG{n+nx}{\PYGZus{}message}\PYG{o}{:}\PYG{+w}{ }\PYG{p}{\PYGZob{}}
\PYG{+w}{      }\PYG{l+s+s2}{\PYGZdq{}label\PYGZdq{}}\PYG{o}{:}\PYG{+w}{ }\PYG{l+s+s2}{\PYGZdq{}Message\PYGZdq{}}\PYG{p}{,}
\PYG{+w}{      }\PYG{l+s+s2}{\PYGZdq{}type\PYGZdq{}}\PYG{o}{:}\PYG{+w}{ }\PYG{l+s+s2}{\PYGZdq{}text\PYGZdq{}}\PYG{p}{,}
\PYG{+w}{      }\PYG{l+s+s2}{\PYGZdq{}default\PYGZdq{}}\PYG{o}{:}\PYG{+w}{ }\PYG{l+s+s2}{\PYGZdq{}Hello world\PYGZdq{}}
\PYG{+w}{    }\PYG{p}{\PYGZcb{}}\PYG{p}{,}
\PYG{+w}{    }\PYG{n+nx}{\PYGZus{}testButton}\PYG{o}{:}\PYG{+w}{ }\PYG{p}{\PYGZob{}}
\PYG{+w}{      }\PYG{l+s+s2}{\PYGZdq{}label\PYGZdq{}}\PYG{o}{:}\PYG{+w}{ }\PYG{l+s+s2}{\PYGZdq{}Test button\PYGZdq{}}\PYG{p}{,}
\PYG{+w}{      }\PYG{l+s+s2}{\PYGZdq{}type\PYGZdq{}}\PYG{o}{:}\PYG{+w}{ }\PYG{l+s+s2}{\PYGZdq{}button\PYGZdq{}}\PYG{p}{,}
\PYG{+w}{    }\PYG{p}{\PYGZcb{}}\PYG{p}{,}
\PYG{+w}{  }\PYG{p}{\PYGZcb{}}
\PYG{p}{\PYGZcb{}}\PYG{p}{;}

\PYG{c+cm}{/**}
\PYG{c+cm}{ * This method is called when the document is loaded.}
\PYG{c+cm}{ * The container element is a div where the plugin options will be displayed.}
\PYG{c+cm}{ * @summary After setting up the tmapp object, initialize it*/}
\PYG{n+nx}{Plugin\PYGZus{}template}\PYG{p}{.}\PYG{n+nx}{init}\PYG{+w}{ }\PYG{o}{=}\PYG{+w}{ }\PYG{k+kd}{function}\PYG{+w}{ }\PYG{p}{(}\PYG{n+nx}{container}\PYG{p}{)}\PYG{+w}{ }\PYG{p}{\PYGZob{}}
\PYG{+w}{  }\PYG{n+nx}{interfaceUtils}\PYG{p}{.}\PYG{n+nx}{alert}\PYG{p}{(}\PYG{l+s+s2}{\PYGZdq{}The plugin has been loaded\PYGZdq{}}\PYG{p}{)}\PYG{p}{;}
\PYG{p}{\PYGZcb{}}\PYG{p}{;}

\PYG{c+cm}{/**}
\PYG{c+cm}{ * This method is called when a button is clicked or a parameter value is changed*/}
\PYG{n+nx}{Plugin\PYGZus{}template}\PYG{p}{.}\PYG{n+nx}{inputTrigger}\PYG{+w}{ }\PYG{o}{=}\PYG{+w}{ }\PYG{k+kd}{function}\PYG{+w}{ }\PYG{p}{(}\PYG{n+nx}{input}\PYG{p}{)}\PYG{+w}{ }\PYG{p}{\PYGZob{}}
\PYG{+w}{  }\PYG{n+nx}{console}\PYG{p}{.}\PYG{n+nx}{log}\PYG{p}{(}\PYG{l+s+s2}{\PYGZdq{}inputTrigger\PYGZdq{}}\PYG{p}{,}\PYG{+w}{ }\PYG{n+nx}{input}\PYG{p}{)}\PYG{p}{;}
\PYG{+w}{  }\PYG{k}{if}\PYG{+w}{ }\PYG{p}{(}\PYG{n+nx}{input}\PYG{+w}{ }\PYG{o}{===}\PYG{+w}{ }\PYG{l+s+s2}{\PYGZdq{}\PYGZus{}testButton\PYGZdq{}}\PYG{p}{)}\PYG{+w}{ }\PYG{p}{\PYGZob{}}
\PYG{+w}{    }\PYG{k+kd}{let}\PYG{+w}{ }\PYG{n+nx}{message}\PYG{+w}{ }\PYG{o}{=}\PYG{+w}{ }\PYG{n+nx}{Plugin\PYGZus{}template}\PYG{p}{.}\PYG{n+nx}{get}\PYG{p}{(}\PYG{l+s+s2}{\PYGZdq{}\PYGZus{}message\PYGZdq{}}\PYG{p}{)}\PYG{p}{;}
\PYG{+w}{    }\PYG{n+nx}{Plugin\PYGZus{}template}\PYG{p}{.}\PYG{n+nx}{demo}\PYG{p}{(}\PYG{n+nx}{message}\PYG{p}{)}\PYG{p}{;}
\PYG{+w}{  }\PYG{p}{\PYGZcb{}}
\PYG{p}{\PYGZcb{}}

\PYG{n+nx}{Plugin\PYGZus{}template}\PYG{p}{.}\PYG{n+nx}{demo}\PYG{+w}{ }\PYG{o}{=}\PYG{+w}{ }\PYG{k+kd}{function}\PYG{+w}{ }\PYG{p}{(}\PYG{n+nx}{message}\PYG{p}{)}\PYG{+w}{ }\PYG{p}{\PYGZob{}}
\PYG{+w}{  }\PYG{k+kd}{let}\PYG{+w}{ }\PYG{n+nx}{successCallback}\PYG{+w}{ }\PYG{o}{=}\PYG{+w}{ }\PYG{k+kd}{function}\PYG{+w}{ }\PYG{p}{(}\PYG{n+nx}{data}\PYG{p}{)}\PYG{+w}{ }\PYG{p}{\PYGZob{}}
\PYG{+w}{    }\PYG{n+nx}{interfaceUtils}\PYG{p}{.}\PYG{n+nx}{alert}\PYG{p}{(}\PYG{n+nx}{data}\PYG{p}{)}\PYG{p}{;}
\PYG{+w}{  }\PYG{p}{\PYGZcb{}}\PYG{p}{;}
\PYG{+w}{  }\PYG{k+kd}{let}\PYG{+w}{ }\PYG{n+nx}{errorCallback}\PYG{+w}{ }\PYG{o}{=}\PYG{+w}{ }\PYG{k+kd}{function}\PYG{+w}{ }\PYG{p}{(}\PYG{n+nx}{data}\PYG{p}{)}\PYG{+w}{ }\PYG{p}{\PYGZob{}}
\PYG{+w}{    }\PYG{n+nx}{console}\PYG{p}{.}\PYG{n+nx}{log}\PYG{p}{(}\PYG{l+s+s2}{\PYGZdq{}Error:\PYGZdq{}}\PYG{p}{,}\PYG{+w}{ }\PYG{n+nx}{data}\PYG{p}{)}\PYG{p}{;}
\PYG{+w}{  }\PYG{p}{\PYGZcb{}}\PYG{p}{;}
\PYG{+w}{  }\PYG{c+c1}{// Call the Python API endpoint \PYGZdq{}server\PYGZus{}demo\PYGZdq{}}
\PYG{+w}{  }\PYG{n+nx}{Plugin\PYGZus{}template}\PYG{p}{.}\PYG{n+nx}{api}\PYG{p}{(}
\PYG{+w}{    }\PYG{l+s+s2}{\PYGZdq{}server\PYGZus{}demo\PYGZdq{}}\PYG{p}{,}
\PYG{+w}{    }\PYG{p}{\PYGZob{}}\PYG{n+nx}{message}\PYG{o}{:}\PYG{+w}{ }\PYG{n+nx}{message}\PYG{p}{\PYGZcb{}}\PYG{p}{,}
\PYG{+w}{    }\PYG{n+nx}{successCallback}\PYG{p}{,}
\PYG{+w}{    }\PYG{n+nx}{errorCallback}\PYG{p}{,}
\PYG{+w}{  }\PYG{p}{)}\PYG{p}{;}
\PYG{p}{\PYGZcb{}}\PYG{p}{;}
\end{sphinxVerbatim}


\paragraph{Plugin parameters}
\label{\detokenize{docs/starting/plugins:plugin-parameters}}
\sphinxAtStartPar
The \sphinxcode{\sphinxupquote{parameters}} object contains the parameters of your plugin. Each parameter is an object with the following properties:
\begin{itemize}
\item {} 
\sphinxAtStartPar
\sphinxcode{\sphinxupquote{label}}: the text that will be displayed in front of the parameter.

\item {} 
\sphinxAtStartPar
\sphinxcode{\sphinxupquote{type}}: the type of parameter. It can be one of the following:
\begin{itemize}
\item {} 
\sphinxAtStartPar
\sphinxcode{\sphinxupquote{button}}: an input button.

\item {} 
\sphinxAtStartPar
\sphinxcode{\sphinxupquote{checkbox}}: a checkbox input.

\item {} 
\sphinxAtStartPar
\sphinxcode{\sphinxupquote{text}}: a text input.

\item {} 
\sphinxAtStartPar
\sphinxcode{\sphinxupquote{number}}: a number input.

\item {} 
\sphinxAtStartPar
\sphinxcode{\sphinxupquote{select}}: a dropdown list.

\item {} 
\sphinxAtStartPar
\sphinxcode{\sphinxupquote{label}}: a text label.

\item {} 
\sphinxAtStartPar
\sphinxcode{\sphinxupquote{section}}: a section that can be collapsed or expanded.

\end{itemize}

\item {} 
\sphinxAtStartPar
\sphinxcode{\sphinxupquote{default}}: the default value of the parameter {[}for \sphinxcode{\sphinxupquote{type}} in \sphinxcode{\sphinxupquote{checkbox}}, \sphinxcode{\sphinxupquote{number}}, \sphinxcode{\sphinxupquote{select}}, \sphinxcode{\sphinxupquote{text}}{]}.

\item {} 
\sphinxAtStartPar
\sphinxcode{\sphinxupquote{options}}: an array of options {[}for \sphinxcode{\sphinxupquote{type}} in \sphinxcode{\sphinxupquote{select}}{]}.

\item {} 
\sphinxAtStartPar
\sphinxcode{\sphinxupquote{title}}: the title of a section {[}for \sphinxcode{\sphinxupquote{type}} in \sphinxcode{\sphinxupquote{section}}{]}.

\item {} 
\sphinxAtStartPar
\sphinxcode{\sphinxupquote{collapsed}}: a boolean indicating if a section is collapsed or expanded by default {[}for \sphinxcode{\sphinxupquote{type}} in \sphinxcode{\sphinxupquote{section}}{]}.

\end{itemize}


\paragraph{Javascript API}
\label{\detokenize{docs/starting/plugins:javascript-api}}
\sphinxAtStartPar
TissUUmaps offers helper functions to help you create your plugin:
\begin{itemize}
\item {} 
\sphinxAtStartPar
\sphinxcode{\sphinxupquote{My\_plugin\_name.set(paramName, value)}}: set the value of a parameter.

\item {} 
\sphinxAtStartPar
\sphinxcode{\sphinxupquote{My\_plugin\_name.get(paramName)}}: get the value of a parameter.

\item {} 
\sphinxAtStartPar
\sphinxcode{\sphinxupquote{My\_plugin\_name.getInputID(paramName)}}: get the id of the html element associated with a parameter.

\item {} 
\sphinxAtStartPar
\sphinxcode{\sphinxupquote{My\_plugin\_name.api(endpoint, data, successCallback, errorCallback)}}: call a Python method from Javascript. The endpoint is the name of the method, and the data is an object that will be passed as a parameter to the method. The successCallback and errorCallback are functions that will be called after the method has been called. The successCallback will be called with the result of the method as a parameter, and the errorCallback will be called with the error message as a parameter.

\end{itemize}

\sphinxAtStartPar
You can access the complete TissUUmaps javascript API \sphinxhref{https://tissuumaps.github.io/TissUUmapsCore/}{here}.


\subsubsection{Python file}
\label{\detokenize{docs/starting/plugins:python-file}}
\sphinxAtStartPar
You only need to use the Python file if your plugin needs to do processing on the server side. For pure javascript plugins, you can leave this file empty.

\sphinxAtStartPar
The python file should implement the class \sphinxcode{\sphinxupquote{Plugin}}:

\begin{sphinxVerbatim}[commandchars=\\\{\}]
\PYG{k+kn}{import} \PYG{n+nn}{logging}
\PYG{k+kn}{import} \PYG{n+nn}{time}
\PYG{k+kn}{from} \PYG{n+nn}{flask} \PYG{k+kn}{import} \PYG{n}{abort}\PYG{p}{,} \PYG{n}{make\PYGZus{}response}

\PYG{k}{class} \PYG{n+nc}{Plugin}\PYG{p}{:}
    \PYG{k}{def} \PYG{n+nf+fm}{\PYGZus{}\PYGZus{}init\PYGZus{}\PYGZus{}}\PYG{p}{(}\PYG{n+nb+bp}{self}\PYG{p}{,} \PYG{n}{app}\PYG{p}{)}\PYG{p}{:}
        \PYG{n+nb+bp}{self}\PYG{o}{.}\PYG{n}{app} \PYG{o}{=} \PYG{n}{app}

    \PYG{k}{def} \PYG{n+nf}{server\PYGZus{}demo}\PYG{p}{(}\PYG{n+nb+bp}{self}\PYG{p}{,} \PYG{n}{jsonParam}\PYG{p}{)}\PYG{p}{:}
        \PYG{k}{if} \PYG{o+ow}{not} \PYG{n}{jsonParam}\PYG{p}{:}
            \PYG{n}{logging}\PYG{o}{.}\PYG{n}{error}\PYG{p}{(}\PYG{l+s+s2}{\PYGZdq{}}\PYG{l+s+s2}{No arguments, aborting.}\PYG{l+s+s2}{\PYGZdq{}}\PYG{p}{)}
            \PYG{n}{abort}\PYG{p}{(}\PYG{l+m+mi}{500}\PYG{p}{)}
        \PYG{n}{resp} \PYG{o}{=} \PYG{n}{make\PYGZus{}response}\PYG{p}{(}\PYG{l+s+s2}{\PYGZdq{}}\PYG{l+s+s2}{The server received the message: }\PYG{l+s+s2}{\PYGZdq{}} \PYG{o}{+} \PYG{n}{jsonParam}\PYG{p}{[}\PYG{l+s+s2}{\PYGZdq{}}\PYG{l+s+s2}{message}\PYG{l+s+s2}{\PYGZdq{}}\PYG{p}{]}\PYG{p}{)}
        \PYG{k}{return} \PYG{n}{resp}

\end{sphinxVerbatim}

\sphinxAtStartPar
The \sphinxcode{\sphinxupquote{app}} object being the flask application running the TissUUmaps server.

\sphinxAtStartPar
You can call a Python method inside the \sphinxcode{\sphinxupquote{Plugin}} class from Javascript using Ajax and the Python API. The endpoint for a method \sphinxcode{\sphinxupquote{methodName}} of the plugin \sphinxcode{\sphinxupquote{PluginName}} will be: \sphinxcode{\sphinxupquote{/plugins/methodName/functionName}}. Data can be transmitted through Ajax as stringified JSON, and will be available as a parameter inside the method.

\sphinxAtStartPar
See the Plugin Template for a working example of Javascript / Python communication.

\sphinxstepscope


\section{Shortcuts}
\label{\detokenize{docs/starting/shortcuts:shortcuts}}\label{\detokenize{docs/starting/shortcuts::doc}}
\sphinxAtStartPar
TissUUmaps contains following keyboard shorcuts:
\begin{itemize}
\item {} 
\sphinxAtStartPar
\sphinxcode{\sphinxupquote{M}} \sphinxhyphen{} to toggle the markers

\item {} 
\sphinxAtStartPar
\sphinxcode{\sphinxupquote{F}} \sphinxhyphen{} to expand to fullscreen

\item {} 
\sphinxAtStartPar
\sphinxcode{\sphinxupquote{0}} \sphinxhyphen{} to hide right menu

\item {} 
\sphinxAtStartPar
\sphinxcode{\sphinxupquote{R}} \sphinxhyphen{} to toggle regions

\item {} 
\sphinxAtStartPar
\sphinxcode{\sphinxupquote{Escape}} \sphinxhyphen{} to exit any active region tool

\end{itemize}

\sphinxAtStartPar
When running TissUUmaps from the graphical interface, you can also use the following shortcuts:
\begin{itemize}
\item {} 
\sphinxAtStartPar
\sphinxcode{\sphinxupquote{Ctrl O}} \sphinxhyphen{} to open a new image

\item {} 
\sphinxAtStartPar
\sphinxcode{\sphinxupquote{Ctrl S}} \sphinxhyphen{} to save the current project

\item {} 
\sphinxAtStartPar
\sphinxcode{\sphinxupquote{Ctrl Q}} \sphinxhyphen{} to quit TissUUmaps

\end{itemize}

\sphinxstepscope


\chapter{Sharing projects}
\label{\detokenize{docs/sharing/index:sharing-projects}}\label{\detokenize{docs/sharing/index::doc}}
\sphinxstepscope


\section{Apache server}
\label{\detokenize{docs/sharing/apache:apache-server}}\label{\detokenize{docs/sharing/apache::doc}}
\sphinxAtStartPar
TissUUmaps projects can be exported into static webpages, that can be uploaded to any Apache server.
\begin{enumerate}
\sphinxsetlistlabels{\arabic}{enumi}{enumii}{}{.}%
\item {} 
\sphinxAtStartPar
Save your project from TissUUmaps (\sphinxcode{\sphinxupquote{menu \textgreater{} File \textgreater{} Save project}})

\item {} 
\sphinxAtStartPar
Export to static page (\sphinxcode{\sphinxupquote{menu \textgreater{} File \textgreater{} Export to static webpage}})

\item {} 
\sphinxAtStartPar
Copy the exported folder on your Apache server

\end{enumerate}

\sphinxstepscope


\section{Docker container}
\label{\detokenize{docs/sharing/docker:docker-container}}\label{\detokenize{docs/sharing/docker::doc}}

\subsection{Start a TissUUmaps docker instance}
\label{\detokenize{docs/sharing/docker:start-a-tissuumaps-docker-instance}}\begin{enumerate}
\sphinxsetlistlabels{\arabic}{enumi}{enumii}{}{.}%
\item {} 
\sphinxAtStartPar
Start the docker container \sphinxcode{\sphinxupquote{cavenel/tissuumaps:latest}} from Docker Hub:

\end{enumerate}

\begin{sphinxVerbatim}[commandchars=\\\{\}]
docker\PYG{+w}{ }run\PYG{+w}{ }\PYGZhy{}it\PYG{+w}{ }\PYGZhy{}p\PYG{+w}{ }\PYG{l+m}{56733}:80\PYG{+w}{ }\PYGZhy{}\PYGZhy{}name\PYG{o}{=}tissuumaps\PYG{+w}{ }\PYGZhy{}v\PYG{+w}{ }/path/to/local/images:/mnt/data\PYG{+w}{ }cavenel/tissuumaps:latest
\end{sphinxVerbatim}
\begin{enumerate}
\sphinxsetlistlabels{\arabic}{enumi}{enumii}{}{.}%
\item {} 
\sphinxAtStartPar
Place your images in the local folder \sphinxcode{\sphinxupquote{/path/to/local/images/share}}.

\item {} 
\sphinxAtStartPar
Open \sphinxurl{http://127.0.0.1:56733/} in your favorite browser.

\end{enumerate}


\subsection{Define TissUUmaps service in a Compose file}
\label{\detokenize{docs/sharing/docker:define-tissuumaps-service-in-a-compose-file}}
\sphinxAtStartPar
Compose is a tool for defining and running multi\sphinxhyphen{}container Docker applications. With Compose, you use a YAML file to configure your application’s services. Then, with a single command, you create and start all the services from your configuration.

\sphinxAtStartPar
\sphinxstylestrong{If you use compose, you don’t need to start the docker container from the previous section.}
\begin{enumerate}
\sphinxsetlistlabels{\arabic}{enumi}{enumii}{}{.}%
\item {} 
\sphinxAtStartPar
Install \sphinxhref{https://docs.docker.com/compose/install/}{docker\sphinxhyphen{}compose}.

\item {} 
\sphinxAtStartPar
Create a file called \sphinxcode{\sphinxupquote{docker\sphinxhyphen{}compose.yml}} in your project directory and paste the following:

\begin{sphinxVerbatim}[commandchars=\\\{\}]
\PYG{n}{version}\PYG{p}{:} \PYG{l+s+s2}{\PYGZdq{}}\PYG{l+s+s2}{3.9}\PYG{l+s+s2}{\PYGZdq{}}
\PYG{n}{services}\PYG{p}{:}
  \PYG{n}{backend}\PYG{p}{:}
    \PYG{n}{image}\PYG{p}{:} \PYG{n}{docker}\PYG{o}{.}\PYG{n}{io}\PYG{o}{/}\PYG{n}{cavenel}\PYG{o}{/}\PYG{n}{tissuumaps}\PYG{p}{:}\PYG{n}{latest}
    \PYG{n}{volumes}\PYG{p}{:}
      \PYG{o}{\PYGZhy{}} \PYG{n+nb}{type}\PYG{p}{:} \PYG{n}{bind}
        \PYG{n}{source}\PYG{p}{:} \PYG{o}{/}\PYG{n}{jail}
        \PYG{n}{target}\PYG{p}{:} \PYG{o}{/}\PYG{n}{mnt}\PYG{o}{/}\PYG{n}{data}
    \PYG{n}{restart}\PYG{p}{:} \PYG{n}{on}\PYG{o}{\PYGZhy{}}\PYG{n}{failure}
    \PYG{n}{ports}\PYG{p}{:}
      \PYG{o}{\PYGZhy{}} \PYG{l+m+mf}{127.0}\PYG{l+m+mf}{.0}\PYG{l+m+mf}{.1}\PYG{p}{:}\PYG{l+m+mi}{8050}\PYG{p}{:}\PYG{l+m+mi}{80}
\end{sphinxVerbatim}

\sphinxAtStartPar
This \sphinxcode{\sphinxupquote{Compose}} file defines TissUUmaps backend service on port 8050.
The web service uses an image that’s built from docker.io hub. It then binds the container and the host machine to the exposed port, 8050.

\item {} 
\sphinxAtStartPar
Put your data in the source folder (here \sphinxcode{\sphinxupquote{/jail/shared/}}). You can change the source path in the \sphinxcode{\sphinxupquote{docker\sphinxhyphen{}compose.yml}} file.

\item {} 
\sphinxAtStartPar
From your project directory, start up your application by running \sphinxcode{\sphinxupquote{docker compose up}}.

\begin{sphinxVerbatim}[commandchars=\\\{\}]
\PYGZdl{}\PYG{+w}{ }docker\PYGZhy{}compose\PYG{+w}{ }up
Creating\PYG{+w}{ }network\PYG{+w}{ }\PYG{l+s+s2}{\PYGZdq{}tmap\PYGZus{}compose\PYGZus{}default\PYGZdq{}}\PYG{+w}{ }with\PYG{+w}{ }the\PYG{+w}{ }default\PYG{+w}{ }driver
Creating\PYG{+w}{ }tmap\PYGZus{}compose\PYGZus{}backend\PYGZus{}1\PYG{+w}{ }...\PYG{+w}{ }\PYG{k}{done}
Attaching\PYG{+w}{ }to\PYG{+w}{ }tmap\PYGZus{}compose\PYGZus{}backend\PYGZus{}1
backend\PYGZus{}1\PYG{+w}{  }\PYG{p}{|}\PYG{+w}{ }\PYG{o}{[}\PYG{l+m}{2023}\PYGZhy{}03\PYGZhy{}27\PYG{+w}{ }\PYG{l+m}{11}:23:57\PYG{+w}{ }+0000\PYG{o}{]}\PYG{+w}{ }\PYG{o}{[}\PYG{l+m}{1}\PYG{o}{]}\PYG{+w}{ }\PYG{o}{[}INFO\PYG{o}{]}\PYG{+w}{ }Starting\PYG{+w}{ }gunicorn\PYG{+w}{ }\PYG{l+m}{20}.1.0
backend\PYGZus{}1\PYG{+w}{  }\PYG{p}{|}\PYG{+w}{ }\PYG{o}{[}\PYG{l+m}{2023}\PYGZhy{}03\PYGZhy{}27\PYG{+w}{ }\PYG{l+m}{11}:23:57\PYG{+w}{ }+0000\PYG{o}{]}\PYG{+w}{ }\PYG{o}{[}\PYG{l+m}{1}\PYG{o}{]}\PYG{+w}{ }\PYG{o}{[}INFO\PYG{o}{]}\PYG{+w}{ }Listening\PYG{+w}{ }at:\PYG{+w}{ }http://0.0.0.0:80\PYG{+w}{ }\PYG{o}{(}\PYG{l+m}{1}\PYG{o}{)}
backend\PYGZus{}1\PYG{+w}{  }\PYG{p}{|}\PYG{+w}{ }\PYG{o}{[}\PYG{l+m}{2023}\PYGZhy{}03\PYGZhy{}27\PYG{+w}{ }\PYG{l+m}{11}:23:57\PYG{+w}{ }+0000\PYG{o}{]}\PYG{+w}{ }\PYG{o}{[}\PYG{l+m}{1}\PYG{o}{]}\PYG{+w}{ }\PYG{o}{[}INFO\PYG{o}{]}\PYG{+w}{ }Using\PYG{+w}{ }worker:\PYG{+w}{ }gevent
backend\PYGZus{}1\PYG{+w}{  }\PYG{p}{|}\PYG{+w}{ }\PYG{o}{[}\PYG{l+m}{2023}\PYGZhy{}03\PYGZhy{}27\PYG{+w}{ }\PYG{l+m}{11}:23:57\PYG{+w}{ }+0000\PYG{o}{]}\PYG{+w}{ }\PYG{o}{[}\PYG{l+m}{7}\PYG{o}{]}\PYG{+w}{ }\PYG{o}{[}INFO\PYG{o}{]}\PYG{+w}{ }Booting\PYG{+w}{ }worker\PYG{+w}{ }with\PYG{+w}{ }pid:\PYG{+w}{ }\PYG{l+m}{7}
backend\PYGZus{}1\PYG{+w}{  }\PYG{p}{|}\PYG{+w}{ }\PYG{o}{[}\PYG{l+m}{2023}\PYGZhy{}03\PYGZhy{}27\PYG{+w}{ }\PYG{l+m}{11}:23:57\PYG{+w}{ }+0000\PYG{o}{]}\PYG{+w}{ }\PYG{o}{[}\PYG{l+m}{8}\PYG{o}{]}\PYG{+w}{ }\PYG{o}{[}INFO\PYG{o}{]}\PYG{+w}{ }Booting\PYG{+w}{ }worker\PYG{+w}{ }with\PYG{+w}{ }pid:\PYG{+w}{ }\PYG{l+m}{8}
backend\PYGZus{}1\PYG{+w}{  }\PYG{p}{|}\PYG{+w}{ }\PYG{o}{[}\PYG{l+m}{2023}\PYGZhy{}03\PYGZhy{}27\PYG{+w}{ }\PYG{l+m}{11}:23:57\PYG{+w}{ }+0000\PYG{o}{]}\PYG{+w}{ }\PYG{o}{[}\PYG{l+m}{9}\PYG{o}{]}\PYG{+w}{ }\PYG{o}{[}INFO\PYG{o}{]}\PYG{+w}{ }Booting\PYG{+w}{ }worker\PYG{+w}{ }with\PYG{+w}{ }pid:\PYG{+w}{ }\PYG{l+m}{9}
backend\PYGZus{}1\PYG{+w}{  }\PYG{p}{|}\PYG{+w}{ }\PYG{o}{[}\PYG{l+m}{2023}\PYGZhy{}03\PYGZhy{}27\PYG{+w}{ }\PYG{l+m}{11}:23:57\PYG{+w}{ }+0000\PYG{o}{]}\PYG{+w}{ }\PYG{o}{[}\PYG{l+m}{10}\PYG{o}{]}\PYG{+w}{ }\PYG{o}{[}INFO\PYG{o}{]}\PYG{+w}{ }Booting\PYG{+w}{ }worker\PYG{+w}{ }with\PYG{+w}{ }pid:\PYG{+w}{ }\PYG{l+m}{10}
backend\PYGZus{}1\PYG{+w}{  }\PYG{p}{|}\PYG{+w}{ }\PYG{o}{[}\PYG{l+m}{2023}\PYGZhy{}03\PYGZhy{}27\PYG{+w}{ }\PYG{l+m}{11}:23:57\PYG{+w}{ }+0000\PYG{o}{]}\PYG{+w}{ }\PYG{o}{[}\PYG{l+m}{11}\PYG{o}{]}\PYG{+w}{ }\PYG{o}{[}INFO\PYG{o}{]}\PYG{+w}{ }Booting\PYG{+w}{ }worker\PYG{+w}{ }with\PYG{+w}{ }pid:\PYG{+w}{ }\PYG{l+m}{11}
backend\PYGZus{}1\PYG{+w}{  }\PYG{p}{|}\PYG{+w}{ }\PYG{o}{[}\PYG{l+m}{2023}\PYGZhy{}03\PYGZhy{}27\PYG{+w}{ }\PYG{l+m}{11}:23:57\PYG{+w}{ }+0000\PYG{o}{]}\PYG{+w}{ }\PYG{o}{[}\PYG{l+m}{12}\PYG{o}{]}\PYG{+w}{ }\PYG{o}{[}INFO\PYG{o}{]}\PYG{+w}{ }Booting\PYG{+w}{ }worker\PYG{+w}{ }with\PYG{+w}{ }pid:\PYG{+w}{ }\PYG{l+m}{12}
backend\PYGZus{}1\PYG{+w}{  }\PYG{p}{|}\PYG{+w}{ }\PYG{o}{[}\PYG{l+m}{2023}\PYGZhy{}03\PYGZhy{}27\PYG{+w}{ }\PYG{l+m}{11}:23:57\PYG{+w}{ }+0000\PYG{o}{]}\PYG{+w}{ }\PYG{o}{[}\PYG{l+m}{13}\PYG{o}{]}\PYG{+w}{ }\PYG{o}{[}INFO\PYG{o}{]}\PYG{+w}{ }Booting\PYG{+w}{ }worker\PYG{+w}{ }with\PYG{+w}{ }pid:\PYG{+w}{ }\PYG{l+m}{13}
backend\PYGZus{}1\PYG{+w}{  }\PYG{p}{|}\PYG{+w}{ }\PYG{o}{[}\PYG{l+m}{2023}\PYGZhy{}03\PYGZhy{}27\PYG{+w}{ }\PYG{l+m}{11}:23:57\PYG{+w}{ }+0000\PYG{o}{]}\PYG{+w}{ }\PYG{o}{[}\PYG{l+m}{14}\PYG{o}{]}\PYG{+w}{ }\PYG{o}{[}INFO\PYG{o}{]}\PYG{+w}{ }Booting\PYG{+w}{ }worker\PYG{+w}{ }with\PYG{+w}{ }pid:\PYG{+w}{ }\PYG{l+m}{14}
backend\PYGZus{}1\PYG{+w}{  }\PYG{p}{|}\PYG{+w}{ }INFO:root:\PYG{+w}{ }*\PYG{+w}{ }TissUUmaps\PYG{+w}{ }version:\PYG{+w}{ }\PYG{l+m}{3}.1.0.1
backend\PYGZus{}1\PYG{+w}{  }\PYG{p}{|}\PYG{+w}{ }INFO:root:\PYG{+w}{ }*\PYG{+w}{ }TissUUmaps\PYG{+w}{ }version:\PYG{+w}{ }\PYG{l+m}{3}.1.0.1
backend\PYGZus{}1\PYG{+w}{  }\PYG{p}{|}\PYG{+w}{ }INFO:root:\PYG{+w}{ }*\PYG{+w}{ }TissUUmaps\PYG{+w}{ }version:\PYG{+w}{ }\PYG{l+m}{3}.1.0.1
backend\PYGZus{}1\PYG{+w}{  }\PYG{p}{|}\PYG{+w}{ }INFO:root:\PYG{+w}{ }*\PYG{+w}{ }TissUUmaps\PYG{+w}{ }version:\PYG{+w}{ }\PYG{l+m}{3}.1.0.1
backend\PYGZus{}1\PYG{+w}{  }\PYG{p}{|}\PYG{+w}{ }INFO:root:\PYG{+w}{ }*\PYG{+w}{ }TissUUmaps\PYG{+w}{ }version:\PYG{+w}{ }\PYG{l+m}{3}.1.0.1
backend\PYGZus{}1\PYG{+w}{  }\PYG{p}{|}\PYG{+w}{ }INFO:root:\PYG{+w}{ }*\PYG{+w}{ }TissUUmaps\PYG{+w}{ }version:\PYG{+w}{ }\PYG{l+m}{3}.1.0.1
backend\PYGZus{}1\PYG{+w}{  }\PYG{p}{|}\PYG{+w}{ }INFO:root:\PYG{+w}{ }*\PYG{+w}{ }TissUUmaps\PYG{+w}{ }version:\PYG{+w}{ }\PYG{l+m}{3}.1.0.1
backend\PYGZus{}1\PYG{+w}{  }\PYG{p}{|}\PYG{+w}{ }INFO:root:\PYG{+w}{ }*\PYG{+w}{ }TissUUmaps\PYG{+w}{ }version:\PYG{+w}{ }\PYG{l+m}{3}.1.0.1
\end{sphinxVerbatim}

\item {} 
\sphinxAtStartPar
Enter http://localhost:8050 in a browser to see TissUUmaps application running.

\end{enumerate}


\subsection{Configure sftp multi\sphinxhyphen{}user access (Optional)}
\label{\detokenize{docs/sharing/docker:configure-sftp-multi-user-access-optional}}
\sphinxAtStartPar
Add the following lines to \sphinxcode{\sphinxupquote{/etc/ssh/sshd\_config}}:

\begin{sphinxVerbatim}[commandchars=\\\{\}]
\PYG{n}{Subsystem} \PYG{n}{sftp} \PYG{n}{internal}\PYG{o}{\PYGZhy{}}\PYG{n}{sftp}
\PYG{n}{Match} \PYG{n}{Group} \PYG{n}{sftpusers}
    \PYG{n}{ChrootDirectory} \PYG{o}{/}\PYG{n}{jail}\PYG{o}{/}\PYG{n}{shared}\PYG{o}{/}\PYG{o}{\PYGZpc{}}\PYG{n}{u}
    \PYG{n}{X11Forwarding} \PYG{n}{no}
    \PYG{n}{AllowTcpForwarding} \PYG{n}{no}
    \PYG{n}{ForceCommand} \PYG{n}{internal}\PYG{o}{\PYGZhy{}}\PYG{n}{sftp}
    \PYG{n}{PasswordAuthentication} \PYG{n}{yes}
\end{sphinxVerbatim}

\sphinxAtStartPar
Use the following bash script to create new sftp users \sphinxstylestrong{(as root)}:

\begin{sphinxVerbatim}[commandchars=\\\{\}]
\PYG{c+ch}{\PYGZsh{}! /bin/bash}

\PYG{n+nb}{read}\PYG{+w}{ }\PYGZhy{}p\PYG{+w}{ }\PYG{l+s+s1}{\PYGZsq{}Username: \PYGZsq{}}\PYG{+w}{ }uservar
\PYG{c+c1}{\PYGZsh{} Use the path mounted in docker:}
\PYG{n+nv}{MOUNT\PYGZus{}TMAP}\PYG{o}{=}\PYG{l+s+s1}{\PYGZsq{}/jail\PYGZsq{}}

\PYG{c+c1}{\PYGZsh{} Create sftpusers group and \PYGZdl{}uservar user}
groupadd\PYG{+w}{ }sftpusers
adduser\PYG{+w}{ }\PYGZhy{}\PYGZhy{}gecos\PYG{+w}{ }\PYG{l+s+s2}{\PYGZdq{}\PYGZdq{}}\PYG{+w}{ }\PYGZhy{}\PYGZhy{}home\PYG{+w}{ }\PYG{n+nv}{\PYGZdl{}MOUNT\PYGZus{}TMAP}/shared/\PYG{n+nv}{\PYGZdl{}uservar}\PYG{+w}{ }\PYG{n+nv}{\PYGZdl{}uservar}
usermod\PYG{+w}{ }\PYGZhy{}G\PYG{+w}{ }sftpusers\PYG{+w}{ }\PYG{n+nv}{\PYGZdl{}uservar}\PYG{+w}{ }

\PYG{c+c1}{\PYGZsh{} Create home for user in \PYGZdl{}MOUNT\PYGZus{}TMAP/shared/ with correct permissions}
mkdir\PYG{+w}{ }\PYGZhy{}p\PYG{+w}{ }\PYG{n+nv}{\PYGZdl{}MOUNT\PYGZus{}TMAP}/shared/\PYG{n+nv}{\PYGZdl{}uservar}/files
chmod\PYG{+w}{ }\PYG{l+m}{755}\PYG{+w}{ }\PYG{n+nv}{\PYGZdl{}MOUNT\PYGZus{}TMAP}/shared/\PYG{n+nv}{\PYGZdl{}uservar}/
chown\PYG{+w}{ }root:root\PYG{+w}{ }\PYG{n+nv}{\PYGZdl{}MOUNT\PYGZus{}TMAP}/shared/\PYG{n+nv}{\PYGZdl{}uservar}/
chmod\PYG{+w}{ }\PYG{l+m}{750}\PYG{+w}{ }\PYG{n+nv}{\PYGZdl{}MOUNT\PYGZus{}TMAP}/shared/\PYG{n+nv}{\PYGZdl{}uservar}/files/
chown\PYG{+w}{ }\PYG{n+nv}{\PYGZdl{}uservar}:\PYG{n+nv}{\PYGZdl{}uservar}\PYG{+w}{ }\PYG{n+nv}{\PYGZdl{}MOUNT\PYGZus{}TMAP}/shared/\PYG{n+nv}{\PYGZdl{}uservar}/files

\PYG{c+c1}{\PYGZsh{} Make sure all files created in the files folder are writable by \PYGZdl{}uservar}
setfacl\PYG{+w}{ }\PYGZhy{}m\PYG{+w}{ }d:u:\PYG{n+nv}{\PYGZdl{}uservar}:rwx\PYG{+w}{ }\PYG{n+nv}{\PYGZdl{}MOUNT\PYGZus{}TMAP}/shared/\PYG{n+nv}{\PYGZdl{}uservar}/files/

\PYG{n+nb}{echo}\PYG{+w}{ }\PYG{l+s+s1}{\PYGZsq{}To access files placed in the /files sftp folder, use:\PYGZsq{}}
\PYG{n+nb}{echo}\PYG{+w}{ }\PYG{l+s+s1}{\PYGZsq{}    http://TMAP\PYGZus{}URL/FILENAME?path=\PYGZsq{}}\PYG{n+nv}{\PYGZdl{}uservar}\PYG{l+s+s1}{\PYGZsq{}/files\PYGZsq{}}
\end{sphinxVerbatim}

\sphinxAtStartPar
The new users will be able to connect through sftp, and only see the files in \sphinxcode{\sphinxupquote{/jail/shared/USER\_NAME/files}}. Files will be available on the web using \sphinxcode{\sphinxupquote{http://localhost:8050/MYFILE.tmap?path=/USER\_NAME/files}}

\sphinxAtStartPar
Restart the ssh daemon with \sphinxcode{\sphinxupquote{sudo service ssh restart}}.


\subsection{Password protection of files hosted on TissUUmaps docker}
\label{\detokenize{docs/sharing/docker:password-protection-of-files-hosted-on-tissuumaps-docker}}
\sphinxAtStartPar
You can add login and password for any folder hosted on the TissUUmaps server by adding a \sphinxcode{\sphinxupquote{auth}} file containing your login and password in the format:

\begin{sphinxVerbatim}[commandchars=\\\{\}]
\PYG{n}{login}\PYG{p}{:}\PYG{n}{password}
\end{sphinxVerbatim}

\sphinxstepscope


\chapter{Advanced usage}
\label{\detokenize{docs/advanced/index:advanced-usage}}\label{\detokenize{docs/advanced/index::doc}}
\sphinxstepscope


\section{Jupyter notebooks}
\label{\detokenize{docs/advanced/jupyter:jupyter-notebooks}}\label{\detokenize{docs/advanced/jupyter::doc}}
\sphinxAtStartPar
TissUUmaps can easily be used inside a Jupyter Notebook or Jupyter Lab.

\sphinxAtStartPar
Simple example to load an image in TissUUmaps:

\begin{sphinxVerbatim}[commandchars=\\\{\}]
\PYG{k+kn}{import} \PYG{n+nn}{tissuumaps}\PYG{n+nn}{.}\PYG{n+nn}{jupyter} \PYG{k}{as} \PYG{n+nn}{tj}
\PYG{n}{viewer} \PYG{o}{=} \PYG{n}{tj}\PYG{o}{.}\PYG{n}{loaddata}\PYG{p}{(}\PYG{p}{[}\PYG{l+s+s2}{\PYGZdq{}}\PYG{l+s+s2}{image.png}\PYG{l+s+s2}{\PYGZdq{}}\PYG{p}{]}\PYG{p}{)}

\PYG{n}{viewer}\PYG{o}{.}\PYG{n}{screenshot}\PYG{p}{(}\PYG{p}{)}
\end{sphinxVerbatim}
\phantomsection\label{\detokenize{docs/advanced/jupyter:module-tissuumaps.jupyter}}\index{module@\spxentry{module}!tissuumaps.jupyter@\spxentry{tissuumaps.jupyter}}\index{tissuumaps.jupyter@\spxentry{tissuumaps.jupyter}!module@\spxentry{module}}

\subsection{tissuumaps.jupyter}
\label{\detokenize{docs/advanced/jupyter:tissuumaps-jupyter}}
\sphinxAtStartPar
Module used to run TissUUmaps from a Jupyter Notebook or from Jupyter Lab.
\index{opentmap() (in module tissuumaps.jupyter)@\spxentry{opentmap()}\spxextra{in module tissuumaps.jupyter}}

\begin{fulllineitems}
\phantomsection\label{\detokenize{docs/advanced/jupyter:tissuumaps.jupyter.opentmap}}
\pysigstartsignatures
\pysiglinewithargsret{\sphinxcode{\sphinxupquote{tissuumaps.jupyter.}}\sphinxbfcode{\sphinxupquote{opentmap}}}{\emph{\DUrole{n}{path}}, \emph{\DUrole{n}{port}\DUrole{o}{=}\DUrole{default_value}{5100}}, \emph{\DUrole{n}{host}\DUrole{o}{=}\DUrole{default_value}{\textquotesingle{}localhost\textquotesingle{}}}, \emph{\DUrole{n}{height}\DUrole{o}{=}\DUrole{default_value}{700}}}{}
\pysigstopsignatures
\sphinxAtStartPar
Open a tmap project
\begin{quote}\begin{description}
\item[{Parameters}] \leavevmode\begin{itemize}
\item {} 
\sphinxAtStartPar
\sphinxstyleliteralstrong{\sphinxupquote{path}} (\sphinxstyleliteralemphasis{\sphinxupquote{str}}) \textendash{} The path to a tmap file

\item {} 
\sphinxAtStartPar
\sphinxstyleliteralstrong{\sphinxupquote{port}} (\sphinxstyleliteralemphasis{\sphinxupquote{int}}) \textendash{} The port to run the TissUUmaps server

\item {} 
\sphinxAtStartPar
\sphinxstyleliteralstrong{\sphinxupquote{host}} (\sphinxstyleliteralemphasis{\sphinxupquote{str}}) \textendash{} The host to run the TissUUmaps server

\item {} 
\sphinxAtStartPar
\sphinxstyleliteralstrong{\sphinxupquote{height}} (\sphinxstyleliteralemphasis{\sphinxupquote{int}}) \textendash{} The height of the jupyter iframe

\end{itemize}

\item[{Returns}] \leavevmode
\sphinxAtStartPar
The TissUUmaps viewer

\item[{Return type}] \leavevmode
\sphinxAtStartPar
{\hyperref[\detokenize{docs/advanced/jupyter:tissuumaps.jupyter.TissUUmapsViewer}]{\sphinxcrossref{TissUUmapsViewer}}}

\end{description}\end{quote}

\end{fulllineitems}

\index{loaddata() (in module tissuumaps.jupyter)@\spxentry{loaddata()}\spxextra{in module tissuumaps.jupyter}}

\begin{fulllineitems}
\phantomsection\label{\detokenize{docs/advanced/jupyter:tissuumaps.jupyter.loaddata}}
\pysigstartsignatures
\pysiglinewithargsret{\sphinxcode{\sphinxupquote{tissuumaps.jupyter.}}\sphinxbfcode{\sphinxupquote{loaddata}}}{\emph{\DUrole{n}{images}\DUrole{o}{=}\DUrole{default_value}{{[}{]}}}, \emph{\DUrole{n}{csvFiles}\DUrole{o}{=}\DUrole{default_value}{{[}{]}}}, \emph{\DUrole{n}{xSelector}\DUrole{o}{=}\DUrole{default_value}{\textquotesingle{}x\textquotesingle{}}}, \emph{\DUrole{n}{ySelector}\DUrole{o}{=}\DUrole{default_value}{\textquotesingle{}y\textquotesingle{}}}, \emph{\DUrole{n}{keySelector}\DUrole{o}{=}\DUrole{default_value}{None}}, \emph{\DUrole{n}{nameSelector}\DUrole{o}{=}\DUrole{default_value}{None}}, \emph{\DUrole{n}{colorSelector}\DUrole{o}{=}\DUrole{default_value}{None}}, \emph{\DUrole{n}{piechartSelector}\DUrole{o}{=}\DUrole{default_value}{None}}, \emph{\DUrole{n}{shapeSelector}\DUrole{o}{=}\DUrole{default_value}{None}}, \emph{\DUrole{n}{scaleSelector}\DUrole{o}{=}\DUrole{default_value}{None}}, \emph{\DUrole{n}{fixedShape}\DUrole{o}{=}\DUrole{default_value}{None}}, \emph{\DUrole{n}{scaleFactor}\DUrole{o}{=}\DUrole{default_value}{1}}, \emph{\DUrole{n}{colormap}\DUrole{o}{=}\DUrole{default_value}{None}}, \emph{\DUrole{n}{compositeMode}\DUrole{o}{=}\DUrole{default_value}{\textquotesingle{}source\sphinxhyphen{}over\textquotesingle{}}}, \emph{\DUrole{n}{boundingBox}\DUrole{o}{=}\DUrole{default_value}{None}}, \emph{\DUrole{n}{port}\DUrole{o}{=}\DUrole{default_value}{5100}}, \emph{\DUrole{n}{host}\DUrole{o}{=}\DUrole{default_value}{\textquotesingle{}localhost\textquotesingle{}}}, \emph{\DUrole{n}{height}\DUrole{o}{=}\DUrole{default_value}{700}}, \emph{\DUrole{n}{tmapFilename}\DUrole{o}{=}\DUrole{default_value}{\textquotesingle{}\_project\textquotesingle{}}}, \emph{\DUrole{n}{plugins}\DUrole{o}{=}\DUrole{default_value}{{[}{]}}}}{}
\pysigstopsignatures
\sphinxAtStartPar
Load data in TissUUmaps
\begin{quote}\begin{description}
\item[{Parameters}] \leavevmode\begin{itemize}
\item {} 
\sphinxAtStartPar
\sphinxstyleliteralstrong{\sphinxupquote{images}} (\sphinxstyleliteralemphasis{\sphinxupquote{list}}\sphinxstyleliteralemphasis{\sphinxupquote{ | }}\sphinxstyleliteralemphasis{\sphinxupquote{str}}) \textendash{} List of images or single image to display

\item {} 
\sphinxAtStartPar
\sphinxstyleliteralstrong{\sphinxupquote{csvFiles}} (\sphinxstyleliteralemphasis{\sphinxupquote{list}}\sphinxstyleliteralemphasis{\sphinxupquote{ | }}\sphinxstyleliteralemphasis{\sphinxupquote{str}}) \textendash{} List of csv files or single csv file to display

\item {} 
\sphinxAtStartPar
\sphinxstyleliteralstrong{\sphinxupquote{xSelector}} (\sphinxstyleliteralemphasis{\sphinxupquote{str}}) \textendash{} Name of the csv column defining the X coordinates

\item {} 
\sphinxAtStartPar
\sphinxstyleliteralstrong{\sphinxupquote{ySelector}} (\sphinxstyleliteralemphasis{\sphinxupquote{str}}) \textendash{} Name of the csv column defining the Y coordinates

\item {} 
\sphinxAtStartPar
\sphinxstyleliteralstrong{\sphinxupquote{keySelector}} (\sphinxstyleliteralemphasis{\sphinxupquote{str}}) \textendash{} Name of the csv column defining the grouping key

\item {} 
\sphinxAtStartPar
\sphinxstyleliteralstrong{\sphinxupquote{nameSelector}} (\sphinxstyleliteralemphasis{\sphinxupquote{str}}) \textendash{} Name of the csv column defining the group name

\item {} 
\sphinxAtStartPar
\sphinxstyleliteralstrong{\sphinxupquote{colorSelector}} (\sphinxstyleliteralemphasis{\sphinxupquote{str}}) \textendash{} Name of the csv column defining the group color

\item {} 
\sphinxAtStartPar
\sphinxstyleliteralstrong{\sphinxupquote{piechartSelector}} (\sphinxstyleliteralemphasis{\sphinxupquote{str}}) \textendash{} Name of the csv column defining pie\sphinxhyphen{}charts

\item {} 
\sphinxAtStartPar
\sphinxstyleliteralstrong{\sphinxupquote{shapeSelector}} (\sphinxstyleliteralemphasis{\sphinxupquote{str}}) \textendash{} Name of the csv column defining markers’ shape

\item {} 
\sphinxAtStartPar
\sphinxstyleliteralstrong{\sphinxupquote{scaleSelector}} (\sphinxstyleliteralemphasis{\sphinxupquote{str}}) \textendash{} Name of the csv column defining markers’ scale

\item {} 
\sphinxAtStartPar
\sphinxstyleliteralstrong{\sphinxupquote{fixedShape}} (\sphinxstyleliteralemphasis{\sphinxupquote{int}}) \textendash{} Name of the markers’ shape

\item {} 
\sphinxAtStartPar
\sphinxstyleliteralstrong{\sphinxupquote{scaleFactor}} (\sphinxstyleliteralemphasis{\sphinxupquote{int}}) \textendash{} Global scale of markers

\item {} 
\sphinxAtStartPar
\sphinxstyleliteralstrong{\sphinxupquote{colormap}} (\sphinxstyleliteralemphasis{\sphinxupquote{str}}) \textendash{} Name of the colormap used if colorSelector is set

\item {} 
\sphinxAtStartPar
\sphinxstyleliteralstrong{\sphinxupquote{compositeMode}} \textendash{} (str): Composite mode used for images

\item {} 
\sphinxAtStartPar
\sphinxstyleliteralstrong{\sphinxupquote{boundingBox}} (\sphinxstyleliteralemphasis{\sphinxupquote{list}}) \textendash{} {[}X,Y,W,H{]} of the bounding box to display

\item {} 
\sphinxAtStartPar
\sphinxstyleliteralstrong{\sphinxupquote{port}} (\sphinxstyleliteralemphasis{\sphinxupquote{int}}) \textendash{} The port to run the TissUUmaps server

\item {} 
\sphinxAtStartPar
\sphinxstyleliteralstrong{\sphinxupquote{host}} (\sphinxstyleliteralemphasis{\sphinxupquote{str}}) \textendash{} The host to run the TissUUmaps server

\item {} 
\sphinxAtStartPar
\sphinxstyleliteralstrong{\sphinxupquote{height}} (\sphinxstyleliteralemphasis{\sphinxupquote{int}}) \textendash{} The height of the jupyter iframe

\item {} 
\sphinxAtStartPar
\sphinxstyleliteralstrong{\sphinxupquote{tmapFilename}} (\sphinxstyleliteralemphasis{\sphinxupquote{str}}) \textendash{} Name of the project file that will be created

\item {} 
\sphinxAtStartPar
\sphinxstyleliteralstrong{\sphinxupquote{plugins}} (\sphinxstyleliteralemphasis{\sphinxupquote{list}}) \textendash{} List of plugins to add to the tmap project

\end{itemize}

\item[{Returns}] \leavevmode
\sphinxAtStartPar
The TissUUmaps viewer

\item[{Return type}] \leavevmode
\sphinxAtStartPar
{\hyperref[\detokenize{docs/advanced/jupyter:tissuumaps.jupyter.TissUUmapsViewer}]{\sphinxcrossref{TissUUmapsViewer}}}

\end{description}\end{quote}

\end{fulllineitems}

\index{TissUUmapsViewer (class in tissuumaps.jupyter)@\spxentry{TissUUmapsViewer}\spxextra{class in tissuumaps.jupyter}}

\begin{fulllineitems}
\phantomsection\label{\detokenize{docs/advanced/jupyter:tissuumaps.jupyter.TissUUmapsViewer}}
\pysigstartsignatures
\pysiglinewithargsret{\sphinxbfcode{\sphinxupquote{class\DUrole{w}{  }}}\sphinxcode{\sphinxupquote{tissuumaps.jupyter.}}\sphinxbfcode{\sphinxupquote{TissUUmapsViewer}}}{\emph{\DUrole{n}{server}}, \emph{\DUrole{n}{image}}, \emph{\DUrole{n}{height}\DUrole{o}{=}\DUrole{default_value}{700}}}{}
\pysigstopsignatures
\sphinxAtStartPar
Class representing a TissUUmaps viewer instance
\index{screenshot() (tissuumaps.jupyter.TissUUmapsViewer method)@\spxentry{screenshot()}\spxextra{tissuumaps.jupyter.TissUUmapsViewer method}}

\begin{fulllineitems}
\phantomsection\label{\detokenize{docs/advanced/jupyter:tissuumaps.jupyter.TissUUmapsViewer.screenshot}}
\pysigstartsignatures
\pysiglinewithargsret{\sphinxbfcode{\sphinxupquote{screenshot}}}{}{}
\pysigstopsignatures
\sphinxAtStartPar
Capture the TissUUmaps viewport and display image in the Notebook.

\end{fulllineitems}


\end{fulllineitems}

\index{TissUUmapsServer (class in tissuumaps.jupyter)@\spxentry{TissUUmapsServer}\spxextra{class in tissuumaps.jupyter}}

\begin{fulllineitems}
\phantomsection\label{\detokenize{docs/advanced/jupyter:tissuumaps.jupyter.TissUUmapsServer}}
\pysigstartsignatures
\pysiglinewithargsret{\sphinxbfcode{\sphinxupquote{class\DUrole{w}{  }}}\sphinxcode{\sphinxupquote{tissuumaps.jupyter.}}\sphinxbfcode{\sphinxupquote{TissUUmapsServer}}}{\emph{\DUrole{n}{slideDir}}, \emph{\DUrole{n}{port}\DUrole{o}{=}\DUrole{default_value}{5000}}, \emph{\DUrole{n}{host}\DUrole{o}{=}\DUrole{default_value}{\textquotesingle{}0.0.0.0\textquotesingle{}}}}{}
\pysigstopsignatures
\sphinxAtStartPar
Class representing a TissUUmaps server instance

\end{fulllineitems}


\sphinxstepscope


\section{Napari}
\label{\detokenize{docs/advanced/napari:napari}}\label{\detokenize{docs/advanced/napari::doc}}
\sphinxAtStartPar
Napari features an important hub containing 118 plugins at the time of writing, many of them expanding further the capabilities of Napari when dealing with biomedical imaging. We thus created our own plugin to allow users to work in Napari, benefit from the tools, scripting and existing plugins, and easily visualize and share the output of their research through TissUUmaps.

\sphinxAtStartPar
The \sphinxhref{https://github.com/TissUUmaps/napari-tissuumaps}{Napari\sphinxhyphen{}TissUUmaps plugin} is available on Napari Hub which makes the installation trivial: from the Napari install/uninstall plugins menu, the \sphinxcode{\sphinxupquote{napari\sphinxhyphen{}tissuumaps}} appears in the list and can be installed with a single click. Alternatively, the plugin can be installed with the Python package manager: \sphinxcode{\sphinxupquote{pip install napari\sphinxhyphen{}tissuumaps}}.

\sphinxAtStartPar
The plugin can export all standard Napari layers, such as images, labels, points, and shapes and preserves the metadata (opacity, visibility), but also the objects parameters (e.g.: label colors, marker colors and symbols, etc…). To export a TissUUmaps project, care must be taken to save all layers of interest and type in a name with the extension \sphinxcode{\sphinxupquote{.tmap}}, e.g.: \sphinxcode{\sphinxupquote{myProject.tmap}}. This is important for Napari to delegate the saving of the files to the plugin. A folder is created and contains all the necessary files and can be loaded in the TissUUmaps server, software, Jupyter Notebook, or shared with the community.

\sphinxAtStartPar
The project folders generated by the plugin contain the metadata in a \sphinxcode{\sphinxupquote{main.tmap}} file, along with folders for each Napari layer types: images, labels, points and regions. Images and labels are saved as plain tif images, points are saved as CSV files, and shapes are saved as GeoJSON. We hope that the use of a simple structure and widespread file formats can simplify the modifying and updating of the TissUUmaps project when prototyping with e.g. Jupyter Notebooks.
The source code is available at \sphinxurl{https://github.com/TissUUmaps/napari-tissuumaps} under the permissive MIT license.
A demonstration of the Cellpose plugin of Napari being exported to the TissUUmaps web viewer is available at: \sphinxurl{https://tissuumaps.github.io/tutorials/\#napari}.

\sphinxstepscope


\section{AnnData Handling in TissUUmaps}
\label{\detokenize{docs/advanced/anndata:anndata-handling-in-tissuumaps}}\label{\detokenize{docs/advanced/anndata::doc}}
\sphinxAtStartPar
To visualize and manipulate annotated data matrices, TissUUmaps leverages the capabilities of the \sphinxstylestrong{AnnData} (Annotated Data) Python package, stored in the h5ad format. AnnData provides a flexible and efficient framework for working with annotated data in memory and on disk, bridging the functionality of pandas and xarray. Notable features of AnnData include support for sparse data, lazy operations, and a PyTorch interface.


\subsection{Loading AnnData Objects}
\label{\detokenize{docs/advanced/anndata:loading-anndata-objects}}
\sphinxAtStartPar
You can load AnnData objects in the same way as you would load TissUUmaps projects or images, by dragging and dropping the file into the TissUUmaps interface. TissUUmaps will automatically detect the file type and load the AnnData object into the interface. You can also load AnnData objects by clicking on the “Open” button in the “File” menu and selecting the file from your computer.


\subsection{AnnData Specification in TissUUmaps}
\label{\detokenize{docs/advanced/anndata:anndata-specification-in-tissuumaps}}
\sphinxAtStartPar
TissUUmaps adheres to the specifications defined by the SCVerse community for AnnData objects. TissUUmaps can load AnnData objects stored in the HDF5 format.

\sphinxAtStartPar
The key components of the AnnData structure in TissUUmaps are as follows:


\subsubsection{Spatial Coordinates}
\label{\detokenize{docs/advanced/anndata:spatial-coordinates}}
\sphinxAtStartPar
Spatial coordinates are crucial for spatial omics analysis. In TissUUmaps, these coordinates are stored as a 2\sphinxhyphen{}column matrix in one of the following locations:
\begin{itemize}
\item {} 
\sphinxAtStartPar
\sphinxcode{\sphinxupquote{/obsm/spatial}}

\item {} 
\sphinxAtStartPar
\sphinxcode{\sphinxupquote{/obsm/X\_spatial}}

\item {} 
\sphinxAtStartPar
\sphinxcode{\sphinxupquote{/obsm/X\_umap}}

\item {} 
\sphinxAtStartPar
\sphinxcode{\sphinxupquote{/obsm/tSNE}}

\item {} 
\sphinxAtStartPar
\sphinxcode{\sphinxupquote{/obsm/UMAP}}

\end{itemize}


\subsubsection{Observations}
\label{\detokenize{docs/advanced/anndata:observations}}
\sphinxAtStartPar
Observations are stored in the \sphinxcode{\sphinxupquote{/obs}} section of the AnnData object. This section contains information related to individual observations.


\subsubsection{Gene Expressions}
\label{\detokenize{docs/advanced/anndata:gene-expressions}}
\sphinxAtStartPar
Gene expressions are stored in the \sphinxcode{\sphinxupquote{X}} section of the AnnData object. It’s important to note that the gene expression matrix should be in sparse Compressed Sparse Column (CSC) format. If the matrix is not in CSC format, TissUUmaps will create a new AnnData object with the gene expression matrix in CSC sparse format.


\subsubsection{Variable Information}
\label{\detokenize{docs/advanced/anndata:variable-information}}
\sphinxAtStartPar
Variable information is stored in the \sphinxcode{\sphinxupquote{/var}} section of the AnnData object.


\subsubsection{Image Loading}
\label{\detokenize{docs/advanced/anndata:image-loading}}
\sphinxAtStartPar
TissUUmaps is equipped to load images stored in the following location:
\begin{itemize}
\item {} 
\sphinxAtStartPar
\sphinxcode{\sphinxupquote{/uns/spatial/\{library\_id\}/images/hires}}

\end{itemize}

\sphinxAtStartPar
Here, \sphinxcode{\sphinxupquote{\{library\_id\}}} is an observation column obtained from \sphinxcode{\sphinxupquote{/obs/library\_id}}. Additionally, TissUUmaps utilizes a scale factor from \sphinxcode{\sphinxupquote{/uns/spatial/\{library\_id\}/scalefactors/tissue\_hires\_scalef}}.


\subsection{Dropdowns}
\label{\detokenize{docs/advanced/anndata:dropdowns}}
\sphinxAtStartPar
Once loaded into TissUUmaps, each observation is accessible through dropdown menus. There are separate dropdowns for numerical and categorical observations, providing a convenient interface for users to explore and analyze their spatial omics data.

\sphinxAtStartPar
The gene expression dropdown menu contains the X matrix.

\sphinxAtStartPar
\sphinxincludegraphics{{AnnData_Dropdowns}.png}


\subsection{Example}
\label{\detokenize{docs/advanced/anndata:example}}
\sphinxAtStartPar
Here is an example of the TissUUmaps interface when loading an AnnData object with three samples. The dropdown menus are populated with the observations from the AnnData object. Here, the “celltype\_annotation” is selected.

\sphinxAtStartPar
\sphinxincludegraphics{{AnnData_Preview}.png}

\sphinxstepscope


\section{The TMAP file format}
\label{\detokenize{docs/advanced/tmap:the-tmap-file-format}}\label{\detokenize{docs/advanced/tmap::doc}}
\sphinxAtStartPar
The TMAP format contains a description of image layers, markers, regions, and settings. It is highly recommended to create .tmap files by saving projects from TissUUmaps, but you can also edit the files manually to add or change projects’ settings, or generate them as exported data from other software for import in TissUUmaps.

\sphinxAtStartPar
For more information on the TMAPS format, see the \sphinxhref{https://github.com/TissUUmaps/TissUUmaps-schema}{TissUUmaps\sphinxhyphen{}schema github page}.

\sphinxAtStartPar
The TMAP format uses JSON, with the following specifications:


\begin{savenotes}\sphinxatlongtablestart\begin{longtable}[c]{|*{4}{\X{1}{4}|}}
\hline\noalign{\phantomsection\label{\detokenize{docs/advanced/tmap:project-json}}}%

\endfirsthead

\multicolumn{4}{c}%
{\makebox[0pt]{\sphinxtablecontinued{\tablename\ \thetable{} \textendash{} continued from previous page}}}\\
\hline

\endhead

\hline
\multicolumn{4}{r}{\makebox[0pt][r]{\sphinxtablecontinued{continues on next page}}}\\
\endfoot

\endlastfoot

\sphinxAtStartPar
type
&\sphinxstartmulticolumn{3}%
\begin{varwidth}[t]{\sphinxcolwidth{3}{4}}
\sphinxAtStartPar
\sphinxstyleemphasis{object}
\par
\vskip-\baselineskip\vbox{\hbox{\strut}}\end{varwidth}%
\sphinxstopmulticolumn
\\
\hline\sphinxstartmulticolumn{4}%
\begin{varwidth}[t]{\sphinxcolwidth{4}{4}}
\sphinxAtStartPar
properties
\par
\vskip-\baselineskip\vbox{\hbox{\strut}}\end{varwidth}%
\sphinxstopmulticolumn
\\
\hline\sphinxmultirow{2}{4}{%
\begin{varwidth}[t]{\sphinxcolwidth{1}{4}}
\begin{itemize}
\item {} 
\sphinxAtStartPar
schemaVersion

\end{itemize}
\par
\vskip-\baselineskip\vbox{\hbox{\strut}}\end{varwidth}%
}%
&
\sphinxAtStartPar
type
&\sphinxstartmulticolumn{2}%
\begin{varwidth}[t]{\sphinxcolwidth{2}{4}}
\sphinxAtStartPar
\sphinxstyleemphasis{string}
\par
\vskip-\baselineskip\vbox{\hbox{\strut}}\end{varwidth}%
\sphinxstopmulticolumn
\\
\cline{2-4}\sphinxtablestrut{4}&
\sphinxAtStartPar
default
&\sphinxstartmulticolumn{2}%
\begin{varwidth}[t]{\sphinxcolwidth{2}{4}}
\sphinxAtStartPar
1.2
\par
\vskip-\baselineskip\vbox{\hbox{\strut}}\end{varwidth}%
\sphinxstopmulticolumn
\\
\hline\sphinxmultirow{4}{9}{%
\begin{varwidth}[t]{\sphinxcolwidth{1}{4}}
\begin{itemize}
\item {} 
\sphinxAtStartPar
filename

\end{itemize}
\par
\vskip-\baselineskip\vbox{\hbox{\strut}}\end{varwidth}%
}%
&\sphinxstartmulticolumn{3}%
\begin{varwidth}[t]{\sphinxcolwidth{3}{4}}
\sphinxAtStartPar
Name of the project.
\par
\vskip-\baselineskip\vbox{\hbox{\strut}}\end{varwidth}%
\sphinxstopmulticolumn
\\
\cline{2-4}\sphinxtablestrut{9}&
\sphinxAtStartPar
default
&\sphinxstartmulticolumn{2}%
\begin{varwidth}[t]{\sphinxcolwidth{2}{4}}
\sphinxAtStartPar
null
\par
\vskip-\baselineskip\vbox{\hbox{\strut}}\end{varwidth}%
\sphinxstopmulticolumn
\\
\cline{2-4}\sphinxtablestrut{9}&\sphinxmultirow{2}{13}{%
\begin{varwidth}[t]{\sphinxcolwidth{1}{4}}
\sphinxAtStartPar
anyOf
\par
\vskip-\baselineskip\vbox{\hbox{\strut}}\end{varwidth}%
}%
&
\sphinxAtStartPar
type
&
\sphinxAtStartPar
\sphinxstyleemphasis{string}
\\
\cline{3-4}\sphinxtablestrut{9}&\sphinxtablestrut{13}&
\sphinxAtStartPar
type
&
\sphinxAtStartPar
\sphinxstyleemphasis{null}
\\
\hline\sphinxmultirow{4}{18}{%
\begin{varwidth}[t]{\sphinxcolwidth{1}{4}}
\begin{itemize}
\item {} 
\sphinxAtStartPar
link

\end{itemize}
\par
\vskip-\baselineskip\vbox{\hbox{\strut}}\end{varwidth}%
}%
&\sphinxstartmulticolumn{3}%
\begin{varwidth}[t]{\sphinxcolwidth{3}{4}}
\sphinxAtStartPar
Url to a publication or other external resource: a click on the filename will open this link.
\par
\vskip-\baselineskip\vbox{\hbox{\strut}}\end{varwidth}%
\sphinxstopmulticolumn
\\
\cline{2-4}\sphinxtablestrut{18}&
\sphinxAtStartPar
default
&\sphinxstartmulticolumn{2}%
\begin{varwidth}[t]{\sphinxcolwidth{2}{4}}
\sphinxAtStartPar
null
\par
\vskip-\baselineskip\vbox{\hbox{\strut}}\end{varwidth}%
\sphinxstopmulticolumn
\\
\cline{2-4}\sphinxtablestrut{18}&\sphinxmultirow{2}{22}{%
\begin{varwidth}[t]{\sphinxcolwidth{1}{4}}
\sphinxAtStartPar
anyOf
\par
\vskip-\baselineskip\vbox{\hbox{\strut}}\end{varwidth}%
}%
&
\sphinxAtStartPar
type
&
\sphinxAtStartPar
\sphinxstyleemphasis{string}
\\
\cline{3-4}\sphinxtablestrut{18}&\sphinxtablestrut{22}&
\sphinxAtStartPar
type
&
\sphinxAtStartPar
\sphinxstyleemphasis{null}
\\
\hline\sphinxmultirow{3}{27}{%
\begin{varwidth}[t]{\sphinxcolwidth{1}{4}}
\begin{itemize}
\item {} 
\sphinxAtStartPar
layers

\end{itemize}
\par
\vskip-\baselineskip\vbox{\hbox{\strut}}\end{varwidth}%
}%
&
\sphinxAtStartPar
type
&\sphinxstartmulticolumn{2}%
\begin{varwidth}[t]{\sphinxcolwidth{2}{4}}
\sphinxAtStartPar
\sphinxstyleemphasis{array}
\par
\vskip-\baselineskip\vbox{\hbox{\strut}}\end{varwidth}%
\sphinxstopmulticolumn
\\
\cline{2-4}\sphinxtablestrut{27}&
\sphinxAtStartPar
default
&\sphinxstartmulticolumn{2}%
\begin{varwidth}[t]{\sphinxcolwidth{2}{4}}
\par
\vskip-\baselineskip\vbox{\hbox{\strut}}\end{varwidth}%
\sphinxstopmulticolumn
\\
\cline{2-4}\sphinxtablestrut{27}&
\sphinxAtStartPar
items
&\sphinxstartmulticolumn{2}%
\begin{varwidth}[t]{\sphinxcolwidth{2}{4}}
\sphinxAtStartPar
{\hyperref[\detokenize{docs/advanced/tmap:layer}]{\sphinxcrossref{Layer}}}
\par
\vskip-\baselineskip\vbox{\hbox{\strut}}\end{varwidth}%
\sphinxstopmulticolumn
\\
\hline\sphinxmultirow{3}{34}{%
\begin{varwidth}[t]{\sphinxcolwidth{1}{4}}
\begin{itemize}
\item {} 
\sphinxAtStartPar
layerOpacities

\end{itemize}
\par
\vskip-\baselineskip\vbox{\hbox{\strut}}\end{varwidth}%
}%
&
\sphinxAtStartPar
type
&\sphinxstartmulticolumn{2}%
\begin{varwidth}[t]{\sphinxcolwidth{2}{4}}
\sphinxAtStartPar
\sphinxstyleemphasis{object}
\par
\vskip-\baselineskip\vbox{\hbox{\strut}}\end{varwidth}%
\sphinxstopmulticolumn
\\
\cline{2-4}\sphinxtablestrut{34}&
\sphinxAtStartPar
default
&\sphinxstartmulticolumn{2}%
\begin{varwidth}[t]{\sphinxcolwidth{2}{4}}
\par
\vskip-\baselineskip\vbox{\hbox{\strut}}\end{varwidth}%
\sphinxstopmulticolumn
\\
\cline{2-4}\sphinxtablestrut{34}&
\sphinxAtStartPar
additionalProperties
&
\sphinxAtStartPar
type
&
\sphinxAtStartPar
\sphinxstyleemphasis{number}
\\
\hline\sphinxmultirow{3}{42}{%
\begin{varwidth}[t]{\sphinxcolwidth{1}{4}}
\begin{itemize}
\item {} 
\sphinxAtStartPar
layerVisibilities

\end{itemize}
\par
\vskip-\baselineskip\vbox{\hbox{\strut}}\end{varwidth}%
}%
&
\sphinxAtStartPar
type
&\sphinxstartmulticolumn{2}%
\begin{varwidth}[t]{\sphinxcolwidth{2}{4}}
\sphinxAtStartPar
\sphinxstyleemphasis{object}
\par
\vskip-\baselineskip\vbox{\hbox{\strut}}\end{varwidth}%
\sphinxstopmulticolumn
\\
\cline{2-4}\sphinxtablestrut{42}&
\sphinxAtStartPar
default
&\sphinxstartmulticolumn{2}%
\begin{varwidth}[t]{\sphinxcolwidth{2}{4}}
\par
\vskip-\baselineskip\vbox{\hbox{\strut}}\end{varwidth}%
\sphinxstopmulticolumn
\\
\cline{2-4}\sphinxtablestrut{42}&
\sphinxAtStartPar
additionalProperties
&
\sphinxAtStartPar
type
&
\sphinxAtStartPar
\sphinxstyleemphasis{boolean}
\\
\hline\sphinxmultirow{5}{50}{%
\begin{varwidth}[t]{\sphinxcolwidth{1}{4}}
\begin{itemize}
\item {} 
\sphinxAtStartPar
layerFilters

\end{itemize}
\par
\vskip-\baselineskip\vbox{\hbox{\strut}}\end{varwidth}%
}%
&\sphinxstartmulticolumn{3}%
\begin{varwidth}[t]{\sphinxcolwidth{3}{4}}
\sphinxAtStartPar
Image filters to be applied to pixels in image layers.
\par
\vskip-\baselineskip\vbox{\hbox{\strut}}\end{varwidth}%
\sphinxstopmulticolumn
\\
\cline{2-4}\sphinxtablestrut{50}&
\sphinxAtStartPar
type
&\sphinxstartmulticolumn{2}%
\begin{varwidth}[t]{\sphinxcolwidth{2}{4}}
\sphinxAtStartPar
\sphinxstyleemphasis{object}
\par
\vskip-\baselineskip\vbox{\hbox{\strut}}\end{varwidth}%
\sphinxstopmulticolumn
\\
\cline{2-4}\sphinxtablestrut{50}&
\sphinxAtStartPar
default
&\sphinxstartmulticolumn{2}%
\begin{varwidth}[t]{\sphinxcolwidth{2}{4}}
\par
\vskip-\baselineskip\vbox{\hbox{\strut}}\end{varwidth}%
\sphinxstopmulticolumn
\\
\cline{2-4}\sphinxtablestrut{50}&\sphinxmultirow{2}{56}{%
\begin{varwidth}[t]{\sphinxcolwidth{1}{4}}
\sphinxAtStartPar
additionalProperties
\par
\vskip-\baselineskip\vbox{\hbox{\strut}}\end{varwidth}%
}%
&
\sphinxAtStartPar
type
&
\sphinxAtStartPar
\sphinxstyleemphasis{array}
\\
\cline{3-4}\sphinxtablestrut{50}&\sphinxtablestrut{56}&
\sphinxAtStartPar
items
&
\sphinxAtStartPar
{\hyperref[\detokenize{docs/advanced/tmap:layerfilter}]{\sphinxcrossref{LayerFilter}}}
\\
\hline\sphinxmultirow{6}{61}{%
\begin{varwidth}[t]{\sphinxcolwidth{1}{4}}
\begin{itemize}
\item {} 
\sphinxAtStartPar
filters

\end{itemize}
\par
\vskip-\baselineskip\vbox{\hbox{\strut}}\end{varwidth}%
}%
&\sphinxstartmulticolumn{3}%
\begin{varwidth}[t]{\sphinxcolwidth{3}{4}}
\sphinxAtStartPar
List of filters shown as active filters in the GUI under the Image layers tab.
\par
\vskip-\baselineskip\vbox{\hbox{\strut}}\end{varwidth}%
\sphinxstopmulticolumn
\\
\cline{2-4}\sphinxtablestrut{61}&
\sphinxAtStartPar
type
&\sphinxstartmulticolumn{2}%
\begin{varwidth}[t]{\sphinxcolwidth{2}{4}}
\sphinxAtStartPar
\sphinxstyleemphasis{array}
\par
\vskip-\baselineskip\vbox{\hbox{\strut}}\end{varwidth}%
\sphinxstopmulticolumn
\\
\cline{2-4}\sphinxtablestrut{61}&\sphinxmultirow{3}{65}{%
\begin{varwidth}[t]{\sphinxcolwidth{1}{4}}
\sphinxAtStartPar
default
\par
\vskip-\baselineskip\vbox{\hbox{\strut}}\end{varwidth}%
}%
&\sphinxstartmulticolumn{2}%
\begin{varwidth}[t]{\sphinxcolwidth{2}{4}}
\sphinxAtStartPar
Saturation
\par
\vskip-\baselineskip\vbox{\hbox{\strut}}\end{varwidth}%
\sphinxstopmulticolumn
\\
\cline{3-4}\sphinxtablestrut{61}&\sphinxtablestrut{65}&\sphinxstartmulticolumn{2}%
\begin{varwidth}[t]{\sphinxcolwidth{2}{4}}
\sphinxAtStartPar
Brightness
\par
\vskip-\baselineskip\vbox{\hbox{\strut}}\end{varwidth}%
\sphinxstopmulticolumn
\\
\cline{3-4}\sphinxtablestrut{61}&\sphinxtablestrut{65}&\sphinxstartmulticolumn{2}%
\begin{varwidth}[t]{\sphinxcolwidth{2}{4}}
\sphinxAtStartPar
Contrast
\par
\vskip-\baselineskip\vbox{\hbox{\strut}}\end{varwidth}%
\sphinxstopmulticolumn
\\
\cline{2-4}\sphinxtablestrut{61}&
\sphinxAtStartPar
items
&\sphinxstartmulticolumn{2}%
\begin{varwidth}[t]{\sphinxcolwidth{2}{4}}
\sphinxAtStartPar
{\hyperref[\detokenize{docs/advanced/tmap:filter}]{\sphinxcrossref{Filter}}}
\par
\vskip-\baselineskip\vbox{\hbox{\strut}}\end{varwidth}%
\sphinxstopmulticolumn
\\
\hline\sphinxmultirow{3}{71}{%
\begin{varwidth}[t]{\sphinxcolwidth{1}{4}}
\begin{itemize}
\item {} 
\sphinxAtStartPar
compositeMode

\end{itemize}
\par
\vskip-\baselineskip\vbox{\hbox{\strut}}\end{varwidth}%
}%
&\sphinxstartmulticolumn{3}%
\begin{varwidth}[t]{\sphinxcolwidth{3}{4}}
\sphinxAtStartPar
Mode defining how image layers will be merged (composited) with each other. Valid string values are ‘source\sphinxhyphen{}over’ and ‘lighter’, which correspond to ‘Channels’ and ‘Composite’ in the GUI.
\par
\vskip-\baselineskip\vbox{\hbox{\strut}}\end{varwidth}%
\sphinxstopmulticolumn
\\
\cline{2-4}\sphinxtablestrut{71}&
\sphinxAtStartPar
default
&\sphinxstartmulticolumn{2}%
\begin{varwidth}[t]{\sphinxcolwidth{2}{4}}
\sphinxAtStartPar
source\sphinxhyphen{}over
\par
\vskip-\baselineskip\vbox{\hbox{\strut}}\end{varwidth}%
\sphinxstopmulticolumn
\\
\cline{2-4}\sphinxtablestrut{71}&
\sphinxAtStartPar
allOf
&\sphinxstartmulticolumn{2}%
\begin{varwidth}[t]{\sphinxcolwidth{2}{4}}
\sphinxAtStartPar
{\hyperref[\detokenize{docs/advanced/tmap:compositemode}]{\sphinxcrossref{CompositeMode}}}
\par
\vskip-\baselineskip\vbox{\hbox{\strut}}\end{varwidth}%
\sphinxstopmulticolumn
\\
\hline\sphinxmultirow{4}{77}{%
\begin{varwidth}[t]{\sphinxcolwidth{1}{4}}
\begin{itemize}
\item {} 
\sphinxAtStartPar
collectionLayout

\end{itemize}
\par
\vskip-\baselineskip\vbox{\hbox{\strut}}\end{varwidth}%
}%
&\sphinxstartmulticolumn{3}%
\begin{varwidth}[t]{\sphinxcolwidth{3}{4}}
\sphinxAtStartPar
Options to be passed to OpenSeadragon arrange method when in collectionmode. See (\sphinxurl{https://openseadragon.github.io/docs/OpenSeadragon.World.html\#arrange})
\par
\vskip-\baselineskip\vbox{\hbox{\strut}}\end{varwidth}%
\sphinxstopmulticolumn
\\
\cline{2-4}\sphinxtablestrut{77}&
\sphinxAtStartPar
default
&\sphinxstartmulticolumn{2}%
\begin{varwidth}[t]{\sphinxcolwidth{2}{4}}
\sphinxAtStartPar
null
\par
\vskip-\baselineskip\vbox{\hbox{\strut}}\end{varwidth}%
\sphinxstopmulticolumn
\\
\cline{2-4}\sphinxtablestrut{77}&\sphinxmultirow{2}{81}{%
\begin{varwidth}[t]{\sphinxcolwidth{1}{4}}
\sphinxAtStartPar
anyOf
\par
\vskip-\baselineskip\vbox{\hbox{\strut}}\end{varwidth}%
}%
&\sphinxstartmulticolumn{2}%
\begin{varwidth}[t]{\sphinxcolwidth{2}{4}}
\sphinxAtStartPar
{\hyperref[\detokenize{docs/advanced/tmap:collectionlayout}]{\sphinxcrossref{CollectionLayout}}}
\par
\vskip-\baselineskip\vbox{\hbox{\strut}}\end{varwidth}%
\sphinxstopmulticolumn
\\
\cline{3-4}\sphinxtablestrut{77}&\sphinxtablestrut{81}&
\sphinxAtStartPar
type
&
\sphinxAtStartPar
\sphinxstyleemphasis{null}
\\
\hline\sphinxmultirow{4}{85}{%
\begin{varwidth}[t]{\sphinxcolwidth{1}{4}}
\begin{itemize}
\item {} 
\sphinxAtStartPar
mpp

\end{itemize}
\par
\vskip-\baselineskip\vbox{\hbox{\strut}}\end{varwidth}%
}%
&\sphinxstartmulticolumn{3}%
\begin{varwidth}[t]{\sphinxcolwidth{3}{4}}
\sphinxAtStartPar
The image scale in Microns Per Pixels. If not null, then adds a scale bar to the viewer. Set to 0 to display the scale bar in pixels.
\par
\vskip-\baselineskip\vbox{\hbox{\strut}}\end{varwidth}%
\sphinxstopmulticolumn
\\
\cline{2-4}\sphinxtablestrut{85}&
\sphinxAtStartPar
default
&\sphinxstartmulticolumn{2}%
\begin{varwidth}[t]{\sphinxcolwidth{2}{4}}
\sphinxAtStartPar
null
\par
\vskip-\baselineskip\vbox{\hbox{\strut}}\end{varwidth}%
\sphinxstopmulticolumn
\\
\cline{2-4}\sphinxtablestrut{85}&\sphinxmultirow{2}{89}{%
\begin{varwidth}[t]{\sphinxcolwidth{1}{4}}
\sphinxAtStartPar
anyOf
\par
\vskip-\baselineskip\vbox{\hbox{\strut}}\end{varwidth}%
}%
&
\sphinxAtStartPar
type
&
\sphinxAtStartPar
\sphinxstyleemphasis{number}
\\
\cline{3-4}\sphinxtablestrut{85}&\sphinxtablestrut{89}&
\sphinxAtStartPar
type
&
\sphinxAtStartPar
\sphinxstyleemphasis{null}
\\
\hline\sphinxmultirow{4}{94}{%
\begin{varwidth}[t]{\sphinxcolwidth{1}{4}}
\begin{itemize}
\item {} 
\sphinxAtStartPar
boundingBox

\end{itemize}
\par
\vskip-\baselineskip\vbox{\hbox{\strut}}\end{varwidth}%
}%
&\sphinxstartmulticolumn{3}%
\begin{varwidth}[t]{\sphinxcolwidth{3}{4}}
\sphinxAtStartPar
Bounding box used to set initial zoom and pan on the view when loading the project.
\par
\vskip-\baselineskip\vbox{\hbox{\strut}}\end{varwidth}%
\sphinxstopmulticolumn
\\
\cline{2-4}\sphinxtablestrut{94}&
\sphinxAtStartPar
default
&\sphinxstartmulticolumn{2}%
\begin{varwidth}[t]{\sphinxcolwidth{2}{4}}
\sphinxAtStartPar
null
\par
\vskip-\baselineskip\vbox{\hbox{\strut}}\end{varwidth}%
\sphinxstopmulticolumn
\\
\cline{2-4}\sphinxtablestrut{94}&\sphinxmultirow{2}{98}{%
\begin{varwidth}[t]{\sphinxcolwidth{1}{4}}
\sphinxAtStartPar
anyOf
\par
\vskip-\baselineskip\vbox{\hbox{\strut}}\end{varwidth}%
}%
&\sphinxstartmulticolumn{2}%
\begin{varwidth}[t]{\sphinxcolwidth{2}{4}}
\sphinxAtStartPar
{\hyperref[\detokenize{docs/advanced/tmap:boundingbox}]{\sphinxcrossref{BoundingBox}}}
\par
\vskip-\baselineskip\vbox{\hbox{\strut}}\end{varwidth}%
\sphinxstopmulticolumn
\\
\cline{3-4}\sphinxtablestrut{94}&\sphinxtablestrut{98}&
\sphinxAtStartPar
type
&
\sphinxAtStartPar
\sphinxstyleemphasis{null}
\\
\hline\sphinxmultirow{3}{102}{%
\begin{varwidth}[t]{\sphinxcolwidth{1}{4}}
\begin{itemize}
\item {} 
\sphinxAtStartPar
rotate

\end{itemize}
\par
\vskip-\baselineskip\vbox{\hbox{\strut}}\end{varwidth}%
}%
&\sphinxstartmulticolumn{3}%
\begin{varwidth}[t]{\sphinxcolwidth{3}{4}}
\sphinxAtStartPar
Angle of rotation of the view in degrees. Only multiples of 90 degrees are supported.
\par
\vskip-\baselineskip\vbox{\hbox{\strut}}\end{varwidth}%
\sphinxstopmulticolumn
\\
\cline{2-4}\sphinxtablestrut{102}&
\sphinxAtStartPar
type
&\sphinxstartmulticolumn{2}%
\begin{varwidth}[t]{\sphinxcolwidth{2}{4}}
\sphinxAtStartPar
\sphinxstyleemphasis{integer}
\par
\vskip-\baselineskip\vbox{\hbox{\strut}}\end{varwidth}%
\sphinxstopmulticolumn
\\
\cline{2-4}\sphinxtablestrut{102}&
\sphinxAtStartPar
default
&\sphinxstartmulticolumn{2}%
\begin{varwidth}[t]{\sphinxcolwidth{2}{4}}
\sphinxAtStartPar
0
\par
\vskip-\baselineskip\vbox{\hbox{\strut}}\end{varwidth}%
\sphinxstopmulticolumn
\\
\hline\sphinxmultirow{3}{108}{%
\begin{varwidth}[t]{\sphinxcolwidth{1}{4}}
\begin{itemize}
\item {} 
\sphinxAtStartPar
markerFiles

\end{itemize}
\par
\vskip-\baselineskip\vbox{\hbox{\strut}}\end{varwidth}%
}%
&
\sphinxAtStartPar
type
&\sphinxstartmulticolumn{2}%
\begin{varwidth}[t]{\sphinxcolwidth{2}{4}}
\sphinxAtStartPar
\sphinxstyleemphasis{array}
\par
\vskip-\baselineskip\vbox{\hbox{\strut}}\end{varwidth}%
\sphinxstopmulticolumn
\\
\cline{2-4}\sphinxtablestrut{108}&
\sphinxAtStartPar
default
&\sphinxstartmulticolumn{2}%
\begin{varwidth}[t]{\sphinxcolwidth{2}{4}}
\par
\vskip-\baselineskip\vbox{\hbox{\strut}}\end{varwidth}%
\sphinxstopmulticolumn
\\
\cline{2-4}\sphinxtablestrut{108}&
\sphinxAtStartPar
items
&\sphinxstartmulticolumn{2}%
\begin{varwidth}[t]{\sphinxcolwidth{2}{4}}
\sphinxAtStartPar
{\hyperref[\detokenize{docs/advanced/tmap:markerfile}]{\sphinxcrossref{MarkerFile}}}
\par
\vskip-\baselineskip\vbox{\hbox{\strut}}\end{varwidth}%
\sphinxstopmulticolumn
\\
\hline\sphinxmultirow{3}{115}{%
\begin{varwidth}[t]{\sphinxcolwidth{1}{4}}
\begin{itemize}
\item {} 
\sphinxAtStartPar
regions

\end{itemize}
\par
\vskip-\baselineskip\vbox{\hbox{\strut}}\end{varwidth}%
}%
&\sphinxstartmulticolumn{3}%
\begin{varwidth}[t]{\sphinxcolwidth{3}{4}}
\sphinxAtStartPar
GeoJSON object.
\par
\vskip-\baselineskip\vbox{\hbox{\strut}}\end{varwidth}%
\sphinxstopmulticolumn
\\
\cline{2-4}\sphinxtablestrut{115}&
\sphinxAtStartPar
type
&\sphinxstartmulticolumn{2}%
\begin{varwidth}[t]{\sphinxcolwidth{2}{4}}
\sphinxAtStartPar
\sphinxstyleemphasis{object}
\par
\vskip-\baselineskip\vbox{\hbox{\strut}}\end{varwidth}%
\sphinxstopmulticolumn
\\
\cline{2-4}\sphinxtablestrut{115}&
\sphinxAtStartPar
default
&\sphinxstartmulticolumn{2}%
\begin{varwidth}[t]{\sphinxcolwidth{2}{4}}
\par
\vskip-\baselineskip\vbox{\hbox{\strut}}\end{varwidth}%
\sphinxstopmulticolumn
\\
\hline\sphinxmultirow{4}{121}{%
\begin{varwidth}[t]{\sphinxcolwidth{1}{4}}
\begin{itemize}
\item {} 
\sphinxAtStartPar
regionFile

\end{itemize}
\par
\vskip-\baselineskip\vbox{\hbox{\strut}}\end{varwidth}%
}%
&\sphinxstartmulticolumn{3}%
\begin{varwidth}[t]{\sphinxcolwidth{3}{4}}
\sphinxAtStartPar
\sphinxstylestrong{{[}Deprecated{]}} GeoJSON region file loaded on project initialization. Use regionFiles instead.
\par
\vskip-\baselineskip\vbox{\hbox{\strut}}\end{varwidth}%
\sphinxstopmulticolumn
\\
\cline{2-4}\sphinxtablestrut{121}&
\sphinxAtStartPar
default
&\sphinxstartmulticolumn{2}%
\begin{varwidth}[t]{\sphinxcolwidth{2}{4}}
\sphinxAtStartPar
null
\par
\vskip-\baselineskip\vbox{\hbox{\strut}}\end{varwidth}%
\sphinxstopmulticolumn
\\
\cline{2-4}\sphinxtablestrut{121}&\sphinxmultirow{2}{125}{%
\begin{varwidth}[t]{\sphinxcolwidth{1}{4}}
\sphinxAtStartPar
anyOf
\par
\vskip-\baselineskip\vbox{\hbox{\strut}}\end{varwidth}%
}%
&
\sphinxAtStartPar
type
&
\sphinxAtStartPar
\sphinxstyleemphasis{string}
\\
\cline{3-4}\sphinxtablestrut{121}&\sphinxtablestrut{125}&
\sphinxAtStartPar
type
&
\sphinxAtStartPar
\sphinxstyleemphasis{null}
\\
\hline\sphinxmultirow{3}{130}{%
\begin{varwidth}[t]{\sphinxcolwidth{1}{4}}
\begin{itemize}
\item {} 
\sphinxAtStartPar
regionFiles

\end{itemize}
\par
\vskip-\baselineskip\vbox{\hbox{\strut}}\end{varwidth}%
}%
&
\sphinxAtStartPar
type
&\sphinxstartmulticolumn{2}%
\begin{varwidth}[t]{\sphinxcolwidth{2}{4}}
\sphinxAtStartPar
\sphinxstyleemphasis{array}
\par
\vskip-\baselineskip\vbox{\hbox{\strut}}\end{varwidth}%
\sphinxstopmulticolumn
\\
\cline{2-4}\sphinxtablestrut{130}&
\sphinxAtStartPar
default
&\sphinxstartmulticolumn{2}%
\begin{varwidth}[t]{\sphinxcolwidth{2}{4}}
\par
\vskip-\baselineskip\vbox{\hbox{\strut}}\end{varwidth}%
\sphinxstopmulticolumn
\\
\cline{2-4}\sphinxtablestrut{130}&
\sphinxAtStartPar
items
&\sphinxstartmulticolumn{2}%
\begin{varwidth}[t]{\sphinxcolwidth{2}{4}}
\sphinxAtStartPar
{\hyperref[\detokenize{docs/advanced/tmap:regionfile}]{\sphinxcrossref{RegionFile}}}
\par
\vskip-\baselineskip\vbox{\hbox{\strut}}\end{varwidth}%
\sphinxstopmulticolumn
\\
\hline\sphinxmultirow{4}{137}{%
\begin{varwidth}[t]{\sphinxcolwidth{1}{4}}
\begin{itemize}
\item {} 
\sphinxAtStartPar
plugins

\end{itemize}
\par
\vskip-\baselineskip\vbox{\hbox{\strut}}\end{varwidth}%
}%
&\sphinxstartmulticolumn{3}%
\begin{varwidth}[t]{\sphinxcolwidth{3}{4}}
\sphinxAtStartPar
List of plugins to load with the project.
\par
\vskip-\baselineskip\vbox{\hbox{\strut}}\end{varwidth}%
\sphinxstopmulticolumn
\\
\cline{2-4}\sphinxtablestrut{137}&
\sphinxAtStartPar
type
&\sphinxstartmulticolumn{2}%
\begin{varwidth}[t]{\sphinxcolwidth{2}{4}}
\sphinxAtStartPar
\sphinxstyleemphasis{array}
\par
\vskip-\baselineskip\vbox{\hbox{\strut}}\end{varwidth}%
\sphinxstopmulticolumn
\\
\cline{2-4}\sphinxtablestrut{137}&
\sphinxAtStartPar
default
&\sphinxstartmulticolumn{2}%
\begin{varwidth}[t]{\sphinxcolwidth{2}{4}}
\par
\vskip-\baselineskip\vbox{\hbox{\strut}}\end{varwidth}%
\sphinxstopmulticolumn
\\
\cline{2-4}\sphinxtablestrut{137}&
\sphinxAtStartPar
items
&
\sphinxAtStartPar
type
&
\sphinxAtStartPar
\sphinxstyleemphasis{string}
\\
\hline\sphinxmultirow{3}{146}{%
\begin{varwidth}[t]{\sphinxcolwidth{1}{4}}
\begin{itemize}
\item {} 
\sphinxAtStartPar
hideTabs

\end{itemize}
\par
\vskip-\baselineskip\vbox{\hbox{\strut}}\end{varwidth}%
}%
&\sphinxstartmulticolumn{3}%
\begin{varwidth}[t]{\sphinxcolwidth{3}{4}}
\sphinxAtStartPar
Hide tabs of markers dataset. Only use when you have a unique marker tab.
\par
\vskip-\baselineskip\vbox{\hbox{\strut}}\end{varwidth}%
\sphinxstopmulticolumn
\\
\cline{2-4}\sphinxtablestrut{146}&
\sphinxAtStartPar
type
&\sphinxstartmulticolumn{2}%
\begin{varwidth}[t]{\sphinxcolwidth{2}{4}}
\sphinxAtStartPar
\sphinxstyleemphasis{boolean}
\par
\vskip-\baselineskip\vbox{\hbox{\strut}}\end{varwidth}%
\sphinxstopmulticolumn
\\
\cline{2-4}\sphinxtablestrut{146}&
\sphinxAtStartPar
default
&\sphinxstartmulticolumn{2}%
\begin{varwidth}[t]{\sphinxcolwidth{2}{4}}
\sphinxAtStartPar
False
\par
\vskip-\baselineskip\vbox{\hbox{\strut}}\end{varwidth}%
\sphinxstopmulticolumn
\\
\hline\sphinxmultirow{3}{152}{%
\begin{varwidth}[t]{\sphinxcolwidth{1}{4}}
\begin{itemize}
\item {} 
\sphinxAtStartPar
hideChannelRange

\end{itemize}
\par
\vskip-\baselineskip\vbox{\hbox{\strut}}\end{varwidth}%
}%
&\sphinxstartmulticolumn{3}%
\begin{varwidth}[t]{\sphinxcolwidth{3}{4}}
\sphinxAtStartPar
Hide input range of channels. Only use when you have a unique image layer.
\par
\vskip-\baselineskip\vbox{\hbox{\strut}}\end{varwidth}%
\sphinxstopmulticolumn
\\
\cline{2-4}\sphinxtablestrut{152}&
\sphinxAtStartPar
type
&\sphinxstartmulticolumn{2}%
\begin{varwidth}[t]{\sphinxcolwidth{2}{4}}
\sphinxAtStartPar
\sphinxstyleemphasis{boolean}
\par
\vskip-\baselineskip\vbox{\hbox{\strut}}\end{varwidth}%
\sphinxstopmulticolumn
\\
\cline{2-4}\sphinxtablestrut{152}&
\sphinxAtStartPar
default
&\sphinxstartmulticolumn{2}%
\begin{varwidth}[t]{\sphinxcolwidth{2}{4}}
\sphinxAtStartPar
False
\par
\vskip-\baselineskip\vbox{\hbox{\strut}}\end{varwidth}%
\sphinxstopmulticolumn
\\
\hline\sphinxmultirow{3}{158}{%
\begin{varwidth}[t]{\sphinxcolwidth{1}{4}}
\begin{itemize}
\item {} 
\sphinxAtStartPar
hideNavigator

\end{itemize}
\par
\vskip-\baselineskip\vbox{\hbox{\strut}}\end{varwidth}%
}%
&\sphinxstartmulticolumn{3}%
\begin{varwidth}[t]{\sphinxcolwidth{3}{4}}
\sphinxAtStartPar
Hide navigator of the viewer.
\par
\vskip-\baselineskip\vbox{\hbox{\strut}}\end{varwidth}%
\sphinxstopmulticolumn
\\
\cline{2-4}\sphinxtablestrut{158}&
\sphinxAtStartPar
type
&\sphinxstartmulticolumn{2}%
\begin{varwidth}[t]{\sphinxcolwidth{2}{4}}
\sphinxAtStartPar
\sphinxstyleemphasis{boolean}
\par
\vskip-\baselineskip\vbox{\hbox{\strut}}\end{varwidth}%
\sphinxstopmulticolumn
\\
\cline{2-4}\sphinxtablestrut{158}&
\sphinxAtStartPar
default
&\sphinxstartmulticolumn{2}%
\begin{varwidth}[t]{\sphinxcolwidth{2}{4}}
\sphinxAtStartPar
False
\par
\vskip-\baselineskip\vbox{\hbox{\strut}}\end{varwidth}%
\sphinxstopmulticolumn
\\
\hline\sphinxmultirow{3}{164}{%
\begin{varwidth}[t]{\sphinxcolwidth{1}{4}}
\begin{itemize}
\item {} 
\sphinxAtStartPar
collectionMode

\end{itemize}
\par
\vskip-\baselineskip\vbox{\hbox{\strut}}\end{varwidth}%
}%
&\sphinxstartmulticolumn{3}%
\begin{varwidth}[t]{\sphinxcolwidth{3}{4}}
\sphinxAtStartPar
If true, then the viewer will be in collection mode, which puts all layers in a grid next to each other.
\par
\vskip-\baselineskip\vbox{\hbox{\strut}}\end{varwidth}%
\sphinxstopmulticolumn
\\
\cline{2-4}\sphinxtablestrut{164}&
\sphinxAtStartPar
type
&\sphinxstartmulticolumn{2}%
\begin{varwidth}[t]{\sphinxcolwidth{2}{4}}
\sphinxAtStartPar
\sphinxstyleemphasis{boolean}
\par
\vskip-\baselineskip\vbox{\hbox{\strut}}\end{varwidth}%
\sphinxstopmulticolumn
\\
\cline{2-4}\sphinxtablestrut{164}&
\sphinxAtStartPar
default
&\sphinxstartmulticolumn{2}%
\begin{varwidth}[t]{\sphinxcolwidth{2}{4}}
\sphinxAtStartPar
False
\par
\vskip-\baselineskip\vbox{\hbox{\strut}}\end{varwidth}%
\sphinxstopmulticolumn
\\
\hline\sphinxmultirow{4}{170}{%
\begin{varwidth}[t]{\sphinxcolwidth{1}{4}}
\begin{itemize}
\item {} 
\sphinxAtStartPar
backgroundColor

\end{itemize}
\par
\vskip-\baselineskip\vbox{\hbox{\strut}}\end{varwidth}%
}%
&\sphinxstartmulticolumn{3}%
\begin{varwidth}[t]{\sphinxcolwidth{3}{4}}
\sphinxAtStartPar
Background color of the viewer.
\par
\vskip-\baselineskip\vbox{\hbox{\strut}}\end{varwidth}%
\sphinxstopmulticolumn
\\
\cline{2-4}\sphinxtablestrut{170}&
\sphinxAtStartPar
default
&\sphinxstartmulticolumn{2}%
\begin{varwidth}[t]{\sphinxcolwidth{2}{4}}
\sphinxAtStartPar
null
\par
\vskip-\baselineskip\vbox{\hbox{\strut}}\end{varwidth}%
\sphinxstopmulticolumn
\\
\cline{2-4}\sphinxtablestrut{170}&\sphinxmultirow{2}{174}{%
\begin{varwidth}[t]{\sphinxcolwidth{1}{4}}
\sphinxAtStartPar
anyOf
\par
\vskip-\baselineskip\vbox{\hbox{\strut}}\end{varwidth}%
}%
&
\sphinxAtStartPar
type
&
\sphinxAtStartPar
\sphinxstyleemphasis{string}
\\
\cline{3-4}\sphinxtablestrut{170}&\sphinxtablestrut{174}&
\sphinxAtStartPar
type
&
\sphinxAtStartPar
\sphinxstyleemphasis{null}
\\
\hline\sphinxmultirow{5}{179}{%
\begin{varwidth}[t]{\sphinxcolwidth{1}{4}}
\begin{itemize}
\item {} 
\sphinxAtStartPar
menuButtons

\end{itemize}
\par
\vskip-\baselineskip\vbox{\hbox{\strut}}\end{varwidth}%
}%
&\sphinxstartmulticolumn{3}%
\begin{varwidth}[t]{\sphinxcolwidth{3}{4}}
\sphinxAtStartPar
List of menu items to be added to the menu bar.
\par
\vskip-\baselineskip\vbox{\hbox{\strut}}\end{varwidth}%
\sphinxstopmulticolumn
\\
\cline{2-4}\sphinxtablestrut{179}&
\sphinxAtStartPar
default
&\sphinxstartmulticolumn{2}%
\begin{varwidth}[t]{\sphinxcolwidth{2}{4}}
\sphinxAtStartPar
null
\par
\vskip-\baselineskip\vbox{\hbox{\strut}}\end{varwidth}%
\sphinxstopmulticolumn
\\
\cline{2-4}\sphinxtablestrut{179}&\sphinxmultirow{3}{183}{%
\begin{varwidth}[t]{\sphinxcolwidth{1}{4}}
\sphinxAtStartPar
anyOf
\par
\vskip-\baselineskip\vbox{\hbox{\strut}}\end{varwidth}%
}%
&
\sphinxAtStartPar
type
&
\sphinxAtStartPar
\sphinxstyleemphasis{array}
\\
\cline{3-4}\sphinxtablestrut{179}&\sphinxtablestrut{183}&
\sphinxAtStartPar
items
&
\sphinxAtStartPar
{\hyperref[\detokenize{docs/advanced/tmap:menubutton}]{\sphinxcrossref{menuButton}}}
\\
\cline{3-4}\sphinxtablestrut{179}&\sphinxtablestrut{183}&
\sphinxAtStartPar
type
&
\sphinxAtStartPar
\sphinxstyleemphasis{null}
\\
\hline\sphinxmultirow{3}{190}{%
\begin{varwidth}[t]{\sphinxcolwidth{1}{4}}
\begin{itemize}
\item {} 
\sphinxAtStartPar
settings

\end{itemize}
\par
\vskip-\baselineskip\vbox{\hbox{\strut}}\end{varwidth}%
}%
&
\sphinxAtStartPar
type
&\sphinxstartmulticolumn{2}%
\begin{varwidth}[t]{\sphinxcolwidth{2}{4}}
\sphinxAtStartPar
\sphinxstyleemphasis{array}
\par
\vskip-\baselineskip\vbox{\hbox{\strut}}\end{varwidth}%
\sphinxstopmulticolumn
\\
\cline{2-4}\sphinxtablestrut{190}&
\sphinxAtStartPar
default
&\sphinxstartmulticolumn{2}%
\begin{varwidth}[t]{\sphinxcolwidth{2}{4}}
\par
\vskip-\baselineskip\vbox{\hbox{\strut}}\end{varwidth}%
\sphinxstopmulticolumn
\\
\cline{2-4}\sphinxtablestrut{190}&
\sphinxAtStartPar
items
&\sphinxstartmulticolumn{2}%
\begin{varwidth}[t]{\sphinxcolwidth{2}{4}}
\sphinxAtStartPar
{\hyperref[\detokenize{docs/advanced/tmap:setting}]{\sphinxcrossref{Setting}}}
\par
\vskip-\baselineskip\vbox{\hbox{\strut}}\end{varwidth}%
\sphinxstopmulticolumn
\\
\hline
\sphinxAtStartPar
additionalProperties
&\sphinxstartmulticolumn{3}%
\begin{varwidth}[t]{\sphinxcolwidth{3}{4}}
\sphinxAtStartPar
False
\par
\vskip-\baselineskip\vbox{\hbox{\strut}}\end{varwidth}%
\sphinxstopmulticolumn
\\
\hline
\end{longtable}\sphinxatlongtableend\end{savenotes}


\subsection{BoundingBox}
\label{\detokenize{docs/advanced/tmap:boundingbox}}\label{\detokenize{docs/advanced/tmap:boundingbox}}

\begin{savenotes}\sphinxattablestart
\centering
\begin{tabular}[t]{|*{3}{\X{1}{3}|}}
\hline

\sphinxAtStartPar
type
&\sphinxstartmulticolumn{2}%
\begin{varwidth}[t]{\sphinxcolwidth{2}{3}}
\sphinxAtStartPar
\sphinxstyleemphasis{object}
\par
\vskip-\baselineskip\vbox{\hbox{\strut}}\end{varwidth}%
\sphinxstopmulticolumn
\\
\hline\sphinxstartmulticolumn{3}%
\begin{varwidth}[t]{\sphinxcolwidth{3}{3}}
\sphinxAtStartPar
properties
\par
\vskip-\baselineskip\vbox{\hbox{\strut}}\end{varwidth}%
\sphinxstopmulticolumn
\\
\hline\sphinxmultirow{2}{4}{%
\begin{varwidth}[t]{\sphinxcolwidth{1}{3}}
\begin{itemize}
\item {} 
\sphinxAtStartPar
\sphinxstylestrong{x}

\end{itemize}
\par
\vskip-\baselineskip\vbox{\hbox{\strut}}\end{varwidth}%
}%
&\sphinxstartmulticolumn{2}%
\begin{varwidth}[t]{\sphinxcolwidth{2}{3}}
\sphinxAtStartPar
Left coordinate of the bounding box in viewport coordinate.
\par
\vskip-\baselineskip\vbox{\hbox{\strut}}\end{varwidth}%
\sphinxstopmulticolumn
\\
\cline{2-3}\sphinxtablestrut{4}&
\sphinxAtStartPar
type
&
\sphinxAtStartPar
\sphinxstyleemphasis{number}
\\
\hline\sphinxmultirow{2}{8}{%
\begin{varwidth}[t]{\sphinxcolwidth{1}{3}}
\begin{itemize}
\item {} 
\sphinxAtStartPar
\sphinxstylestrong{y}

\end{itemize}
\par
\vskip-\baselineskip\vbox{\hbox{\strut}}\end{varwidth}%
}%
&\sphinxstartmulticolumn{2}%
\begin{varwidth}[t]{\sphinxcolwidth{2}{3}}
\sphinxAtStartPar
Top coordinate of the bounding box in viewport coordinate.
\par
\vskip-\baselineskip\vbox{\hbox{\strut}}\end{varwidth}%
\sphinxstopmulticolumn
\\
\cline{2-3}\sphinxtablestrut{8}&
\sphinxAtStartPar
type
&
\sphinxAtStartPar
\sphinxstyleemphasis{number}
\\
\hline\sphinxmultirow{2}{12}{%
\begin{varwidth}[t]{\sphinxcolwidth{1}{3}}
\begin{itemize}
\item {} 
\sphinxAtStartPar
\sphinxstylestrong{width}

\end{itemize}
\par
\vskip-\baselineskip\vbox{\hbox{\strut}}\end{varwidth}%
}%
&\sphinxstartmulticolumn{2}%
\begin{varwidth}[t]{\sphinxcolwidth{2}{3}}
\sphinxAtStartPar
Width of the bounding box in viewport coordinate.
\par
\vskip-\baselineskip\vbox{\hbox{\strut}}\end{varwidth}%
\sphinxstopmulticolumn
\\
\cline{2-3}\sphinxtablestrut{12}&
\sphinxAtStartPar
type
&
\sphinxAtStartPar
\sphinxstyleemphasis{number}
\\
\hline\sphinxmultirow{2}{16}{%
\begin{varwidth}[t]{\sphinxcolwidth{1}{3}}
\begin{itemize}
\item {} 
\sphinxAtStartPar
\sphinxstylestrong{height}

\end{itemize}
\par
\vskip-\baselineskip\vbox{\hbox{\strut}}\end{varwidth}%
}%
&\sphinxstartmulticolumn{2}%
\begin{varwidth}[t]{\sphinxcolwidth{2}{3}}
\sphinxAtStartPar
Height of the bounding box in viewport coordinate.
\par
\vskip-\baselineskip\vbox{\hbox{\strut}}\end{varwidth}%
\sphinxstopmulticolumn
\\
\cline{2-3}\sphinxtablestrut{16}&
\sphinxAtStartPar
type
&
\sphinxAtStartPar
\sphinxstyleemphasis{number}
\\
\hline
\sphinxAtStartPar
additionalProperties
&\sphinxstartmulticolumn{2}%
\begin{varwidth}[t]{\sphinxcolwidth{2}{3}}
\sphinxAtStartPar
False
\par
\vskip-\baselineskip\vbox{\hbox{\strut}}\end{varwidth}%
\sphinxstopmulticolumn
\\
\hline
\end{tabular}
\par
\sphinxattableend\end{savenotes}


\subsection{CollectionLayout}
\label{\detokenize{docs/advanced/tmap:collectionlayout}}\label{\detokenize{docs/advanced/tmap:collectionlayout}}

\begin{savenotes}\sphinxattablestart
\centering
\begin{tabular}[t]{|*{4}{\X{1}{4}|}}
\hline

\sphinxAtStartPar
type
&\sphinxstartmulticolumn{3}%
\begin{varwidth}[t]{\sphinxcolwidth{3}{4}}
\sphinxAtStartPar
\sphinxstyleemphasis{object}
\par
\vskip-\baselineskip\vbox{\hbox{\strut}}\end{varwidth}%
\sphinxstopmulticolumn
\\
\hline\sphinxstartmulticolumn{4}%
\begin{varwidth}[t]{\sphinxcolwidth{4}{4}}
\sphinxAtStartPar
properties
\par
\vskip-\baselineskip\vbox{\hbox{\strut}}\end{varwidth}%
\sphinxstopmulticolumn
\\
\hline\sphinxmultirow{4}{4}{%
\begin{varwidth}[t]{\sphinxcolwidth{1}{4}}
\begin{itemize}
\item {} 
\sphinxAtStartPar
immediately

\end{itemize}
\par
\vskip-\baselineskip\vbox{\hbox{\strut}}\end{varwidth}%
}%
&\sphinxstartmulticolumn{3}%
\begin{varwidth}[t]{\sphinxcolwidth{3}{4}}
\sphinxAtStartPar
Whether to animate to the new arrangement.
\par
\vskip-\baselineskip\vbox{\hbox{\strut}}\end{varwidth}%
\sphinxstopmulticolumn
\\
\cline{2-4}\sphinxtablestrut{4}&
\sphinxAtStartPar
default
&\sphinxstartmulticolumn{2}%
\begin{varwidth}[t]{\sphinxcolwidth{2}{4}}
\sphinxAtStartPar
False
\par
\vskip-\baselineskip\vbox{\hbox{\strut}}\end{varwidth}%
\sphinxstopmulticolumn
\\
\cline{2-4}\sphinxtablestrut{4}&\sphinxmultirow{2}{8}{%
\begin{varwidth}[t]{\sphinxcolwidth{1}{4}}
\sphinxAtStartPar
anyOf
\par
\vskip-\baselineskip\vbox{\hbox{\strut}}\end{varwidth}%
}%
&
\sphinxAtStartPar
type
&
\sphinxAtStartPar
\sphinxstyleemphasis{boolean}
\\
\cline{3-4}\sphinxtablestrut{4}&\sphinxtablestrut{8}&
\sphinxAtStartPar
type
&
\sphinxAtStartPar
\sphinxstyleemphasis{null}
\\
\hline\sphinxmultirow{4}{13}{%
\begin{varwidth}[t]{\sphinxcolwidth{1}{4}}
\begin{itemize}
\item {} 
\sphinxAtStartPar
layout

\end{itemize}
\par
\vskip-\baselineskip\vbox{\hbox{\strut}}\end{varwidth}%
}%
&\sphinxstartmulticolumn{3}%
\begin{varwidth}[t]{\sphinxcolwidth{3}{4}}
\sphinxAtStartPar
See collectionLayout in OpenSeadragon.Options.
\par
\vskip-\baselineskip\vbox{\hbox{\strut}}\end{varwidth}%
\sphinxstopmulticolumn
\\
\cline{2-4}\sphinxtablestrut{13}&
\sphinxAtStartPar
default
&\sphinxstartmulticolumn{2}%
\begin{varwidth}[t]{\sphinxcolwidth{2}{4}}
\sphinxAtStartPar
null
\par
\vskip-\baselineskip\vbox{\hbox{\strut}}\end{varwidth}%
\sphinxstopmulticolumn
\\
\cline{2-4}\sphinxtablestrut{13}&\sphinxmultirow{2}{17}{%
\begin{varwidth}[t]{\sphinxcolwidth{1}{4}}
\sphinxAtStartPar
anyOf
\par
\vskip-\baselineskip\vbox{\hbox{\strut}}\end{varwidth}%
}%
&\sphinxstartmulticolumn{2}%
\begin{varwidth}[t]{\sphinxcolwidth{2}{4}}
\sphinxAtStartPar
{\hyperref[\detokenize{docs/advanced/tmap:layoutaxis}]{\sphinxcrossref{LayoutAxis}}}
\par
\vskip-\baselineskip\vbox{\hbox{\strut}}\end{varwidth}%
\sphinxstopmulticolumn
\\
\cline{3-4}\sphinxtablestrut{13}&\sphinxtablestrut{17}&
\sphinxAtStartPar
type
&
\sphinxAtStartPar
\sphinxstyleemphasis{null}
\\
\hline\sphinxmultirow{4}{21}{%
\begin{varwidth}[t]{\sphinxcolwidth{1}{4}}
\begin{itemize}
\item {} 
\sphinxAtStartPar
rows

\end{itemize}
\par
\vskip-\baselineskip\vbox{\hbox{\strut}}\end{varwidth}%
}%
&\sphinxstartmulticolumn{3}%
\begin{varwidth}[t]{\sphinxcolwidth{3}{4}}
\sphinxAtStartPar
See collectionRows in OpenSeadragon.Options.
\par
\vskip-\baselineskip\vbox{\hbox{\strut}}\end{varwidth}%
\sphinxstopmulticolumn
\\
\cline{2-4}\sphinxtablestrut{21}&
\sphinxAtStartPar
default
&\sphinxstartmulticolumn{2}%
\begin{varwidth}[t]{\sphinxcolwidth{2}{4}}
\sphinxAtStartPar
null
\par
\vskip-\baselineskip\vbox{\hbox{\strut}}\end{varwidth}%
\sphinxstopmulticolumn
\\
\cline{2-4}\sphinxtablestrut{21}&\sphinxmultirow{2}{25}{%
\begin{varwidth}[t]{\sphinxcolwidth{1}{4}}
\sphinxAtStartPar
anyOf
\par
\vskip-\baselineskip\vbox{\hbox{\strut}}\end{varwidth}%
}%
&
\sphinxAtStartPar
type
&
\sphinxAtStartPar
\sphinxstyleemphasis{integer}
\\
\cline{3-4}\sphinxtablestrut{21}&\sphinxtablestrut{25}&
\sphinxAtStartPar
type
&
\sphinxAtStartPar
\sphinxstyleemphasis{null}
\\
\hline\sphinxmultirow{4}{30}{%
\begin{varwidth}[t]{\sphinxcolwidth{1}{4}}
\begin{itemize}
\item {} 
\sphinxAtStartPar
columns

\end{itemize}
\par
\vskip-\baselineskip\vbox{\hbox{\strut}}\end{varwidth}%
}%
&\sphinxstartmulticolumn{3}%
\begin{varwidth}[t]{\sphinxcolwidth{3}{4}}
\sphinxAtStartPar
See collectionColumns in OpenSeadragon.Options.
\par
\vskip-\baselineskip\vbox{\hbox{\strut}}\end{varwidth}%
\sphinxstopmulticolumn
\\
\cline{2-4}\sphinxtablestrut{30}&
\sphinxAtStartPar
default
&\sphinxstartmulticolumn{2}%
\begin{varwidth}[t]{\sphinxcolwidth{2}{4}}
\sphinxAtStartPar
null
\par
\vskip-\baselineskip\vbox{\hbox{\strut}}\end{varwidth}%
\sphinxstopmulticolumn
\\
\cline{2-4}\sphinxtablestrut{30}&\sphinxmultirow{2}{34}{%
\begin{varwidth}[t]{\sphinxcolwidth{1}{4}}
\sphinxAtStartPar
anyOf
\par
\vskip-\baselineskip\vbox{\hbox{\strut}}\end{varwidth}%
}%
&
\sphinxAtStartPar
type
&
\sphinxAtStartPar
\sphinxstyleemphasis{integer}
\\
\cline{3-4}\sphinxtablestrut{30}&\sphinxtablestrut{34}&
\sphinxAtStartPar
type
&
\sphinxAtStartPar
\sphinxstyleemphasis{null}
\\
\hline\sphinxmultirow{4}{39}{%
\begin{varwidth}[t]{\sphinxcolwidth{1}{4}}
\begin{itemize}
\item {} 
\sphinxAtStartPar
tileSize

\end{itemize}
\par
\vskip-\baselineskip\vbox{\hbox{\strut}}\end{varwidth}%
}%
&\sphinxstartmulticolumn{3}%
\begin{varwidth}[t]{\sphinxcolwidth{3}{4}}
\sphinxAtStartPar
See collectionTileSize in OpenSeadragon.Options.
\par
\vskip-\baselineskip\vbox{\hbox{\strut}}\end{varwidth}%
\sphinxstopmulticolumn
\\
\cline{2-4}\sphinxtablestrut{39}&
\sphinxAtStartPar
default
&\sphinxstartmulticolumn{2}%
\begin{varwidth}[t]{\sphinxcolwidth{2}{4}}
\sphinxAtStartPar
null
\par
\vskip-\baselineskip\vbox{\hbox{\strut}}\end{varwidth}%
\sphinxstopmulticolumn
\\
\cline{2-4}\sphinxtablestrut{39}&\sphinxmultirow{2}{43}{%
\begin{varwidth}[t]{\sphinxcolwidth{1}{4}}
\sphinxAtStartPar
anyOf
\par
\vskip-\baselineskip\vbox{\hbox{\strut}}\end{varwidth}%
}%
&
\sphinxAtStartPar
type
&
\sphinxAtStartPar
\sphinxstyleemphasis{number}
\\
\cline{3-4}\sphinxtablestrut{39}&\sphinxtablestrut{43}&
\sphinxAtStartPar
type
&
\sphinxAtStartPar
\sphinxstyleemphasis{null}
\\
\hline\sphinxmultirow{4}{48}{%
\begin{varwidth}[t]{\sphinxcolwidth{1}{4}}
\begin{itemize}
\item {} 
\sphinxAtStartPar
tileMargin

\end{itemize}
\par
\vskip-\baselineskip\vbox{\hbox{\strut}}\end{varwidth}%
}%
&\sphinxstartmulticolumn{3}%
\begin{varwidth}[t]{\sphinxcolwidth{3}{4}}
\sphinxAtStartPar
See collectionTileMargin in OpenSeadragon.Options.
\par
\vskip-\baselineskip\vbox{\hbox{\strut}}\end{varwidth}%
\sphinxstopmulticolumn
\\
\cline{2-4}\sphinxtablestrut{48}&
\sphinxAtStartPar
default
&\sphinxstartmulticolumn{2}%
\begin{varwidth}[t]{\sphinxcolwidth{2}{4}}
\sphinxAtStartPar
null
\par
\vskip-\baselineskip\vbox{\hbox{\strut}}\end{varwidth}%
\sphinxstopmulticolumn
\\
\cline{2-4}\sphinxtablestrut{48}&\sphinxmultirow{2}{52}{%
\begin{varwidth}[t]{\sphinxcolwidth{1}{4}}
\sphinxAtStartPar
anyOf
\par
\vskip-\baselineskip\vbox{\hbox{\strut}}\end{varwidth}%
}%
&
\sphinxAtStartPar
type
&
\sphinxAtStartPar
\sphinxstyleemphasis{number}
\\
\cline{3-4}\sphinxtablestrut{48}&\sphinxtablestrut{52}&
\sphinxAtStartPar
type
&
\sphinxAtStartPar
\sphinxstyleemphasis{null}
\\
\hline
\sphinxAtStartPar
additionalProperties
&\sphinxstartmulticolumn{3}%
\begin{varwidth}[t]{\sphinxcolwidth{3}{4}}
\sphinxAtStartPar
False
\par
\vskip-\baselineskip\vbox{\hbox{\strut}}\end{varwidth}%
\sphinxstopmulticolumn
\\
\hline
\end{tabular}
\par
\sphinxattableend\end{savenotes}


\subsection{CompositeMode}
\label{\detokenize{docs/advanced/tmap:compositemode}}\label{\detokenize{docs/advanced/tmap:compositemode}}

\begin{savenotes}\sphinxattablestart
\centering
\begin{tabulary}{\linewidth}[t]{|T|T|}
\hline

\sphinxAtStartPar
type
&
\sphinxAtStartPar
\sphinxstyleemphasis{string}
\\
\hline
\sphinxAtStartPar
enum
&
\sphinxAtStartPar
source\sphinxhyphen{}over, lighter, darken, source\sphinxhyphen{}atop, source\sphinxhyphen{}in, source\sphinxhyphen{}out, destination\sphinxhyphen{}over, destination\sphinxhyphen{}atop, destination\sphinxhyphen{}in, destination\sphinxhyphen{}out, copy, xor, multiply, screen, overlay, color\sphinxhyphen{}dodge, color\sphinxhyphen{}burn, hard\sphinxhyphen{}light, soft\sphinxhyphen{}light, difference, exclusion, hue, saturation, color, luminosity
\\
\hline
\end{tabulary}
\par
\sphinxattableend\end{savenotes}


\subsection{DropdownOption}
\label{\detokenize{docs/advanced/tmap:dropdownoption}}\label{\detokenize{docs/advanced/tmap:dropdownoption}}

\begin{savenotes}\sphinxattablestart
\centering
\begin{tabular}[t]{|*{3}{\X{1}{3}|}}
\hline

\sphinxAtStartPar
type
&\sphinxstartmulticolumn{2}%
\begin{varwidth}[t]{\sphinxcolwidth{2}{3}}
\sphinxAtStartPar
\sphinxstyleemphasis{object}
\par
\vskip-\baselineskip\vbox{\hbox{\strut}}\end{varwidth}%
\sphinxstopmulticolumn
\\
\hline\sphinxstartmulticolumn{3}%
\begin{varwidth}[t]{\sphinxcolwidth{3}{3}}
\sphinxAtStartPar
properties
\par
\vskip-\baselineskip\vbox{\hbox{\strut}}\end{varwidth}%
\sphinxstopmulticolumn
\\
\hline\sphinxmultirow{2}{4}{%
\begin{varwidth}[t]{\sphinxcolwidth{1}{3}}
\begin{itemize}
\item {} 
\sphinxAtStartPar
\sphinxstylestrong{optionName}

\end{itemize}
\par
\vskip-\baselineskip\vbox{\hbox{\strut}}\end{varwidth}%
}%
&\sphinxstartmulticolumn{2}%
\begin{varwidth}[t]{\sphinxcolwidth{2}{3}}
\sphinxAtStartPar
Name displayed in the dropdown menu.
\par
\vskip-\baselineskip\vbox{\hbox{\strut}}\end{varwidth}%
\sphinxstopmulticolumn
\\
\cline{2-3}\sphinxtablestrut{4}&
\sphinxAtStartPar
type
&
\sphinxAtStartPar
\sphinxstyleemphasis{string}
\\
\hline\sphinxmultirow{2}{8}{%
\begin{varwidth}[t]{\sphinxcolwidth{1}{3}}
\begin{itemize}
\item {} 
\sphinxAtStartPar
\sphinxstylestrong{name}

\end{itemize}
\par
\vskip-\baselineskip\vbox{\hbox{\strut}}\end{varwidth}%
}%
&\sphinxstartmulticolumn{2}%
\begin{varwidth}[t]{\sphinxcolwidth{2}{3}}
\sphinxAtStartPar
Name of the tab to be loaded when the option is selected.
\par
\vskip-\baselineskip\vbox{\hbox{\strut}}\end{varwidth}%
\sphinxstopmulticolumn
\\
\cline{2-3}\sphinxtablestrut{8}&
\sphinxAtStartPar
type
&
\sphinxAtStartPar
\sphinxstyleemphasis{string}
\\
\hline
\sphinxAtStartPar
additionalProperties
&\sphinxstartmulticolumn{2}%
\begin{varwidth}[t]{\sphinxcolwidth{2}{3}}
\sphinxAtStartPar
True
\par
\vskip-\baselineskip\vbox{\hbox{\strut}}\end{varwidth}%
\sphinxstopmulticolumn
\\
\hline
\end{tabular}
\par
\sphinxattableend\end{savenotes}


\subsection{ExpectedHeader}
\label{\detokenize{docs/advanced/tmap:expectedheader}}\label{\detokenize{docs/advanced/tmap:expectedheader}}

\begin{savenotes}\sphinxatlongtablestart\begin{longtable}[c]{|*{5}{\X{1}{5}|}}
\hline

\endfirsthead

\multicolumn{5}{c}%
{\makebox[0pt]{\sphinxtablecontinued{\tablename\ \thetable{} \textendash{} continued from previous page}}}\\
\hline

\endhead

\hline
\multicolumn{5}{r}{\makebox[0pt][r]{\sphinxtablecontinued{continues on next page}}}\\
\endfoot

\endlastfoot

\sphinxAtStartPar
type
&\sphinxstartmulticolumn{4}%
\begin{varwidth}[t]{\sphinxcolwidth{4}{5}}
\sphinxAtStartPar
\sphinxstyleemphasis{object}
\par
\vskip-\baselineskip\vbox{\hbox{\strut}}\end{varwidth}%
\sphinxstopmulticolumn
\\
\hline\sphinxstartmulticolumn{5}%
\begin{varwidth}[t]{\sphinxcolwidth{5}{5}}
\sphinxAtStartPar
properties
\par
\vskip-\baselineskip\vbox{\hbox{\strut}}\end{varwidth}%
\sphinxstopmulticolumn
\\
\hline\sphinxmultirow{2}{4}{%
\begin{varwidth}[t]{\sphinxcolwidth{1}{5}}
\begin{itemize}
\item {} 
\sphinxAtStartPar
\sphinxstylestrong{X}

\end{itemize}
\par
\vskip-\baselineskip\vbox{\hbox{\strut}}\end{varwidth}%
}%
&\sphinxstartmulticolumn{4}%
\begin{varwidth}[t]{\sphinxcolwidth{4}{5}}
\sphinxAtStartPar
Name of CSV column to use as X\sphinxhyphen{}coordinate.
\par
\vskip-\baselineskip\vbox{\hbox{\strut}}\end{varwidth}%
\sphinxstopmulticolumn
\\
\cline{2-5}\sphinxtablestrut{4}&
\sphinxAtStartPar
type
&\sphinxstartmulticolumn{3}%
\begin{varwidth}[t]{\sphinxcolwidth{3}{5}}
\sphinxAtStartPar
\sphinxstyleemphasis{string}
\par
\vskip-\baselineskip\vbox{\hbox{\strut}}\end{varwidth}%
\sphinxstopmulticolumn
\\
\hline\sphinxmultirow{2}{8}{%
\begin{varwidth}[t]{\sphinxcolwidth{1}{5}}
\begin{itemize}
\item {} 
\sphinxAtStartPar
\sphinxstylestrong{Y}

\end{itemize}
\par
\vskip-\baselineskip\vbox{\hbox{\strut}}\end{varwidth}%
}%
&\sphinxstartmulticolumn{4}%
\begin{varwidth}[t]{\sphinxcolwidth{4}{5}}
\sphinxAtStartPar
Name of CSV column to use as Y\sphinxhyphen{}coordinate.
\par
\vskip-\baselineskip\vbox{\hbox{\strut}}\end{varwidth}%
\sphinxstopmulticolumn
\\
\cline{2-5}\sphinxtablestrut{8}&
\sphinxAtStartPar
type
&\sphinxstartmulticolumn{3}%
\begin{varwidth}[t]{\sphinxcolwidth{3}{5}}
\sphinxAtStartPar
\sphinxstyleemphasis{string}
\par
\vskip-\baselineskip\vbox{\hbox{\strut}}\end{varwidth}%
\sphinxstopmulticolumn
\\
\hline\sphinxmultirow{4}{12}{%
\begin{varwidth}[t]{\sphinxcolwidth{1}{5}}
\begin{itemize}
\item {} 
\sphinxAtStartPar
gb\_col

\end{itemize}
\par
\vskip-\baselineskip\vbox{\hbox{\strut}}\end{varwidth}%
}%
&\sphinxstartmulticolumn{4}%
\begin{varwidth}[t]{\sphinxcolwidth{4}{5}}
\sphinxAtStartPar
Name of CSV column to use as key to group markers by.
\par
\vskip-\baselineskip\vbox{\hbox{\strut}}\end{varwidth}%
\sphinxstopmulticolumn
\\
\cline{2-5}\sphinxtablestrut{12}&
\sphinxAtStartPar
default
&\sphinxstartmulticolumn{3}%
\begin{varwidth}[t]{\sphinxcolwidth{3}{5}}
\sphinxAtStartPar
null
\par
\vskip-\baselineskip\vbox{\hbox{\strut}}\end{varwidth}%
\sphinxstopmulticolumn
\\
\cline{2-5}\sphinxtablestrut{12}&\sphinxmultirow{2}{16}{%
\begin{varwidth}[t]{\sphinxcolwidth{1}{5}}
\sphinxAtStartPar
anyOf
\par
\vskip-\baselineskip\vbox{\hbox{\strut}}\end{varwidth}%
}%
&
\sphinxAtStartPar
type
&\sphinxstartmulticolumn{2}%
\begin{varwidth}[t]{\sphinxcolwidth{2}{5}}
\sphinxAtStartPar
\sphinxstyleemphasis{string}
\par
\vskip-\baselineskip\vbox{\hbox{\strut}}\end{varwidth}%
\sphinxstopmulticolumn
\\
\cline{3-5}\sphinxtablestrut{12}&\sphinxtablestrut{16}&
\sphinxAtStartPar
type
&\sphinxstartmulticolumn{2}%
\begin{varwidth}[t]{\sphinxcolwidth{2}{5}}
\sphinxAtStartPar
\sphinxstyleemphasis{null}
\par
\vskip-\baselineskip\vbox{\hbox{\strut}}\end{varwidth}%
\sphinxstopmulticolumn
\\
\hline\sphinxmultirow{4}{21}{%
\begin{varwidth}[t]{\sphinxcolwidth{1}{5}}
\begin{itemize}
\item {} 
\sphinxAtStartPar
gb\_name

\end{itemize}
\par
\vskip-\baselineskip\vbox{\hbox{\strut}}\end{varwidth}%
}%
&\sphinxstartmulticolumn{4}%
\begin{varwidth}[t]{\sphinxcolwidth{4}{5}}
\sphinxAtStartPar
Name of CSV column to display for groups instead of group key.
\par
\vskip-\baselineskip\vbox{\hbox{\strut}}\end{varwidth}%
\sphinxstopmulticolumn
\\
\cline{2-5}\sphinxtablestrut{21}&
\sphinxAtStartPar
default
&\sphinxstartmulticolumn{3}%
\begin{varwidth}[t]{\sphinxcolwidth{3}{5}}
\sphinxAtStartPar
null
\par
\vskip-\baselineskip\vbox{\hbox{\strut}}\end{varwidth}%
\sphinxstopmulticolumn
\\
\cline{2-5}\sphinxtablestrut{21}&\sphinxmultirow{2}{25}{%
\begin{varwidth}[t]{\sphinxcolwidth{1}{5}}
\sphinxAtStartPar
anyOf
\par
\vskip-\baselineskip\vbox{\hbox{\strut}}\end{varwidth}%
}%
&
\sphinxAtStartPar
type
&\sphinxstartmulticolumn{2}%
\begin{varwidth}[t]{\sphinxcolwidth{2}{5}}
\sphinxAtStartPar
\sphinxstyleemphasis{string}
\par
\vskip-\baselineskip\vbox{\hbox{\strut}}\end{varwidth}%
\sphinxstopmulticolumn
\\
\cline{3-5}\sphinxtablestrut{21}&\sphinxtablestrut{25}&
\sphinxAtStartPar
type
&\sphinxstartmulticolumn{2}%
\begin{varwidth}[t]{\sphinxcolwidth{2}{5}}
\sphinxAtStartPar
\sphinxstyleemphasis{null}
\par
\vskip-\baselineskip\vbox{\hbox{\strut}}\end{varwidth}%
\sphinxstopmulticolumn
\\
\hline\sphinxmultirow{4}{30}{%
\begin{varwidth}[t]{\sphinxcolwidth{1}{5}}
\begin{itemize}
\item {} 
\sphinxAtStartPar
cb\_cmap

\end{itemize}
\par
\vskip-\baselineskip\vbox{\hbox{\strut}}\end{varwidth}%
}%
&\sphinxstartmulticolumn{4}%
\begin{varwidth}[t]{\sphinxcolwidth{4}{5}}
\sphinxAtStartPar
Name of D3 color scale to be used for color mapping.
\par
\vskip-\baselineskip\vbox{\hbox{\strut}}\end{varwidth}%
\sphinxstopmulticolumn
\\
\cline{2-5}\sphinxtablestrut{30}&
\sphinxAtStartPar
default
&\sphinxstartmulticolumn{3}%
\begin{varwidth}[t]{\sphinxcolwidth{3}{5}}
\sphinxAtStartPar
null
\par
\vskip-\baselineskip\vbox{\hbox{\strut}}\end{varwidth}%
\sphinxstopmulticolumn
\\
\cline{2-5}\sphinxtablestrut{30}&\sphinxmultirow{2}{34}{%
\begin{varwidth}[t]{\sphinxcolwidth{1}{5}}
\sphinxAtStartPar
anyOf
\par
\vskip-\baselineskip\vbox{\hbox{\strut}}\end{varwidth}%
}%
&
\sphinxAtStartPar
type
&\sphinxstartmulticolumn{2}%
\begin{varwidth}[t]{\sphinxcolwidth{2}{5}}
\sphinxAtStartPar
\sphinxstyleemphasis{string}
\par
\vskip-\baselineskip\vbox{\hbox{\strut}}\end{varwidth}%
\sphinxstopmulticolumn
\\
\cline{3-5}\sphinxtablestrut{30}&\sphinxtablestrut{34}&
\sphinxAtStartPar
type
&\sphinxstartmulticolumn{2}%
\begin{varwidth}[t]{\sphinxcolwidth{2}{5}}
\sphinxAtStartPar
\sphinxstyleemphasis{null}
\par
\vskip-\baselineskip\vbox{\hbox{\strut}}\end{varwidth}%
\sphinxstopmulticolumn
\\
\hline\sphinxmultirow{4}{39}{%
\begin{varwidth}[t]{\sphinxcolwidth{1}{5}}
\begin{itemize}
\item {} 
\sphinxAtStartPar
cb\_col

\end{itemize}
\par
\vskip-\baselineskip\vbox{\hbox{\strut}}\end{varwidth}%
}%
&\sphinxstartmulticolumn{4}%
\begin{varwidth}[t]{\sphinxcolwidth{4}{5}}
\sphinxAtStartPar
Name of CSV column containing scalar values for color mapping or hexadecimal RGB colors in format ‘\#ff0000’.
\par
\vskip-\baselineskip\vbox{\hbox{\strut}}\end{varwidth}%
\sphinxstopmulticolumn
\\
\cline{2-5}\sphinxtablestrut{39}&
\sphinxAtStartPar
default
&\sphinxstartmulticolumn{3}%
\begin{varwidth}[t]{\sphinxcolwidth{3}{5}}
\sphinxAtStartPar
null
\par
\vskip-\baselineskip\vbox{\hbox{\strut}}\end{varwidth}%
\sphinxstopmulticolumn
\\
\cline{2-5}\sphinxtablestrut{39}&\sphinxmultirow{2}{43}{%
\begin{varwidth}[t]{\sphinxcolwidth{1}{5}}
\sphinxAtStartPar
anyOf
\par
\vskip-\baselineskip\vbox{\hbox{\strut}}\end{varwidth}%
}%
&
\sphinxAtStartPar
type
&\sphinxstartmulticolumn{2}%
\begin{varwidth}[t]{\sphinxcolwidth{2}{5}}
\sphinxAtStartPar
\sphinxstyleemphasis{string}
\par
\vskip-\baselineskip\vbox{\hbox{\strut}}\end{varwidth}%
\sphinxstopmulticolumn
\\
\cline{3-5}\sphinxtablestrut{39}&\sphinxtablestrut{43}&
\sphinxAtStartPar
type
&\sphinxstartmulticolumn{2}%
\begin{varwidth}[t]{\sphinxcolwidth{2}{5}}
\sphinxAtStartPar
\sphinxstyleemphasis{null}
\par
\vskip-\baselineskip\vbox{\hbox{\strut}}\end{varwidth}%
\sphinxstopmulticolumn
\\
\hline\sphinxmultirow{6}{48}{%
\begin{varwidth}[t]{\sphinxcolwidth{1}{5}}
\begin{itemize}
\item {} 
\sphinxAtStartPar
cb\_gr\_dict

\end{itemize}
\par
\vskip-\baselineskip\vbox{\hbox{\strut}}\end{varwidth}%
}%
&\sphinxstartmulticolumn{4}%
\begin{varwidth}[t]{\sphinxcolwidth{4}{5}}
\sphinxAtStartPar
JSON string specifying a custom dictionary for mapping group keys to group colors. Example: \sphinxcode{\sphinxupquote{"\{\textquotesingle{}key1\textquotesingle{}: \textquotesingle{}\#ff0000\textquotesingle{}, \textquotesingle{}key2\textquotesingle{}: \textquotesingle{}\#00ff00\textquotesingle{}, \textquotesingle{}key3\textquotesingle{}: \textquotesingle{}\#0000ff\textquotesingle{}\}"}}.
\par
\vskip-\baselineskip\vbox{\hbox{\strut}}\end{varwidth}%
\sphinxstopmulticolumn
\\
\cline{2-5}\sphinxtablestrut{48}&
\sphinxAtStartPar
default
&\sphinxstartmulticolumn{3}%
\begin{varwidth}[t]{\sphinxcolwidth{3}{5}}
\par
\vskip-\baselineskip\vbox{\hbox{\strut}}\end{varwidth}%
\sphinxstopmulticolumn
\\
\cline{2-5}\sphinxtablestrut{48}&\sphinxmultirow{4}{52}{%
\begin{varwidth}[t]{\sphinxcolwidth{1}{5}}
\sphinxAtStartPar
anyOf
\par
\vskip-\baselineskip\vbox{\hbox{\strut}}\end{varwidth}%
}%
&
\sphinxAtStartPar
type
&\sphinxstartmulticolumn{2}%
\begin{varwidth}[t]{\sphinxcolwidth{2}{5}}
\sphinxAtStartPar
\sphinxstyleemphasis{string}
\par
\vskip-\baselineskip\vbox{\hbox{\strut}}\end{varwidth}%
\sphinxstopmulticolumn
\\
\cline{3-5}\sphinxtablestrut{48}&\sphinxtablestrut{52}&
\sphinxAtStartPar
type
&\sphinxstartmulticolumn{2}%
\begin{varwidth}[t]{\sphinxcolwidth{2}{5}}
\sphinxAtStartPar
\sphinxstyleemphasis{object}
\par
\vskip-\baselineskip\vbox{\hbox{\strut}}\end{varwidth}%
\sphinxstopmulticolumn
\\
\cline{3-5}\sphinxtablestrut{48}&\sphinxtablestrut{52}&
\sphinxAtStartPar
type
&\sphinxstartmulticolumn{2}%
\begin{varwidth}[t]{\sphinxcolwidth{2}{5}}
\sphinxAtStartPar
\sphinxstyleemphasis{array}
\par
\vskip-\baselineskip\vbox{\hbox{\strut}}\end{varwidth}%
\sphinxstopmulticolumn
\\
\cline{3-5}\sphinxtablestrut{48}&\sphinxtablestrut{52}&
\sphinxAtStartPar
items
&
\sphinxAtStartPar
type
&
\sphinxAtStartPar
\sphinxstyleemphasis{string}
\\
\hline\sphinxmultirow{4}{62}{%
\begin{varwidth}[t]{\sphinxcolwidth{1}{5}}
\begin{itemize}
\item {} 
\sphinxAtStartPar
scale\_col

\end{itemize}
\par
\vskip-\baselineskip\vbox{\hbox{\strut}}\end{varwidth}%
}%
&\sphinxstartmulticolumn{4}%
\begin{varwidth}[t]{\sphinxcolwidth{4}{5}}
\sphinxAtStartPar
Name of CSV column containing scalar values for changing the size of markers.
\par
\vskip-\baselineskip\vbox{\hbox{\strut}}\end{varwidth}%
\sphinxstopmulticolumn
\\
\cline{2-5}\sphinxtablestrut{62}&
\sphinxAtStartPar
default
&\sphinxstartmulticolumn{3}%
\begin{varwidth}[t]{\sphinxcolwidth{3}{5}}
\sphinxAtStartPar
null
\par
\vskip-\baselineskip\vbox{\hbox{\strut}}\end{varwidth}%
\sphinxstopmulticolumn
\\
\cline{2-5}\sphinxtablestrut{62}&\sphinxmultirow{2}{66}{%
\begin{varwidth}[t]{\sphinxcolwidth{1}{5}}
\sphinxAtStartPar
anyOf
\par
\vskip-\baselineskip\vbox{\hbox{\strut}}\end{varwidth}%
}%
&
\sphinxAtStartPar
type
&\sphinxstartmulticolumn{2}%
\begin{varwidth}[t]{\sphinxcolwidth{2}{5}}
\sphinxAtStartPar
\sphinxstyleemphasis{string}
\par
\vskip-\baselineskip\vbox{\hbox{\strut}}\end{varwidth}%
\sphinxstopmulticolumn
\\
\cline{3-5}\sphinxtablestrut{62}&\sphinxtablestrut{66}&
\sphinxAtStartPar
type
&\sphinxstartmulticolumn{2}%
\begin{varwidth}[t]{\sphinxcolwidth{2}{5}}
\sphinxAtStartPar
\sphinxstyleemphasis{null}
\par
\vskip-\baselineskip\vbox{\hbox{\strut}}\end{varwidth}%
\sphinxstopmulticolumn
\\
\hline\sphinxmultirow{3}{71}{%
\begin{varwidth}[t]{\sphinxcolwidth{1}{5}}
\begin{itemize}
\item {} 
\sphinxAtStartPar
scale\_factor

\end{itemize}
\par
\vskip-\baselineskip\vbox{\hbox{\strut}}\end{varwidth}%
}%
&\sphinxstartmulticolumn{4}%
\begin{varwidth}[t]{\sphinxcolwidth{4}{5}}
\sphinxAtStartPar
Numerical value for a fixed scale factor to be applied to markers.
\par
\vskip-\baselineskip\vbox{\hbox{\strut}}\end{varwidth}%
\sphinxstopmulticolumn
\\
\cline{2-5}\sphinxtablestrut{71}&
\sphinxAtStartPar
type
&\sphinxstartmulticolumn{3}%
\begin{varwidth}[t]{\sphinxcolwidth{3}{5}}
\sphinxAtStartPar
\sphinxstyleemphasis{number}
\par
\vskip-\baselineskip\vbox{\hbox{\strut}}\end{varwidth}%
\sphinxstopmulticolumn
\\
\cline{2-5}\sphinxtablestrut{71}&
\sphinxAtStartPar
default
&\sphinxstartmulticolumn{3}%
\begin{varwidth}[t]{\sphinxcolwidth{3}{5}}
\sphinxAtStartPar
1.0
\par
\vskip-\baselineskip\vbox{\hbox{\strut}}\end{varwidth}%
\sphinxstopmulticolumn
\\
\hline\sphinxmultirow{3}{77}{%
\begin{varwidth}[t]{\sphinxcolwidth{1}{5}}
\begin{itemize}
\item {} 
\sphinxAtStartPar
coord\_factor

\end{itemize}
\par
\vskip-\baselineskip\vbox{\hbox{\strut}}\end{varwidth}%
}%
&\sphinxstartmulticolumn{4}%
\begin{varwidth}[t]{\sphinxcolwidth{4}{5}}
\sphinxAtStartPar
Numerical value for a fixed scale factor to be applied to marker coordinates.
\par
\vskip-\baselineskip\vbox{\hbox{\strut}}\end{varwidth}%
\sphinxstopmulticolumn
\\
\cline{2-5}\sphinxtablestrut{77}&
\sphinxAtStartPar
type
&\sphinxstartmulticolumn{3}%
\begin{varwidth}[t]{\sphinxcolwidth{3}{5}}
\sphinxAtStartPar
\sphinxstyleemphasis{number}
\par
\vskip-\baselineskip\vbox{\hbox{\strut}}\end{varwidth}%
\sphinxstopmulticolumn
\\
\cline{2-5}\sphinxtablestrut{77}&
\sphinxAtStartPar
default
&\sphinxstartmulticolumn{3}%
\begin{varwidth}[t]{\sphinxcolwidth{3}{5}}
\sphinxAtStartPar
1.0
\par
\vskip-\baselineskip\vbox{\hbox{\strut}}\end{varwidth}%
\sphinxstopmulticolumn
\\
\hline\sphinxmultirow{4}{83}{%
\begin{varwidth}[t]{\sphinxcolwidth{1}{5}}
\begin{itemize}
\item {} 
\sphinxAtStartPar
pie\_col

\end{itemize}
\par
\vskip-\baselineskip\vbox{\hbox{\strut}}\end{varwidth}%
}%
&\sphinxstartmulticolumn{4}%
\begin{varwidth}[t]{\sphinxcolwidth{4}{5}}
\sphinxAtStartPar
Name of CSV column containing data for pie chart sectors. TissUUmaps expects labels and numerical values for sectors to be separated by ‘:’ characters in the CSV column data.
\par
\vskip-\baselineskip\vbox{\hbox{\strut}}\end{varwidth}%
\sphinxstopmulticolumn
\\
\cline{2-5}\sphinxtablestrut{83}&
\sphinxAtStartPar
default
&\sphinxstartmulticolumn{3}%
\begin{varwidth}[t]{\sphinxcolwidth{3}{5}}
\sphinxAtStartPar
null
\par
\vskip-\baselineskip\vbox{\hbox{\strut}}\end{varwidth}%
\sphinxstopmulticolumn
\\
\cline{2-5}\sphinxtablestrut{83}&\sphinxmultirow{2}{87}{%
\begin{varwidth}[t]{\sphinxcolwidth{1}{5}}
\sphinxAtStartPar
anyOf
\par
\vskip-\baselineskip\vbox{\hbox{\strut}}\end{varwidth}%
}%
&
\sphinxAtStartPar
type
&\sphinxstartmulticolumn{2}%
\begin{varwidth}[t]{\sphinxcolwidth{2}{5}}
\sphinxAtStartPar
\sphinxstyleemphasis{string}
\par
\vskip-\baselineskip\vbox{\hbox{\strut}}\end{varwidth}%
\sphinxstopmulticolumn
\\
\cline{3-5}\sphinxtablestrut{83}&\sphinxtablestrut{87}&
\sphinxAtStartPar
type
&\sphinxstartmulticolumn{2}%
\begin{varwidth}[t]{\sphinxcolwidth{2}{5}}
\sphinxAtStartPar
\sphinxstyleemphasis{null}
\par
\vskip-\baselineskip\vbox{\hbox{\strut}}\end{varwidth}%
\sphinxstopmulticolumn
\\
\hline\sphinxmultirow{6}{92}{%
\begin{varwidth}[t]{\sphinxcolwidth{1}{5}}
\begin{itemize}
\item {} 
\sphinxAtStartPar
pie\_dict

\end{itemize}
\par
\vskip-\baselineskip\vbox{\hbox{\strut}}\end{varwidth}%
}%
&\sphinxstartmulticolumn{4}%
\begin{varwidth}[t]{\sphinxcolwidth{4}{5}}
\sphinxAtStartPar
JSON string specifying a custom dictionary for mapping pie chart sector indices to colors. Example: \sphinxcode{\sphinxupquote{"\{0: \textquotesingle{}\#ff0000\textquotesingle{}, 1: \textquotesingle{}\#00ff00\textquotesingle{}, 2: \textquotesingle{}\#0000ff\textquotesingle{}\}"}}. If no dictionary is specified, TissUUmaps will use a default color palette instead.
\par
\vskip-\baselineskip\vbox{\hbox{\strut}}\end{varwidth}%
\sphinxstopmulticolumn
\\
\cline{2-5}\sphinxtablestrut{92}&
\sphinxAtStartPar
default
&\sphinxstartmulticolumn{3}%
\begin{varwidth}[t]{\sphinxcolwidth{3}{5}}
\par
\vskip-\baselineskip\vbox{\hbox{\strut}}\end{varwidth}%
\sphinxstopmulticolumn
\\
\cline{2-5}\sphinxtablestrut{92}&\sphinxmultirow{4}{96}{%
\begin{varwidth}[t]{\sphinxcolwidth{1}{5}}
\sphinxAtStartPar
anyOf
\par
\vskip-\baselineskip\vbox{\hbox{\strut}}\end{varwidth}%
}%
&
\sphinxAtStartPar
type
&\sphinxstartmulticolumn{2}%
\begin{varwidth}[t]{\sphinxcolwidth{2}{5}}
\sphinxAtStartPar
\sphinxstyleemphasis{string}
\par
\vskip-\baselineskip\vbox{\hbox{\strut}}\end{varwidth}%
\sphinxstopmulticolumn
\\
\cline{3-5}\sphinxtablestrut{92}&\sphinxtablestrut{96}&
\sphinxAtStartPar
type
&\sphinxstartmulticolumn{2}%
\begin{varwidth}[t]{\sphinxcolwidth{2}{5}}
\sphinxAtStartPar
\sphinxstyleemphasis{object}
\par
\vskip-\baselineskip\vbox{\hbox{\strut}}\end{varwidth}%
\sphinxstopmulticolumn
\\
\cline{3-5}\sphinxtablestrut{92}&\sphinxtablestrut{96}&
\sphinxAtStartPar
type
&\sphinxstartmulticolumn{2}%
\begin{varwidth}[t]{\sphinxcolwidth{2}{5}}
\sphinxAtStartPar
\sphinxstyleemphasis{array}
\par
\vskip-\baselineskip\vbox{\hbox{\strut}}\end{varwidth}%
\sphinxstopmulticolumn
\\
\cline{3-5}\sphinxtablestrut{92}&\sphinxtablestrut{96}&
\sphinxAtStartPar
items
&
\sphinxAtStartPar
type
&
\sphinxAtStartPar
\sphinxstyleemphasis{string}
\\
\hline\sphinxmultirow{4}{106}{%
\begin{varwidth}[t]{\sphinxcolwidth{1}{5}}
\begin{itemize}
\item {} 
\sphinxAtStartPar
shape\_col

\end{itemize}
\par
\vskip-\baselineskip\vbox{\hbox{\strut}}\end{varwidth}%
}%
&\sphinxstartmulticolumn{4}%
\begin{varwidth}[t]{\sphinxcolwidth{4}{5}}
\sphinxAtStartPar
Name of CSV column containing a name or an index for marker shape.
\par
\vskip-\baselineskip\vbox{\hbox{\strut}}\end{varwidth}%
\sphinxstopmulticolumn
\\
\cline{2-5}\sphinxtablestrut{106}&
\sphinxAtStartPar
default
&\sphinxstartmulticolumn{3}%
\begin{varwidth}[t]{\sphinxcolwidth{3}{5}}
\sphinxAtStartPar
null
\par
\vskip-\baselineskip\vbox{\hbox{\strut}}\end{varwidth}%
\sphinxstopmulticolumn
\\
\cline{2-5}\sphinxtablestrut{106}&\sphinxmultirow{2}{110}{%
\begin{varwidth}[t]{\sphinxcolwidth{1}{5}}
\sphinxAtStartPar
anyOf
\par
\vskip-\baselineskip\vbox{\hbox{\strut}}\end{varwidth}%
}%
&
\sphinxAtStartPar
type
&\sphinxstartmulticolumn{2}%
\begin{varwidth}[t]{\sphinxcolwidth{2}{5}}
\sphinxAtStartPar
\sphinxstyleemphasis{string}
\par
\vskip-\baselineskip\vbox{\hbox{\strut}}\end{varwidth}%
\sphinxstopmulticolumn
\\
\cline{3-5}\sphinxtablestrut{106}&\sphinxtablestrut{110}&
\sphinxAtStartPar
type
&\sphinxstartmulticolumn{2}%
\begin{varwidth}[t]{\sphinxcolwidth{2}{5}}
\sphinxAtStartPar
\sphinxstyleemphasis{null}
\par
\vskip-\baselineskip\vbox{\hbox{\strut}}\end{varwidth}%
\sphinxstopmulticolumn
\\
\hline\sphinxmultirow{3}{115}{%
\begin{varwidth}[t]{\sphinxcolwidth{1}{5}}
\begin{itemize}
\item {} 
\sphinxAtStartPar
shape\_fixed

\end{itemize}
\par
\vskip-\baselineskip\vbox{\hbox{\strut}}\end{varwidth}%
}%
&\sphinxstartmulticolumn{4}%
\begin{varwidth}[t]{\sphinxcolwidth{4}{5}}
\sphinxAtStartPar
Name or index of a single fixed shape to be used for all markers.
\par
\vskip-\baselineskip\vbox{\hbox{\strut}}\end{varwidth}%
\sphinxstopmulticolumn
\\
\cline{2-5}\sphinxtablestrut{115}&
\sphinxAtStartPar
type
&\sphinxstartmulticolumn{3}%
\begin{varwidth}[t]{\sphinxcolwidth{3}{5}}
\sphinxAtStartPar
\sphinxstyleemphasis{string}
\par
\vskip-\baselineskip\vbox{\hbox{\strut}}\end{varwidth}%
\sphinxstopmulticolumn
\\
\cline{2-5}\sphinxtablestrut{115}&
\sphinxAtStartPar
default
&\sphinxstartmulticolumn{3}%
\begin{varwidth}[t]{\sphinxcolwidth{3}{5}}
\sphinxAtStartPar
cross
\par
\vskip-\baselineskip\vbox{\hbox{\strut}}\end{varwidth}%
\sphinxstopmulticolumn
\\
\hline\sphinxmultirow{6}{121}{%
\begin{varwidth}[t]{\sphinxcolwidth{1}{5}}
\begin{itemize}
\item {} 
\sphinxAtStartPar
shape\_gr\_dict

\end{itemize}
\par
\vskip-\baselineskip\vbox{\hbox{\strut}}\end{varwidth}%
}%
&\sphinxstartmulticolumn{4}%
\begin{varwidth}[t]{\sphinxcolwidth{4}{5}}
\sphinxAtStartPar
JSON string specifying a custom dictionary for mapping group keys to group shapes. Example: \sphinxcode{\sphinxupquote{"\{\textquotesingle{}key1\textquotesingle{}: \textquotesingle{}square\textquotesingle{}, \textquotesingle{}key2\textquotesingle{}: \textquotesingle{}diamond\textquotesingle{}, \textquotesingle{}key3\textquotesingle{}: \textquotesingle{}triangle up\textquotesingle{}\}"}}.
\par
\vskip-\baselineskip\vbox{\hbox{\strut}}\end{varwidth}%
\sphinxstopmulticolumn
\\
\cline{2-5}\sphinxtablestrut{121}&
\sphinxAtStartPar
default
&\sphinxstartmulticolumn{3}%
\begin{varwidth}[t]{\sphinxcolwidth{3}{5}}
\par
\vskip-\baselineskip\vbox{\hbox{\strut}}\end{varwidth}%
\sphinxstopmulticolumn
\\
\cline{2-5}\sphinxtablestrut{121}&\sphinxmultirow{4}{125}{%
\begin{varwidth}[t]{\sphinxcolwidth{1}{5}}
\sphinxAtStartPar
anyOf
\par
\vskip-\baselineskip\vbox{\hbox{\strut}}\end{varwidth}%
}%
&
\sphinxAtStartPar
type
&\sphinxstartmulticolumn{2}%
\begin{varwidth}[t]{\sphinxcolwidth{2}{5}}
\sphinxAtStartPar
\sphinxstyleemphasis{string}
\par
\vskip-\baselineskip\vbox{\hbox{\strut}}\end{varwidth}%
\sphinxstopmulticolumn
\\
\cline{3-5}\sphinxtablestrut{121}&\sphinxtablestrut{125}&
\sphinxAtStartPar
type
&\sphinxstartmulticolumn{2}%
\begin{varwidth}[t]{\sphinxcolwidth{2}{5}}
\sphinxAtStartPar
\sphinxstyleemphasis{object}
\par
\vskip-\baselineskip\vbox{\hbox{\strut}}\end{varwidth}%
\sphinxstopmulticolumn
\\
\cline{3-5}\sphinxtablestrut{121}&\sphinxtablestrut{125}&
\sphinxAtStartPar
type
&\sphinxstartmulticolumn{2}%
\begin{varwidth}[t]{\sphinxcolwidth{2}{5}}
\sphinxAtStartPar
\sphinxstyleemphasis{array}
\par
\vskip-\baselineskip\vbox{\hbox{\strut}}\end{varwidth}%
\sphinxstopmulticolumn
\\
\cline{3-5}\sphinxtablestrut{121}&\sphinxtablestrut{125}&
\sphinxAtStartPar
items
&
\sphinxAtStartPar
type
&
\sphinxAtStartPar
\sphinxstyleemphasis{string}
\\
\hline\sphinxmultirow{4}{135}{%
\begin{varwidth}[t]{\sphinxcolwidth{1}{5}}
\begin{itemize}
\item {} 
\sphinxAtStartPar
edges\_col

\end{itemize}
\par
\vskip-\baselineskip\vbox{\hbox{\strut}}\end{varwidth}%
}%
&\sphinxstartmulticolumn{4}%
\begin{varwidth}[t]{\sphinxcolwidth{4}{5}}
\sphinxAtStartPar
Name of CSV column containing a name or an index for marker edges in Network Diagram mode.
\par
\vskip-\baselineskip\vbox{\hbox{\strut}}\end{varwidth}%
\sphinxstopmulticolumn
\\
\cline{2-5}\sphinxtablestrut{135}&
\sphinxAtStartPar
default
&\sphinxstartmulticolumn{3}%
\begin{varwidth}[t]{\sphinxcolwidth{3}{5}}
\sphinxAtStartPar
null
\par
\vskip-\baselineskip\vbox{\hbox{\strut}}\end{varwidth}%
\sphinxstopmulticolumn
\\
\cline{2-5}\sphinxtablestrut{135}&\sphinxmultirow{2}{139}{%
\begin{varwidth}[t]{\sphinxcolwidth{1}{5}}
\sphinxAtStartPar
anyOf
\par
\vskip-\baselineskip\vbox{\hbox{\strut}}\end{varwidth}%
}%
&
\sphinxAtStartPar
type
&\sphinxstartmulticolumn{2}%
\begin{varwidth}[t]{\sphinxcolwidth{2}{5}}
\sphinxAtStartPar
\sphinxstyleemphasis{string}
\par
\vskip-\baselineskip\vbox{\hbox{\strut}}\end{varwidth}%
\sphinxstopmulticolumn
\\
\cline{3-5}\sphinxtablestrut{135}&\sphinxtablestrut{139}&
\sphinxAtStartPar
type
&\sphinxstartmulticolumn{2}%
\begin{varwidth}[t]{\sphinxcolwidth{2}{5}}
\sphinxAtStartPar
\sphinxstyleemphasis{null}
\par
\vskip-\baselineskip\vbox{\hbox{\strut}}\end{varwidth}%
\sphinxstopmulticolumn
\\
\hline\sphinxmultirow{4}{144}{%
\begin{varwidth}[t]{\sphinxcolwidth{1}{5}}
\begin{itemize}
\item {} 
\sphinxAtStartPar
collectionItem\_col

\end{itemize}
\par
\vskip-\baselineskip\vbox{\hbox{\strut}}\end{varwidth}%
}%
&\sphinxstartmulticolumn{4}%
\begin{varwidth}[t]{\sphinxcolwidth{4}{5}}
\sphinxAtStartPar
Name of CSV column containing a name or an index for marker collection items in Collection mode.
\par
\vskip-\baselineskip\vbox{\hbox{\strut}}\end{varwidth}%
\sphinxstopmulticolumn
\\
\cline{2-5}\sphinxtablestrut{144}&
\sphinxAtStartPar
default
&\sphinxstartmulticolumn{3}%
\begin{varwidth}[t]{\sphinxcolwidth{3}{5}}
\sphinxAtStartPar
null
\par
\vskip-\baselineskip\vbox{\hbox{\strut}}\end{varwidth}%
\sphinxstopmulticolumn
\\
\cline{2-5}\sphinxtablestrut{144}&\sphinxmultirow{2}{148}{%
\begin{varwidth}[t]{\sphinxcolwidth{1}{5}}
\sphinxAtStartPar
anyOf
\par
\vskip-\baselineskip\vbox{\hbox{\strut}}\end{varwidth}%
}%
&
\sphinxAtStartPar
type
&\sphinxstartmulticolumn{2}%
\begin{varwidth}[t]{\sphinxcolwidth{2}{5}}
\sphinxAtStartPar
\sphinxstyleemphasis{string}
\par
\vskip-\baselineskip\vbox{\hbox{\strut}}\end{varwidth}%
\sphinxstopmulticolumn
\\
\cline{3-5}\sphinxtablestrut{144}&\sphinxtablestrut{148}&
\sphinxAtStartPar
type
&\sphinxstartmulticolumn{2}%
\begin{varwidth}[t]{\sphinxcolwidth{2}{5}}
\sphinxAtStartPar
\sphinxstyleemphasis{null}
\par
\vskip-\baselineskip\vbox{\hbox{\strut}}\end{varwidth}%
\sphinxstopmulticolumn
\\
\hline\sphinxmultirow{4}{153}{%
\begin{varwidth}[t]{\sphinxcolwidth{1}{5}}
\begin{itemize}
\item {} 
\sphinxAtStartPar
collectionItem\_fixed

\end{itemize}
\par
\vskip-\baselineskip\vbox{\hbox{\strut}}\end{varwidth}%
}%
&\sphinxstartmulticolumn{4}%
\begin{varwidth}[t]{\sphinxcolwidth{4}{5}}
\sphinxAtStartPar
Name or index of a single fixed collection item to be used for all markers in Collection mode.
\par
\vskip-\baselineskip\vbox{\hbox{\strut}}\end{varwidth}%
\sphinxstopmulticolumn
\\
\cline{2-5}\sphinxtablestrut{153}&
\sphinxAtStartPar
default
&\sphinxstartmulticolumn{3}%
\begin{varwidth}[t]{\sphinxcolwidth{3}{5}}
\sphinxAtStartPar
0
\par
\vskip-\baselineskip\vbox{\hbox{\strut}}\end{varwidth}%
\sphinxstopmulticolumn
\\
\cline{2-5}\sphinxtablestrut{153}&\sphinxmultirow{2}{157}{%
\begin{varwidth}[t]{\sphinxcolwidth{1}{5}}
\sphinxAtStartPar
anyOf
\par
\vskip-\baselineskip\vbox{\hbox{\strut}}\end{varwidth}%
}%
&
\sphinxAtStartPar
type
&\sphinxstartmulticolumn{2}%
\begin{varwidth}[t]{\sphinxcolwidth{2}{5}}
\sphinxAtStartPar
\sphinxstyleemphasis{string}
\par
\vskip-\baselineskip\vbox{\hbox{\strut}}\end{varwidth}%
\sphinxstopmulticolumn
\\
\cline{3-5}\sphinxtablestrut{153}&\sphinxtablestrut{157}&
\sphinxAtStartPar
type
&\sphinxstartmulticolumn{2}%
\begin{varwidth}[t]{\sphinxcolwidth{2}{5}}
\sphinxAtStartPar
\sphinxstyleemphasis{integer}
\par
\vskip-\baselineskip\vbox{\hbox{\strut}}\end{varwidth}%
\sphinxstopmulticolumn
\\
\hline\sphinxmultirow{4}{162}{%
\begin{varwidth}[t]{\sphinxcolwidth{1}{5}}
\begin{itemize}
\item {} 
\sphinxAtStartPar
opacity\_col

\end{itemize}
\par
\vskip-\baselineskip\vbox{\hbox{\strut}}\end{varwidth}%
}%
&\sphinxstartmulticolumn{4}%
\begin{varwidth}[t]{\sphinxcolwidth{4}{5}}
\sphinxAtStartPar
Name of CSV column containing scalar values for opacities.
\par
\vskip-\baselineskip\vbox{\hbox{\strut}}\end{varwidth}%
\sphinxstopmulticolumn
\\
\cline{2-5}\sphinxtablestrut{162}&
\sphinxAtStartPar
default
&\sphinxstartmulticolumn{3}%
\begin{varwidth}[t]{\sphinxcolwidth{3}{5}}
\sphinxAtStartPar
null
\par
\vskip-\baselineskip\vbox{\hbox{\strut}}\end{varwidth}%
\sphinxstopmulticolumn
\\
\cline{2-5}\sphinxtablestrut{162}&\sphinxmultirow{2}{166}{%
\begin{varwidth}[t]{\sphinxcolwidth{1}{5}}
\sphinxAtStartPar
anyOf
\par
\vskip-\baselineskip\vbox{\hbox{\strut}}\end{varwidth}%
}%
&
\sphinxAtStartPar
type
&\sphinxstartmulticolumn{2}%
\begin{varwidth}[t]{\sphinxcolwidth{2}{5}}
\sphinxAtStartPar
\sphinxstyleemphasis{string}
\par
\vskip-\baselineskip\vbox{\hbox{\strut}}\end{varwidth}%
\sphinxstopmulticolumn
\\
\cline{3-5}\sphinxtablestrut{162}&\sphinxtablestrut{166}&
\sphinxAtStartPar
type
&\sphinxstartmulticolumn{2}%
\begin{varwidth}[t]{\sphinxcolwidth{2}{5}}
\sphinxAtStartPar
\sphinxstyleemphasis{null}
\par
\vskip-\baselineskip\vbox{\hbox{\strut}}\end{varwidth}%
\sphinxstopmulticolumn
\\
\hline\sphinxmultirow{3}{171}{%
\begin{varwidth}[t]{\sphinxcolwidth{1}{5}}
\begin{itemize}
\item {} 
\sphinxAtStartPar
opacity

\end{itemize}
\par
\vskip-\baselineskip\vbox{\hbox{\strut}}\end{varwidth}%
}%
&\sphinxstartmulticolumn{4}%
\begin{varwidth}[t]{\sphinxcolwidth{4}{5}}
\sphinxAtStartPar
Numerical value for a fixed opacity factor to be applied to markers.
\par
\vskip-\baselineskip\vbox{\hbox{\strut}}\end{varwidth}%
\sphinxstopmulticolumn
\\
\cline{2-5}\sphinxtablestrut{171}&
\sphinxAtStartPar
type
&\sphinxstartmulticolumn{3}%
\begin{varwidth}[t]{\sphinxcolwidth{3}{5}}
\sphinxAtStartPar
\sphinxstyleemphasis{number}
\par
\vskip-\baselineskip\vbox{\hbox{\strut}}\end{varwidth}%
\sphinxstopmulticolumn
\\
\cline{2-5}\sphinxtablestrut{171}&
\sphinxAtStartPar
default
&\sphinxstartmulticolumn{3}%
\begin{varwidth}[t]{\sphinxcolwidth{3}{5}}
\sphinxAtStartPar
1.0
\par
\vskip-\baselineskip\vbox{\hbox{\strut}}\end{varwidth}%
\sphinxstopmulticolumn
\\
\hline\sphinxmultirow{3}{177}{%
\begin{varwidth}[t]{\sphinxcolwidth{1}{5}}
\begin{itemize}
\item {} 
\sphinxAtStartPar
stroke\_width

\end{itemize}
\par
\vskip-\baselineskip\vbox{\hbox{\strut}}\end{varwidth}%
}%
&\sphinxstartmulticolumn{4}%
\begin{varwidth}[t]{\sphinxcolwidth{4}{5}}
\sphinxAtStartPar
Numerical value for the marker stroke width.
\par
\vskip-\baselineskip\vbox{\hbox{\strut}}\end{varwidth}%
\sphinxstopmulticolumn
\\
\cline{2-5}\sphinxtablestrut{177}&
\sphinxAtStartPar
type
&\sphinxstartmulticolumn{3}%
\begin{varwidth}[t]{\sphinxcolwidth{3}{5}}
\sphinxAtStartPar
\sphinxstyleemphasis{number}
\par
\vskip-\baselineskip\vbox{\hbox{\strut}}\end{varwidth}%
\sphinxstopmulticolumn
\\
\cline{2-5}\sphinxtablestrut{177}&
\sphinxAtStartPar
default
&\sphinxstartmulticolumn{3}%
\begin{varwidth}[t]{\sphinxcolwidth{3}{5}}
\sphinxAtStartPar
2.5
\par
\vskip-\baselineskip\vbox{\hbox{\strut}}\end{varwidth}%
\sphinxstopmulticolumn
\\
\hline\sphinxmultirow{4}{183}{%
\begin{varwidth}[t]{\sphinxcolwidth{1}{5}}
\begin{itemize}
\item {} 
\sphinxAtStartPar
sortby\_col

\end{itemize}
\par
\vskip-\baselineskip\vbox{\hbox{\strut}}\end{varwidth}%
}%
&\sphinxstartmulticolumn{4}%
\begin{varwidth}[t]{\sphinxcolwidth{4}{5}}
\sphinxAtStartPar
Name of CSV column containing scalar values for sorting markers.
\par
\vskip-\baselineskip\vbox{\hbox{\strut}}\end{varwidth}%
\sphinxstopmulticolumn
\\
\cline{2-5}\sphinxtablestrut{183}&
\sphinxAtStartPar
default
&\sphinxstartmulticolumn{3}%
\begin{varwidth}[t]{\sphinxcolwidth{3}{5}}
\sphinxAtStartPar
null
\par
\vskip-\baselineskip\vbox{\hbox{\strut}}\end{varwidth}%
\sphinxstopmulticolumn
\\
\cline{2-5}\sphinxtablestrut{183}&\sphinxmultirow{2}{187}{%
\begin{varwidth}[t]{\sphinxcolwidth{1}{5}}
\sphinxAtStartPar
anyOf
\par
\vskip-\baselineskip\vbox{\hbox{\strut}}\end{varwidth}%
}%
&
\sphinxAtStartPar
type
&\sphinxstartmulticolumn{2}%
\begin{varwidth}[t]{\sphinxcolwidth{2}{5}}
\sphinxAtStartPar
\sphinxstyleemphasis{string}
\par
\vskip-\baselineskip\vbox{\hbox{\strut}}\end{varwidth}%
\sphinxstopmulticolumn
\\
\cline{3-5}\sphinxtablestrut{183}&\sphinxtablestrut{187}&
\sphinxAtStartPar
type
&\sphinxstartmulticolumn{2}%
\begin{varwidth}[t]{\sphinxcolwidth{2}{5}}
\sphinxAtStartPar
\sphinxstyleemphasis{null}
\par
\vskip-\baselineskip\vbox{\hbox{\strut}}\end{varwidth}%
\sphinxstopmulticolumn
\\
\hline\sphinxmultirow{3}{192}{%
\begin{varwidth}[t]{\sphinxcolwidth{1}{5}}
\begin{itemize}
\item {} 
\sphinxAtStartPar
z\_order

\end{itemize}
\par
\vskip-\baselineskip\vbox{\hbox{\strut}}\end{varwidth}%
}%
&\sphinxstartmulticolumn{4}%
\begin{varwidth}[t]{\sphinxcolwidth{4}{5}}
\sphinxAtStartPar
Numerical value of z\sphinxhyphen{}order to be used for all markers.
\par
\vskip-\baselineskip\vbox{\hbox{\strut}}\end{varwidth}%
\sphinxstopmulticolumn
\\
\cline{2-5}\sphinxtablestrut{192}&
\sphinxAtStartPar
type
&\sphinxstartmulticolumn{3}%
\begin{varwidth}[t]{\sphinxcolwidth{3}{5}}
\sphinxAtStartPar
\sphinxstyleemphasis{number}
\par
\vskip-\baselineskip\vbox{\hbox{\strut}}\end{varwidth}%
\sphinxstopmulticolumn
\\
\cline{2-5}\sphinxtablestrut{192}&
\sphinxAtStartPar
default
&\sphinxstartmulticolumn{3}%
\begin{varwidth}[t]{\sphinxcolwidth{3}{5}}
\sphinxAtStartPar
1.0
\par
\vskip-\baselineskip\vbox{\hbox{\strut}}\end{varwidth}%
\sphinxstopmulticolumn
\\
\hline\sphinxmultirow{3}{198}{%
\begin{varwidth}[t]{\sphinxcolwidth{1}{5}}
\begin{itemize}
\item {} 
\sphinxAtStartPar
tooltip\_fmt

\end{itemize}
\par
\vskip-\baselineskip\vbox{\hbox{\strut}}\end{varwidth}%
}%
&\sphinxstartmulticolumn{4}%
\begin{varwidth}[t]{\sphinxcolwidth{4}{5}}
\sphinxAtStartPar
Custom formatting string used for displaying metadata about a selected marker. See \sphinxurl{https://github.com/TissUUmaps/TissUUmaps/issues/2} for an overview of the grammer and keywords. If no string is specified, TissUUmaps will show default metadata depending on the context.
\par
\vskip-\baselineskip\vbox{\hbox{\strut}}\end{varwidth}%
\sphinxstopmulticolumn
\\
\cline{2-5}\sphinxtablestrut{198}&
\sphinxAtStartPar
type
&\sphinxstartmulticolumn{3}%
\begin{varwidth}[t]{\sphinxcolwidth{3}{5}}
\sphinxAtStartPar
\sphinxstyleemphasis{string}
\par
\vskip-\baselineskip\vbox{\hbox{\strut}}\end{varwidth}%
\sphinxstopmulticolumn
\\
\cline{2-5}\sphinxtablestrut{198}&
\sphinxAtStartPar
default
&\sphinxstartmulticolumn{3}%
\begin{varwidth}[t]{\sphinxcolwidth{3}{5}}
\par
\vskip-\baselineskip\vbox{\hbox{\strut}}\end{varwidth}%
\sphinxstopmulticolumn
\\
\hline
\sphinxAtStartPar
additionalProperties
&\sphinxstartmulticolumn{4}%
\begin{varwidth}[t]{\sphinxcolwidth{4}{5}}
\sphinxAtStartPar
False
\par
\vskip-\baselineskip\vbox{\hbox{\strut}}\end{varwidth}%
\sphinxstopmulticolumn
\\
\hline
\end{longtable}\sphinxatlongtableend\end{savenotes}


\subsection{ExpectedRadios}
\label{\detokenize{docs/advanced/tmap:expectedradios}}\label{\detokenize{docs/advanced/tmap:expectedradios}}

\begin{savenotes}\sphinxatlongtablestart\begin{longtable}[c]{|*{3}{\X{1}{3}|}}
\hline

\endfirsthead

\multicolumn{3}{c}%
{\makebox[0pt]{\sphinxtablecontinued{\tablename\ \thetable{} \textendash{} continued from previous page}}}\\
\hline

\endhead

\hline
\multicolumn{3}{r}{\makebox[0pt][r]{\sphinxtablecontinued{continues on next page}}}\\
\endfoot

\endlastfoot

\sphinxAtStartPar
type
&\sphinxstartmulticolumn{2}%
\begin{varwidth}[t]{\sphinxcolwidth{2}{3}}
\sphinxAtStartPar
\sphinxstyleemphasis{object}
\par
\vskip-\baselineskip\vbox{\hbox{\strut}}\end{varwidth}%
\sphinxstopmulticolumn
\\
\hline\sphinxstartmulticolumn{3}%
\begin{varwidth}[t]{\sphinxcolwidth{3}{3}}
\sphinxAtStartPar
properties
\par
\vskip-\baselineskip\vbox{\hbox{\strut}}\end{varwidth}%
\sphinxstopmulticolumn
\\
\hline\sphinxmultirow{3}{4}{%
\begin{varwidth}[t]{\sphinxcolwidth{1}{3}}
\begin{itemize}
\item {} 
\sphinxAtStartPar
cb\_col

\end{itemize}
\par
\vskip-\baselineskip\vbox{\hbox{\strut}}\end{varwidth}%
}%
&\sphinxstartmulticolumn{2}%
\begin{varwidth}[t]{\sphinxcolwidth{2}{3}}
\sphinxAtStartPar
If markers should be colored by data in CSV column.
\par
\vskip-\baselineskip\vbox{\hbox{\strut}}\end{varwidth}%
\sphinxstopmulticolumn
\\
\cline{2-3}\sphinxtablestrut{4}&
\sphinxAtStartPar
type
&
\sphinxAtStartPar
\sphinxstyleemphasis{boolean}
\\
\cline{2-3}\sphinxtablestrut{4}&
\sphinxAtStartPar
default
&
\sphinxAtStartPar
False
\\
\hline\sphinxmultirow{3}{10}{%
\begin{varwidth}[t]{\sphinxcolwidth{1}{3}}
\begin{itemize}
\item {} 
\sphinxAtStartPar
cb\_gr

\end{itemize}
\par
\vskip-\baselineskip\vbox{\hbox{\strut}}\end{varwidth}%
}%
&\sphinxstartmulticolumn{2}%
\begin{varwidth}[t]{\sphinxcolwidth{2}{3}}
\sphinxAtStartPar
If markers should be colored by group.
\par
\vskip-\baselineskip\vbox{\hbox{\strut}}\end{varwidth}%
\sphinxstopmulticolumn
\\
\cline{2-3}\sphinxtablestrut{10}&
\sphinxAtStartPar
type
&
\sphinxAtStartPar
\sphinxstyleemphasis{boolean}
\\
\cline{2-3}\sphinxtablestrut{10}&
\sphinxAtStartPar
default
&
\sphinxAtStartPar
True
\\
\hline\sphinxmultirow{3}{16}{%
\begin{varwidth}[t]{\sphinxcolwidth{1}{3}}
\begin{itemize}
\item {} 
\sphinxAtStartPar
cb\_gr\_rand

\end{itemize}
\par
\vskip-\baselineskip\vbox{\hbox{\strut}}\end{varwidth}%
}%
&\sphinxstartmulticolumn{2}%
\begin{varwidth}[t]{\sphinxcolwidth{2}{3}}
\sphinxAtStartPar
If group color should be generated randomly.
\par
\vskip-\baselineskip\vbox{\hbox{\strut}}\end{varwidth}%
\sphinxstopmulticolumn
\\
\cline{2-3}\sphinxtablestrut{16}&
\sphinxAtStartPar
type
&
\sphinxAtStartPar
\sphinxstyleemphasis{boolean}
\\
\cline{2-3}\sphinxtablestrut{16}&
\sphinxAtStartPar
default
&
\sphinxAtStartPar
False
\\
\hline\sphinxmultirow{3}{22}{%
\begin{varwidth}[t]{\sphinxcolwidth{1}{3}}
\begin{itemize}
\item {} 
\sphinxAtStartPar
cb\_gr\_dict

\end{itemize}
\par
\vskip-\baselineskip\vbox{\hbox{\strut}}\end{varwidth}%
}%
&\sphinxstartmulticolumn{2}%
\begin{varwidth}[t]{\sphinxcolwidth{2}{3}}
\sphinxAtStartPar
If group color should be read from custom dictionary.
\par
\vskip-\baselineskip\vbox{\hbox{\strut}}\end{varwidth}%
\sphinxstopmulticolumn
\\
\cline{2-3}\sphinxtablestrut{22}&
\sphinxAtStartPar
type
&
\sphinxAtStartPar
\sphinxstyleemphasis{boolean}
\\
\cline{2-3}\sphinxtablestrut{22}&
\sphinxAtStartPar
default
&
\sphinxAtStartPar
False
\\
\hline\sphinxmultirow{3}{28}{%
\begin{varwidth}[t]{\sphinxcolwidth{1}{3}}
\begin{itemize}
\item {} 
\sphinxAtStartPar
cb\_gr\_key

\end{itemize}
\par
\vskip-\baselineskip\vbox{\hbox{\strut}}\end{varwidth}%
}%
&\sphinxstartmulticolumn{2}%
\begin{varwidth}[t]{\sphinxcolwidth{2}{3}}
\sphinxAtStartPar
If group color should be generated from group key.
\par
\vskip-\baselineskip\vbox{\hbox{\strut}}\end{varwidth}%
\sphinxstopmulticolumn
\\
\cline{2-3}\sphinxtablestrut{28}&
\sphinxAtStartPar
type
&
\sphinxAtStartPar
\sphinxstyleemphasis{boolean}
\\
\cline{2-3}\sphinxtablestrut{28}&
\sphinxAtStartPar
default
&
\sphinxAtStartPar
True
\\
\hline\sphinxmultirow{3}{34}{%
\begin{varwidth}[t]{\sphinxcolwidth{1}{3}}
\begin{itemize}
\item {} 
\sphinxAtStartPar
pie\_check

\end{itemize}
\par
\vskip-\baselineskip\vbox{\hbox{\strut}}\end{varwidth}%
}%
&\sphinxstartmulticolumn{2}%
\begin{varwidth}[t]{\sphinxcolwidth{2}{3}}
\sphinxAtStartPar
If markers should be rendered as pie charts.
\par
\vskip-\baselineskip\vbox{\hbox{\strut}}\end{varwidth}%
\sphinxstopmulticolumn
\\
\cline{2-3}\sphinxtablestrut{34}&
\sphinxAtStartPar
type
&
\sphinxAtStartPar
\sphinxstyleemphasis{boolean}
\\
\cline{2-3}\sphinxtablestrut{34}&
\sphinxAtStartPar
default
&
\sphinxAtStartPar
False
\\
\hline\sphinxmultirow{3}{40}{%
\begin{varwidth}[t]{\sphinxcolwidth{1}{3}}
\begin{itemize}
\item {} 
\sphinxAtStartPar
scale\_check

\end{itemize}
\par
\vskip-\baselineskip\vbox{\hbox{\strut}}\end{varwidth}%
}%
&\sphinxstartmulticolumn{2}%
\begin{varwidth}[t]{\sphinxcolwidth{2}{3}}
\sphinxAtStartPar
If markers should be scaled by data in CSV column.
\par
\vskip-\baselineskip\vbox{\hbox{\strut}}\end{varwidth}%
\sphinxstopmulticolumn
\\
\cline{2-3}\sphinxtablestrut{40}&
\sphinxAtStartPar
type
&
\sphinxAtStartPar
\sphinxstyleemphasis{boolean}
\\
\cline{2-3}\sphinxtablestrut{40}&
\sphinxAtStartPar
default
&
\sphinxAtStartPar
False
\\
\hline\sphinxmultirow{3}{46}{%
\begin{varwidth}[t]{\sphinxcolwidth{1}{3}}
\begin{itemize}
\item {} 
\sphinxAtStartPar
shape\_col

\end{itemize}
\par
\vskip-\baselineskip\vbox{\hbox{\strut}}\end{varwidth}%
}%
&\sphinxstartmulticolumn{2}%
\begin{varwidth}[t]{\sphinxcolwidth{2}{3}}
\sphinxAtStartPar
If markers should get their shape from data in CSV column.
\par
\vskip-\baselineskip\vbox{\hbox{\strut}}\end{varwidth}%
\sphinxstopmulticolumn
\\
\cline{2-3}\sphinxtablestrut{46}&
\sphinxAtStartPar
type
&
\sphinxAtStartPar
\sphinxstyleemphasis{boolean}
\\
\cline{2-3}\sphinxtablestrut{46}&
\sphinxAtStartPar
default
&
\sphinxAtStartPar
False
\\
\hline\sphinxmultirow{3}{52}{%
\begin{varwidth}[t]{\sphinxcolwidth{1}{3}}
\begin{itemize}
\item {} 
\sphinxAtStartPar
shape\_gr

\end{itemize}
\par
\vskip-\baselineskip\vbox{\hbox{\strut}}\end{varwidth}%
}%
&\sphinxstartmulticolumn{2}%
\begin{varwidth}[t]{\sphinxcolwidth{2}{3}}
\sphinxAtStartPar
If markers should get their shape from group.
\par
\vskip-\baselineskip\vbox{\hbox{\strut}}\end{varwidth}%
\sphinxstopmulticolumn
\\
\cline{2-3}\sphinxtablestrut{52}&
\sphinxAtStartPar
type
&
\sphinxAtStartPar
\sphinxstyleemphasis{boolean}
\\
\cline{2-3}\sphinxtablestrut{52}&
\sphinxAtStartPar
default
&
\sphinxAtStartPar
True
\\
\hline\sphinxmultirow{3}{58}{%
\begin{varwidth}[t]{\sphinxcolwidth{1}{3}}
\begin{itemize}
\item {} 
\sphinxAtStartPar
shape\_gr\_rand

\end{itemize}
\par
\vskip-\baselineskip\vbox{\hbox{\strut}}\end{varwidth}%
}%
&\sphinxstartmulticolumn{2}%
\begin{varwidth}[t]{\sphinxcolwidth{2}{3}}
\sphinxAtStartPar
If group shape should be generated randomly.
\par
\vskip-\baselineskip\vbox{\hbox{\strut}}\end{varwidth}%
\sphinxstopmulticolumn
\\
\cline{2-3}\sphinxtablestrut{58}&
\sphinxAtStartPar
type
&
\sphinxAtStartPar
\sphinxstyleemphasis{boolean}
\\
\cline{2-3}\sphinxtablestrut{58}&
\sphinxAtStartPar
default
&
\sphinxAtStartPar
True
\\
\hline\sphinxmultirow{3}{64}{%
\begin{varwidth}[t]{\sphinxcolwidth{1}{3}}
\begin{itemize}
\item {} 
\sphinxAtStartPar
shape\_gr\_dict

\end{itemize}
\par
\vskip-\baselineskip\vbox{\hbox{\strut}}\end{varwidth}%
}%
&\sphinxstartmulticolumn{2}%
\begin{varwidth}[t]{\sphinxcolwidth{2}{3}}
\sphinxAtStartPar
If group shape should be read from custom dictionary.
\par
\vskip-\baselineskip\vbox{\hbox{\strut}}\end{varwidth}%
\sphinxstopmulticolumn
\\
\cline{2-3}\sphinxtablestrut{64}&
\sphinxAtStartPar
type
&
\sphinxAtStartPar
\sphinxstyleemphasis{boolean}
\\
\cline{2-3}\sphinxtablestrut{64}&
\sphinxAtStartPar
default
&
\sphinxAtStartPar
False
\\
\hline\sphinxmultirow{3}{70}{%
\begin{varwidth}[t]{\sphinxcolwidth{1}{3}}
\begin{itemize}
\item {} 
\sphinxAtStartPar
shape\_fixed

\end{itemize}
\par
\vskip-\baselineskip\vbox{\hbox{\strut}}\end{varwidth}%
}%
&\sphinxstartmulticolumn{2}%
\begin{varwidth}[t]{\sphinxcolwidth{2}{3}}
\sphinxAtStartPar
If a single fixed shape should be used for all markers.
\par
\vskip-\baselineskip\vbox{\hbox{\strut}}\end{varwidth}%
\sphinxstopmulticolumn
\\
\cline{2-3}\sphinxtablestrut{70}&
\sphinxAtStartPar
type
&
\sphinxAtStartPar
\sphinxstyleemphasis{boolean}
\\
\cline{2-3}\sphinxtablestrut{70}&
\sphinxAtStartPar
default
&
\sphinxAtStartPar
False
\\
\hline\sphinxmultirow{3}{76}{%
\begin{varwidth}[t]{\sphinxcolwidth{1}{3}}
\begin{itemize}
\item {} 
\sphinxAtStartPar
opacity\_check

\end{itemize}
\par
\vskip-\baselineskip\vbox{\hbox{\strut}}\end{varwidth}%
}%
&\sphinxstartmulticolumn{2}%
\begin{varwidth}[t]{\sphinxcolwidth{2}{3}}
\sphinxAtStartPar
If markers should get their opacities from data in CSV column.
\par
\vskip-\baselineskip\vbox{\hbox{\strut}}\end{varwidth}%
\sphinxstopmulticolumn
\\
\cline{2-3}\sphinxtablestrut{76}&
\sphinxAtStartPar
type
&
\sphinxAtStartPar
\sphinxstyleemphasis{boolean}
\\
\cline{2-3}\sphinxtablestrut{76}&
\sphinxAtStartPar
default
&
\sphinxAtStartPar
False
\\
\hline\sphinxmultirow{3}{82}{%
\begin{varwidth}[t]{\sphinxcolwidth{1}{3}}
\begin{itemize}
\item {} 
\sphinxAtStartPar
\_no\_outline

\end{itemize}
\par
\vskip-\baselineskip\vbox{\hbox{\strut}}\end{varwidth}%
}%
&\sphinxstartmulticolumn{2}%
\begin{varwidth}[t]{\sphinxcolwidth{2}{3}}
\sphinxAtStartPar
If marker shapes should be rendered without outline.
\par
\vskip-\baselineskip\vbox{\hbox{\strut}}\end{varwidth}%
\sphinxstopmulticolumn
\\
\cline{2-3}\sphinxtablestrut{82}&
\sphinxAtStartPar
type
&
\sphinxAtStartPar
\sphinxstyleemphasis{boolean}
\\
\cline{2-3}\sphinxtablestrut{82}&
\sphinxAtStartPar
default
&
\sphinxAtStartPar
False
\\
\hline\sphinxmultirow{3}{88}{%
\begin{varwidth}[t]{\sphinxcolwidth{1}{3}}
\begin{itemize}
\item {} 
\sphinxAtStartPar
no\_fill

\end{itemize}
\par
\vskip-\baselineskip\vbox{\hbox{\strut}}\end{varwidth}%
}%
&\sphinxstartmulticolumn{2}%
\begin{varwidth}[t]{\sphinxcolwidth{2}{3}}
\sphinxAtStartPar
If marker shapes should be rendered without filling.
\par
\vskip-\baselineskip\vbox{\hbox{\strut}}\end{varwidth}%
\sphinxstopmulticolumn
\\
\cline{2-3}\sphinxtablestrut{88}&
\sphinxAtStartPar
type
&
\sphinxAtStartPar
\sphinxstyleemphasis{boolean}
\\
\cline{2-3}\sphinxtablestrut{88}&
\sphinxAtStartPar
default
&
\sphinxAtStartPar
False
\\
\hline\sphinxmultirow{3}{94}{%
\begin{varwidth}[t]{\sphinxcolwidth{1}{3}}
\begin{itemize}
\item {} 
\sphinxAtStartPar
collectionItem\_col

\end{itemize}
\par
\vskip-\baselineskip\vbox{\hbox{\strut}}\end{varwidth}%
}%
&\sphinxstartmulticolumn{2}%
\begin{varwidth}[t]{\sphinxcolwidth{2}{3}}
\sphinxAtStartPar
If markers should get their collection item from data in CSV column.
\par
\vskip-\baselineskip\vbox{\hbox{\strut}}\end{varwidth}%
\sphinxstopmulticolumn
\\
\cline{2-3}\sphinxtablestrut{94}&
\sphinxAtStartPar
type
&
\sphinxAtStartPar
\sphinxstyleemphasis{boolean}
\\
\cline{2-3}\sphinxtablestrut{94}&
\sphinxAtStartPar
default
&
\sphinxAtStartPar
False
\\
\hline\sphinxmultirow{3}{100}{%
\begin{varwidth}[t]{\sphinxcolwidth{1}{3}}
\begin{itemize}
\item {} 
\sphinxAtStartPar
collectionItem\_fixed

\end{itemize}
\par
\vskip-\baselineskip\vbox{\hbox{\strut}}\end{varwidth}%
}%
&\sphinxstartmulticolumn{2}%
\begin{varwidth}[t]{\sphinxcolwidth{2}{3}}
\sphinxAtStartPar
If a single fixed collection item should be used for all markers.
\par
\vskip-\baselineskip\vbox{\hbox{\strut}}\end{varwidth}%
\sphinxstopmulticolumn
\\
\cline{2-3}\sphinxtablestrut{100}&
\sphinxAtStartPar
type
&
\sphinxAtStartPar
\sphinxstyleemphasis{boolean}
\\
\cline{2-3}\sphinxtablestrut{100}&
\sphinxAtStartPar
default
&
\sphinxAtStartPar
True
\\
\hline\sphinxmultirow{3}{106}{%
\begin{varwidth}[t]{\sphinxcolwidth{1}{3}}
\begin{itemize}
\item {} 
\sphinxAtStartPar
sortby\_check

\end{itemize}
\par
\vskip-\baselineskip\vbox{\hbox{\strut}}\end{varwidth}%
}%
&\sphinxstartmulticolumn{2}%
\begin{varwidth}[t]{\sphinxcolwidth{2}{3}}
\sphinxAtStartPar
If markers should be sorted by data in CSV column.
\par
\vskip-\baselineskip\vbox{\hbox{\strut}}\end{varwidth}%
\sphinxstopmulticolumn
\\
\cline{2-3}\sphinxtablestrut{106}&
\sphinxAtStartPar
type
&
\sphinxAtStartPar
\sphinxstyleemphasis{boolean}
\\
\cline{2-3}\sphinxtablestrut{106}&
\sphinxAtStartPar
default
&
\sphinxAtStartPar
False
\\
\hline\sphinxmultirow{3}{112}{%
\begin{varwidth}[t]{\sphinxcolwidth{1}{3}}
\begin{itemize}
\item {} 
\sphinxAtStartPar
sortby\_desc\_check

\end{itemize}
\par
\vskip-\baselineskip\vbox{\hbox{\strut}}\end{varwidth}%
}%
&\sphinxstartmulticolumn{2}%
\begin{varwidth}[t]{\sphinxcolwidth{2}{3}}
\sphinxAtStartPar
If markers should be sorted in descending order.
\par
\vskip-\baselineskip\vbox{\hbox{\strut}}\end{varwidth}%
\sphinxstopmulticolumn
\\
\cline{2-3}\sphinxtablestrut{112}&
\sphinxAtStartPar
type
&
\sphinxAtStartPar
\sphinxstyleemphasis{boolean}
\\
\cline{2-3}\sphinxtablestrut{112}&
\sphinxAtStartPar
default
&
\sphinxAtStartPar
False
\\
\hline\sphinxmultirow{3}{118}{%
\begin{varwidth}[t]{\sphinxcolwidth{1}{3}}
\begin{itemize}
\item {} 
\sphinxAtStartPar
edges\_check

\end{itemize}
\par
\vskip-\baselineskip\vbox{\hbox{\strut}}\end{varwidth}%
}%
&\sphinxstartmulticolumn{2}%
\begin{varwidth}[t]{\sphinxcolwidth{2}{3}}
\sphinxAtStartPar
If markers should be connected by edges in Network Diagram mode.
\par
\vskip-\baselineskip\vbox{\hbox{\strut}}\end{varwidth}%
\sphinxstopmulticolumn
\\
\cline{2-3}\sphinxtablestrut{118}&
\sphinxAtStartPar
type
&
\sphinxAtStartPar
\sphinxstyleemphasis{boolean}
\\
\cline{2-3}\sphinxtablestrut{118}&
\sphinxAtStartPar
default
&
\sphinxAtStartPar
False
\\
\hline
\sphinxAtStartPar
additionalProperties
&\sphinxstartmulticolumn{2}%
\begin{varwidth}[t]{\sphinxcolwidth{2}{3}}
\sphinxAtStartPar
False
\par
\vskip-\baselineskip\vbox{\hbox{\strut}}\end{varwidth}%
\sphinxstopmulticolumn
\\
\hline
\end{longtable}\sphinxatlongtableend\end{savenotes}


\subsection{Filter}
\label{\detokenize{docs/advanced/tmap:filter}}\label{\detokenize{docs/advanced/tmap:filter}}

\begin{savenotes}\sphinxattablestart
\centering
\begin{tabulary}{\linewidth}[t]{|T|T|}
\hline

\sphinxAtStartPar
type
&
\sphinxAtStartPar
\sphinxstyleemphasis{string}
\\
\hline
\sphinxAtStartPar
enum
&
\sphinxAtStartPar
Color, Brightness, Exposure, Hue, Contrast, Vibrance, Noise, Saturation, Gamma, Invert, Greyscale, Threshold, Erosion, Dilation, SplitChannel, Colormap
\\
\hline
\end{tabulary}
\par
\sphinxattableend\end{savenotes}


\subsection{Layer}
\label{\detokenize{docs/advanced/tmap:layer}}\label{\detokenize{docs/advanced/tmap:layer}}

\begin{savenotes}\sphinxattablestart
\centering
\begin{tabular}[t]{|*{4}{\X{1}{4}|}}
\hline

\sphinxAtStartPar
type
&\sphinxstartmulticolumn{3}%
\begin{varwidth}[t]{\sphinxcolwidth{3}{4}}
\sphinxAtStartPar
\sphinxstyleemphasis{object}
\par
\vskip-\baselineskip\vbox{\hbox{\strut}}\end{varwidth}%
\sphinxstopmulticolumn
\\
\hline\sphinxstartmulticolumn{4}%
\begin{varwidth}[t]{\sphinxcolwidth{4}{4}}
\sphinxAtStartPar
properties
\par
\vskip-\baselineskip\vbox{\hbox{\strut}}\end{varwidth}%
\sphinxstopmulticolumn
\\
\hline\sphinxmultirow{2}{4}{%
\begin{varwidth}[t]{\sphinxcolwidth{1}{4}}
\begin{itemize}
\item {} 
\sphinxAtStartPar
\sphinxstylestrong{name}

\end{itemize}
\par
\vskip-\baselineskip\vbox{\hbox{\strut}}\end{varwidth}%
}%
&\sphinxstartmulticolumn{3}%
\begin{varwidth}[t]{\sphinxcolwidth{3}{4}}
\sphinxAtStartPar
Name of the image layer
\par
\vskip-\baselineskip\vbox{\hbox{\strut}}\end{varwidth}%
\sphinxstopmulticolumn
\\
\cline{2-4}\sphinxtablestrut{4}&
\sphinxAtStartPar
type
&\sphinxstartmulticolumn{2}%
\begin{varwidth}[t]{\sphinxcolwidth{2}{4}}
\sphinxAtStartPar
\sphinxstyleemphasis{string}
\par
\vskip-\baselineskip\vbox{\hbox{\strut}}\end{varwidth}%
\sphinxstopmulticolumn
\\
\hline\sphinxmultirow{2}{8}{%
\begin{varwidth}[t]{\sphinxcolwidth{1}{4}}
\begin{itemize}
\item {} 
\sphinxAtStartPar
\sphinxstylestrong{tileSource}

\end{itemize}
\par
\vskip-\baselineskip\vbox{\hbox{\strut}}\end{varwidth}%
}%
&\sphinxstartmulticolumn{3}%
\begin{varwidth}[t]{\sphinxcolwidth{3}{4}}
\sphinxAtStartPar
Relative path to an image file in a supported format.
\par
\vskip-\baselineskip\vbox{\hbox{\strut}}\end{varwidth}%
\sphinxstopmulticolumn
\\
\cline{2-4}\sphinxtablestrut{8}&
\sphinxAtStartPar
type
&\sphinxstartmulticolumn{2}%
\begin{varwidth}[t]{\sphinxcolwidth{2}{4}}
\sphinxAtStartPar
\sphinxstyleemphasis{string}
\par
\vskip-\baselineskip\vbox{\hbox{\strut}}\end{varwidth}%
\sphinxstopmulticolumn
\\
\hline\sphinxmultirow{4}{12}{%
\begin{varwidth}[t]{\sphinxcolwidth{1}{4}}
\begin{itemize}
\item {} 
\sphinxAtStartPar
x

\end{itemize}
\par
\vskip-\baselineskip\vbox{\hbox{\strut}}\end{varwidth}%
}%
&\sphinxstartmulticolumn{3}%
\begin{varwidth}[t]{\sphinxcolwidth{3}{4}}
\sphinxAtStartPar
Left coordinate of the image in viewport coordinate.
\par
\vskip-\baselineskip\vbox{\hbox{\strut}}\end{varwidth}%
\sphinxstopmulticolumn
\\
\cline{2-4}\sphinxtablestrut{12}&
\sphinxAtStartPar
default
&\sphinxstartmulticolumn{2}%
\begin{varwidth}[t]{\sphinxcolwidth{2}{4}}
\sphinxAtStartPar
null
\par
\vskip-\baselineskip\vbox{\hbox{\strut}}\end{varwidth}%
\sphinxstopmulticolumn
\\
\cline{2-4}\sphinxtablestrut{12}&\sphinxmultirow{2}{16}{%
\begin{varwidth}[t]{\sphinxcolwidth{1}{4}}
\sphinxAtStartPar
anyOf
\par
\vskip-\baselineskip\vbox{\hbox{\strut}}\end{varwidth}%
}%
&
\sphinxAtStartPar
type
&
\sphinxAtStartPar
\sphinxstyleemphasis{number}
\\
\cline{3-4}\sphinxtablestrut{12}&\sphinxtablestrut{16}&
\sphinxAtStartPar
type
&
\sphinxAtStartPar
\sphinxstyleemphasis{null}
\\
\hline\sphinxmultirow{4}{21}{%
\begin{varwidth}[t]{\sphinxcolwidth{1}{4}}
\begin{itemize}
\item {} 
\sphinxAtStartPar
y

\end{itemize}
\par
\vskip-\baselineskip\vbox{\hbox{\strut}}\end{varwidth}%
}%
&\sphinxstartmulticolumn{3}%
\begin{varwidth}[t]{\sphinxcolwidth{3}{4}}
\sphinxAtStartPar
Top coordinate of the image in viewport coordinate.
\par
\vskip-\baselineskip\vbox{\hbox{\strut}}\end{varwidth}%
\sphinxstopmulticolumn
\\
\cline{2-4}\sphinxtablestrut{21}&
\sphinxAtStartPar
default
&\sphinxstartmulticolumn{2}%
\begin{varwidth}[t]{\sphinxcolwidth{2}{4}}
\sphinxAtStartPar
null
\par
\vskip-\baselineskip\vbox{\hbox{\strut}}\end{varwidth}%
\sphinxstopmulticolumn
\\
\cline{2-4}\sphinxtablestrut{21}&\sphinxmultirow{2}{25}{%
\begin{varwidth}[t]{\sphinxcolwidth{1}{4}}
\sphinxAtStartPar
anyOf
\par
\vskip-\baselineskip\vbox{\hbox{\strut}}\end{varwidth}%
}%
&
\sphinxAtStartPar
type
&
\sphinxAtStartPar
\sphinxstyleemphasis{number}
\\
\cline{3-4}\sphinxtablestrut{21}&\sphinxtablestrut{25}&
\sphinxAtStartPar
type
&
\sphinxAtStartPar
\sphinxstyleemphasis{null}
\\
\hline\sphinxmultirow{4}{30}{%
\begin{varwidth}[t]{\sphinxcolwidth{1}{4}}
\begin{itemize}
\item {} 
\sphinxAtStartPar
rotation

\end{itemize}
\par
\vskip-\baselineskip\vbox{\hbox{\strut}}\end{varwidth}%
}%
&\sphinxstartmulticolumn{3}%
\begin{varwidth}[t]{\sphinxcolwidth{3}{4}}
\sphinxAtStartPar
Rotation of the image in degrees.
\par
\vskip-\baselineskip\vbox{\hbox{\strut}}\end{varwidth}%
\sphinxstopmulticolumn
\\
\cline{2-4}\sphinxtablestrut{30}&
\sphinxAtStartPar
default
&\sphinxstartmulticolumn{2}%
\begin{varwidth}[t]{\sphinxcolwidth{2}{4}}
\sphinxAtStartPar
null
\par
\vskip-\baselineskip\vbox{\hbox{\strut}}\end{varwidth}%
\sphinxstopmulticolumn
\\
\cline{2-4}\sphinxtablestrut{30}&\sphinxmultirow{2}{34}{%
\begin{varwidth}[t]{\sphinxcolwidth{1}{4}}
\sphinxAtStartPar
anyOf
\par
\vskip-\baselineskip\vbox{\hbox{\strut}}\end{varwidth}%
}%
&
\sphinxAtStartPar
type
&
\sphinxAtStartPar
\sphinxstyleemphasis{number}
\\
\cline{3-4}\sphinxtablestrut{30}&\sphinxtablestrut{34}&
\sphinxAtStartPar
type
&
\sphinxAtStartPar
\sphinxstyleemphasis{null}
\\
\hline\sphinxmultirow{3}{39}{%
\begin{varwidth}[t]{\sphinxcolwidth{1}{4}}
\begin{itemize}
\item {} 
\sphinxAtStartPar
flip

\end{itemize}
\par
\vskip-\baselineskip\vbox{\hbox{\strut}}\end{varwidth}%
}%
&\sphinxstartmulticolumn{3}%
\begin{varwidth}[t]{\sphinxcolwidth{3}{4}}
\sphinxAtStartPar
Flip the image horizontally.
\par
\vskip-\baselineskip\vbox{\hbox{\strut}}\end{varwidth}%
\sphinxstopmulticolumn
\\
\cline{2-4}\sphinxtablestrut{39}&
\sphinxAtStartPar
type
&\sphinxstartmulticolumn{2}%
\begin{varwidth}[t]{\sphinxcolwidth{2}{4}}
\sphinxAtStartPar
\sphinxstyleemphasis{boolean}
\par
\vskip-\baselineskip\vbox{\hbox{\strut}}\end{varwidth}%
\sphinxstopmulticolumn
\\
\cline{2-4}\sphinxtablestrut{39}&
\sphinxAtStartPar
default
&\sphinxstartmulticolumn{2}%
\begin{varwidth}[t]{\sphinxcolwidth{2}{4}}
\sphinxAtStartPar
False
\par
\vskip-\baselineskip\vbox{\hbox{\strut}}\end{varwidth}%
\sphinxstopmulticolumn
\\
\hline\sphinxmultirow{4}{45}{%
\begin{varwidth}[t]{\sphinxcolwidth{1}{4}}
\begin{itemize}
\item {} 
\sphinxAtStartPar
scale

\end{itemize}
\par
\vskip-\baselineskip\vbox{\hbox{\strut}}\end{varwidth}%
}%
&\sphinxstartmulticolumn{3}%
\begin{varwidth}[t]{\sphinxcolwidth{3}{4}}
\sphinxAtStartPar
Scale of the image.
\par
\vskip-\baselineskip\vbox{\hbox{\strut}}\end{varwidth}%
\sphinxstopmulticolumn
\\
\cline{2-4}\sphinxtablestrut{45}&
\sphinxAtStartPar
default
&\sphinxstartmulticolumn{2}%
\begin{varwidth}[t]{\sphinxcolwidth{2}{4}}
\sphinxAtStartPar
null
\par
\vskip-\baselineskip\vbox{\hbox{\strut}}\end{varwidth}%
\sphinxstopmulticolumn
\\
\cline{2-4}\sphinxtablestrut{45}&\sphinxmultirow{2}{49}{%
\begin{varwidth}[t]{\sphinxcolwidth{1}{4}}
\sphinxAtStartPar
anyOf
\par
\vskip-\baselineskip\vbox{\hbox{\strut}}\end{varwidth}%
}%
&
\sphinxAtStartPar
type
&
\sphinxAtStartPar
\sphinxstyleemphasis{number}
\\
\cline{3-4}\sphinxtablestrut{45}&\sphinxtablestrut{49}&
\sphinxAtStartPar
type
&
\sphinxAtStartPar
\sphinxstyleemphasis{null}
\\
\hline\sphinxmultirow{4}{54}{%
\begin{varwidth}[t]{\sphinxcolwidth{1}{4}}
\begin{itemize}
\item {} 
\sphinxAtStartPar
clip

\end{itemize}
\par
\vskip-\baselineskip\vbox{\hbox{\strut}}\end{varwidth}%
}%
&\sphinxstartmulticolumn{3}%
\begin{varwidth}[t]{\sphinxcolwidth{3}{4}}
\sphinxAtStartPar
Bounding box used to clip image in image pixel coordinate. If not specified, the whole image is shown.
\par
\vskip-\baselineskip\vbox{\hbox{\strut}}\end{varwidth}%
\sphinxstopmulticolumn
\\
\cline{2-4}\sphinxtablestrut{54}&
\sphinxAtStartPar
default
&\sphinxstartmulticolumn{2}%
\begin{varwidth}[t]{\sphinxcolwidth{2}{4}}
\sphinxAtStartPar
null
\par
\vskip-\baselineskip\vbox{\hbox{\strut}}\end{varwidth}%
\sphinxstopmulticolumn
\\
\cline{2-4}\sphinxtablestrut{54}&\sphinxmultirow{2}{58}{%
\begin{varwidth}[t]{\sphinxcolwidth{1}{4}}
\sphinxAtStartPar
anyOf
\par
\vskip-\baselineskip\vbox{\hbox{\strut}}\end{varwidth}%
}%
&\sphinxstartmulticolumn{2}%
\begin{varwidth}[t]{\sphinxcolwidth{2}{4}}
\sphinxAtStartPar
{\hyperref[\detokenize{docs/advanced/tmap:layerclip}]{\sphinxcrossref{LayerClip}}}
\par
\vskip-\baselineskip\vbox{\hbox{\strut}}\end{varwidth}%
\sphinxstopmulticolumn
\\
\cline{3-4}\sphinxtablestrut{54}&\sphinxtablestrut{58}&
\sphinxAtStartPar
type
&
\sphinxAtStartPar
\sphinxstyleemphasis{null}
\\
\hline
\sphinxAtStartPar
additionalProperties
&\sphinxstartmulticolumn{3}%
\begin{varwidth}[t]{\sphinxcolwidth{3}{4}}
\sphinxAtStartPar
False
\par
\vskip-\baselineskip\vbox{\hbox{\strut}}\end{varwidth}%
\sphinxstopmulticolumn
\\
\hline
\end{tabular}
\par
\sphinxattableend\end{savenotes}


\subsection{LayerClip}
\label{\detokenize{docs/advanced/tmap:layerclip}}\label{\detokenize{docs/advanced/tmap:layerclip}}

\begin{savenotes}\sphinxattablestart
\centering
\begin{tabular}[t]{|*{3}{\X{1}{3}|}}
\hline

\sphinxAtStartPar
type
&\sphinxstartmulticolumn{2}%
\begin{varwidth}[t]{\sphinxcolwidth{2}{3}}
\sphinxAtStartPar
\sphinxstyleemphasis{object}
\par
\vskip-\baselineskip\vbox{\hbox{\strut}}\end{varwidth}%
\sphinxstopmulticolumn
\\
\hline\sphinxstartmulticolumn{3}%
\begin{varwidth}[t]{\sphinxcolwidth{3}{3}}
\sphinxAtStartPar
properties
\par
\vskip-\baselineskip\vbox{\hbox{\strut}}\end{varwidth}%
\sphinxstopmulticolumn
\\
\hline\sphinxmultirow{2}{4}{%
\begin{varwidth}[t]{\sphinxcolwidth{1}{3}}
\begin{itemize}
\item {} 
\sphinxAtStartPar
\sphinxstylestrong{x}

\end{itemize}
\par
\vskip-\baselineskip\vbox{\hbox{\strut}}\end{varwidth}%
}%
&\sphinxstartmulticolumn{2}%
\begin{varwidth}[t]{\sphinxcolwidth{2}{3}}
\sphinxAtStartPar
Left coordinate of the clip in image pixel coordinate.
\par
\vskip-\baselineskip\vbox{\hbox{\strut}}\end{varwidth}%
\sphinxstopmulticolumn
\\
\cline{2-3}\sphinxtablestrut{4}&
\sphinxAtStartPar
type
&
\sphinxAtStartPar
\sphinxstyleemphasis{number}
\\
\hline\sphinxmultirow{2}{8}{%
\begin{varwidth}[t]{\sphinxcolwidth{1}{3}}
\begin{itemize}
\item {} 
\sphinxAtStartPar
\sphinxstylestrong{y}

\end{itemize}
\par
\vskip-\baselineskip\vbox{\hbox{\strut}}\end{varwidth}%
}%
&\sphinxstartmulticolumn{2}%
\begin{varwidth}[t]{\sphinxcolwidth{2}{3}}
\sphinxAtStartPar
Top coordinate of the clip in image pixel coordinate.
\par
\vskip-\baselineskip\vbox{\hbox{\strut}}\end{varwidth}%
\sphinxstopmulticolumn
\\
\cline{2-3}\sphinxtablestrut{8}&
\sphinxAtStartPar
type
&
\sphinxAtStartPar
\sphinxstyleemphasis{number}
\\
\hline\sphinxmultirow{2}{12}{%
\begin{varwidth}[t]{\sphinxcolwidth{1}{3}}
\begin{itemize}
\item {} 
\sphinxAtStartPar
\sphinxstylestrong{w}

\end{itemize}
\par
\vskip-\baselineskip\vbox{\hbox{\strut}}\end{varwidth}%
}%
&\sphinxstartmulticolumn{2}%
\begin{varwidth}[t]{\sphinxcolwidth{2}{3}}
\sphinxAtStartPar
Width of the clip in image pixel coordinate.
\par
\vskip-\baselineskip\vbox{\hbox{\strut}}\end{varwidth}%
\sphinxstopmulticolumn
\\
\cline{2-3}\sphinxtablestrut{12}&
\sphinxAtStartPar
type
&
\sphinxAtStartPar
\sphinxstyleemphasis{number}
\\
\hline\sphinxmultirow{2}{16}{%
\begin{varwidth}[t]{\sphinxcolwidth{1}{3}}
\begin{itemize}
\item {} 
\sphinxAtStartPar
\sphinxstylestrong{h}

\end{itemize}
\par
\vskip-\baselineskip\vbox{\hbox{\strut}}\end{varwidth}%
}%
&\sphinxstartmulticolumn{2}%
\begin{varwidth}[t]{\sphinxcolwidth{2}{3}}
\sphinxAtStartPar
Height of the clip in image pixel coordinate.
\par
\vskip-\baselineskip\vbox{\hbox{\strut}}\end{varwidth}%
\sphinxstopmulticolumn
\\
\cline{2-3}\sphinxtablestrut{16}&
\sphinxAtStartPar
type
&
\sphinxAtStartPar
\sphinxstyleemphasis{number}
\\
\hline
\sphinxAtStartPar
additionalProperties
&\sphinxstartmulticolumn{2}%
\begin{varwidth}[t]{\sphinxcolwidth{2}{3}}
\sphinxAtStartPar
False
\par
\vskip-\baselineskip\vbox{\hbox{\strut}}\end{varwidth}%
\sphinxstopmulticolumn
\\
\hline
\end{tabular}
\par
\sphinxattableend\end{savenotes}


\subsection{LayerFilter}
\label{\detokenize{docs/advanced/tmap:layerfilter}}\label{\detokenize{docs/advanced/tmap:layerfilter}}

\begin{savenotes}\sphinxattablestart
\centering
\begin{tabular}[t]{|*{4}{\X{1}{4}|}}
\hline

\sphinxAtStartPar
type
&\sphinxstartmulticolumn{3}%
\begin{varwidth}[t]{\sphinxcolwidth{3}{4}}
\sphinxAtStartPar
\sphinxstyleemphasis{object}
\par
\vskip-\baselineskip\vbox{\hbox{\strut}}\end{varwidth}%
\sphinxstopmulticolumn
\\
\hline\sphinxstartmulticolumn{4}%
\begin{varwidth}[t]{\sphinxcolwidth{4}{4}}
\sphinxAtStartPar
properties
\par
\vskip-\baselineskip\vbox{\hbox{\strut}}\end{varwidth}%
\sphinxstopmulticolumn
\\
\hline\sphinxmultirow{2}{4}{%
\begin{varwidth}[t]{\sphinxcolwidth{1}{4}}
\begin{itemize}
\item {} 
\sphinxAtStartPar
\sphinxstylestrong{name}

\end{itemize}
\par
\vskip-\baselineskip\vbox{\hbox{\strut}}\end{varwidth}%
}%
&\sphinxstartmulticolumn{3}%
\begin{varwidth}[t]{\sphinxcolwidth{3}{4}}
\sphinxAtStartPar
Filter name.
\par
\vskip-\baselineskip\vbox{\hbox{\strut}}\end{varwidth}%
\sphinxstopmulticolumn
\\
\cline{2-4}\sphinxtablestrut{4}&
\sphinxAtStartPar
allOf
&\sphinxstartmulticolumn{2}%
\begin{varwidth}[t]{\sphinxcolwidth{2}{4}}
\sphinxAtStartPar
{\hyperref[\detokenize{docs/advanced/tmap:filter}]{\sphinxcrossref{Filter}}}
\par
\vskip-\baselineskip\vbox{\hbox{\strut}}\end{varwidth}%
\sphinxstopmulticolumn
\\
\hline\sphinxmultirow{5}{8}{%
\begin{varwidth}[t]{\sphinxcolwidth{1}{4}}
\begin{itemize}
\item {} 
\sphinxAtStartPar
\sphinxstylestrong{value}

\end{itemize}
\par
\vskip-\baselineskip\vbox{\hbox{\strut}}\end{varwidth}%
}%
&\sphinxstartmulticolumn{3}%
\begin{varwidth}[t]{\sphinxcolwidth{3}{4}}
\sphinxAtStartPar
Filter parameter.
\par
\vskip-\baselineskip\vbox{\hbox{\strut}}\end{varwidth}%
\sphinxstopmulticolumn
\\
\cline{2-4}\sphinxtablestrut{8}&\sphinxmultirow{4}{10}{%
\begin{varwidth}[t]{\sphinxcolwidth{1}{4}}
\sphinxAtStartPar
anyOf
\par
\vskip-\baselineskip\vbox{\hbox{\strut}}\end{varwidth}%
}%
&
\sphinxAtStartPar
type
&
\sphinxAtStartPar
\sphinxstyleemphasis{string}
\\
\cline{3-4}\sphinxtablestrut{8}&\sphinxtablestrut{10}&
\sphinxAtStartPar
type
&
\sphinxAtStartPar
\sphinxstyleemphasis{boolean}
\\
\cline{3-4}\sphinxtablestrut{8}&\sphinxtablestrut{10}&
\sphinxAtStartPar
type
&
\sphinxAtStartPar
\sphinxstyleemphasis{integer}
\\
\cline{3-4}\sphinxtablestrut{8}&\sphinxtablestrut{10}&
\sphinxAtStartPar
type
&
\sphinxAtStartPar
\sphinxstyleemphasis{number}
\\
\hline
\sphinxAtStartPar
additionalProperties
&\sphinxstartmulticolumn{3}%
\begin{varwidth}[t]{\sphinxcolwidth{3}{4}}
\sphinxAtStartPar
False
\par
\vskip-\baselineskip\vbox{\hbox{\strut}}\end{varwidth}%
\sphinxstopmulticolumn
\\
\hline
\end{tabular}
\par
\sphinxattableend\end{savenotes}


\subsection{LayoutAxis}
\label{\detokenize{docs/advanced/tmap:layoutaxis}}\label{\detokenize{docs/advanced/tmap:layoutaxis}}

\begin{savenotes}\sphinxattablestart
\centering
\begin{tabulary}{\linewidth}[t]{|T|T|}
\hline

\sphinxAtStartPar
type
&
\sphinxAtStartPar
\sphinxstyleemphasis{string}
\\
\hline
\sphinxAtStartPar
enum
&
\sphinxAtStartPar
horizontally, vertically
\\
\hline
\end{tabulary}
\par
\sphinxattableend\end{savenotes}


\subsection{MarkerFile}
\label{\detokenize{docs/advanced/tmap:markerfile}}\label{\detokenize{docs/advanced/tmap:markerfile}}

\begin{savenotes}\sphinxatlongtablestart\begin{longtable}[c]{|*{5}{\X{1}{5}|}}
\hline

\endfirsthead

\multicolumn{5}{c}%
{\makebox[0pt]{\sphinxtablecontinued{\tablename\ \thetable{} \textendash{} continued from previous page}}}\\
\hline

\endhead

\hline
\multicolumn{5}{r}{\makebox[0pt][r]{\sphinxtablecontinued{continues on next page}}}\\
\endfoot

\endlastfoot

\sphinxAtStartPar
type
&\sphinxstartmulticolumn{4}%
\begin{varwidth}[t]{\sphinxcolwidth{4}{5}}
\sphinxAtStartPar
\sphinxstyleemphasis{object}
\par
\vskip-\baselineskip\vbox{\hbox{\strut}}\end{varwidth}%
\sphinxstopmulticolumn
\\
\hline\sphinxstartmulticolumn{5}%
\begin{varwidth}[t]{\sphinxcolwidth{5}{5}}
\sphinxAtStartPar
properties
\par
\vskip-\baselineskip\vbox{\hbox{\strut}}\end{varwidth}%
\sphinxstopmulticolumn
\\
\hline\sphinxmultirow{4}{4}{%
\begin{varwidth}[t]{\sphinxcolwidth{1}{5}}
\begin{itemize}
\item {} 
\sphinxAtStartPar
comment

\end{itemize}
\par
\vskip-\baselineskip\vbox{\hbox{\strut}}\end{varwidth}%
}%
&\sphinxstartmulticolumn{4}%
\begin{varwidth}[t]{\sphinxcolwidth{4}{5}}
\sphinxAtStartPar
Optional description text shown next to marker button.
\par
\vskip-\baselineskip\vbox{\hbox{\strut}}\end{varwidth}%
\sphinxstopmulticolumn
\\
\cline{2-5}\sphinxtablestrut{4}&
\sphinxAtStartPar
default
&\sphinxstartmulticolumn{3}%
\begin{varwidth}[t]{\sphinxcolwidth{3}{5}}
\sphinxAtStartPar
null
\par
\vskip-\baselineskip\vbox{\hbox{\strut}}\end{varwidth}%
\sphinxstopmulticolumn
\\
\cline{2-5}\sphinxtablestrut{4}&\sphinxmultirow{2}{8}{%
\begin{varwidth}[t]{\sphinxcolwidth{1}{5}}
\sphinxAtStartPar
anyOf
\par
\vskip-\baselineskip\vbox{\hbox{\strut}}\end{varwidth}%
}%
&
\sphinxAtStartPar
type
&\sphinxstartmulticolumn{2}%
\begin{varwidth}[t]{\sphinxcolwidth{2}{5}}
\sphinxAtStartPar
\sphinxstyleemphasis{string}
\par
\vskip-\baselineskip\vbox{\hbox{\strut}}\end{varwidth}%
\sphinxstopmulticolumn
\\
\cline{3-5}\sphinxtablestrut{4}&\sphinxtablestrut{8}&
\sphinxAtStartPar
type
&\sphinxstartmulticolumn{2}%
\begin{varwidth}[t]{\sphinxcolwidth{2}{5}}
\sphinxAtStartPar
\sphinxstyleemphasis{null}
\par
\vskip-\baselineskip\vbox{\hbox{\strut}}\end{varwidth}%
\sphinxstopmulticolumn
\\
\hline\sphinxmultirow{4}{13}{%
\begin{varwidth}[t]{\sphinxcolwidth{1}{5}}
\begin{itemize}
\item {} 
\sphinxAtStartPar
name

\end{itemize}
\par
\vskip-\baselineskip\vbox{\hbox{\strut}}\end{varwidth}%
}%
&\sphinxstartmulticolumn{4}%
\begin{varwidth}[t]{\sphinxcolwidth{4}{5}}
\sphinxAtStartPar
Name of marker tab.
\par
\vskip-\baselineskip\vbox{\hbox{\strut}}\end{varwidth}%
\sphinxstopmulticolumn
\\
\cline{2-5}\sphinxtablestrut{13}&
\sphinxAtStartPar
default
&\sphinxstartmulticolumn{3}%
\begin{varwidth}[t]{\sphinxcolwidth{3}{5}}
\sphinxAtStartPar
null
\par
\vskip-\baselineskip\vbox{\hbox{\strut}}\end{varwidth}%
\sphinxstopmulticolumn
\\
\cline{2-5}\sphinxtablestrut{13}&\sphinxmultirow{2}{17}{%
\begin{varwidth}[t]{\sphinxcolwidth{1}{5}}
\sphinxAtStartPar
anyOf
\par
\vskip-\baselineskip\vbox{\hbox{\strut}}\end{varwidth}%
}%
&
\sphinxAtStartPar
type
&\sphinxstartmulticolumn{2}%
\begin{varwidth}[t]{\sphinxcolwidth{2}{5}}
\sphinxAtStartPar
\sphinxstyleemphasis{string}
\par
\vskip-\baselineskip\vbox{\hbox{\strut}}\end{varwidth}%
\sphinxstopmulticolumn
\\
\cline{3-5}\sphinxtablestrut{13}&\sphinxtablestrut{17}&
\sphinxAtStartPar
type
&\sphinxstartmulticolumn{2}%
\begin{varwidth}[t]{\sphinxcolwidth{2}{5}}
\sphinxAtStartPar
\sphinxstyleemphasis{null}
\par
\vskip-\baselineskip\vbox{\hbox{\strut}}\end{varwidth}%
\sphinxstopmulticolumn
\\
\hline\sphinxmultirow{4}{22}{%
\begin{varwidth}[t]{\sphinxcolwidth{1}{5}}
\begin{itemize}
\item {} 
\sphinxAtStartPar
autoLoad

\end{itemize}
\par
\vskip-\baselineskip\vbox{\hbox{\strut}}\end{varwidth}%
}%
&\sphinxstartmulticolumn{4}%
\begin{varwidth}[t]{\sphinxcolwidth{4}{5}}
\sphinxAtStartPar
If the CSV file for the marker dataset should be automatically loaded when the TMAP project is opened. If this is false, the user instead has to click on the marker button in the GUI to load the dataset. If this is an integer, the n\sphinxhyphen{}th marker dataset is automatically loaded.
\par
\vskip-\baselineskip\vbox{\hbox{\strut}}\end{varwidth}%
\sphinxstopmulticolumn
\\
\cline{2-5}\sphinxtablestrut{22}&
\sphinxAtStartPar
default
&\sphinxstartmulticolumn{3}%
\begin{varwidth}[t]{\sphinxcolwidth{3}{5}}
\sphinxAtStartPar
False
\par
\vskip-\baselineskip\vbox{\hbox{\strut}}\end{varwidth}%
\sphinxstopmulticolumn
\\
\cline{2-5}\sphinxtablestrut{22}&\sphinxmultirow{2}{26}{%
\begin{varwidth}[t]{\sphinxcolwidth{1}{5}}
\sphinxAtStartPar
anyOf
\par
\vskip-\baselineskip\vbox{\hbox{\strut}}\end{varwidth}%
}%
&
\sphinxAtStartPar
type
&\sphinxstartmulticolumn{2}%
\begin{varwidth}[t]{\sphinxcolwidth{2}{5}}
\sphinxAtStartPar
\sphinxstyleemphasis{boolean}
\par
\vskip-\baselineskip\vbox{\hbox{\strut}}\end{varwidth}%
\sphinxstopmulticolumn
\\
\cline{3-5}\sphinxtablestrut{22}&\sphinxtablestrut{26}&
\sphinxAtStartPar
type
&\sphinxstartmulticolumn{2}%
\begin{varwidth}[t]{\sphinxcolwidth{2}{5}}
\sphinxAtStartPar
\sphinxstyleemphasis{integer}
\par
\vskip-\baselineskip\vbox{\hbox{\strut}}\end{varwidth}%
\sphinxstopmulticolumn
\\
\hline\sphinxmultirow{3}{31}{%
\begin{varwidth}[t]{\sphinxcolwidth{1}{5}}
\begin{itemize}
\item {} 
\sphinxAtStartPar
hideSettings

\end{itemize}
\par
\vskip-\baselineskip\vbox{\hbox{\strut}}\end{varwidth}%
}%
&\sphinxstartmulticolumn{4}%
\begin{varwidth}[t]{\sphinxcolwidth{4}{5}}
\sphinxAtStartPar
Hide markers’ settings and add a toggle button instead.
\par
\vskip-\baselineskip\vbox{\hbox{\strut}}\end{varwidth}%
\sphinxstopmulticolumn
\\
\cline{2-5}\sphinxtablestrut{31}&
\sphinxAtStartPar
type
&\sphinxstartmulticolumn{3}%
\begin{varwidth}[t]{\sphinxcolwidth{3}{5}}
\sphinxAtStartPar
\sphinxstyleemphasis{boolean}
\par
\vskip-\baselineskip\vbox{\hbox{\strut}}\end{varwidth}%
\sphinxstopmulticolumn
\\
\cline{2-5}\sphinxtablestrut{31}&
\sphinxAtStartPar
default
&\sphinxstartmulticolumn{3}%
\begin{varwidth}[t]{\sphinxcolwidth{3}{5}}
\sphinxAtStartPar
False
\par
\vskip-\baselineskip\vbox{\hbox{\strut}}\end{varwidth}%
\sphinxstopmulticolumn
\\
\hline\sphinxmultirow{4}{37}{%
\begin{varwidth}[t]{\sphinxcolwidth{1}{5}}
\begin{itemize}
\item {} 
\sphinxAtStartPar
uid

\end{itemize}
\par
\vskip-\baselineskip\vbox{\hbox{\strut}}\end{varwidth}%
}%
&\sphinxstartmulticolumn{4}%
\begin{varwidth}[t]{\sphinxcolwidth{4}{5}}
\sphinxAtStartPar
A unique identifier used internally by TissUUmaps to reference the marker dataset.
\par
\vskip-\baselineskip\vbox{\hbox{\strut}}\end{varwidth}%
\sphinxstopmulticolumn
\\
\cline{2-5}\sphinxtablestrut{37}&
\sphinxAtStartPar
default
&\sphinxstartmulticolumn{3}%
\begin{varwidth}[t]{\sphinxcolwidth{3}{5}}
\sphinxAtStartPar
null
\par
\vskip-\baselineskip\vbox{\hbox{\strut}}\end{varwidth}%
\sphinxstopmulticolumn
\\
\cline{2-5}\sphinxtablestrut{37}&\sphinxmultirow{2}{41}{%
\begin{varwidth}[t]{\sphinxcolwidth{1}{5}}
\sphinxAtStartPar
anyOf
\par
\vskip-\baselineskip\vbox{\hbox{\strut}}\end{varwidth}%
}%
&
\sphinxAtStartPar
type
&\sphinxstartmulticolumn{2}%
\begin{varwidth}[t]{\sphinxcolwidth{2}{5}}
\sphinxAtStartPar
\sphinxstyleemphasis{string}
\par
\vskip-\baselineskip\vbox{\hbox{\strut}}\end{varwidth}%
\sphinxstopmulticolumn
\\
\cline{3-5}\sphinxtablestrut{37}&\sphinxtablestrut{41}&
\sphinxAtStartPar
type
&\sphinxstartmulticolumn{2}%
\begin{varwidth}[t]{\sphinxcolwidth{2}{5}}
\sphinxAtStartPar
\sphinxstyleemphasis{null}
\par
\vskip-\baselineskip\vbox{\hbox{\strut}}\end{varwidth}%
\sphinxstopmulticolumn
\\
\hline\begin{itemize}
\item {} 
\sphinxAtStartPar
\sphinxstylestrong{expectedHeader}

\end{itemize}
&\sphinxstartmulticolumn{4}%
\begin{varwidth}[t]{\sphinxcolwidth{4}{5}}
\sphinxAtStartPar
{\hyperref[\detokenize{docs/advanced/tmap:expectedheader}]{\sphinxcrossref{ExpectedHeader}}}
\par
\vskip-\baselineskip\vbox{\hbox{\strut}}\end{varwidth}%
\sphinxstopmulticolumn
\\
\hline\begin{itemize}
\item {} 
\sphinxAtStartPar
expectedRadios

\end{itemize}
&\sphinxstartmulticolumn{4}%
\begin{varwidth}[t]{\sphinxcolwidth{4}{5}}
\sphinxAtStartPar
{\hyperref[\detokenize{docs/advanced/tmap:expectedradios}]{\sphinxcrossref{ExpectedRadios}}}
\par
\vskip-\baselineskip\vbox{\hbox{\strut}}\end{varwidth}%
\sphinxstopmulticolumn
\\
\hline\sphinxmultirow{4}{50}{%
\begin{varwidth}[t]{\sphinxcolwidth{1}{5}}
\begin{itemize}
\item {} 
\sphinxAtStartPar
\sphinxstylestrong{path}

\end{itemize}
\par
\vskip-\baselineskip\vbox{\hbox{\strut}}\end{varwidth}%
}%
&\sphinxstartmulticolumn{4}%
\begin{varwidth}[t]{\sphinxcolwidth{4}{5}}
\sphinxAtStartPar
Relative file path to CSV file in which marker data is stored. If array of string, then a dropdown is created instead of a button.
\par
\vskip-\baselineskip\vbox{\hbox{\strut}}\end{varwidth}%
\sphinxstopmulticolumn
\\
\cline{2-5}\sphinxtablestrut{50}&\sphinxmultirow{3}{52}{%
\begin{varwidth}[t]{\sphinxcolwidth{1}{5}}
\sphinxAtStartPar
anyOf
\par
\vskip-\baselineskip\vbox{\hbox{\strut}}\end{varwidth}%
}%
&
\sphinxAtStartPar
type
&\sphinxstartmulticolumn{2}%
\begin{varwidth}[t]{\sphinxcolwidth{2}{5}}
\sphinxAtStartPar
\sphinxstyleemphasis{string}
\par
\vskip-\baselineskip\vbox{\hbox{\strut}}\end{varwidth}%
\sphinxstopmulticolumn
\\
\cline{3-5}\sphinxtablestrut{50}&\sphinxtablestrut{52}&
\sphinxAtStartPar
type
&\sphinxstartmulticolumn{2}%
\begin{varwidth}[t]{\sphinxcolwidth{2}{5}}
\sphinxAtStartPar
\sphinxstyleemphasis{array}
\par
\vskip-\baselineskip\vbox{\hbox{\strut}}\end{varwidth}%
\sphinxstopmulticolumn
\\
\cline{3-5}\sphinxtablestrut{50}&\sphinxtablestrut{52}&
\sphinxAtStartPar
items
&
\sphinxAtStartPar
type
&
\sphinxAtStartPar
\sphinxstyleemphasis{string}
\\
\hline\sphinxmultirow{5}{60}{%
\begin{varwidth}[t]{\sphinxcolwidth{1}{5}}
\begin{itemize}
\item {} 
\sphinxAtStartPar
dropdownOptions

\end{itemize}
\par
\vskip-\baselineskip\vbox{\hbox{\strut}}\end{varwidth}%
}%
&\sphinxstartmulticolumn{4}%
\begin{varwidth}[t]{\sphinxcolwidth{4}{5}}
\sphinxAtStartPar
List of dropdown options. Each option is a dictionary with the keys ‘title’ and ‘path’.
\par
\vskip-\baselineskip\vbox{\hbox{\strut}}\end{varwidth}%
\sphinxstopmulticolumn
\\
\cline{2-5}\sphinxtablestrut{60}&
\sphinxAtStartPar
default
&\sphinxstartmulticolumn{3}%
\begin{varwidth}[t]{\sphinxcolwidth{3}{5}}
\sphinxAtStartPar
null
\par
\vskip-\baselineskip\vbox{\hbox{\strut}}\end{varwidth}%
\sphinxstopmulticolumn
\\
\cline{2-5}\sphinxtablestrut{60}&\sphinxmultirow{3}{64}{%
\begin{varwidth}[t]{\sphinxcolwidth{1}{5}}
\sphinxAtStartPar
anyOf
\par
\vskip-\baselineskip\vbox{\hbox{\strut}}\end{varwidth}%
}%
&
\sphinxAtStartPar
type
&\sphinxstartmulticolumn{2}%
\begin{varwidth}[t]{\sphinxcolwidth{2}{5}}
\sphinxAtStartPar
\sphinxstyleemphasis{array}
\par
\vskip-\baselineskip\vbox{\hbox{\strut}}\end{varwidth}%
\sphinxstopmulticolumn
\\
\cline{3-5}\sphinxtablestrut{60}&\sphinxtablestrut{64}&
\sphinxAtStartPar
items
&\sphinxstartmulticolumn{2}%
\begin{varwidth}[t]{\sphinxcolwidth{2}{5}}
\sphinxAtStartPar
{\hyperref[\detokenize{docs/advanced/tmap:dropdownoption}]{\sphinxcrossref{DropdownOption}}}
\par
\vskip-\baselineskip\vbox{\hbox{\strut}}\end{varwidth}%
\sphinxstopmulticolumn
\\
\cline{3-5}\sphinxtablestrut{60}&\sphinxtablestrut{64}&
\sphinxAtStartPar
type
&\sphinxstartmulticolumn{2}%
\begin{varwidth}[t]{\sphinxcolwidth{2}{5}}
\sphinxAtStartPar
\sphinxstyleemphasis{null}
\par
\vskip-\baselineskip\vbox{\hbox{\strut}}\end{varwidth}%
\sphinxstopmulticolumn
\\
\hline\sphinxmultirow{3}{71}{%
\begin{varwidth}[t]{\sphinxcolwidth{1}{5}}
\begin{itemize}
\item {} 
\sphinxAtStartPar
settings

\end{itemize}
\par
\vskip-\baselineskip\vbox{\hbox{\strut}}\end{varwidth}%
}%
&
\sphinxAtStartPar
type
&\sphinxstartmulticolumn{3}%
\begin{varwidth}[t]{\sphinxcolwidth{3}{5}}
\sphinxAtStartPar
\sphinxstyleemphasis{array}
\par
\vskip-\baselineskip\vbox{\hbox{\strut}}\end{varwidth}%
\sphinxstopmulticolumn
\\
\cline{2-5}\sphinxtablestrut{71}&
\sphinxAtStartPar
default
&\sphinxstartmulticolumn{3}%
\begin{varwidth}[t]{\sphinxcolwidth{3}{5}}
\par
\vskip-\baselineskip\vbox{\hbox{\strut}}\end{varwidth}%
\sphinxstopmulticolumn
\\
\cline{2-5}\sphinxtablestrut{71}&
\sphinxAtStartPar
items
&\sphinxstartmulticolumn{3}%
\begin{varwidth}[t]{\sphinxcolwidth{3}{5}}
\sphinxAtStartPar
{\hyperref[\detokenize{docs/advanced/tmap:setting}]{\sphinxcrossref{Setting}}}
\par
\vskip-\baselineskip\vbox{\hbox{\strut}}\end{varwidth}%
\sphinxstopmulticolumn
\\
\hline\sphinxmultirow{4}{78}{%
\begin{varwidth}[t]{\sphinxcolwidth{1}{5}}
\begin{itemize}
\item {} 
\sphinxAtStartPar
fromButton

\end{itemize}
\par
\vskip-\baselineskip\vbox{\hbox{\strut}}\end{varwidth}%
}%
&\sphinxstartmulticolumn{4}%
\begin{varwidth}[t]{\sphinxcolwidth{4}{5}}
\sphinxAtStartPar
If this is an integer, then the marker dataset is loaded from the n\sphinxhyphen{}th marker button.
\par
\vskip-\baselineskip\vbox{\hbox{\strut}}\end{varwidth}%
\sphinxstopmulticolumn
\\
\cline{2-5}\sphinxtablestrut{78}&
\sphinxAtStartPar
default
&\sphinxstartmulticolumn{3}%
\begin{varwidth}[t]{\sphinxcolwidth{3}{5}}
\sphinxAtStartPar
null
\par
\vskip-\baselineskip\vbox{\hbox{\strut}}\end{varwidth}%
\sphinxstopmulticolumn
\\
\cline{2-5}\sphinxtablestrut{78}&\sphinxmultirow{2}{82}{%
\begin{varwidth}[t]{\sphinxcolwidth{1}{5}}
\sphinxAtStartPar
anyOf
\par
\vskip-\baselineskip\vbox{\hbox{\strut}}\end{varwidth}%
}%
&
\sphinxAtStartPar
type
&\sphinxstartmulticolumn{2}%
\begin{varwidth}[t]{\sphinxcolwidth{2}{5}}
\sphinxAtStartPar
\sphinxstyleemphasis{integer}
\par
\vskip-\baselineskip\vbox{\hbox{\strut}}\end{varwidth}%
\sphinxstopmulticolumn
\\
\cline{3-5}\sphinxtablestrut{78}&\sphinxtablestrut{82}&
\sphinxAtStartPar
type
&\sphinxstartmulticolumn{2}%
\begin{varwidth}[t]{\sphinxcolwidth{2}{5}}
\sphinxAtStartPar
\sphinxstyleemphasis{null}
\par
\vskip-\baselineskip\vbox{\hbox{\strut}}\end{varwidth}%
\sphinxstopmulticolumn
\\
\hline
\sphinxAtStartPar
additionalProperties
&\sphinxstartmulticolumn{4}%
\begin{varwidth}[t]{\sphinxcolwidth{4}{5}}
\sphinxAtStartPar
False
\par
\vskip-\baselineskip\vbox{\hbox{\strut}}\end{varwidth}%
\sphinxstopmulticolumn
\\
\hline
\end{longtable}\sphinxatlongtableend\end{savenotes}


\subsection{RegionFile}
\label{\detokenize{docs/advanced/tmap:regionfile}}\label{\detokenize{docs/advanced/tmap:regionfile}}

\begin{savenotes}\sphinxattablestart
\centering
\begin{tabular}[t]{|*{5}{\X{1}{5}|}}
\hline

\sphinxAtStartPar
type
&\sphinxstartmulticolumn{4}%
\begin{varwidth}[t]{\sphinxcolwidth{4}{5}}
\sphinxAtStartPar
\sphinxstyleemphasis{object}
\par
\vskip-\baselineskip\vbox{\hbox{\strut}}\end{varwidth}%
\sphinxstopmulticolumn
\\
\hline\sphinxstartmulticolumn{5}%
\begin{varwidth}[t]{\sphinxcolwidth{5}{5}}
\sphinxAtStartPar
properties
\par
\vskip-\baselineskip\vbox{\hbox{\strut}}\end{varwidth}%
\sphinxstopmulticolumn
\\
\hline\sphinxmultirow{4}{4}{%
\begin{varwidth}[t]{\sphinxcolwidth{1}{5}}
\begin{itemize}
\item {} 
\sphinxAtStartPar
comment

\end{itemize}
\par
\vskip-\baselineskip\vbox{\hbox{\strut}}\end{varwidth}%
}%
&\sphinxstartmulticolumn{4}%
\begin{varwidth}[t]{\sphinxcolwidth{4}{5}}
\sphinxAtStartPar
Optional description text shown next to region button.
\par
\vskip-\baselineskip\vbox{\hbox{\strut}}\end{varwidth}%
\sphinxstopmulticolumn
\\
\cline{2-5}\sphinxtablestrut{4}&
\sphinxAtStartPar
default
&\sphinxstartmulticolumn{3}%
\begin{varwidth}[t]{\sphinxcolwidth{3}{5}}
\sphinxAtStartPar
null
\par
\vskip-\baselineskip\vbox{\hbox{\strut}}\end{varwidth}%
\sphinxstopmulticolumn
\\
\cline{2-5}\sphinxtablestrut{4}&\sphinxmultirow{2}{8}{%
\begin{varwidth}[t]{\sphinxcolwidth{1}{5}}
\sphinxAtStartPar
anyOf
\par
\vskip-\baselineskip\vbox{\hbox{\strut}}\end{varwidth}%
}%
&
\sphinxAtStartPar
type
&\sphinxstartmulticolumn{2}%
\begin{varwidth}[t]{\sphinxcolwidth{2}{5}}
\sphinxAtStartPar
\sphinxstyleemphasis{string}
\par
\vskip-\baselineskip\vbox{\hbox{\strut}}\end{varwidth}%
\sphinxstopmulticolumn
\\
\cline{3-5}\sphinxtablestrut{4}&\sphinxtablestrut{8}&
\sphinxAtStartPar
type
&\sphinxstartmulticolumn{2}%
\begin{varwidth}[t]{\sphinxcolwidth{2}{5}}
\sphinxAtStartPar
\sphinxstyleemphasis{null}
\par
\vskip-\baselineskip\vbox{\hbox{\strut}}\end{varwidth}%
\sphinxstopmulticolumn
\\
\hline\sphinxmultirow{3}{13}{%
\begin{varwidth}[t]{\sphinxcolwidth{1}{5}}
\begin{itemize}
\item {} 
\sphinxAtStartPar
autoLoad

\end{itemize}
\par
\vskip-\baselineskip\vbox{\hbox{\strut}}\end{varwidth}%
}%
&\sphinxstartmulticolumn{4}%
\begin{varwidth}[t]{\sphinxcolwidth{4}{5}}
\sphinxAtStartPar
If the regions should be automatically loaded when the TMAP project is opened. If this is false, the user instead has to click on the region button in the GUI to load the regions.
\par
\vskip-\baselineskip\vbox{\hbox{\strut}}\end{varwidth}%
\sphinxstopmulticolumn
\\
\cline{2-5}\sphinxtablestrut{13}&
\sphinxAtStartPar
type
&\sphinxstartmulticolumn{3}%
\begin{varwidth}[t]{\sphinxcolwidth{3}{5}}
\sphinxAtStartPar
\sphinxstyleemphasis{boolean}
\par
\vskip-\baselineskip\vbox{\hbox{\strut}}\end{varwidth}%
\sphinxstopmulticolumn
\\
\cline{2-5}\sphinxtablestrut{13}&
\sphinxAtStartPar
default
&\sphinxstartmulticolumn{3}%
\begin{varwidth}[t]{\sphinxcolwidth{3}{5}}
\sphinxAtStartPar
False
\par
\vskip-\baselineskip\vbox{\hbox{\strut}}\end{varwidth}%
\sphinxstopmulticolumn
\\
\hline\sphinxmultirow{4}{19}{%
\begin{varwidth}[t]{\sphinxcolwidth{1}{5}}
\begin{itemize}
\item {} 
\sphinxAtStartPar
\sphinxstylestrong{path}

\end{itemize}
\par
\vskip-\baselineskip\vbox{\hbox{\strut}}\end{varwidth}%
}%
&\sphinxstartmulticolumn{4}%
\begin{varwidth}[t]{\sphinxcolwidth{4}{5}}
\sphinxAtStartPar
Relative file path to GeoJSON file in which marker data is stored. If array of string, then a dropdown is created instead of a button.
\par
\vskip-\baselineskip\vbox{\hbox{\strut}}\end{varwidth}%
\sphinxstopmulticolumn
\\
\cline{2-5}\sphinxtablestrut{19}&\sphinxmultirow{3}{21}{%
\begin{varwidth}[t]{\sphinxcolwidth{1}{5}}
\sphinxAtStartPar
anyOf
\par
\vskip-\baselineskip\vbox{\hbox{\strut}}\end{varwidth}%
}%
&
\sphinxAtStartPar
type
&\sphinxstartmulticolumn{2}%
\begin{varwidth}[t]{\sphinxcolwidth{2}{5}}
\sphinxAtStartPar
\sphinxstyleemphasis{string}
\par
\vskip-\baselineskip\vbox{\hbox{\strut}}\end{varwidth}%
\sphinxstopmulticolumn
\\
\cline{3-5}\sphinxtablestrut{19}&\sphinxtablestrut{21}&
\sphinxAtStartPar
type
&\sphinxstartmulticolumn{2}%
\begin{varwidth}[t]{\sphinxcolwidth{2}{5}}
\sphinxAtStartPar
\sphinxstyleemphasis{array}
\par
\vskip-\baselineskip\vbox{\hbox{\strut}}\end{varwidth}%
\sphinxstopmulticolumn
\\
\cline{3-5}\sphinxtablestrut{19}&\sphinxtablestrut{21}&
\sphinxAtStartPar
items
&
\sphinxAtStartPar
type
&
\sphinxAtStartPar
\sphinxstyleemphasis{string}
\\
\hline\sphinxmultirow{3}{29}{%
\begin{varwidth}[t]{\sphinxcolwidth{1}{5}}
\begin{itemize}
\item {} 
\sphinxAtStartPar
settings

\end{itemize}
\par
\vskip-\baselineskip\vbox{\hbox{\strut}}\end{varwidth}%
}%
&
\sphinxAtStartPar
type
&\sphinxstartmulticolumn{3}%
\begin{varwidth}[t]{\sphinxcolwidth{3}{5}}
\sphinxAtStartPar
\sphinxstyleemphasis{array}
\par
\vskip-\baselineskip\vbox{\hbox{\strut}}\end{varwidth}%
\sphinxstopmulticolumn
\\
\cline{2-5}\sphinxtablestrut{29}&
\sphinxAtStartPar
default
&\sphinxstartmulticolumn{3}%
\begin{varwidth}[t]{\sphinxcolwidth{3}{5}}
\par
\vskip-\baselineskip\vbox{\hbox{\strut}}\end{varwidth}%
\sphinxstopmulticolumn
\\
\cline{2-5}\sphinxtablestrut{29}&
\sphinxAtStartPar
items
&\sphinxstartmulticolumn{3}%
\begin{varwidth}[t]{\sphinxcolwidth{3}{5}}
\sphinxAtStartPar
{\hyperref[\detokenize{docs/advanced/tmap:setting}]{\sphinxcrossref{Setting}}}
\par
\vskip-\baselineskip\vbox{\hbox{\strut}}\end{varwidth}%
\sphinxstopmulticolumn
\\
\hline
\sphinxAtStartPar
additionalProperties
&\sphinxstartmulticolumn{4}%
\begin{varwidth}[t]{\sphinxcolwidth{4}{5}}
\sphinxAtStartPar
False
\par
\vskip-\baselineskip\vbox{\hbox{\strut}}\end{varwidth}%
\sphinxstopmulticolumn
\\
\hline
\end{tabular}
\par
\sphinxattableend\end{savenotes}


\subsection{Setting}
\label{\detokenize{docs/advanced/tmap:setting}}\label{\detokenize{docs/advanced/tmap:setting}}

\begin{savenotes}\sphinxattablestart
\centering
\begin{tabular}[t]{|*{3}{\X{1}{3}|}}
\hline

\sphinxAtStartPar
type
&\sphinxstartmulticolumn{2}%
\begin{varwidth}[t]{\sphinxcolwidth{2}{3}}
\sphinxAtStartPar
\sphinxstyleemphasis{object}
\par
\vskip-\baselineskip\vbox{\hbox{\strut}}\end{varwidth}%
\sphinxstopmulticolumn
\\
\hline\sphinxstartmulticolumn{3}%
\begin{varwidth}[t]{\sphinxcolwidth{3}{3}}
\sphinxAtStartPar
properties
\par
\vskip-\baselineskip\vbox{\hbox{\strut}}\end{varwidth}%
\sphinxstopmulticolumn
\\
\hline\sphinxmultirow{2}{4}{%
\begin{varwidth}[t]{\sphinxcolwidth{1}{3}}
\begin{itemize}
\item {} 
\sphinxAtStartPar
\sphinxstylestrong{module}

\end{itemize}
\par
\vskip-\baselineskip\vbox{\hbox{\strut}}\end{varwidth}%
}%
&\sphinxstartmulticolumn{2}%
\begin{varwidth}[t]{\sphinxcolwidth{2}{3}}
\sphinxAtStartPar
Module where the function or property lies.
\par
\vskip-\baselineskip\vbox{\hbox{\strut}}\end{varwidth}%
\sphinxstopmulticolumn
\\
\cline{2-3}\sphinxtablestrut{4}&
\sphinxAtStartPar
type
&
\sphinxAtStartPar
\sphinxstyleemphasis{string}
\\
\hline\sphinxmultirow{2}{8}{%
\begin{varwidth}[t]{\sphinxcolwidth{1}{3}}
\begin{itemize}
\item {} 
\sphinxAtStartPar
\sphinxstylestrong{function}

\end{itemize}
\par
\vskip-\baselineskip\vbox{\hbox{\strut}}\end{varwidth}%
}%
&\sphinxstartmulticolumn{2}%
\begin{varwidth}[t]{\sphinxcolwidth{2}{3}}
\sphinxAtStartPar
Function or property of the given module.
\par
\vskip-\baselineskip\vbox{\hbox{\strut}}\end{varwidth}%
\sphinxstopmulticolumn
\\
\cline{2-3}\sphinxtablestrut{8}&
\sphinxAtStartPar
type
&
\sphinxAtStartPar
\sphinxstyleemphasis{string}
\\
\hline\sphinxstartmulticolumn{3}%
\begin{varwidth}[t]{\sphinxcolwidth{3}{3}}
\begin{itemize}
\item {} 
\sphinxAtStartPar
\sphinxstylestrong{value}

\end{itemize}
\par
\vskip-\baselineskip\vbox{\hbox{\strut}}\end{varwidth}%
\sphinxstopmulticolumn
\\
\hline
\sphinxAtStartPar
additionalProperties
&\sphinxstartmulticolumn{2}%
\begin{varwidth}[t]{\sphinxcolwidth{2}{3}}
\sphinxAtStartPar
False
\par
\vskip-\baselineskip\vbox{\hbox{\strut}}\end{varwidth}%
\sphinxstopmulticolumn
\\
\hline
\end{tabular}
\par
\sphinxattableend\end{savenotes}


\subsection{menuButton}
\label{\detokenize{docs/advanced/tmap:menubutton}}\label{\detokenize{docs/advanced/tmap:menubutton}}

\begin{savenotes}\sphinxattablestart
\centering
\begin{tabular}[t]{|*{5}{\X{1}{5}|}}
\hline

\sphinxAtStartPar
type
&\sphinxstartmulticolumn{4}%
\begin{varwidth}[t]{\sphinxcolwidth{4}{5}}
\sphinxAtStartPar
\sphinxstyleemphasis{object}
\par
\vskip-\baselineskip\vbox{\hbox{\strut}}\end{varwidth}%
\sphinxstopmulticolumn
\\
\hline\sphinxstartmulticolumn{5}%
\begin{varwidth}[t]{\sphinxcolwidth{5}{5}}
\sphinxAtStartPar
properties
\par
\vskip-\baselineskip\vbox{\hbox{\strut}}\end{varwidth}%
\sphinxstopmulticolumn
\\
\hline\sphinxmultirow{4}{4}{%
\begin{varwidth}[t]{\sphinxcolwidth{1}{5}}
\begin{itemize}
\item {} 
\sphinxAtStartPar
\sphinxstylestrong{text}

\end{itemize}
\par
\vskip-\baselineskip\vbox{\hbox{\strut}}\end{varwidth}%
}%
&\sphinxstartmulticolumn{4}%
\begin{varwidth}[t]{\sphinxcolwidth{4}{5}}
\sphinxAtStartPar
Text of the menu item. If list, then a nested menu is created.
\par
\vskip-\baselineskip\vbox{\hbox{\strut}}\end{varwidth}%
\sphinxstopmulticolumn
\\
\cline{2-5}\sphinxtablestrut{4}&\sphinxmultirow{3}{6}{%
\begin{varwidth}[t]{\sphinxcolwidth{1}{5}}
\sphinxAtStartPar
anyOf
\par
\vskip-\baselineskip\vbox{\hbox{\strut}}\end{varwidth}%
}%
&
\sphinxAtStartPar
type
&\sphinxstartmulticolumn{2}%
\begin{varwidth}[t]{\sphinxcolwidth{2}{5}}
\sphinxAtStartPar
\sphinxstyleemphasis{array}
\par
\vskip-\baselineskip\vbox{\hbox{\strut}}\end{varwidth}%
\sphinxstopmulticolumn
\\
\cline{3-5}\sphinxtablestrut{4}&\sphinxtablestrut{6}&
\sphinxAtStartPar
items
&
\sphinxAtStartPar
type
&
\sphinxAtStartPar
\sphinxstyleemphasis{string}
\\
\cline{3-5}\sphinxtablestrut{4}&\sphinxtablestrut{6}&
\sphinxAtStartPar
type
&\sphinxstartmulticolumn{2}%
\begin{varwidth}[t]{\sphinxcolwidth{2}{5}}
\sphinxAtStartPar
\sphinxstyleemphasis{string}
\par
\vskip-\baselineskip\vbox{\hbox{\strut}}\end{varwidth}%
\sphinxstopmulticolumn
\\
\hline\sphinxmultirow{2}{14}{%
\begin{varwidth}[t]{\sphinxcolwidth{1}{5}}
\begin{itemize}
\item {} 
\sphinxAtStartPar
\sphinxstylestrong{url}

\end{itemize}
\par
\vskip-\baselineskip\vbox{\hbox{\strut}}\end{varwidth}%
}%
&\sphinxstartmulticolumn{4}%
\begin{varwidth}[t]{\sphinxcolwidth{4}{5}}
\sphinxAtStartPar
Url of the menu item.
\par
\vskip-\baselineskip\vbox{\hbox{\strut}}\end{varwidth}%
\sphinxstopmulticolumn
\\
\cline{2-5}\sphinxtablestrut{14}&
\sphinxAtStartPar
type
&\sphinxstartmulticolumn{3}%
\begin{varwidth}[t]{\sphinxcolwidth{3}{5}}
\sphinxAtStartPar
\sphinxstyleemphasis{string}
\par
\vskip-\baselineskip\vbox{\hbox{\strut}}\end{varwidth}%
\sphinxstopmulticolumn
\\
\hline
\sphinxAtStartPar
additionalProperties
&\sphinxstartmulticolumn{4}%
\begin{varwidth}[t]{\sphinxcolwidth{4}{5}}
\sphinxAtStartPar
False
\par
\vskip-\baselineskip\vbox{\hbox{\strut}}\end{varwidth}%
\sphinxstopmulticolumn
\\
\hline
\end{tabular}
\par
\sphinxattableend\end{savenotes}


\subsection{Example of a .tmap file}
\label{\detokenize{docs/advanced/tmap:example-of-a-tmap-file}}
\begin{sphinxVerbatim}[commandchars=\\\{\}]
\PYG{p}{\PYGZob{}}
\PYG{+w}{    }\PYG{n+nt}{\PYGZdq{}filename\PYGZdq{}}\PYG{p}{:}\PYG{+w}{ }\PYG{l+s+s2}{\PYGZdq{}TissUUmaps\PYGZus{}Example.tmap\PYGZdq{}}\PYG{p}{,}
\PYG{+w}{    }\PYG{n+nt}{\PYGZdq{}layers\PYGZdq{}}\PYG{p}{:}\PYG{+w}{ }\PYG{p}{[}
\PYG{+w}{        }\PYG{p}{\PYGZob{}}
\PYG{+w}{            }\PYG{n+nt}{\PYGZdq{}name\PYGZdq{}}\PYG{p}{:}\PYG{+w}{ }\PYG{l+s+s2}{\PYGZdq{}Round1\PYGZus{}A.tif\PYGZdq{}}\PYG{p}{,}
\PYG{+w}{            }\PYG{n+nt}{\PYGZdq{}tileSource\PYGZdq{}}\PYG{p}{:}\PYG{+w}{ }\PYG{l+s+s2}{\PYGZdq{}images/Round1\PYGZus{}A.tif.dzi\PYGZdq{}}
\PYG{+w}{        }\PYG{p}{\PYGZcb{},}
\PYG{+w}{        }\PYG{p}{\PYGZob{}}
\PYG{+w}{            }\PYG{n+nt}{\PYGZdq{}name\PYGZdq{}}\PYG{p}{:}\PYG{+w}{ }\PYG{l+s+s2}{\PYGZdq{}Round1\PYGZus{}C.tif\PYGZdq{}}\PYG{p}{,}
\PYG{+w}{            }\PYG{n+nt}{\PYGZdq{}tileSource\PYGZdq{}}\PYG{p}{:}\PYG{+w}{ }\PYG{l+s+s2}{\PYGZdq{}images/Round1\PYGZus{}C.tif.dzi\PYGZdq{}}
\PYG{+w}{        }\PYG{p}{\PYGZcb{}}
\PYG{+w}{    }\PYG{p}{],}
\PYG{+w}{    }\PYG{n+nt}{\PYGZdq{}layerOpacities\PYGZdq{}}\PYG{p}{:}\PYG{+w}{ }\PYG{p}{\PYGZob{}}
\PYG{+w}{        }\PYG{n+nt}{\PYGZdq{}0\PYGZdq{}}\PYG{p}{:}\PYG{+w}{ }\PYG{l+s+s2}{\PYGZdq{}1\PYGZdq{}}\PYG{p}{,}
\PYG{+w}{        }\PYG{n+nt}{\PYGZdq{}1\PYGZdq{}}\PYG{p}{:}\PYG{+w}{ }\PYG{l+s+s2}{\PYGZdq{}1\PYGZdq{}}
\PYG{+w}{    }\PYG{p}{\PYGZcb{},}
\PYG{+w}{    }\PYG{n+nt}{\PYGZdq{}layerVisibilities\PYGZdq{}}\PYG{p}{:}\PYG{+w}{ }\PYG{p}{\PYGZob{}}
\PYG{+w}{        }\PYG{n+nt}{\PYGZdq{}0\PYGZdq{}}\PYG{p}{:}\PYG{+w}{ }\PYG{k+kc}{true}\PYG{p}{,}
\PYG{+w}{        }\PYG{n+nt}{\PYGZdq{}1\PYGZdq{}}\PYG{p}{:}\PYG{+w}{ }\PYG{k+kc}{false}\PYG{p}{,}
\PYG{+w}{    }\PYG{p}{\PYGZcb{},}
\PYG{+w}{    }\PYG{n+nt}{\PYGZdq{}layerFilters\PYGZdq{}}\PYG{p}{:}\PYG{+w}{ }\PYG{p}{\PYGZob{}}
\PYG{+w}{        }\PYG{n+nt}{\PYGZdq{}0\PYGZdq{}}\PYG{p}{:}\PYG{+w}{ }\PYG{p}{[}
\PYG{+w}{            }\PYG{p}{\PYGZob{}}
\PYG{+w}{                }\PYG{n+nt}{\PYGZdq{}name\PYGZdq{}}\PYG{p}{:}\PYG{+w}{ }\PYG{l+s+s2}{\PYGZdq{}Color\PYGZdq{}}\PYG{p}{,}
\PYG{+w}{                }\PYG{n+nt}{\PYGZdq{}value\PYGZdq{}}\PYG{p}{:}\PYG{+w}{ }\PYG{l+s+s2}{\PYGZdq{}0,100,0\PYGZdq{}}
\PYG{+w}{            }\PYG{p}{\PYGZcb{}}
\PYG{+w}{        }\PYG{p}{],}
\PYG{+w}{        }\PYG{n+nt}{\PYGZdq{}1\PYGZdq{}}\PYG{p}{:}\PYG{+w}{ }\PYG{p}{[}
\PYG{+w}{            }\PYG{p}{\PYGZob{}}
\PYG{+w}{                }\PYG{n+nt}{\PYGZdq{}name\PYGZdq{}}\PYG{p}{:}\PYG{+w}{ }\PYG{l+s+s2}{\PYGZdq{}Color\PYGZdq{}}\PYG{p}{,}
\PYG{+w}{                }\PYG{n+nt}{\PYGZdq{}value\PYGZdq{}}\PYG{p}{:}\PYG{+w}{ }\PYG{l+s+s2}{\PYGZdq{}0,100,0\PYGZdq{}}
\PYG{+w}{            }\PYG{p}{\PYGZcb{}}
\PYG{+w}{        }\PYG{p}{]}
\PYG{+w}{    }\PYG{p}{\PYGZcb{},}
\PYG{+w}{    }\PYG{n+nt}{\PYGZdq{}filters\PYGZdq{}}\PYG{p}{:}\PYG{+w}{ }\PYG{p}{[}
\PYG{+w}{        }\PYG{l+s+s2}{\PYGZdq{}Color\PYGZdq{}}
\PYG{+w}{    }\PYG{p}{],}
\PYG{+w}{    }\PYG{n+nt}{\PYGZdq{}compositeMode\PYGZdq{}}\PYG{p}{:}\PYG{+w}{ }\PYG{l+s+s2}{\PYGZdq{}lighter\PYGZdq{}}\PYG{p}{,}
\PYG{+w}{    }\PYG{n+nt}{\PYGZdq{}markerFiles\PYGZdq{}}\PYG{p}{:}\PYG{+w}{ }\PYG{p}{[}
\PYG{+w}{        }\PYG{p}{\PYGZob{}}
\PYG{+w}{            }\PYG{n+nt}{\PYGZdq{}autoLoad\PYGZdq{}}\PYG{p}{:}\PYG{+w}{ }\PYG{k+kc}{false}\PYG{p}{,}
\PYG{+w}{            }\PYG{n+nt}{\PYGZdq{}comment\PYGZdq{}}\PYG{p}{:}\PYG{+w}{ }\PYG{l+s+s2}{\PYGZdq{}\PYGZdq{}}\PYG{p}{,}
\PYG{+w}{            }\PYG{n+nt}{\PYGZdq{}expectedHeader\PYGZdq{}}\PYG{p}{:}\PYG{+w}{ }\PYG{p}{\PYGZob{}}
\PYG{+w}{                }\PYG{n+nt}{\PYGZdq{}X\PYGZdq{}}\PYG{p}{:}\PYG{+w}{ }\PYG{l+s+s2}{\PYGZdq{}global\PYGZus{}x\PYGZdq{}}\PYG{p}{,}
\PYG{+w}{                }\PYG{n+nt}{\PYGZdq{}Y\PYGZdq{}}\PYG{p}{:}\PYG{+w}{ }\PYG{l+s+s2}{\PYGZdq{}global\PYGZus{}y\PYGZdq{}}\PYG{p}{,}
\PYG{+w}{                }\PYG{n+nt}{\PYGZdq{}cb\PYGZus{}cmap\PYGZdq{}}\PYG{p}{:}\PYG{+w}{ }\PYG{l+s+s2}{\PYGZdq{}\PYGZdq{}}\PYG{p}{,}
\PYG{+w}{                }\PYG{n+nt}{\PYGZdq{}cb\PYGZus{}col\PYGZdq{}}\PYG{p}{:}\PYG{+w}{ }\PYG{l+s+s2}{\PYGZdq{}null\PYGZdq{}}\PYG{p}{,}
\PYG{+w}{                }\PYG{n+nt}{\PYGZdq{}cb\PYGZus{}gr\PYGZus{}dict\PYGZdq{}}\PYG{p}{:}\PYG{+w}{ }\PYG{l+s+s2}{\PYGZdq{}\PYGZdq{}}\PYG{p}{,}
\PYG{+w}{                }\PYG{n+nt}{\PYGZdq{}gb\PYGZus{}col\PYGZdq{}}\PYG{p}{:}\PYG{+w}{ }\PYG{l+s+s2}{\PYGZdq{}Gene\PYGZdq{}}\PYG{p}{,}
\PYG{+w}{                }\PYG{n+nt}{\PYGZdq{}gb\PYGZus{}name\PYGZdq{}}\PYG{p}{:}\PYG{+w}{ }\PYG{l+s+s2}{\PYGZdq{}\PYGZdq{}}\PYG{p}{,}
\PYG{+w}{                }\PYG{n+nt}{\PYGZdq{}opacity\PYGZdq{}}\PYG{p}{:}\PYG{+w}{ }\PYG{l+s+s2}{\PYGZdq{}1\PYGZdq{}}\PYG{p}{,}
\PYG{+w}{                }\PYG{n+nt}{\PYGZdq{}opacity\PYGZus{}col\PYGZdq{}}\PYG{p}{:}\PYG{+w}{ }\PYG{l+s+s2}{\PYGZdq{}null\PYGZdq{}}\PYG{p}{,}
\PYG{+w}{                }\PYG{n+nt}{\PYGZdq{}pie\PYGZus{}col\PYGZdq{}}\PYG{p}{:}\PYG{+w}{ }\PYG{l+s+s2}{\PYGZdq{}null\PYGZdq{}}\PYG{p}{,}
\PYG{+w}{                }\PYG{n+nt}{\PYGZdq{}pie\PYGZus{}dict\PYGZdq{}}\PYG{p}{:}\PYG{+w}{ }\PYG{l+s+s2}{\PYGZdq{}\PYGZdq{}}\PYG{p}{,}
\PYG{+w}{                }\PYG{n+nt}{\PYGZdq{}scale\PYGZus{}col\PYGZdq{}}\PYG{p}{:}\PYG{+w}{ }\PYG{l+s+s2}{\PYGZdq{}null\PYGZdq{}}\PYG{p}{,}
\PYG{+w}{                }\PYG{n+nt}{\PYGZdq{}scale\PYGZus{}factor\PYGZdq{}}\PYG{p}{:}\PYG{+w}{ }\PYG{l+s+s2}{\PYGZdq{}0.5\PYGZdq{}}\PYG{p}{,}
\PYG{+w}{                }\PYG{n+nt}{\PYGZdq{}shape\PYGZus{}col\PYGZdq{}}\PYG{p}{:}\PYG{+w}{ }\PYG{l+s+s2}{\PYGZdq{}null\PYGZdq{}}\PYG{p}{,}
\PYG{+w}{                }\PYG{n+nt}{\PYGZdq{}shape\PYGZus{}fixed\PYGZdq{}}\PYG{p}{:}\PYG{+w}{ }\PYG{l+s+s2}{\PYGZdq{}cross\PYGZdq{}}\PYG{p}{,}
\PYG{+w}{                }\PYG{n+nt}{\PYGZdq{}shape\PYGZus{}gr\PYGZus{}dict\PYGZdq{}}\PYG{p}{:}\PYG{+w}{ }\PYG{l+s+s2}{\PYGZdq{}\PYGZdq{}}\PYG{p}{,}
\PYG{+w}{                }\PYG{n+nt}{\PYGZdq{}tooltip\PYGZus{}fmt\PYGZdq{}}\PYG{p}{:}\PYG{+w}{ }\PYG{l+s+s2}{\PYGZdq{}\PYGZdq{}}
\PYG{+w}{            }\PYG{p}{\PYGZcb{},}
\PYG{+w}{            }\PYG{n+nt}{\PYGZdq{}expectedRadios\PYGZdq{}}\PYG{p}{:}\PYG{+w}{ }\PYG{p}{\PYGZob{}}
\PYG{+w}{                }\PYG{n+nt}{\PYGZdq{}cb\PYGZus{}col\PYGZdq{}}\PYG{p}{:}\PYG{+w}{ }\PYG{k+kc}{false}\PYG{p}{,}
\PYG{+w}{                }\PYG{n+nt}{\PYGZdq{}cb\PYGZus{}gr\PYGZdq{}}\PYG{p}{:}\PYG{+w}{ }\PYG{k+kc}{true}\PYG{p}{,}
\PYG{+w}{                }\PYG{n+nt}{\PYGZdq{}cb\PYGZus{}gr\PYGZus{}dict\PYGZdq{}}\PYG{p}{:}\PYG{+w}{ }\PYG{k+kc}{false}\PYG{p}{,}
\PYG{+w}{                }\PYG{n+nt}{\PYGZdq{}cb\PYGZus{}gr\PYGZus{}key\PYGZdq{}}\PYG{p}{:}\PYG{+w}{ }\PYG{k+kc}{true}\PYG{p}{,}
\PYG{+w}{                }\PYG{n+nt}{\PYGZdq{}cb\PYGZus{}gr\PYGZus{}rand\PYGZdq{}}\PYG{p}{:}\PYG{+w}{ }\PYG{k+kc}{false}\PYG{p}{,}
\PYG{+w}{                }\PYG{n+nt}{\PYGZdq{}pie\PYGZus{}check\PYGZdq{}}\PYG{p}{:}\PYG{+w}{ }\PYG{k+kc}{false}\PYG{p}{,}
\PYG{+w}{                }\PYG{n+nt}{\PYGZdq{}scale\PYGZus{}check\PYGZdq{}}\PYG{p}{:}\PYG{+w}{ }\PYG{k+kc}{false}\PYG{p}{,}
\PYG{+w}{                }\PYG{n+nt}{\PYGZdq{}shape\PYGZus{}col\PYGZdq{}}\PYG{p}{:}\PYG{+w}{ }\PYG{k+kc}{false}\PYG{p}{,}
\PYG{+w}{                }\PYG{n+nt}{\PYGZdq{}shape\PYGZus{}fixed\PYGZdq{}}\PYG{p}{:}\PYG{+w}{ }\PYG{k+kc}{false}\PYG{p}{,}
\PYG{+w}{                }\PYG{n+nt}{\PYGZdq{}shape\PYGZus{}gr\PYGZdq{}}\PYG{p}{:}\PYG{+w}{ }\PYG{k+kc}{true}\PYG{p}{,}
\PYG{+w}{                }\PYG{n+nt}{\PYGZdq{}shape\PYGZus{}gr\PYGZus{}dict\PYGZdq{}}\PYG{p}{:}\PYG{+w}{ }\PYG{k+kc}{false}\PYG{p}{,}
\PYG{+w}{                }\PYG{n+nt}{\PYGZdq{}shape\PYGZus{}gr\PYGZus{}rand\PYGZdq{}}\PYG{p}{:}\PYG{+w}{ }\PYG{k+kc}{true}\PYG{p}{,}
\PYG{+w}{                }\PYG{n+nt}{\PYGZdq{}opacity\PYGZus{}check\PYGZdq{}}\PYG{p}{:}\PYG{+w}{ }\PYG{k+kc}{false}
\PYG{+w}{            }\PYG{p}{\PYGZcb{},}
\PYG{+w}{            }\PYG{n+nt}{\PYGZdq{}name\PYGZdq{}}\PYG{p}{:}\PYG{+w}{ }\PYG{l+s+s2}{\PYGZdq{} markers\PYGZdq{}}\PYG{p}{,}
\PYG{+w}{            }\PYG{n+nt}{\PYGZdq{}path\PYGZdq{}}\PYG{p}{:}\PYG{+w}{ }\PYG{l+s+s2}{\PYGZdq{}./istdeco\PYGZus{}codes\PYGZus{}n.csv\PYGZdq{}}\PYG{p}{,}
\PYG{+w}{            }\PYG{n+nt}{\PYGZdq{}title\PYGZdq{}}\PYG{p}{:}\PYG{+w}{ }\PYG{l+s+s2}{\PYGZdq{}Download markers\PYGZdq{}}\PYG{p}{,}
\PYG{+w}{            }\PYG{n+nt}{\PYGZdq{}uid\PYGZdq{}}\PYG{p}{:}\PYG{+w}{ }\PYG{l+s+s2}{\PYGZdq{}uniquetab\PYGZdq{}}
\PYG{+w}{        }\PYG{p}{\PYGZcb{}}
\PYG{+w}{    }\PYG{p}{],}
\PYG{+w}{    }\PYG{n+nt}{\PYGZdq{}regions\PYGZdq{}}\PYG{p}{:}\PYG{+w}{ }\PYG{p}{\PYGZob{}\PYGZcb{},}
\PYG{+w}{    }\PYG{n+nt}{\PYGZdq{}plugins\PYGZdq{}}\PYG{p}{:}\PYG{+w}{ }\PYG{p}{[}
\PYG{+w}{        }\PYG{l+s+s2}{\PYGZdq{}Spot\PYGZus{}Inspector\PYGZdq{}}
\PYG{+w}{    }\PYG{p}{],}
\PYG{+w}{    }\PYG{n+nt}{\PYGZdq{}hideTabs\PYGZdq{}}\PYG{p}{:}\PYG{+w}{ }\PYG{k+kc}{true}\PYG{p}{,}
\PYG{+w}{    }\PYG{n+nt}{\PYGZdq{}settings\PYGZdq{}}\PYG{p}{:}\PYG{+w}{ }\PYG{p}{[]}
\PYG{p}{\PYGZcb{}}
\end{sphinxVerbatim}

\sphinxstepscope


\chapter{Support}
\label{\detokenize{docs/support/index:support}}\label{\detokenize{docs/support/index::doc}}

\section{How to seek support and discussion}
\label{\detokenize{docs/support/index:how-to-seek-support-and-discussion}}
\sphinxAtStartPar
If you want to ask questions about TissUUmaps please visit forum.image.sc.


\section{How to issue an error on GitHub}
\label{\detokenize{docs/support/index:how-to-issue-an-error-on-github}}
\sphinxAtStartPar
If you want to report a bug, please do so at issues on GitHub.



\renewcommand{\indexname}{Index}
\printindex
\end{document}