%% Generated by Sphinx.
\def\sphinxdocclass{report}
\documentclass[letterpaper,10pt,english,openany,oneside]{sphinxmanual}
\ifdefined\pdfpxdimen
   \let\sphinxpxdimen\pdfpxdimen\else\newdimen\sphinxpxdimen
\fi \sphinxpxdimen=.75bp\relax
\ifdefined\pdfimageresolution
    \pdfimageresolution= \numexpr \dimexpr1in\relax/\sphinxpxdimen\relax
\fi
%% let collapsible pdf bookmarks panel have high depth per default
\PassOptionsToPackage{bookmarksdepth=5}{hyperref}

\PassOptionsToPackage{warn}{textcomp}
\usepackage[utf8]{inputenc}
\ifdefined\DeclareUnicodeCharacter
% support both utf8 and utf8x syntaxes
  \ifdefined\DeclareUnicodeCharacterAsOptional
    \def\sphinxDUC#1{\DeclareUnicodeCharacter{"#1}}
  \else
    \let\sphinxDUC\DeclareUnicodeCharacter
  \fi
  \sphinxDUC{00A0}{\nobreakspace}
  \sphinxDUC{2500}{\sphinxunichar{2500}}
  \sphinxDUC{2502}{\sphinxunichar{2502}}
  \sphinxDUC{2514}{\sphinxunichar{2514}}
  \sphinxDUC{251C}{\sphinxunichar{251C}}
  \sphinxDUC{2572}{\textbackslash}
\fi
\usepackage{cmap}
\usepackage[T1]{fontenc}
\usepackage{amsmath,amssymb,amstext}
\usepackage{babel}



\usepackage{tgtermes}
\usepackage{tgheros}
\renewcommand{\ttdefault}{txtt}



\usepackage[Bjarne]{fncychap}
\usepackage{sphinx}

\fvset{fontsize=auto}
\usepackage{geometry}


% Include hyperref last.
\usepackage{hyperref}
% Fix anchor placement for figures with captions.
\usepackage{hypcap}% it must be loaded after hyperref.
% Set up styles of URL: it should be placed after hyperref.
\urlstyle{same}

\addto\captionsenglish{\renewcommand{\contentsname}{Contents:}}

\usepackage{sphinxmessages}
\setcounter{tocdepth}{1}



\title{TissUUmaps}
\date{May 10, 2022}
\release{3.0}
\author{Nicolas Pielawski\and Axel Andersson\and Christophe Avenel\and Andrea Behanova\and Eduard Chelebian\and Anna Klemm\and Fredrik Nysjö\and Leslie Solorzano\and Carolina Wählby}
\newcommand{\sphinxlogo}{\vbox{}}
\renewcommand{\releasename}{Release}
\makeindex
\begin{document}

\pagestyle{empty}
\sphinxmaketitle
\pagestyle{plain}
\sphinxtableofcontents
\pagestyle{normal}
\phantomsection\label{\detokenize{index::doc}}


\sphinxAtStartPar
This page hosts the documentation for TissUUmaps 3.0. You can find a pdf version of this documentation \sphinxhref{https://tissuumaps.github.io/TissUUmaps-docs/index.pdf}{here}.

\sphinxAtStartPar
For more information on the TissUUmaps project, including \sphinxhref{https://tissuumaps.github.io/tutorials/}{video tutorials} and \sphinxhref{https://tissuumaps.github.io/gallery/}{demos}, visit our website: \sphinxurl{https://tissuumaps.github.io}.

\begin{sphinxShadowBox}
\sphinxstylesidebartitle{Work in progress!}

\sphinxAtStartPar
We are working actively on writing this documentation, more content will be available soon!
\end{sphinxShadowBox}

\sphinxstepscope


\chapter{Introduction}
\label{\detokenize{docs/intro/index:introduction}}\label{\detokenize{docs/intro/index::doc}}
\sphinxstepscope


\section{About TissUUmaps}
\label{\detokenize{docs/intro/about:about-tissuumaps}}\label{\detokenize{docs/intro/about::doc}}
\sphinxAtStartPar
\sphinxhref{https://tissuumaps.github.io/}{TissUUmaps} is a free and open source browser\sphinxhyphen{}based tool for GPU\sphinxhyphen{}accelerated visualization and interactive exploration of tens of millions of datapoints overlaying tissue samples. Users can visualize markers and regions, explore spatial statistics and quantitative analyses of tissue morphology, and assess the quality of decoding in situ transcriptomics data. TissUUmaps provides instant multi\sphinxhyphen{}resolution image viewing, can be customized, shared, and also integrated in Jupyter Notebooks. We envision TissUUmaps to contribute to broader dissemination and flexible sharing of large\sphinxhyphen{}scale spatial omics data.

\sphinxAtStartPar
Currently, microscopy data can be cumbersome to share: physically transferring the images is often necessary and dedicated software must be installed. Instead, researchers can now share their findings with a simple link to a website running TissUUmaps. The images are loaded in real time, together with annotations, markers, and masks that may also be modified by the user. We also provide tools for quality control and image processing. The software is designed to display and interact with images at multiple resolutions and large numbers of markers, especially data from spatially resolved omics techniques and tissue atlases. TissUUmaps is compatible with many different bioimage informatics tools, and provides new ways to develop insights when exploring and sharing data.

\sphinxAtStartPar
You can access the \sphinxhref{https://tissuumaps.github.io/gallery/}{TissUUmaps project gallery} with interactive examples to explore data from in situ sequencing and spatial transcriptomics experiments and view localized quantification of cell and tissue morphology, including links to publications. For seeing examples of TissUUmaps compatibility with other platforms you can access the \sphinxhref{https://tissuumaps.github.io/tutorials/}{tutorials page}.

\sphinxstepscope


\section{Installation}
\label{\detokenize{docs/intro/installation:installation}}\label{\detokenize{docs/intro/installation::doc}}
\sphinxAtStartPar
\sphinxhref{https://tissuumaps.github.io/}{TissUUmaps} is a browser\sphinxhyphen{}based tool for fast visualization and exploration of millions of data points overlaying a tissue sample. TissUUmaps can be used as a web service or locally in your computer, and allows users to share regions of interest and local statistics.


\subsection{Windows installation}
\label{\detokenize{docs/intro/installation:windows-installation}}\begin{enumerate}
\sphinxsetlistlabels{\arabic}{enumi}{enumii}{}{.}%
\item {} 
\sphinxAtStartPar
Download the Windows Installer from \sphinxhref{https://github.com/TissUUmaps/TissUUmaps/releases/latest}{the last release} and install it. Note that the installer is not signed yet and may trigger warnings from the browser and from the firewall. You can safely pass these warnings.

\end{enumerate}


\subsection{PIP installation (for Linux and Mac)}
\label{\detokenize{docs/intro/installation:pip-installation-for-linux-and-mac}}\begin{enumerate}
\sphinxsetlistlabels{\arabic}{enumi}{enumii}{}{.}%
\item {} 
\sphinxAtStartPar
Install \sphinxcode{\sphinxupquote{libvips}} for your system: \sphinxurl{https://www.libvips.org/install.html}

\sphinxAtStartPar
An easy way to install \sphinxcode{\sphinxupquote{libvips}} is to use an \sphinxhref{https://docs.anaconda.com/anaconda/install/index.html}{Anaconda} environment with \sphinxcode{\sphinxupquote{libvips}}:

\begin{sphinxVerbatim}[commandchars=\\\{\}]
conda create \PYGZhy{}y \PYGZhy{}n tissuumaps\PYGZus{}env \PYGZhy{}c conda\PYGZhy{}forge \PYG{n+nv}{python}\PYG{o}{=}\PYG{l+m}{3}.9 libvips
conda activate tissuumaps\PYGZus{}env
\end{sphinxVerbatim}

\item {} 
\sphinxAtStartPar
Install the TissUUmaps library using \sphinxcode{\sphinxupquote{pip}}:

\begin{sphinxVerbatim}[commandchars=\\\{\}]
pip install \PYG{l+s+s2}{\PYGZdq{}TissUUmaps[full]\PYGZdq{}}
\end{sphinxVerbatim}

\item {} 
\sphinxAtStartPar
Start the TissUUmaps user interface:

\begin{sphinxVerbatim}[commandchars=\\\{\}]
tissuumaps
\end{sphinxVerbatim}

\item {} 
\sphinxAtStartPar
Or start TissUUmaps as a local server:

\begin{sphinxVerbatim}[commandchars=\\\{\}]
tissuumaps\PYGZus{}server path\PYGZus{}to\PYGZus{}your\PYGZus{}images
\end{sphinxVerbatim}

\sphinxAtStartPar
And open \sphinxurl{http://127.0.0.1:5000/} in your favorite browser.

\end{enumerate}

\sphinxstepscope


\section{Citing TissUUmaps}
\label{\detokenize{docs/intro/citing:citing-tissuumaps}}\label{\detokenize{docs/intro/citing::doc}}
\sphinxAtStartPar
Please cite our \sphinxhref{https://www.biorxiv.org/content/10.1101/2022.01.28.478131v1}{preprint} on bioRxiv if using TissUUmaps in your work:

\sphinxAtStartPar
\sphinxstylestrong{TissUUmaps 3: Interactive visualization and quality assessment of large\sphinxhyphen{}scale spatial omics data.} \sphinxstyleemphasis{Nicolas Pielawski, Axel Andersson, Christophe Avenel, Andrea Behanova, Eduard Chelebian, Anna Klemm, Fredrik Nysjö, Leslie Solorzano, Carolina Wählby,} bioRxiv 2022.01.28.478131; doi: \sphinxurl{https://doi.org/10.1101/2022.01.28.478131}.

\sphinxstepscope


\section{Changelog}
\label{\detokenize{docs/intro/versions:changelog}}\label{\detokenize{docs/intro/versions::doc}}

\subsection{3.0.9}
\label{\detokenize{docs/intro/versions:id1}}\begin{itemize}
\item {} 
\sphinxAtStartPar
Move to webgl2

\item {} 
\sphinxAtStartPar
Add Open Recent sub menu in File menu

\item {} 
\sphinxAtStartPar
Fix path for linux and mac in server mode

\item {} 
\sphinxAtStartPar
Minor fixes

\end{itemize}


\subsection{3.0.8.9}
\label{\detokenize{docs/intro/versions:id2}}\begin{itemize}
\item {} 
\sphinxAtStartPar
Make it possible to update to newer version of plugins

\item {} 
\sphinxAtStartPar
Fully support \textendash{}debug option in command line

\item {} 
\sphinxAtStartPar
Add tooltip title for piecharts

\item {} 
\sphinxAtStartPar
Add documentation https://tissuumaps.github.io/TissUUmaps\sphinxhyphen{}docs/

\item {} 
\sphinxAtStartPar
Fix marker picking when pixel ratio != 1

\item {} 
\sphinxAtStartPar
Other minor fixes and cleaning

\end{itemize}


\subsection{3.0.8.5}
\label{\detokenize{docs/intro/versions:id3}}\begin{itemize}
\item {} 
\sphinxAtStartPar
Minor fixes.

\end{itemize}


\subsection{3.0.8.4}
\label{\detokenize{docs/intro/versions:id4}}\begin{itemize}
\item {} 
\sphinxAtStartPar
Add tiling to viewport capture for higher resolution output

\item {} 
\sphinxAtStartPar
Increase resolution of markers on high resolution devices

\item {} 
\sphinxAtStartPar
Fix jumps on pan with mouse gesture (mobile)

\item {} 
\sphinxAtStartPar
Add fix for bright image canvas on Safari

\item {} 
\sphinxAtStartPar
Add an option to remove markers’ outlines.

\end{itemize}


\subsection{3.0.8.3}
\label{\detokenize{docs/intro/versions:id5}}\begin{itemize}
\item {} 
\sphinxAtStartPar
Fix png artifact in Firefox, by generating jpg tiles.

\end{itemize}


\subsection{3.0.8.2}
\label{\detokenize{docs/intro/versions:id6}}\begin{itemize}
\item {} 
\sphinxAtStartPar
Add high resolution capture of viewport, up to 4096x4096 pixels.

\end{itemize}


\subsection{3.0.8.1}
\label{\detokenize{docs/intro/versions:id7}}\begin{itemize}
\item {} 
\sphinxAtStartPar
Fix multiple dataset alignment when no background image

\end{itemize}


\subsection{3.0.8}
\label{\detokenize{docs/intro/versions:id8}}\begin{itemize}
\item {} 
\sphinxAtStartPar
Fix black images generated by VIPS

\item {} 
\sphinxAtStartPar
Fix Linux and Mac open of captures

\item {} 
\sphinxAtStartPar
Auto save datasets as buttons when saving tmap projects

\item {} 
\sphinxAtStartPar
Add \sphinxcode{\sphinxupquote{mpp}} (microns per pixel) option in tmap files, to add scale bar to viewer

\item {} 
\sphinxAtStartPar
Make region line thickness depend on zoom level

\item {} 
\sphinxAtStartPar
Add compatibility with JupyterLab

\item {} 
\sphinxAtStartPar
Add opacity per marker option

\end{itemize}


\subsection{3.0.7}
\label{\detokenize{docs/intro/versions:id9}}\begin{itemize}
\item {} 
\sphinxAtStartPar
Add menu to load plugins through an update\sphinxhyphen{}site

\end{itemize}


\subsection{3.0.6}
\label{\detokenize{docs/intro/versions:id10}}\begin{itemize}
\item {} 
\sphinxAtStartPar
Fix multiple plugins opening always last plugin

\item {} 
\sphinxAtStartPar
Move to OpenSeadragon 3.0.0

\item {} 
\sphinxAtStartPar
Add tooltip format in Advanced Settings

\item {} 
\sphinxAtStartPar
Add drag and drop to open CSV files and images

\item {} 
\sphinxAtStartPar
Add “Add layer” button for flask version

\item {} 
\sphinxAtStartPar
Add viewport capture

\end{itemize}


\subsection{3.0.5}
\label{\detokenize{docs/intro/versions:id11}}\begin{itemize}
\item {} 
\sphinxAtStartPar
Move csv loading to Papa Parse streaming, to allow better memory management

\end{itemize}


\subsection{3.0.4}
\label{\detokenize{docs/intro/versions:id12}}\begin{itemize}
\item {} 
\sphinxAtStartPar
Add filtering of markers

\end{itemize}


\subsection{3.0}
\label{\detokenize{docs/intro/versions:id13}}\begin{itemize}
\item {} 
\sphinxAtStartPar
Add tissuumaps.jupyter module

\end{itemize}

\sphinxstepscope


\chapter{Getting started}
\label{\detokenize{docs/starting/index:getting-started}}\label{\detokenize{docs/starting/index::doc}}
\sphinxstepscope


\section{Images}
\label{\detokenize{docs/starting/images:images}}\label{\detokenize{docs/starting/images::doc}}

\subsection{Supported image formats}
\label{\detokenize{docs/starting/images:supported-image-formats}}
\sphinxAtStartPar
TissUUmaps can read whole slide images in any format recognized by the OpenSlide library:
\begin{itemize}
\item {} 
\sphinxAtStartPar
Aperio (.svs, .tif)

\item {} 
\sphinxAtStartPar
Hamamatsu (.ndpi, .vms, .vmu)

\item {} 
\sphinxAtStartPar
Leica (.scn)

\item {} 
\sphinxAtStartPar
MIRAX (.mrxs)

\item {} 
\sphinxAtStartPar
Philips (.tiff)

\item {} 
\sphinxAtStartPar
Sakura (.svslide)

\item {} 
\sphinxAtStartPar
Trestle (.tif)

\item {} 
\sphinxAtStartPar
Ventana (.bif, .tif)

\item {} 
\sphinxAtStartPar
Generic tiled TIFF (.tif)

\end{itemize}

\sphinxAtStartPar
TissUUmaps will automatically convert any other format into a pyramidal tiff (in a temporary \sphinxcode{\sphinxupquote{.tissuumaps}} folder created in the original image folder) using vips.

\sphinxAtStartPar
If your image fails to open, try converting it to \sphinxcode{\sphinxupquote{tif}} format using an external tool.


\subsection{Load images}
\label{\detokenize{docs/starting/images:load-images}}
\sphinxAtStartPar
You can load the images when you select the Image layer tab as you can see in the figure below:
\sphinxincludegraphics{{image_layers}.png}

\sphinxAtStartPar
Then click the button Add image layer and select the desired image from your computer. Subsequently, the image is listed in the Image layer tab. You can load several images into TissUUmaps.
\sphinxincludegraphics{{image_layers_many}.png}

\sphinxAtStartPar
You can also drag and drop the image from file explorer into TissUUmaps.
\sphinxincludegraphics{{drag_drop_image}.png}


\subsection{Load images using TissUUmaps server}
\label{\detokenize{docs/starting/images:load-images-using-tissuumaps-server}}
\sphinxAtStartPar
If you are running TissUUmaps in server mode and not through the GUI, you must specify an image folder in the command line:

\begin{sphinxVerbatim}[commandchars=\\\{\}]
python \PYGZhy{}m tissuumaps \PYG{l+s+s2}{\PYGZdq{}/home/username/Documents/myImages/\PYGZdq{}} \PYGZhy{}p \PYG{l+m}{5005}
\end{sphinxVerbatim}

\sphinxAtStartPar
You can then access your images from your web browser by accessing the url \sphinxurl{http://localhost:5005}, and using the \sphinxcode{\sphinxupquote{File \textgreater{} Open}} menu.

\sphinxAtStartPar
\sphinxincludegraphics{{server_open_menu1}.png}

\sphinxAtStartPar
\sphinxincludegraphics{{server_open_menu2}.png}


\subsection{Apply filters}
\label{\detokenize{docs/starting/images:apply-filters}}
\sphinxAtStartPar
You can apply several filters to the images. The ones we can be adjusted by default are saturation, brightness, and contrast. Additionally, when opening the Filter settings menu, there are various other filters, such as exposure, noise, erosion, etc. When you check their box, they are automatically added to the filter panel above. The filter’s sliders can be adjusted so that the filter is applied at the desired intensity. Another option in filter settings is merging mode (bottom part), where you can merge the channels as a composite.

\sphinxAtStartPar
\sphinxincludegraphics{{Filters}.png}

\sphinxstepscope


\section{Markers}
\label{\detokenize{docs/starting/markers:markers}}\label{\detokenize{docs/starting/markers::doc}}

\subsection{Supported marker format}
\label{\detokenize{docs/starting/markers:supported-marker-format}}
\sphinxAtStartPar
TissUUmaps can read CSV (Comma Separated Values) files with a header row, and at least spatial coordinate columns (X and Y). CSV files are not limited in the number of columns or number of rows. Other columns can contain information for displaying markers (key to group markers, color, size, shape, piecharts, etc.)

\sphinxAtStartPar
CSV files can be exported from any spreadsheet program, or any programming language (Python, R, etc.)


\subsection{Load markers}
\label{\detokenize{docs/starting/markers:load-markers}}
\sphinxAtStartPar
You can load the markers when you select the \sphinxstyleemphasis{Markers} tab and click the button + as you can see in the figure below. You can click the plus several times to load various marker files.
\sphinxincludegraphics{{markers}.png}

\sphinxAtStartPar
You can also load markers directly using drag and drop from a File Explorer if you are using the TissUUmaps GUI.


\subsection{Markers settings}
\label{\detokenize{docs/starting/markers:markers-settings}}
\sphinxAtStartPar
Before the markers are displayed you have to set up the markers settings.


\subsubsection{File and coordinates}
\label{\detokenize{docs/starting/markers:file-and-coordinates}}
\sphinxAtStartPar
The first step is to select the desired file from your computer under the tab \sphinxstyleemphasis{File and coordinates \sphinxhyphen{} Choose file}.

\sphinxAtStartPar
\sphinxincludegraphics{{load_markers}.png}

\sphinxAtStartPar
You can change the \sphinxstyleemphasis{Tab name} to the desired name, so it is easier to navigate between them when there are more tabs.

\sphinxAtStartPar
\sphinxincludegraphics{{Tab_name}.png}

\sphinxAtStartPar
The next step is to select the column names from the .csv file corresponding to the X and Y coordinates.

\sphinxAtStartPar
\sphinxincludegraphics{{XY_coordinates}.png}


\subsubsection{Render options}
\label{\detokenize{docs/starting/markers:render-options}}
\sphinxAtStartPar
Here, you can define a \sphinxstyleemphasis{key to group by}, what is a column from the .csv file which will be used to display the dataset grouped by different colors and shapes of the markers. In this example, we use the column \sphinxstyleemphasis{genes}, where different colors and shapes of markers represent different genes.

\sphinxAtStartPar
\sphinxincludegraphics{{group_by}.png}

\sphinxAtStartPar
There is an option to display an extra column, for example when the data are clustered but you want to see the original genes and also the cluster names.

\sphinxAtStartPar
\sphinxincludegraphics{{group_by2}.png}

\sphinxAtStartPar
In \sphinxstyleemphasis{Color options}, you can select to color by groups where each group has a different color. Then on the right side, you can select the color palette:
\begin{itemize}
\item {} 
\sphinxAtStartPar
Key value \sphinxhyphen{} Colors are generated from the name of the group (first 4 letters). Groups starting with the same letter have similar colors.

\item {} 
\sphinxAtStartPar
Randomly \sphinxhyphen{} Colors are generated randomly.

\item {} 
\sphinxAtStartPar
Dictionary \sphinxhyphen{} you can insert a specific dictionary in the text area which will be used for generating the colors.

\end{itemize}

\sphinxAtStartPar
\sphinxincludegraphics{{Color_options}.png}

\sphinxAtStartPar
If you want to \sphinxstyleemphasis{color by markers}, you have to select the column from the .csv file which will be used to create the colors, and the colormap, but only if the color column is numeral.

\sphinxAtStartPar
\sphinxincludegraphics{{Color_options2}.png}


\subsubsection{Advanced options}
\label{\detokenize{docs/starting/markers:advanced-options}}
\sphinxAtStartPar
TissUUmaps tool contains also advanced options when working with the data. The first one is adjustable marker size. This is usually done in the right upper corner of the visualization panel. However, in the advanced setting, the user can change the size factor of the slider to any value.

\sphinxAtStartPar
\sphinxincludegraphics{{Advanced_size}.png}

\sphinxAtStartPar
Additionally, there can be used a different size per marker based on a selected column. In the example below, I  chose column \sphinxstyleemphasis{counts} which represents the number of counts in that cell (marker). This means that a larger marker represents a cell that contains more counts in it.

\sphinxAtStartPar
\sphinxincludegraphics{{Advanced_size_ex}.png}

\sphinxAtStartPar
Another advanced option is the choice to display markers as pie\sphinxhyphen{}charts, it can represent the probability of that marker belonging to different groups. The user needs to select the \sphinxstyleemphasis{pie\sphinxhyphen{}chart column}, which contains mentioned probabilities for all the markers. All the probabilities for that specific marker need to be in a row divided by a semicolon.

\sphinxAtStartPar
\sphinxincludegraphics{{Advanced_pie}.png}

\sphinxAtStartPar
In the example below the pie\sphinxhyphen{}charts represent the probability of the marker being of each cell type. In the left upper corner can be seen the legend of the cell types. By default, there are only 10 colors so the colors are used in the loop. This can be changed by using pie\sphinxhyphen{}chart colors from a dictionary.

\sphinxAtStartPar
\sphinxincludegraphics{{Advanced_pie_ex}.png}

\sphinxAtStartPar
The shape of the markers can also be changed. By default, it is set up to be selected by the group, which means that each group has a different marker shape chosen from the list of shapes iteratively. The user can also pre\sphinxhyphen{}define the shapes from the dictionary to ensure visualization robustness in different sessions.

\sphinxAtStartPar
\sphinxincludegraphics{{Advanced_shape}.png}

\sphinxAtStartPar
In the example below each group has specific color as well as a specific marker shape.

\sphinxAtStartPar
\sphinxincludegraphics{{Advanced_shape_ex}.png}

\sphinxAtStartPar
Another option for the marker shape is \sphinxstyleemphasis{shape by marker}. Here, the user needs to select a column with category values, and each category is used for a different shape.

\sphinxAtStartPar
\sphinxincludegraphics{{Advanced_shape2}.png}

\sphinxAtStartPar
In the example below, the selected column is ARIH1, which contains 10 categories, so you can see that there are 10 shapes in the visualization.

\sphinxAtStartPar
\sphinxincludegraphics{{Advanced_shape2_ex}.png}

\sphinxAtStartPar
The third option in the marker shape is to \sphinxstyleemphasis{use a fixed shape}. This can be used if the user is not happy with all the different marker shapes and wants to make it homogeneous.

\sphinxAtStartPar
\sphinxincludegraphics{{Advanced_shape3}.png}

\sphinxAtStartPar
In the example below, the selected shape is a clobber and you can see that all the markers are in the shape of a clobber.

\sphinxAtStartPar
\sphinxincludegraphics{{Advanced_shape3_ex}.png}

\sphinxAtStartPar
The last option in the marker shape is to \sphinxstyleemphasis{remove outline}. This can be used to remove the dark outline of markers when the check box is checked.

\sphinxAtStartPar
\sphinxincludegraphics{{Advanced_shape4}.png}

\sphinxAtStartPar
In the example below, the outline is included on the left side and the outline is removed on the right side.

\sphinxAtStartPar
\sphinxincludegraphics{{Advanced_shape4_ex}.png}

\sphinxAtStartPar
The next advanced option is \sphinxstyleemphasis{marker opacity} which is adjustable. The user can change the opacity in order to display things underneath.

\sphinxAtStartPar
\sphinxincludegraphics{{Advanced_opacity}.png}

\sphinxAtStartPar
In the example below, the opacity value was set to 0.6 which made the markers a bit transparent.

\sphinxAtStartPar
\sphinxincludegraphics{{Advanced_opacity_ex}.png}

\sphinxAtStartPar
In the example below, we checked \sphinxstyleemphasis{use different opacity per marker}. The user needs to select opacity column which will be used for displaying different opacities in markers.

\sphinxAtStartPar
\sphinxincludegraphics{{Advanced_opacity_ex2}.png}

\sphinxAtStartPar
The next option is the \sphinxstyleemphasis{marker tooltip}, which is the text which is displayed when the user clicks on the marker. By default, it displayed the key group the marker belongs to.

\sphinxAtStartPar
\sphinxincludegraphics{{Advanced_tip}.png}

\sphinxAtStartPar
As you can see in the example below, the green marker we clicked on belongs to the cell type Proliferating epithelial 2.

\sphinxAtStartPar
\sphinxincludegraphics{{Advanced_tip_ex}.png}

\sphinxAtStartPar
However, this can be modified by writing text into the text area. In this example, we wrote \{tab\} \sphinxhyphen{} \{key\} \sphinxhyphen{} \{col\_counts\}. \sphinxstyleemphasis{\{tab\}} represents the tab name that we set when we loaded the markers, \sphinxstyleemphasis{\{key\}} represents the key group to which the marker belongs and \sphinxstyleemphasis{\{col\_counts\}} represents the value of that marker in the column called counts. The word counts can be replaced by any column name in order to display it on the tooltip.

\sphinxAtStartPar
\sphinxincludegraphics{{Advanced_tip2}.png}

\sphinxAtStartPar
In the example below, the green dot which we clicked on is from the tab Cell types, belongs to the cell type Proliferating epithelial 2 and it has 127 counts in it.

\sphinxAtStartPar
\sphinxincludegraphics{{Advanced_tip2_ex}.png}

\sphinxAtStartPar
The last advanced option is the button \sphinxstyleemphasis{Generate button from tab}. This button incorporates all the display settings the user set up into a single button.

\sphinxAtStartPar
\sphinxincludegraphics{{Advanced_gen_button}.png}

\sphinxAtStartPar
The user can choose the relative path to the csv file, button inner text and comment which will be displayed next to the button.

\sphinxAtStartPar
\sphinxincludegraphics{{Advanced_gen_win}.png}

\sphinxAtStartPar
In the example below, you can see the generated button \sphinxstyleemphasis{Download data} placed on the top of the tabs panel. On the right of the button is placed text \sphinxstyleemphasis{My settings}.

\sphinxAtStartPar
\sphinxincludegraphics{{Advanced_gen_ex}.png}


\subsubsection{Table of markers}
\label{\detokenize{docs/starting/markers:table-of-markers}}
\sphinxAtStartPar
When the markers are loaded, a table of markers will appear in order to interact with the marker. Each row represents a group of markers with a specific color and shape. In the figure below, column \sphinxstyleemphasis{A)} represents if a specific row of markers is displayed or not, the second column \sphinxstyleemphasis{B)} represents the list of groups, the third column \sphinxstyleemphasis{C)} represents group counts, the fourth column \sphinxstyleemphasis{D)} represents the shape of the group markers, the fifth column \sphinxstyleemphasis{E)} represents the color of the group markers and the sixth column \sphinxstyleemphasis{F)} can display specific group when the cursor is on the eye icon.

\sphinxAtStartPar
\sphinxincludegraphics{{Table_general}.png}

\sphinxAtStartPar
If the check box is checked \sphinxhyphen{} the group is displayed, if the check box is unchecked \sphinxhyphen{} the group is not displayed. In the example below, we checked two groups of cell types: Airway Fibroblast and Airway smooth muscle, and only these two groups are displayed on the left visualization panel. The first checkbox \sphinxstyleemphasis{All} ensures displaying of all the markers.

\sphinxAtStartPar
\sphinxincludegraphics{{Table_check}.png}

\sphinxAtStartPar
In the fourth column \sphinxstyleemphasis{Shape}, the user can select which shape is preferred for each marker group. In the figure below, there is a list of 14 different shapes which can be used.

\sphinxAtStartPar
\sphinxincludegraphics{{Table_Shape}.png}

\sphinxAtStartPar
In the fifth column \sphinxstyleemphasis{Color}, the user can select which color is preferred for each marker group. In the figure below, it is possible to choose from some list of basic colors, select a specific color by the cursor from the palette and also use numbers to generate color, either RGB, HSV, or HTML.

\sphinxAtStartPar
\sphinxincludegraphics{{Table_Color}.png}

\sphinxAtStartPar
In the example below can be seen that if the cursor is placed on the eye icon in the row Airway fibroblast, only markers of this group are displayed on the visualization panel.

\sphinxAtStartPar
\sphinxincludegraphics{{Table_eye}.png}

\sphinxstepscope


\section{Regions}
\label{\detokenize{docs/starting/regions:regions}}\label{\detokenize{docs/starting/regions::doc}}

\subsection{Supported region formats}
\label{\detokenize{docs/starting/regions:supported-region-formats}}
\sphinxAtStartPar
TissUUmaps can read and write region files in the \sphinxhref{https://geojson.org/}{GeoJSON} format.

\sphinxAtStartPar
Only a subset of the GeoJSON format is supported, as TissUUmaps uses only polygonal regions:

\sphinxAtStartPar
\sphinxstylestrong{Main types}:
\begin{itemize}
\item {} 
\sphinxAtStartPar
Feature

\item {} 
\sphinxAtStartPar
FeatureCollection

\item {} 
\sphinxAtStartPar
GeometryCollection

\end{itemize}

\sphinxAtStartPar
\sphinxstylestrong{Geometries}:
\begin{itemize}
\item {} 
\sphinxAtStartPar
Polygon

\item {} 
\sphinxAtStartPar
Multipolygon

\end{itemize}

\sphinxAtStartPar
The coordinate system must be the same as the image and marker coordinate systems.


\subsection{Draw Regions}
\label{\detokenize{docs/starting/regions:draw-regions}}
\sphinxAtStartPar
Regions are polygons that can be drawn by the user or imported from an external file. If the user clicks on the button \sphinxstyleemphasis{Draw regions} shown underneath, the button is checked and the user can draw an unlimited number of regions. The user clicks on the image to outline the region of interest, then click on the first point to close the region and the region will appear in the right panel.

\sphinxAtStartPar
The button \sphinxstyleemphasis{Fill all regions} fills the inner part of all the regions by semitransparent color. This can be done separately by clicking on the check box next to individual regions.

\sphinxAtStartPar
\sphinxincludegraphics{{Regions_Draw}.png}

\sphinxAtStartPar
In the example below are three drawn regions selecting the hippocampus areas, all set to green color. You can set the class name in the \sphinxstyleemphasis{Class} column. You can see that the drawn regions are categorized into two main groups Hippocampus and White matter. The user can create an unlimited number of groups depending on his interest. This helps to have the regions organized and also it is very useful when exporting regions.

\sphinxAtStartPar
\sphinxincludegraphics{{Regions_Draw_ex1}.png}

\sphinxAtStartPar
In the example below are two drawn regions selecting the white matter areas, one is set to yellow and the other one to blue color. The regions are interchangeable between groups, so if you want to move a region from group Hippocampus to group White matter, just change the class name in the region’s row and it will be automatically moved to the desired group.

\sphinxAtStartPar
\sphinxincludegraphics{{Regions_Draw_ex2}.png}


\subsection{Analyze Regions}
\label{\detokenize{docs/starting/regions:analyze-regions}}
\sphinxAtStartPar
The regions can be analyzed, meaning displaying a list of all the marker keys with their counts inside that region (expression). The example below shows the analyzed region1. In this case, the analysis contains gene expression, so the column \sphinxstyleemphasis{Key} contains a list of genes, and column \sphinxstyleemphasis{Name} could show an additional column from the dataset, in this case, it is undefined since we provided only column Genes. The last column \sphinxstyleemphasis{Count} shows the number of each gene inside the analyzed region.

\sphinxAtStartPar
\sphinxincludegraphics{{Regions_Anlayze_ex}.png}


\subsection{Import Regions}
\label{\detokenize{docs/starting/regions:import-regions}}
\sphinxAtStartPar
Regions can be imported from .json file, which could be achieved from an external software or also from TissUUmaps’ plugin \sphinxstyleemphasis{Points2Regions}. The user just click on the tab \sphinxstyleemphasis{Import} \sphinxhyphen{}\textgreater{} \sphinxstyleemphasis{Choose File} and press the button \sphinxstyleemphasis{Import}.

\sphinxAtStartPar
\sphinxincludegraphics{{Regions_Import}.png}

\sphinxAtStartPar
After that, the displayed regions appear in the left panel and the list of regions in the right panel as you can see in the example below. In this case, there are 10 different regions, called clusters. The user can change the color, the name, and the class of the regions if necessary. The user can as well draw some extra regions. These regions can be analyzed to observe the marker expression.

\sphinxAtStartPar
\sphinxincludegraphics{{Regions_Import_ex}.png}


\subsection{Export Regions}
\label{\detokenize{docs/starting/regions:export-regions}}
\sphinxAtStartPar
The regions can be exported by clicking the tab \sphinxstyleemphasis{Export}, there the user can export two types of files. The first one is the .json file and the name can be selected. The second file is the marker expression in the regions which can be exported as .csv file (this is exported only if the regions were analyzed).

\sphinxAtStartPar
\sphinxincludegraphics{{Regions_Export}.png}

\sphinxAtStartPar
In the figure below can be seen an example of the exported .cvs file.

\sphinxAtStartPar
\sphinxincludegraphics{{Regions_Export_ex}.png}

\sphinxstepscope


\section{Projects}
\label{\detokenize{docs/starting/projects:projects}}\label{\detokenize{docs/starting/projects::doc}}

\subsection{Saving projects}
\label{\detokenize{docs/starting/projects:saving-projects}}
\sphinxAtStartPar
When the user has finished the visualization adjustments, region drawings, etc., the project is ready to be saved in order to continue working on it later or just basically to save it as it is for further consistency. The user needs to press \sphinxstyleemphasis{File} in the menu and then \sphinxstyleemphasis{Save project} or Ctrl + S.

\sphinxAtStartPar
\sphinxincludegraphics{{Project_Save}.png}

\sphinxAtStartPar
In order to save the project together with the .csv file, it is necessary to generate a button first. The warning window below appears and the user needs to generate the button. The path to the .csv file needs to be relative to the path of the image. In this example, the image layer and the .csv file are in the exact same directory.

\sphinxAtStartPar
\sphinxincludegraphics{{Project_Save_GenButton}.png}

\sphinxAtStartPar
Then the user selects a suitable directory to save the project and writes the project file name, i.e. My\_project.tmap, and the project is saved.

\sphinxAtStartPar
\sphinxincludegraphics{{Project_Save_dic}.png}


\subsection{Loading projects}
\label{\detokenize{docs/starting/projects:loading-projects}}
\sphinxAtStartPar
The .tmap project can be loaded by two approaches. The first one is opening the TissUUmaps program, click \sphinxstyleemphasis{File} in the menu and then \sphinxstyleemphasis{Open} or Crtl + O. Then the user navigates in the directory and selects the .tmap file. By default, the directory navigates in the recent .tmap project.

\sphinxAtStartPar
\sphinxincludegraphics{{Project_load_open}.png}

\sphinxAtStartPar
The second option is directly double click on the .tmap file in file explorer in your computer.

\sphinxAtStartPar
\sphinxincludegraphics{{Project_load_directly}.png}

\sphinxAtStartPar
After clicking the button \sphinxstyleemphasis{Download data}, both these approaches will lead to loading the project as can be seen in the example below.

\sphinxAtStartPar
\sphinxincludegraphics{{Project_load_result}.png}

\sphinxAtStartPar
For more information on the tmap file format and specifications, see {\hyperref[\detokenize{docs/advanced/tmap:the-tmap-file-format}]{\sphinxcrossref{\DUrole{std,std-ref}{The TMAP file format}}}}.


\subsection{Editing .tmap file manually}
\label{\detokenize{docs/starting/projects:editing-tmap-file-manually}}
\sphinxAtStartPar
Work in progress


\subsection{Existing projects}
\label{\detokenize{docs/starting/projects:existing-projects}}

\subsubsection{Human Developmental Lung Cell Atlas (pcw 5\sphinxhyphen{} pcw 14)}
\label{\detokenize{docs/starting/projects:human-developmental-lung-cell-atlas-pcw-5-pcw-14}}
\sphinxAtStartPar
The human lung is a highly complex tubular organ, whose main function is the gas exchange between blood and breathed air. In contains a large number of specialized cell\sphinxhyphen{}types of epithelial, endothelial, neuronal, stromal and immune cells that are necessary for normal organ function and structural integrity. To understand how this cell heterogeneity develops to create a healthy mature lung, we focused on the 1st trimester of gestation and applied state of art technologies to capture the gene expression profiles of all the cells in the developing organ, in time and space.

\sphinxAtStartPar
\sphinxincludegraphics{{Lung_Cell_Atlas}.png}

\sphinxAtStartPar
\sphinxstylestrong{TissUUmaps interactive viewer}: Single\sphinxhyphen{}cell RNA\sphinxhyphen{}sequencing UMAP representation of single\sphinxhyphen{}cell clusters and sub\sphinxhyphen{}clusters, gene expression and metadata.

\sphinxAtStartPar
\sphinxstylestrong{In situ sequencing data (ISS) \sphinxhyphen{} TissUUmaps interactive viewer}: pcw 5  pcw 6  pcw 13In situ sequencing data. Spot location + identity, per bin pie chart view of cell type probabilities and imputed genes.

\sphinxAtStartPar
\sphinxstylestrong{SCRINSHOT data \sphinxhyphen{} TissUUmaps interactive viewer}: pcw 6  pcw 8  pcw 11  pcw 14SCRINSHOT data. Spot location + identity.

\sphinxAtStartPar
\sphinxstylestrong{Spatial Transcriptomics data \sphinxhyphen{} TissUUmaps interactive viewer}: pcw 6  pcw 8  pcw 10  pcw 11Per gene or pie chart view of gene expression.

\sphinxAtStartPar
More information is available in the original \sphinxhref{https://doi.org/10.1101/2022.01.11.475631}{publication}: A. Sountoulidis, S.M. Salas, E. Braun, C. Avenel, J. Bergenstråhle, M. Vicari, P. Czarnewski, J. Theelke, A. Liontos, X. Abalo, Ž. Andrusivová, M. Asp, X. Li, L. Hu, S. Sariyar, A.M. Casals, B. Ayoglu, A. Firsova, J. Michaëlsson, E. Lundberg, C. Wählby, E. Sundström, S. Linnarsson, J. Lundeberg, M. Nilsson, C. Samakovlis. Developmental origins of cell heterogeneity in the human lung. BioRxiv doi: \sphinxurl{https://doi.org/10.1101/2022.01.11.475631}


\subsubsection{Modelling of cell\sphinxhyphen{}type signatures in the developmental human heart}
\label{\detokenize{docs/starting/projects:modelling-of-cell-type-signatures-in-the-developmental-human-heart}}
\sphinxAtStartPar
With the emergence of high throughput single cell techniques, the understanding of cellular diversity in biologically complex processes has rapidly increased. The next step towards comprehension of e.g. key organs in the mammal development is to obtain spatiotemporal atlases of the cellular diversity. However, targeted cell typing approaches relying on existing single cell data achieve incomplete and biased maps that could mask the molecular and cellular heterogeneity present in a tissue slide. Here we applied spage2vec, a de novo approach to spatially resolve and characterize cellular diversity during human heart development. Data from the original in situ sequencing experiment as well as identified cell types can be viewed in TissUUmaps.

\sphinxAtStartPar
\sphinxincludegraphics{{de_novo}.png}

\sphinxAtStartPar
\sphinxstylestrong{TissUUmaps interactive viewer}:
Human heart

\sphinxAtStartPar
More information is available in the original \sphinxhref{https://doi.org/10.1101/2021.07.10.451822}{publication}: RS. Marco Salas, X. Yuan,  C. Sylven,  M. Nilsson,  C. Wählby and  G.Partel. De novo spatiotemporal modelling of cell\sphinxhyphen{}type signatures identifies novel cell populations in the developmental human heart. BioRxiv doi: \sphinxurl{https://doi.org/10.1101/2021.07.10.451822}


\subsubsection{Automated identification of the mouse brain’s spatial compartments from in situ sequencing data}
\label{\detokenize{docs/starting/projects:automated-identification-of-the-mouse-brain-s-spatial-compartments-from-in-situ-sequencing-data}}
\sphinxAtStartPar
Neuroanatomical compartments of the mouse brain are identified and outlined mainly based on manual annotations of samples using features related to tissue and cellular morphology, taking advantage of publicly available reference atlases. However, this task is challenging since sliced tissue sections are rarely perfectly parallel or angled with respect to sections in the reference atlas and organs from different individuals may vary in size and shape and requires manual annotation. Here, we show how in situ sequencing data combined with dimensionality reduction and unsupervised clustering can be used to identify spatial compartments that correspond to known anatomical compartments of the brain. Here we show results on four different sections of mouse brains.

\sphinxAtStartPar
\sphinxincludegraphics{{automated}.png}

\sphinxAtStartPar
\sphinxstylestrong{TissUUmaps interactive viewer}:
Mouse brain

\sphinxAtStartPar
More information is available in this \sphinxhref{https://doi.org/10.1186/s12915-020-00874-5}{publication}: G. Partel, M.M. Hilscher, G. Milli, L. Solorzano, A.H. Klemm, M. Nilsson, and C. Wählby.  Automated identification of the mouse brain’s spatial compartments from in situ sequencing data.  BMC Biology, \sphinxurl{https://doi.org/10.1186/s12915-020-00874-5}, Oct 2020.

\sphinxAtStartPar
The original raw ISS data was published in Qian, X., Harris, K. D., Hauling, T., Nicoloutsopoulos, D., Muñoz\sphinxhyphen{}Manchado, A. B., Skene, N., … \& Nilsson, M. (2020). Probabilistic cell typing enables fine mapping of closely related cell types in situ. \sphinxhref{https://doi.org/10.1038/s41592-019-0631-4}{Nature methods}, 17(1), 101\sphinxhyphen{}106.

\sphinxAtStartPar
Data and code availability: All software was developed in Python 3 using open source libraries, and data processing of pipeline workflows was carried out using \sphinxhref{https://doi.org/10.1093/bioinformatics/btz133}{Anduril2} analysis framework. The processing pipelines, data, and the software version used to generate the analysis results and figures presented in this paper are available at \sphinxurl{https://doi.org/10.5281/zenodo.3928219} or from our github repository \sphinxurl{https://github.com/wahlby-lab/graph-iss}.


\subsubsection{Spage2vec: Unsupervised representation of localized spatial gene expression signatures}
\label{\detokenize{docs/starting/projects:spage2vec-unsupervised-representation-of-localized-spatial-gene-expression-signatures}}
\sphinxAtStartPar
Spage2vec is an unsupervised segmentation free approach for decrypting the spatial transcriptomic heterogeneity of complex tissues at subcellular resolution. Spage2vec represents the spatial transcriptomic landscape of tissue samples as a graph and leverage powerful machine learning graph representation technique to create a lower dimensional representation of local spatial gene expression. Here we visualize spage2vec localized gene expression signatures of different spatial transcriptomic datasets. We thank Mats Nilsson, Sten Linnarsson and Xiaowei Zhuang for making their datasets publicly available.

\sphinxAtStartPar
\sphinxincludegraphics{{spage2vec}.png}

\sphinxAtStartPar
\sphinxstylestrong{TissUUmaps interactive viewer 1}: In situ sequencing mouse brain hippocampal area CA1 Qian, X., Harris, K. D., Hauling, T., Nicoloutsopoulos, D., Muñoz\sphinxhyphen{}Manchado, A. B., Skene, N., … \& Nilsson, M. (2020). Probabilistic cell typing enables fine mapping of closely related cell types in situ. \sphinxhref{https://doi.org/10.1038/s41592-019-0631-4}{Nature methods}, 17(1), 101\sphinxhyphen{}106.

\sphinxAtStartPar
\sphinxstylestrong{TissUUmaps interactive viewer 2}: osmFISH mouse brain somatosensory cortex Codeluppi, S., Borm, L. E., Zeisel, A., La Manno, G., van Lunteren, J. A., Svensson, C. I., \& Linnarsson, S. (2018). Spatial organization of the somatosensory cortex revealed by osmFISH. \sphinxhref{https://doi.org/10.1038/s41592-018-0175-z}{Nature methods}, 15(11), 932\sphinxhyphen{}935.

\sphinxAtStartPar
\sphinxstylestrong{TissUUmaps interactive viewer 3}: MERFISH mouse brain hypothalamic preoptic area Moffitt, J. R., Bambah\sphinxhyphen{}Mukku, D., Eichhorn, S. W., Vaughn, E., Shekhar, K., Perez, J. D., … \& Zhuang, X. (2018). Molecular, spatial, and functional single\sphinxhyphen{}cell profiling of the hypothalamic preoptic region. \sphinxhref{https://doi.org/10.1126/science.aau5324}{Science}, 362(6416), eaau5324.

\sphinxAtStartPar
\sphinxstylestrong{TissUUmaps interactive viewer 4}: MERFISH human fibroblast cells (IMR90) Chen, K. H., Boettiger, A. N., Moffitt, J. R., Wang, S., \& Zhuang, X. (2015). Spatially resolved, highly multiplexed RNA profiling in single cells. \sphinxhref{https://doi.org/10.1126/science.aaa6090}{Science}, 348(6233), aaa6090.

\sphinxAtStartPar
Data and code availability: Spatial gene expression data are available in Zenodo database at \sphinxurl{https://doi.org/10.5281/zenodo.3897401}.
Source code for reproducing analysis results and figures is available in Zenodo database at \sphinxurl{http://www.doi.org/10.5281/zenodo.4030404}.


\subsubsection{Artificial intelligence for diagnosis and grading of prostate cancer in biopsies: a population\sphinxhyphen{}based}
\label{\detokenize{docs/starting/projects:artificial-intelligence-for-diagnosis-and-grading-of-prostate-cancer-in-biopsies-a-population-based}}
\sphinxAtStartPar
An increasing volume of prostate biopsies and a worldwide shortage of urological pathologists puts a
strain on pathology departments. Additionally, the high intra\sphinxhyphen{}observer and inter\sphinxhyphen{}observer variability in grading can
result in overtreatment and undertreatment of prostate cancer. To alleviate these problems, we aimed to develop an
artificial intelligence (AI) system with clinically acceptable accuracy for prostate cancer detection, localisation, and
Gleason grading. Here we show examples of full\sphinxhyphen{}resolution digitized biopsies and corresponding AI\sphinxhyphen{}based grading.

\sphinxAtStartPar
\sphinxincludegraphics{{prostate}.png}

\sphinxAtStartPar
An overview of all sample \sphinxstylestrong{datasets} can be found here: Prostate cancer in biopsies 

\sphinxAtStartPar
More information is available in this \sphinxhref{https://www.sciencedirect.com/science/article/pii/S1470204519307387}{publication}: P. Ström, K. Kartasalo, H. Olsson, L. Solorzano et al. Artificial intelligence for diagnosis and grading of prostate cancer in biopsies: a population\sphinxhyphen{}based, diagnostic study. The Lancet Oncology, Volume 21, Issue 2, 2020, Pages 222\sphinxhyphen{}232, ISSN 1470\sphinxhyphen{}2045,  doi: 10.1016/S1470\sphinxhyphen{}2045(19)30738\sphinxhyphen{}7, url: \sphinxurl{https://www.sciencedirect.com/science/article/pii/S1470204519307387}

\sphinxstepscope


\section{Exporting screenshots}
\label{\detokenize{docs/starting/capture:exporting-screenshots}}\label{\detokenize{docs/starting/capture::doc}}
\sphinxAtStartPar
TissUUmaps allows high resolution capture of the image viewport. Go to \sphinxcode{\sphinxupquote{Menu \textgreater{} File \textgreater{} Capture viewport}} and chose a zoom factor for export (1 = screen resolution).

\sphinxAtStartPar
The screen capture will contain all filtered layers, markers, and regions. Note that legends will not be part of the export and must be added manually.

\sphinxstepscope


\section{Plugins}
\label{\detokenize{docs/starting/plugins:plugins}}\label{\detokenize{docs/starting/plugins::doc}}

\subsection{Load plugins}
\label{\detokenize{docs/starting/plugins:load-plugins}}

\subsection{Make your own plugin}
\label{\detokenize{docs/starting/plugins:make-your-own-plugin}}
\sphinxAtStartPar
Download the Plugin Template python and javascript files from the \sphinxhref{https://tissuumaps.github.io/TissUUmaps/plugins/}{Plugin Update Site} and put both files in your local folder \sphinxcode{\sphinxupquote{\$USER\_PATH/.tissuumaps/plugins/}}. You can then change the plugin name and add your own options and functions.


\subsubsection{Javascript file}
\label{\detokenize{docs/starting/plugins:javascript-file}}
\sphinxAtStartPar
When loading a plugin, the function \sphinxcode{\sphinxupquote{PluginName.init(container)}} will be called. The \sphinxcode{\sphinxupquote{container}} is an html Element that will be added to the plugin menu. Use this element to add options and texts related to your plugin.

\sphinxAtStartPar
\sphinxincludegraphics{{plugin_container}.png}

\sphinxAtStartPar
Here is a minimal example of plugin:

\begin{sphinxVerbatim}[commandchars=\\\{\}]
\PYG{k+kd}{var}\PYG{+w}{ }\PYG{n+nx}{Plugin\PYGZus{}template}\PYG{p}{;}
\PYG{n+nx}{Plugin\PYGZus{}template}\PYG{+w}{ }\PYG{o}{=}\PYG{+w}{ }\PYG{p}{\PYGZob{}}
\PYG{+w}{    }\PYG{n+nx}{name}\PYG{o}{:}\PYG{l+s+s2}{\PYGZdq{}Template Plugin\PYGZdq{}}
\PYG{p}{\PYGZcb{}}
\PYG{+w}{ }
\PYG{c+cm}{/**}
\PYG{c+cm}{ * This method is called when the document is loaded.}
\PYG{c+cm}{ * The container element is a div where the plugin options will be displayed. */}
\PYG{n+nx}{Plugin\PYGZus{}template}\PYG{p}{.}\PYG{n+nx}{init}\PYG{+w}{ }\PYG{o}{=}\PYG{+w}{ }\PYG{k+kd}{function}\PYG{+w}{ }\PYG{p}{(}\PYG{n+nx}{container}\PYG{p}{)}\PYG{+w}{ }\PYG{p}{\PYGZob{}}
\PYG{+w}{    }\PYG{n+nx}{container}\PYG{p}{.}\PYG{n+nx}{innerHTML}\PYG{+w}{ }\PYG{o}{=}\PYG{+w}{ }\PYG{l+s+s2}{\PYGZdq{}Hello world\PYGZdq{}}\PYG{p}{;}
\PYG{p}{\PYGZcb{}}
\end{sphinxVerbatim}

\sphinxAtStartPar
You can access the TissUUmaps javascript API \sphinxhref{https://tissuumaps.github.io/TissUUmapsCore/}{here}.


\subsubsection{Python file}
\label{\detokenize{docs/starting/plugins:python-file}}
\sphinxAtStartPar
You only need to use the Python file if your plugin needs to do processing on the server side. For pure javascript plugins, you can leave this file empty.

\sphinxAtStartPar
The python file should implement the class \sphinxcode{\sphinxupquote{Plugin}}:

\begin{sphinxVerbatim}[commandchars=\\\{\}]
\PYG{k}{class} \PYG{n+nc}{Plugin} \PYG{p}{(}\PYG{p}{)}\PYG{p}{:}
    \PYG{k}{def} \PYG{n+nf+fm}{\PYGZus{}\PYGZus{}init\PYGZus{}\PYGZus{}}\PYG{p}{(}\PYG{n+nb+bp}{self}\PYG{p}{,} \PYG{n}{app}\PYG{p}{)}\PYG{p}{:}
        \PYG{n+nb+bp}{self}\PYG{o}{.}\PYG{n}{app} \PYG{o}{=} \PYG{n}{app}
\end{sphinxVerbatim}

\sphinxAtStartPar
The \sphinxcode{\sphinxupquote{app}} object being the flask application running the TissUUmaps server.

\sphinxAtStartPar
You can call a Python method inside the \sphinxcode{\sphinxupquote{Plugin}} class from Javascript using Ajax and the Python API. The endpoint for a method \sphinxcode{\sphinxupquote{methodName}} of the plugin \sphinxcode{\sphinxupquote{PluginName}} will be: \sphinxcode{\sphinxupquote{/plugins/methodName/functionName}}. Data can be transmitted through Ajax as stringified JSON, and will be available as a parameter inside the method.

\sphinxAtStartPar
See the Plugin Template for a working example of Javascript / Python communication.

\sphinxstepscope


\chapter{Sharing projects}
\label{\detokenize{docs/sharing/index:sharing-projects}}\label{\detokenize{docs/sharing/index::doc}}
\sphinxstepscope


\section{Apache server}
\label{\detokenize{docs/sharing/apache:apache-server}}\label{\detokenize{docs/sharing/apache::doc}}
\sphinxAtStartPar
TissUUmaps projects can be exported into static webpages, that can be uploaded to any Apache server.
\begin{enumerate}
\sphinxsetlistlabels{\arabic}{enumi}{enumii}{}{.}%
\item {} 
\sphinxAtStartPar
Save your project from TissUUmaps (\sphinxcode{\sphinxupquote{menu \textgreater{} File \textgreater{} Save project}})

\item {} 
\sphinxAtStartPar
Export to static page (\sphinxcode{\sphinxupquote{menu \textgreater{} File \textgreater{} Export to static webpage}})

\item {} 
\sphinxAtStartPar
Copy the exported folder on your Apache server

\end{enumerate}

\sphinxstepscope


\section{Docker container}
\label{\detokenize{docs/sharing/docker:docker-container}}\label{\detokenize{docs/sharing/docker::doc}}\begin{enumerate}
\sphinxsetlistlabels{\arabic}{enumi}{enumii}{}{.}%
\item {} 
\sphinxAtStartPar
Start the docker container \sphinxcode{\sphinxupquote{cavenel/tissuumaps:latest}} from Docker Hub:

\end{enumerate}

\begin{sphinxVerbatim}[commandchars=\\\{\}]
docker run \PYGZhy{}it \PYGZhy{}p \PYG{l+m}{56733}:80 \PYGZhy{}\PYGZhy{}name\PYG{o}{=}tissuumaps \PYGZhy{}v /path/to/local/images:/mnt/data cavenel/tissuumaps:latest
\end{sphinxVerbatim}
\begin{enumerate}
\sphinxsetlistlabels{\arabic}{enumi}{enumii}{}{.}%
\item {} 
\sphinxAtStartPar
Place your images in the local folder \sphinxcode{\sphinxupquote{/path/to/local/images/share}}.

\item {} 
\sphinxAtStartPar
Open \sphinxurl{http://127.0.0.1:56733/} in your favorite browser.

\end{enumerate}

\sphinxstepscope


\chapter{Advanced usage}
\label{\detokenize{docs/advanced/index:advanced-usage}}\label{\detokenize{docs/advanced/index::doc}}
\sphinxstepscope


\section{Jupyter notebooks}
\label{\detokenize{docs/advanced/jupyter:jupyter-notebooks}}\label{\detokenize{docs/advanced/jupyter::doc}}
\sphinxAtStartPar
TissUUmaps can easily be used inside a Jupyter Notebook or Jupyter Lab.

\sphinxAtStartPar
Simple example to load an image in TissUUmaps:

\begin{sphinxVerbatim}[commandchars=\\\{\}]
\PYG{k+kn}{import} \PYG{n+nn}{tissuumaps}\PYG{n+nn}{.}\PYG{n+nn}{jupyter} \PYG{k}{as} \PYG{n+nn}{tj}
\PYG{n}{viewer} \PYG{o}{=} \PYG{n}{tj}\PYG{o}{.}\PYG{n}{loaddata}\PYG{p}{(}\PYG{p}{[}\PYG{l+s+s2}{\PYGZdq{}}\PYG{l+s+s2}{image.png}\PYG{l+s+s2}{\PYGZdq{}}\PYG{p}{]}\PYG{p}{)}

\PYG{n}{viewer}\PYG{o}{.}\PYG{n}{screenshot}\PYG{p}{(}\PYG{p}{)}
\end{sphinxVerbatim}
\phantomsection\label{\detokenize{docs/advanced/jupyter:module-tissuumaps.jupyter}}\index{module@\spxentry{module}!tissuumaps.jupyter@\spxentry{tissuumaps.jupyter}}\index{tissuumaps.jupyter@\spxentry{tissuumaps.jupyter}!module@\spxentry{module}}

\subsection{tissuumaps.jupyter}
\label{\detokenize{docs/advanced/jupyter:tissuumaps-jupyter}}
\sphinxAtStartPar
Module used to run TissUUmaps from a Jupyter Notebook or from Jupyter Lab.
\index{opentmap() (in module tissuumaps.jupyter)@\spxentry{opentmap()}\spxextra{in module tissuumaps.jupyter}}

\begin{fulllineitems}
\phantomsection\label{\detokenize{docs/advanced/jupyter:tissuumaps.jupyter.opentmap}}
\pysigstartsignatures
\pysiglinewithargsret{\sphinxcode{\sphinxupquote{tissuumaps.jupyter.}}\sphinxbfcode{\sphinxupquote{opentmap}}}{\emph{\DUrole{n}{path}}, \emph{\DUrole{n}{port}\DUrole{o}{=}\DUrole{default_value}{5100}}, \emph{\DUrole{n}{host}\DUrole{o}{=}\DUrole{default_value}{\textquotesingle{}localhost\textquotesingle{}}}, \emph{\DUrole{n}{height}\DUrole{o}{=}\DUrole{default_value}{700}}}{}
\pysigstopsignatures
\sphinxAtStartPar
Open a tmap project
\begin{quote}\begin{description}
\item[{Parameters}] \leavevmode\begin{itemize}
\item {} 
\sphinxAtStartPar
\sphinxstyleliteralstrong{\sphinxupquote{path}} (\sphinxstyleliteralemphasis{\sphinxupquote{str}}) \textendash{} The path to a tmap file

\item {} 
\sphinxAtStartPar
\sphinxstyleliteralstrong{\sphinxupquote{port}} (\sphinxstyleliteralemphasis{\sphinxupquote{int}}) \textendash{} The port to run the TissUUmaps server

\item {} 
\sphinxAtStartPar
\sphinxstyleliteralstrong{\sphinxupquote{host}} (\sphinxstyleliteralemphasis{\sphinxupquote{str}}) \textendash{} The host to run the TissUUmaps server

\item {} 
\sphinxAtStartPar
\sphinxstyleliteralstrong{\sphinxupquote{height}} (\sphinxstyleliteralemphasis{\sphinxupquote{int}}) \textendash{} The height of the jupyter iframe

\end{itemize}

\item[{Returns}] \leavevmode
\sphinxAtStartPar
The TissUUmaps viewer

\item[{Return type}] \leavevmode
\sphinxAtStartPar
{\hyperref[\detokenize{docs/advanced/jupyter:tissuumaps.jupyter.TissUUmapsViewer}]{\sphinxcrossref{TissUUmapsViewer}}}

\end{description}\end{quote}

\end{fulllineitems}

\index{loaddata() (in module tissuumaps.jupyter)@\spxentry{loaddata()}\spxextra{in module tissuumaps.jupyter}}

\begin{fulllineitems}
\phantomsection\label{\detokenize{docs/advanced/jupyter:tissuumaps.jupyter.loaddata}}
\pysigstartsignatures
\pysiglinewithargsret{\sphinxcode{\sphinxupquote{tissuumaps.jupyter.}}\sphinxbfcode{\sphinxupquote{loaddata}}}{\emph{\DUrole{n}{images}\DUrole{o}{=}\DUrole{default_value}{{[}{]}}}, \emph{\DUrole{n}{csvFiles}\DUrole{o}{=}\DUrole{default_value}{{[}{]}}}, \emph{\DUrole{n}{xSelector}\DUrole{o}{=}\DUrole{default_value}{\textquotesingle{}x\textquotesingle{}}}, \emph{\DUrole{n}{ySelector}\DUrole{o}{=}\DUrole{default_value}{\textquotesingle{}y\textquotesingle{}}}, \emph{\DUrole{n}{keySelector}\DUrole{o}{=}\DUrole{default_value}{None}}, \emph{\DUrole{n}{nameSelector}\DUrole{o}{=}\DUrole{default_value}{None}}, \emph{\DUrole{n}{colorSelector}\DUrole{o}{=}\DUrole{default_value}{None}}, \emph{\DUrole{n}{piechartSelector}\DUrole{o}{=}\DUrole{default_value}{None}}, \emph{\DUrole{n}{shapeSelector}\DUrole{o}{=}\DUrole{default_value}{None}}, \emph{\DUrole{n}{scaleSelector}\DUrole{o}{=}\DUrole{default_value}{None}}, \emph{\DUrole{n}{fixedShape}\DUrole{o}{=}\DUrole{default_value}{None}}, \emph{\DUrole{n}{scaleFactor}\DUrole{o}{=}\DUrole{default_value}{1}}, \emph{\DUrole{n}{colormap}\DUrole{o}{=}\DUrole{default_value}{None}}, \emph{\DUrole{n}{compositeMode}\DUrole{o}{=}\DUrole{default_value}{\textquotesingle{}source\sphinxhyphen{}over\textquotesingle{}}}, \emph{\DUrole{n}{boundingBox}\DUrole{o}{=}\DUrole{default_value}{None}}, \emph{\DUrole{n}{port}\DUrole{o}{=}\DUrole{default_value}{5100}}, \emph{\DUrole{n}{host}\DUrole{o}{=}\DUrole{default_value}{\textquotesingle{}localhost\textquotesingle{}}}, \emph{\DUrole{n}{height}\DUrole{o}{=}\DUrole{default_value}{700}}, \emph{\DUrole{n}{tmapFilename}\DUrole{o}{=}\DUrole{default_value}{\textquotesingle{}\_project\textquotesingle{}}}, \emph{\DUrole{n}{plugins}\DUrole{o}{=}\DUrole{default_value}{{[}{]}}}}{}
\pysigstopsignatures
\sphinxAtStartPar
Load data in TissUUmaps
\begin{quote}\begin{description}
\item[{Parameters}] \leavevmode\begin{itemize}
\item {} 
\sphinxAtStartPar
\sphinxstyleliteralstrong{\sphinxupquote{images}} (\sphinxstyleliteralemphasis{\sphinxupquote{list}}\sphinxstyleliteralemphasis{\sphinxupquote{ | }}\sphinxstyleliteralemphasis{\sphinxupquote{str}}) \textendash{} List of images or single image to display

\item {} 
\sphinxAtStartPar
\sphinxstyleliteralstrong{\sphinxupquote{csvFiles}} (\sphinxstyleliteralemphasis{\sphinxupquote{list}}\sphinxstyleliteralemphasis{\sphinxupquote{ | }}\sphinxstyleliteralemphasis{\sphinxupquote{str}}) \textendash{} List of csv files or single csv file to display

\item {} 
\sphinxAtStartPar
\sphinxstyleliteralstrong{\sphinxupquote{xSelector}} (\sphinxstyleliteralemphasis{\sphinxupquote{str}}) \textendash{} Name of the csv column defining the X coordinates

\item {} 
\sphinxAtStartPar
\sphinxstyleliteralstrong{\sphinxupquote{ySelector}} (\sphinxstyleliteralemphasis{\sphinxupquote{str}}) \textendash{} Name of the csv column defining the Y coordinates

\item {} 
\sphinxAtStartPar
\sphinxstyleliteralstrong{\sphinxupquote{keySelector}} (\sphinxstyleliteralemphasis{\sphinxupquote{str}}) \textendash{} Name of the csv column defining the grouping key

\item {} 
\sphinxAtStartPar
\sphinxstyleliteralstrong{\sphinxupquote{nameSelector}} (\sphinxstyleliteralemphasis{\sphinxupquote{str}}) \textendash{} Name of the csv column defining the group name

\item {} 
\sphinxAtStartPar
\sphinxstyleliteralstrong{\sphinxupquote{colorSelector}} (\sphinxstyleliteralemphasis{\sphinxupquote{str}}) \textendash{} Name of the csv column defining the group color

\item {} 
\sphinxAtStartPar
\sphinxstyleliteralstrong{\sphinxupquote{piechartSelector}} (\sphinxstyleliteralemphasis{\sphinxupquote{str}}) \textendash{} Name of the csv column defining pie\sphinxhyphen{}charts

\item {} 
\sphinxAtStartPar
\sphinxstyleliteralstrong{\sphinxupquote{shapeSelector}} (\sphinxstyleliteralemphasis{\sphinxupquote{str}}) \textendash{} Name of the csv column defining markers’ shape

\item {} 
\sphinxAtStartPar
\sphinxstyleliteralstrong{\sphinxupquote{scaleSelector}} (\sphinxstyleliteralemphasis{\sphinxupquote{str}}) \textendash{} Name of the csv column defining markers’ scale

\item {} 
\sphinxAtStartPar
\sphinxstyleliteralstrong{\sphinxupquote{fixedShape}} (\sphinxstyleliteralemphasis{\sphinxupquote{int}}) \textendash{} Name of the markers’ shape

\item {} 
\sphinxAtStartPar
\sphinxstyleliteralstrong{\sphinxupquote{scaleFactor}} (\sphinxstyleliteralemphasis{\sphinxupquote{int}}) \textendash{} Global scale of markers

\item {} 
\sphinxAtStartPar
\sphinxstyleliteralstrong{\sphinxupquote{colormap}} (\sphinxstyleliteralemphasis{\sphinxupquote{str}}) \textendash{} Name of the colormap used if colorSelector is set

\item {} 
\sphinxAtStartPar
\sphinxstyleliteralstrong{\sphinxupquote{compositeMode}} \textendash{} (str): Composite mode used for images

\item {} 
\sphinxAtStartPar
\sphinxstyleliteralstrong{\sphinxupquote{boundingBox}} (\sphinxstyleliteralemphasis{\sphinxupquote{list}}) \textendash{} {[}X,Y,W,H{]} of the bounding box to display

\item {} 
\sphinxAtStartPar
\sphinxstyleliteralstrong{\sphinxupquote{port}} (\sphinxstyleliteralemphasis{\sphinxupquote{int}}) \textendash{} The port to run the TissUUmaps server

\item {} 
\sphinxAtStartPar
\sphinxstyleliteralstrong{\sphinxupquote{host}} (\sphinxstyleliteralemphasis{\sphinxupquote{str}}) \textendash{} The host to run the TissUUmaps server

\item {} 
\sphinxAtStartPar
\sphinxstyleliteralstrong{\sphinxupquote{height}} (\sphinxstyleliteralemphasis{\sphinxupquote{int}}) \textendash{} The height of the jupyter iframe

\item {} 
\sphinxAtStartPar
\sphinxstyleliteralstrong{\sphinxupquote{tmapFilename}} (\sphinxstyleliteralemphasis{\sphinxupquote{str}}) \textendash{} Name of the project file that will be created

\item {} 
\sphinxAtStartPar
\sphinxstyleliteralstrong{\sphinxupquote{plugins}} (\sphinxstyleliteralemphasis{\sphinxupquote{list}}) \textendash{} List of plugins to add to the tmap project

\end{itemize}

\item[{Returns}] \leavevmode
\sphinxAtStartPar
The TissUUmaps viewer

\item[{Return type}] \leavevmode
\sphinxAtStartPar
{\hyperref[\detokenize{docs/advanced/jupyter:tissuumaps.jupyter.TissUUmapsViewer}]{\sphinxcrossref{TissUUmapsViewer}}}

\end{description}\end{quote}

\end{fulllineitems}

\index{TissUUmapsViewer (class in tissuumaps.jupyter)@\spxentry{TissUUmapsViewer}\spxextra{class in tissuumaps.jupyter}}

\begin{fulllineitems}
\phantomsection\label{\detokenize{docs/advanced/jupyter:tissuumaps.jupyter.TissUUmapsViewer}}
\pysigstartsignatures
\pysiglinewithargsret{\sphinxbfcode{\sphinxupquote{class\DUrole{w}{  }}}\sphinxcode{\sphinxupquote{tissuumaps.jupyter.}}\sphinxbfcode{\sphinxupquote{TissUUmapsViewer}}}{\emph{\DUrole{n}{server}}, \emph{\DUrole{n}{image}}, \emph{\DUrole{n}{height}\DUrole{o}{=}\DUrole{default_value}{700}}}{}
\pysigstopsignatures
\sphinxAtStartPar
Class representing a TissUUmaps viewer instance
\index{screenshot() (tissuumaps.jupyter.TissUUmapsViewer method)@\spxentry{screenshot()}\spxextra{tissuumaps.jupyter.TissUUmapsViewer method}}

\begin{fulllineitems}
\phantomsection\label{\detokenize{docs/advanced/jupyter:tissuumaps.jupyter.TissUUmapsViewer.screenshot}}
\pysigstartsignatures
\pysiglinewithargsret{\sphinxbfcode{\sphinxupquote{screenshot}}}{}{}
\pysigstopsignatures
\sphinxAtStartPar
Capture the TissUUmaps viewport and display image in the Notebook.

\end{fulllineitems}


\end{fulllineitems}

\index{TissUUmapsServer (class in tissuumaps.jupyter)@\spxentry{TissUUmapsServer}\spxextra{class in tissuumaps.jupyter}}

\begin{fulllineitems}
\phantomsection\label{\detokenize{docs/advanced/jupyter:tissuumaps.jupyter.TissUUmapsServer}}
\pysigstartsignatures
\pysiglinewithargsret{\sphinxbfcode{\sphinxupquote{class\DUrole{w}{  }}}\sphinxcode{\sphinxupquote{tissuumaps.jupyter.}}\sphinxbfcode{\sphinxupquote{TissUUmapsServer}}}{\emph{\DUrole{n}{slideDir}}, \emph{\DUrole{n}{port}\DUrole{o}{=}\DUrole{default_value}{5000}}, \emph{\DUrole{n}{host}\DUrole{o}{=}\DUrole{default_value}{\textquotesingle{}0.0.0.0\textquotesingle{}}}}{}
\pysigstopsignatures
\sphinxAtStartPar
Class representing a TissUUmaps server instance

\end{fulllineitems}


\sphinxstepscope


\section{Napari}
\label{\detokenize{docs/advanced/napari:napari}}\label{\detokenize{docs/advanced/napari::doc}}
\sphinxAtStartPar
Napari features an important hub containing 118 plugins at the time of writing, many of them expanding further the capabilities of Napari when dealing with biomedical imaging. We thus created our own plugin to allow users to work in Napari, benefit from the tools, scripting and existing plugins, and easily visualize and share the output of their research through TissUUmaps.

\sphinxAtStartPar
The \sphinxhref{https://github.com/TissUUmaps/napari-tissuumaps}{Napari\sphinxhyphen{}TissUUmaps plugin} is available on Napari Hub which makes the installation trivial: from the Napari install/uninstall plugins menu, the \sphinxcode{\sphinxupquote{napari\sphinxhyphen{}tissuumaps}} appears in the list and can be installed with a single click. Alternatively, the plugin can be installed with the Python package manager: \sphinxcode{\sphinxupquote{pip install napari\sphinxhyphen{}tissuumaps}}.

\sphinxAtStartPar
The plugin can export all standard Napari layers, such as images, labels, points, and shapes and preserves the metadata (opacity, visibility), but also the objects parameters (e.g.: label colors, marker colors and symbols, etc…). To export a TissUUmaps project, care must be taken to save all layers of interest and type in a name with the extension \sphinxcode{\sphinxupquote{.tmap}}, e.g.: \sphinxcode{\sphinxupquote{myProject.tmap}}. This is important for Napari to delegate the saving of the files to the plugin. A folder is created and contains all the necessary files and can be loaded in the TissUUmaps server, software, Jupyter Notebook, or shared with the community.

\sphinxAtStartPar
The project folders generated by the plugin contain the metadata in a \sphinxcode{\sphinxupquote{main.tmap}} file, along with folders for each Napari layer types: images, labels, points and regions. Images and labels are saved as plain tif images, points are saved as CSV files, and shapes are saved as GeoJSON. We hope that the use of a simple structure and widespread file formats can simplify the modifying and updating of the TissUUmaps project when prototyping with e.g. Jupyter Notebooks.
The source code is available at \sphinxurl{https://github.com/TissUUmaps/napari-tissuumaps} under the permissive MIT license.
A demonstration of the Cellpose plugin of Napari being exported to the TissUUmaps web viewer is available at: \sphinxurl{https://tissuumaps.github.io/tutorials/\#napari}.

\sphinxstepscope


\section{AnnData}
\label{\detokenize{docs/advanced/anndata:anndata}}\label{\detokenize{docs/advanced/anndata::doc}}
\sphinxAtStartPar
Work in progress

\sphinxstepscope


\section{The TMAP file format}
\label{\detokenize{docs/advanced/tmap:the-tmap-file-format}}\label{\detokenize{docs/advanced/tmap::doc}}
\sphinxAtStartPar
The TMAP format contains a description of image layers, markers, regions, and settings. It is highly recommended to create .tmap files by saving projects from TissUUmaps, but you can also edit the files manually to add or change projects’ settings, or generate them as exported data from other software for import in TissUUmaps.

\sphinxAtStartPar
The TMAP format uses JSON, with the following specifications:


\subsection{TMAP project specifications}
\label{\detokenize{docs/advanced/tmap:tmap-project-specifications}}\label{\detokenize{docs/advanced/tmap:tmap-project-specifications}}

\begin{savenotes}\sphinxatlongtablestart\begin{longtable}[c]{|*{4}{\X{1}{4}|}}
\hline

\endfirsthead

\multicolumn{4}{c}%
{\makebox[0pt]{\sphinxtablecontinued{\tablename\ \thetable{} \textendash{} continued from previous page}}}\\
\hline

\endhead

\hline
\multicolumn{4}{r}{\makebox[0pt][r]{\sphinxtablecontinued{continues on next page}}}\\
\endfoot

\endlastfoot
\sphinxstartmulticolumn{4}%
\begin{varwidth}[t]{\sphinxcolwidth{4}{4}}
\sphinxAtStartPar
Description of image layers, markers, regions, and settings of a project. Required properties are shown in \sphinxstylestrong{bold} text
\par
\vskip-\baselineskip\vbox{\hbox{\strut}}\end{varwidth}%
\sphinxstopmulticolumn
\\
\hline
\sphinxAtStartPar
type
&\sphinxstartmulticolumn{3}%
\begin{varwidth}[t]{\sphinxcolwidth{3}{4}}
\sphinxAtStartPar
\sphinxstyleemphasis{object}
\par
\vskip-\baselineskip\vbox{\hbox{\strut}}\end{varwidth}%
\sphinxstopmulticolumn
\\
\hline\sphinxstartmulticolumn{4}%
\begin{varwidth}[t]{\sphinxcolwidth{4}{4}}
\sphinxAtStartPar
properties
\par
\vskip-\baselineskip\vbox{\hbox{\strut}}\end{varwidth}%
\sphinxstopmulticolumn
\\
\hline\sphinxmultirow{2}{5}{%
\begin{varwidth}[t]{\sphinxcolwidth{1}{4}}
\begin{itemize}
\item {} 
\sphinxAtStartPar
\sphinxstylestrong{filename}

\end{itemize}
\par
\vskip-\baselineskip\vbox{\hbox{\strut}}\end{varwidth}%
}%
&\sphinxstartmulticolumn{3}%
\begin{varwidth}[t]{\sphinxcolwidth{3}{4}}
\sphinxAtStartPar
Name of the project
\par
\vskip-\baselineskip\vbox{\hbox{\strut}}\end{varwidth}%
\sphinxstopmulticolumn
\\
\cline{2-4}\sphinxtablestrut{5}&
\sphinxAtStartPar
type
&\sphinxstartmulticolumn{2}%
\begin{varwidth}[t]{\sphinxcolwidth{2}{4}}
\sphinxAtStartPar
\sphinxstyleemphasis{string}
\par
\vskip-\baselineskip\vbox{\hbox{\strut}}\end{varwidth}%
\sphinxstopmulticolumn
\\
\hline\sphinxmultirow{4}{9}{%
\begin{varwidth}[t]{\sphinxcolwidth{1}{4}}
\begin{itemize}
\item {} 
\sphinxAtStartPar
layers

\end{itemize}
\par
\vskip-\baselineskip\vbox{\hbox{\strut}}\end{varwidth}%
}%
&
\sphinxAtStartPar
type
&\sphinxstartmulticolumn{2}%
\begin{varwidth}[t]{\sphinxcolwidth{2}{4}}
\sphinxAtStartPar
\sphinxstyleemphasis{array}
\par
\vskip-\baselineskip\vbox{\hbox{\strut}}\end{varwidth}%
\sphinxstopmulticolumn
\\
\cline{2-4}\sphinxtablestrut{9}&
\sphinxAtStartPar
default
&\sphinxstartmulticolumn{2}%
\begin{varwidth}[t]{\sphinxcolwidth{2}{4}}
\sphinxAtStartPar
{[}{]}
\par
\vskip-\baselineskip\vbox{\hbox{\strut}}\end{varwidth}%
\sphinxstopmulticolumn
\\
\cline{2-4}\sphinxtablestrut{9}&\sphinxstartmulticolumn{3}%
\begin{varwidth}[t]{\sphinxcolwidth{3}{4}}
\sphinxAtStartPar
items
\par
\vskip-\baselineskip\vbox{\hbox{\strut}}\end{varwidth}%
\sphinxstopmulticolumn
\\
\cline{2-4}\sphinxtablestrut{9}&\begin{itemize}
\item {} 
\end{itemize}
&\sphinxstartmulticolumn{2}%
\begin{varwidth}[t]{\sphinxcolwidth{2}{4}}
\sphinxAtStartPar
{\hyperref[\detokenize{docs/advanced/tmap:layer}]{\sphinxcrossref{Layer}}}
\par
\vskip-\baselineskip\vbox{\hbox{\strut}}\end{varwidth}%
\sphinxstopmulticolumn
\\
\hline\sphinxmultirow{3}{17}{%
\begin{varwidth}[t]{\sphinxcolwidth{1}{4}}
\begin{itemize}
\item {} 
\sphinxAtStartPar
layerOpacities

\end{itemize}
\par
\vskip-\baselineskip\vbox{\hbox{\strut}}\end{varwidth}%
}%
&
\sphinxAtStartPar
type
&\sphinxstartmulticolumn{2}%
\begin{varwidth}[t]{\sphinxcolwidth{2}{4}}
\sphinxAtStartPar
\sphinxstyleemphasis{object}
\par
\vskip-\baselineskip\vbox{\hbox{\strut}}\end{varwidth}%
\sphinxstopmulticolumn
\\
\cline{2-4}\sphinxtablestrut{17}&\sphinxstartmulticolumn{3}%
\begin{varwidth}[t]{\sphinxcolwidth{3}{4}}
\sphinxAtStartPar
patternProperties
\par
\vskip-\baselineskip\vbox{\hbox{\strut}}\end{varwidth}%
\sphinxstopmulticolumn
\\
\cline{2-4}\sphinxtablestrut{17}&\begin{itemize}
\item {} 
\sphinxAtStartPar
\textasciicircum{}{[}0\sphinxhyphen{}9{]}+\$

\end{itemize}
&
\sphinxAtStartPar
type
&
\sphinxAtStartPar
\sphinxstyleemphasis{integer}
\\
\hline\sphinxmultirow{3}{24}{%
\begin{varwidth}[t]{\sphinxcolwidth{1}{4}}
\begin{itemize}
\item {} 
\sphinxAtStartPar
layerVisibilities

\end{itemize}
\par
\vskip-\baselineskip\vbox{\hbox{\strut}}\end{varwidth}%
}%
&
\sphinxAtStartPar
type
&\sphinxstartmulticolumn{2}%
\begin{varwidth}[t]{\sphinxcolwidth{2}{4}}
\sphinxAtStartPar
\sphinxstyleemphasis{object}
\par
\vskip-\baselineskip\vbox{\hbox{\strut}}\end{varwidth}%
\sphinxstopmulticolumn
\\
\cline{2-4}\sphinxtablestrut{24}&\sphinxstartmulticolumn{3}%
\begin{varwidth}[t]{\sphinxcolwidth{3}{4}}
\sphinxAtStartPar
patternProperties
\par
\vskip-\baselineskip\vbox{\hbox{\strut}}\end{varwidth}%
\sphinxstopmulticolumn
\\
\cline{2-4}\sphinxtablestrut{24}&\begin{itemize}
\item {} 
\sphinxAtStartPar
\textasciicircum{}{[}0\sphinxhyphen{}9{]}+\$

\end{itemize}
&
\sphinxAtStartPar
type
&
\sphinxAtStartPar
\sphinxstyleemphasis{boolean}
\\
\hline\sphinxmultirow{3}{31}{%
\begin{varwidth}[t]{\sphinxcolwidth{1}{4}}
\begin{itemize}
\item {} 
\sphinxAtStartPar
layerFilters

\end{itemize}
\par
\vskip-\baselineskip\vbox{\hbox{\strut}}\end{varwidth}%
}%
&
\sphinxAtStartPar
type
&\sphinxstartmulticolumn{2}%
\begin{varwidth}[t]{\sphinxcolwidth{2}{4}}
\sphinxAtStartPar
\sphinxstyleemphasis{object}
\par
\vskip-\baselineskip\vbox{\hbox{\strut}}\end{varwidth}%
\sphinxstopmulticolumn
\\
\cline{2-4}\sphinxtablestrut{31}&\sphinxstartmulticolumn{3}%
\begin{varwidth}[t]{\sphinxcolwidth{3}{4}}
\sphinxAtStartPar
patternProperties
\par
\vskip-\baselineskip\vbox{\hbox{\strut}}\end{varwidth}%
\sphinxstopmulticolumn
\\
\cline{2-4}\sphinxtablestrut{31}&\begin{itemize}
\item {} 
\sphinxAtStartPar
\textasciicircum{}{[}0\sphinxhyphen{}9{]}+\$

\end{itemize}
&\sphinxstartmulticolumn{2}%
\begin{varwidth}[t]{\sphinxcolwidth{2}{4}}
\sphinxAtStartPar
{\hyperref[\detokenize{docs/advanced/tmap:layerfilter}]{\sphinxcrossref{LayerFilter}}}
\par
\vskip-\baselineskip\vbox{\hbox{\strut}}\end{varwidth}%
\sphinxstopmulticolumn
\\
\hline\sphinxmultirow{5}{37}{%
\begin{varwidth}[t]{\sphinxcolwidth{1}{4}}
\begin{itemize}
\item {} 
\sphinxAtStartPar
filters

\end{itemize}
\par
\vskip-\baselineskip\vbox{\hbox{\strut}}\end{varwidth}%
}%
&\sphinxstartmulticolumn{3}%
\begin{varwidth}[t]{\sphinxcolwidth{3}{4}}
\sphinxAtStartPar
List of filters shown as active filters in the GUI under the Image layers tab
\par
\vskip-\baselineskip\vbox{\hbox{\strut}}\end{varwidth}%
\sphinxstopmulticolumn
\\
\cline{2-4}\sphinxtablestrut{37}&
\sphinxAtStartPar
type
&\sphinxstartmulticolumn{2}%
\begin{varwidth}[t]{\sphinxcolwidth{2}{4}}
\sphinxAtStartPar
\sphinxstyleemphasis{array}
\par
\vskip-\baselineskip\vbox{\hbox{\strut}}\end{varwidth}%
\sphinxstopmulticolumn
\\
\cline{2-4}\sphinxtablestrut{37}&
\sphinxAtStartPar
default
&\sphinxstartmulticolumn{2}%
\begin{varwidth}[t]{\sphinxcolwidth{2}{4}}
\sphinxAtStartPar
{[}“Saturation”, “Brightness”, “Contrast”{]}
\par
\vskip-\baselineskip\vbox{\hbox{\strut}}\end{varwidth}%
\sphinxstopmulticolumn
\\
\cline{2-4}\sphinxtablestrut{37}&\sphinxstartmulticolumn{3}%
\begin{varwidth}[t]{\sphinxcolwidth{3}{4}}
\sphinxAtStartPar
items
\par
\vskip-\baselineskip\vbox{\hbox{\strut}}\end{varwidth}%
\sphinxstopmulticolumn
\\
\cline{2-4}\sphinxtablestrut{37}&\begin{itemize}
\item {} 
\end{itemize}
&\sphinxstartmulticolumn{2}%
\begin{varwidth}[t]{\sphinxcolwidth{2}{4}}
\sphinxAtStartPar
{\hyperref[\detokenize{docs/advanced/tmap:filter}]{\sphinxcrossref{\DUrole{std,std-ref}{Filter}}}}
\par
\vskip-\baselineskip\vbox{\hbox{\strut}}\end{varwidth}%
\sphinxstopmulticolumn
\\
\hline\sphinxmultirow{3}{46}{%
\begin{varwidth}[t]{\sphinxcolwidth{1}{4}}
\begin{itemize}
\item {} 
\sphinxAtStartPar
compositeMode

\end{itemize}
\par
\vskip-\baselineskip\vbox{\hbox{\strut}}\end{varwidth}%
}%
&\sphinxstartmulticolumn{3}%
\begin{varwidth}[t]{\sphinxcolwidth{3}{4}}
\sphinxAtStartPar
Mode defining how image layers will be merged (composited) with each other. Valid string values are “source\sphinxhyphen{}over” and “lighter”, which correspond to ‘Channels’ and ‘Composite’ in the GUI.
\par
\vskip-\baselineskip\vbox{\hbox{\strut}}\end{varwidth}%
\sphinxstopmulticolumn
\\
\cline{2-4}\sphinxtablestrut{46}&
\sphinxAtStartPar
type
&\sphinxstartmulticolumn{2}%
\begin{varwidth}[t]{\sphinxcolwidth{2}{4}}
\sphinxAtStartPar
\sphinxstyleemphasis{string}
\par
\vskip-\baselineskip\vbox{\hbox{\strut}}\end{varwidth}%
\sphinxstopmulticolumn
\\
\cline{2-4}\sphinxtablestrut{46}&
\sphinxAtStartPar
default
&\sphinxstartmulticolumn{2}%
\begin{varwidth}[t]{\sphinxcolwidth{2}{4}}
\sphinxAtStartPar
source\sphinxhyphen{}over
\par
\vskip-\baselineskip\vbox{\hbox{\strut}}\end{varwidth}%
\sphinxstopmulticolumn
\\
\hline\sphinxmultirow{3}{52}{%
\begin{varwidth}[t]{\sphinxcolwidth{1}{4}}
\begin{itemize}
\item {} 
\sphinxAtStartPar
mpp

\end{itemize}
\par
\vskip-\baselineskip\vbox{\hbox{\strut}}\end{varwidth}%
}%
&\sphinxstartmulticolumn{3}%
\begin{varwidth}[t]{\sphinxcolwidth{3}{4}}
\sphinxAtStartPar
The image scale in Microns Per Pixels. If not null, then adds a scale bar to the viewer. Set to 0 to display the scale bar in pixels.
\par
\vskip-\baselineskip\vbox{\hbox{\strut}}\end{varwidth}%
\sphinxstopmulticolumn
\\
\cline{2-4}\sphinxtablestrut{52}&
\sphinxAtStartPar
type
&\sphinxstartmulticolumn{2}%
\begin{varwidth}[t]{\sphinxcolwidth{2}{4}}
\sphinxAtStartPar
\sphinxstyleemphasis{float}
\par
\vskip-\baselineskip\vbox{\hbox{\strut}}\end{varwidth}%
\sphinxstopmulticolumn
\\
\cline{2-4}\sphinxtablestrut{52}&
\sphinxAtStartPar
default
&\sphinxstartmulticolumn{2}%
\begin{varwidth}[t]{\sphinxcolwidth{2}{4}}
\sphinxAtStartPar
null
\par
\vskip-\baselineskip\vbox{\hbox{\strut}}\end{varwidth}%
\sphinxstopmulticolumn
\\
\hline\sphinxmultirow{12}{58}{%
\begin{varwidth}[t]{\sphinxcolwidth{1}{4}}
\begin{itemize}
\item {} 
\sphinxAtStartPar
boundingBox

\end{itemize}
\par
\vskip-\baselineskip\vbox{\hbox{\strut}}\end{varwidth}%
}%
&\sphinxstartmulticolumn{3}%
\begin{varwidth}[t]{\sphinxcolwidth{3}{4}}
\sphinxAtStartPar
Bounding box used to set initial zoom and pan on the view when loading the project.
\par
\vskip-\baselineskip\vbox{\hbox{\strut}}\end{varwidth}%
\sphinxstopmulticolumn
\\
\cline{2-4}\sphinxtablestrut{58}&
\sphinxAtStartPar
type
&\sphinxstartmulticolumn{2}%
\begin{varwidth}[t]{\sphinxcolwidth{2}{4}}
\sphinxAtStartPar
\sphinxstyleemphasis{object}
\par
\vskip-\baselineskip\vbox{\hbox{\strut}}\end{varwidth}%
\sphinxstopmulticolumn
\\
\cline{2-4}\sphinxtablestrut{58}&
\sphinxAtStartPar
default
&\sphinxstartmulticolumn{2}%
\begin{varwidth}[t]{\sphinxcolwidth{2}{4}}
\sphinxAtStartPar
null
\par
\vskip-\baselineskip\vbox{\hbox{\strut}}\end{varwidth}%
\sphinxstopmulticolumn
\\
\cline{2-4}\sphinxtablestrut{58}&\sphinxstartmulticolumn{3}%
\begin{varwidth}[t]{\sphinxcolwidth{3}{4}}
\sphinxAtStartPar
properties
\par
\vskip-\baselineskip\vbox{\hbox{\strut}}\end{varwidth}%
\sphinxstopmulticolumn
\\
\cline{2-4}\sphinxtablestrut{58}&\sphinxmultirow{2}{65}{%
\begin{varwidth}[t]{\sphinxcolwidth{1}{4}}
\begin{itemize}
\item {} 
\sphinxAtStartPar
\sphinxstylestrong{x}

\end{itemize}
\par
\vskip-\baselineskip\vbox{\hbox{\strut}}\end{varwidth}%
}%
&\sphinxstartmulticolumn{2}%
\begin{varwidth}[t]{\sphinxcolwidth{2}{4}}
\sphinxAtStartPar
Left coordinate of the bounding box in pixels
\par
\vskip-\baselineskip\vbox{\hbox{\strut}}\end{varwidth}%
\sphinxstopmulticolumn
\\
\cline{3-4}\sphinxtablestrut{58}&\sphinxtablestrut{65}&
\sphinxAtStartPar
type
&
\sphinxAtStartPar
\sphinxstyleemphasis{float}
\\
\cline{2-4}\sphinxtablestrut{58}&\sphinxmultirow{2}{69}{%
\begin{varwidth}[t]{\sphinxcolwidth{1}{4}}
\begin{itemize}
\item {} 
\sphinxAtStartPar
\sphinxstylestrong{y}

\end{itemize}
\par
\vskip-\baselineskip\vbox{\hbox{\strut}}\end{varwidth}%
}%
&\sphinxstartmulticolumn{2}%
\begin{varwidth}[t]{\sphinxcolwidth{2}{4}}
\sphinxAtStartPar
Top coordinate of the bounding box in pixels
\par
\vskip-\baselineskip\vbox{\hbox{\strut}}\end{varwidth}%
\sphinxstopmulticolumn
\\
\cline{3-4}\sphinxtablestrut{58}&\sphinxtablestrut{69}&
\sphinxAtStartPar
type
&
\sphinxAtStartPar
\sphinxstyleemphasis{float}
\\
\cline{2-4}\sphinxtablestrut{58}&\sphinxmultirow{2}{73}{%
\begin{varwidth}[t]{\sphinxcolwidth{1}{4}}
\begin{itemize}
\item {} 
\sphinxAtStartPar
\sphinxstylestrong{width}

\end{itemize}
\par
\vskip-\baselineskip\vbox{\hbox{\strut}}\end{varwidth}%
}%
&\sphinxstartmulticolumn{2}%
\begin{varwidth}[t]{\sphinxcolwidth{2}{4}}
\sphinxAtStartPar
Width of the bounding box in pixels
\par
\vskip-\baselineskip\vbox{\hbox{\strut}}\end{varwidth}%
\sphinxstopmulticolumn
\\
\cline{3-4}\sphinxtablestrut{58}&\sphinxtablestrut{73}&
\sphinxAtStartPar
type
&
\sphinxAtStartPar
\sphinxstyleemphasis{float}
\\
\cline{2-4}\sphinxtablestrut{58}&\sphinxmultirow{2}{77}{%
\begin{varwidth}[t]{\sphinxcolwidth{1}{4}}
\begin{itemize}
\item {} 
\sphinxAtStartPar
\sphinxstylestrong{height}

\end{itemize}
\par
\vskip-\baselineskip\vbox{\hbox{\strut}}\end{varwidth}%
}%
&\sphinxstartmulticolumn{2}%
\begin{varwidth}[t]{\sphinxcolwidth{2}{4}}
\sphinxAtStartPar
Height of the bounding box in pixels
\par
\vskip-\baselineskip\vbox{\hbox{\strut}}\end{varwidth}%
\sphinxstopmulticolumn
\\
\cline{3-4}\sphinxtablestrut{58}&\sphinxtablestrut{77}&
\sphinxAtStartPar
type
&
\sphinxAtStartPar
\sphinxstyleemphasis{float}
\\
\hline\sphinxmultirow{3}{81}{%
\begin{varwidth}[t]{\sphinxcolwidth{1}{4}}
\begin{itemize}
\item {} 
\sphinxAtStartPar
rotate

\end{itemize}
\par
\vskip-\baselineskip\vbox{\hbox{\strut}}\end{varwidth}%
}%
&\sphinxstartmulticolumn{3}%
\begin{varwidth}[t]{\sphinxcolwidth{3}{4}}
\sphinxAtStartPar
Angle of rotation of the view in degrees. Only multiples of 90 degrees are supported.
\par
\vskip-\baselineskip\vbox{\hbox{\strut}}\end{varwidth}%
\sphinxstopmulticolumn
\\
\cline{2-4}\sphinxtablestrut{81}&
\sphinxAtStartPar
type
&\sphinxstartmulticolumn{2}%
\begin{varwidth}[t]{\sphinxcolwidth{2}{4}}
\sphinxAtStartPar
\sphinxstyleemphasis{integer}
\par
\vskip-\baselineskip\vbox{\hbox{\strut}}\end{varwidth}%
\sphinxstopmulticolumn
\\
\cline{2-4}\sphinxtablestrut{81}&
\sphinxAtStartPar
default
&\sphinxstartmulticolumn{2}%
\begin{varwidth}[t]{\sphinxcolwidth{2}{4}}
\sphinxAtStartPar
0
\par
\vskip-\baselineskip\vbox{\hbox{\strut}}\end{varwidth}%
\sphinxstopmulticolumn
\\
\hline\sphinxmultirow{4}{87}{%
\begin{varwidth}[t]{\sphinxcolwidth{1}{4}}
\begin{itemize}
\item {} 
\sphinxAtStartPar
markerFiles

\end{itemize}
\par
\vskip-\baselineskip\vbox{\hbox{\strut}}\end{varwidth}%
}%
&
\sphinxAtStartPar
type
&\sphinxstartmulticolumn{2}%
\begin{varwidth}[t]{\sphinxcolwidth{2}{4}}
\sphinxAtStartPar
\sphinxstyleemphasis{array}
\par
\vskip-\baselineskip\vbox{\hbox{\strut}}\end{varwidth}%
\sphinxstopmulticolumn
\\
\cline{2-4}\sphinxtablestrut{87}&
\sphinxAtStartPar
default
&\sphinxstartmulticolumn{2}%
\begin{varwidth}[t]{\sphinxcolwidth{2}{4}}
\sphinxAtStartPar
{[}{]}
\par
\vskip-\baselineskip\vbox{\hbox{\strut}}\end{varwidth}%
\sphinxstopmulticolumn
\\
\cline{2-4}\sphinxtablestrut{87}&\sphinxstartmulticolumn{3}%
\begin{varwidth}[t]{\sphinxcolwidth{3}{4}}
\sphinxAtStartPar
items
\par
\vskip-\baselineskip\vbox{\hbox{\strut}}\end{varwidth}%
\sphinxstopmulticolumn
\\
\cline{2-4}\sphinxtablestrut{87}&\begin{itemize}
\item {} 
\end{itemize}
&\sphinxstartmulticolumn{2}%
\begin{varwidth}[t]{\sphinxcolwidth{2}{4}}
\sphinxAtStartPar
{\hyperref[\detokenize{docs/advanced/tmap:markerfile}]{\sphinxcrossref{MarkerFile}}}
\par
\vskip-\baselineskip\vbox{\hbox{\strut}}\end{varwidth}%
\sphinxstopmulticolumn
\\
\hline\sphinxmultirow{3}{95}{%
\begin{varwidth}[t]{\sphinxcolwidth{1}{4}}
\begin{itemize}
\item {} 
\sphinxAtStartPar
regions

\end{itemize}
\par
\vskip-\baselineskip\vbox{\hbox{\strut}}\end{varwidth}%
}%
&\sphinxstartmulticolumn{3}%
\begin{varwidth}[t]{\sphinxcolwidth{3}{4}}
\sphinxAtStartPar
GeoJSON object, see {\hyperref[\detokenize{docs/starting/regions:regions}]{\sphinxcrossref{\DUrole{std,std-ref}{Regions section}}}}.
\par
\vskip-\baselineskip\vbox{\hbox{\strut}}\end{varwidth}%
\sphinxstopmulticolumn
\\
\cline{2-4}\sphinxtablestrut{95}&
\sphinxAtStartPar
type
&\sphinxstartmulticolumn{2}%
\begin{varwidth}[t]{\sphinxcolwidth{2}{4}}
\sphinxAtStartPar
\sphinxstyleemphasis{object}
\par
\vskip-\baselineskip\vbox{\hbox{\strut}}\end{varwidth}%
\sphinxstopmulticolumn
\\
\cline{2-4}\sphinxtablestrut{95}&
\sphinxAtStartPar
default
&\sphinxstartmulticolumn{2}%
\begin{varwidth}[t]{\sphinxcolwidth{2}{4}}
\sphinxAtStartPar
\{\}
\par
\vskip-\baselineskip\vbox{\hbox{\strut}}\end{varwidth}%
\sphinxstopmulticolumn
\\
\hline\sphinxmultirow{3}{101}{%
\begin{varwidth}[t]{\sphinxcolwidth{1}{4}}
\begin{itemize}
\item {} 
\sphinxAtStartPar
regionFile

\end{itemize}
\par
\vskip-\baselineskip\vbox{\hbox{\strut}}\end{varwidth}%
}%
&\sphinxstartmulticolumn{3}%
\begin{varwidth}[t]{\sphinxcolwidth{3}{4}}
\sphinxAtStartPar
\sphinxstylestrong{(Deprecated)} GeoJSON region file loaded on project initialization. Use regionFiles instead.
\par
\vskip-\baselineskip\vbox{\hbox{\strut}}\end{varwidth}%
\sphinxstopmulticolumn
\\
\cline{2-4}\sphinxtablestrut{101}&
\sphinxAtStartPar
type
&\sphinxstartmulticolumn{2}%
\begin{varwidth}[t]{\sphinxcolwidth{2}{4}}
\sphinxAtStartPar
\sphinxstyleemphasis{string}
\par
\vskip-\baselineskip\vbox{\hbox{\strut}}\end{varwidth}%
\sphinxstopmulticolumn
\\
\cline{2-4}\sphinxtablestrut{101}&
\sphinxAtStartPar
default
&\sphinxstartmulticolumn{2}%
\begin{varwidth}[t]{\sphinxcolwidth{2}{4}}
\par
\vskip-\baselineskip\vbox{\hbox{\strut}}\end{varwidth}%
\sphinxstopmulticolumn
\\
\hline\sphinxmultirow{4}{107}{%
\begin{varwidth}[t]{\sphinxcolwidth{1}{4}}
\begin{itemize}
\item {} 
\sphinxAtStartPar
regionFiles

\end{itemize}
\par
\vskip-\baselineskip\vbox{\hbox{\strut}}\end{varwidth}%
}%
&
\sphinxAtStartPar
type
&\sphinxstartmulticolumn{2}%
\begin{varwidth}[t]{\sphinxcolwidth{2}{4}}
\sphinxAtStartPar
\sphinxstyleemphasis{array}
\par
\vskip-\baselineskip\vbox{\hbox{\strut}}\end{varwidth}%
\sphinxstopmulticolumn
\\
\cline{2-4}\sphinxtablestrut{107}&
\sphinxAtStartPar
default
&\sphinxstartmulticolumn{2}%
\begin{varwidth}[t]{\sphinxcolwidth{2}{4}}
\sphinxAtStartPar
{[}{]}
\par
\vskip-\baselineskip\vbox{\hbox{\strut}}\end{varwidth}%
\sphinxstopmulticolumn
\\
\cline{2-4}\sphinxtablestrut{107}&\sphinxstartmulticolumn{3}%
\begin{varwidth}[t]{\sphinxcolwidth{3}{4}}
\sphinxAtStartPar
items
\par
\vskip-\baselineskip\vbox{\hbox{\strut}}\end{varwidth}%
\sphinxstopmulticolumn
\\
\cline{2-4}\sphinxtablestrut{107}&\begin{itemize}
\item {} 
\end{itemize}
&\sphinxstartmulticolumn{2}%
\begin{varwidth}[t]{\sphinxcolwidth{2}{4}}
\sphinxAtStartPar
{\hyperref[\detokenize{docs/advanced/tmap:regionfile}]{\sphinxcrossref{RegionFile}}}
\par
\vskip-\baselineskip\vbox{\hbox{\strut}}\end{varwidth}%
\sphinxstopmulticolumn
\\
\hline\sphinxmultirow{5}{115}{%
\begin{varwidth}[t]{\sphinxcolwidth{1}{4}}
\begin{itemize}
\item {} 
\sphinxAtStartPar
plugins

\end{itemize}
\par
\vskip-\baselineskip\vbox{\hbox{\strut}}\end{varwidth}%
}%
&\sphinxstartmulticolumn{3}%
\begin{varwidth}[t]{\sphinxcolwidth{3}{4}}
\sphinxAtStartPar
List of plugins to load with the project. See also the {\hyperref[\detokenize{docs/starting/plugins:plugins}]{\sphinxcrossref{\DUrole{std,std-ref}{Plugins section}}}}.
\par
\vskip-\baselineskip\vbox{\hbox{\strut}}\end{varwidth}%
\sphinxstopmulticolumn
\\
\cline{2-4}\sphinxtablestrut{115}&
\sphinxAtStartPar
type
&\sphinxstartmulticolumn{2}%
\begin{varwidth}[t]{\sphinxcolwidth{2}{4}}
\sphinxAtStartPar
\sphinxstyleemphasis{array}
\par
\vskip-\baselineskip\vbox{\hbox{\strut}}\end{varwidth}%
\sphinxstopmulticolumn
\\
\cline{2-4}\sphinxtablestrut{115}&
\sphinxAtStartPar
default
&\sphinxstartmulticolumn{2}%
\begin{varwidth}[t]{\sphinxcolwidth{2}{4}}
\sphinxAtStartPar
{[}{]}
\par
\vskip-\baselineskip\vbox{\hbox{\strut}}\end{varwidth}%
\sphinxstopmulticolumn
\\
\cline{2-4}\sphinxtablestrut{115}&\sphinxstartmulticolumn{3}%
\begin{varwidth}[t]{\sphinxcolwidth{3}{4}}
\sphinxAtStartPar
items
\par
\vskip-\baselineskip\vbox{\hbox{\strut}}\end{varwidth}%
\sphinxstopmulticolumn
\\
\cline{2-4}\sphinxtablestrut{115}&\begin{itemize}
\item {} 
\end{itemize}
&
\sphinxAtStartPar
type
&
\sphinxAtStartPar
\sphinxstyleemphasis{string}
\\
\hline\sphinxmultirow{3}{125}{%
\begin{varwidth}[t]{\sphinxcolwidth{1}{4}}
\begin{itemize}
\item {} 
\sphinxAtStartPar
hideTabs

\end{itemize}
\par
\vskip-\baselineskip\vbox{\hbox{\strut}}\end{varwidth}%
}%
&\sphinxstartmulticolumn{3}%
\begin{varwidth}[t]{\sphinxcolwidth{3}{4}}
\sphinxAtStartPar
Hide tabs of markers dataset. Only use when you have a unique marker tab.
\par
\vskip-\baselineskip\vbox{\hbox{\strut}}\end{varwidth}%
\sphinxstopmulticolumn
\\
\cline{2-4}\sphinxtablestrut{125}&
\sphinxAtStartPar
type
&\sphinxstartmulticolumn{2}%
\begin{varwidth}[t]{\sphinxcolwidth{2}{4}}
\sphinxAtStartPar
\sphinxstyleemphasis{boolean}
\par
\vskip-\baselineskip\vbox{\hbox{\strut}}\end{varwidth}%
\sphinxstopmulticolumn
\\
\cline{2-4}\sphinxtablestrut{125}&
\sphinxAtStartPar
default
&\sphinxstartmulticolumn{2}%
\begin{varwidth}[t]{\sphinxcolwidth{2}{4}}
\sphinxAtStartPar
false
\par
\vskip-\baselineskip\vbox{\hbox{\strut}}\end{varwidth}%
\sphinxstopmulticolumn
\\
\hline\sphinxmultirow{4}{131}{%
\begin{varwidth}[t]{\sphinxcolwidth{1}{4}}
\begin{itemize}
\item {} 
\sphinxAtStartPar
settings

\end{itemize}
\par
\vskip-\baselineskip\vbox{\hbox{\strut}}\end{varwidth}%
}%
&
\sphinxAtStartPar
type
&\sphinxstartmulticolumn{2}%
\begin{varwidth}[t]{\sphinxcolwidth{2}{4}}
\sphinxAtStartPar
\sphinxstyleemphasis{array}
\par
\vskip-\baselineskip\vbox{\hbox{\strut}}\end{varwidth}%
\sphinxstopmulticolumn
\\
\cline{2-4}\sphinxtablestrut{131}&
\sphinxAtStartPar
default
&\sphinxstartmulticolumn{2}%
\begin{varwidth}[t]{\sphinxcolwidth{2}{4}}
\sphinxAtStartPar
{[}{]}
\par
\vskip-\baselineskip\vbox{\hbox{\strut}}\end{varwidth}%
\sphinxstopmulticolumn
\\
\cline{2-4}\sphinxtablestrut{131}&\sphinxstartmulticolumn{3}%
\begin{varwidth}[t]{\sphinxcolwidth{3}{4}}
\sphinxAtStartPar
items
\par
\vskip-\baselineskip\vbox{\hbox{\strut}}\end{varwidth}%
\sphinxstopmulticolumn
\\
\cline{2-4}\sphinxtablestrut{131}&\begin{itemize}
\item {} 
\end{itemize}
&\sphinxstartmulticolumn{2}%
\begin{varwidth}[t]{\sphinxcolwidth{2}{4}}
\sphinxAtStartPar
{\hyperref[\detokenize{docs/advanced/tmap:setting}]{\sphinxcrossref{Setting}}}
\par
\vskip-\baselineskip\vbox{\hbox{\strut}}\end{varwidth}%
\sphinxstopmulticolumn
\\
\hline
\end{longtable}\sphinxatlongtableend\end{savenotes}


\subsubsection{Layer}
\label{\detokenize{docs/advanced/tmap:layer}}\label{\detokenize{docs/advanced/tmap:layer}}

\begin{savenotes}\sphinxattablestart
\centering
\begin{tabular}[t]{|*{3}{\X{1}{3}|}}
\hline
\sphinxstartmulticolumn{3}%
\begin{varwidth}[t]{\sphinxcolwidth{3}{3}}
\sphinxAtStartPar
Description of an image layer. Required properties are shown in \sphinxstylestrong{bold} text
\par
\vskip-\baselineskip\vbox{\hbox{\strut}}\end{varwidth}%
\sphinxstopmulticolumn
\\
\hline
\sphinxAtStartPar
type
&\sphinxstartmulticolumn{2}%
\begin{varwidth}[t]{\sphinxcolwidth{2}{3}}
\sphinxAtStartPar
\sphinxstyleemphasis{object}
\par
\vskip-\baselineskip\vbox{\hbox{\strut}}\end{varwidth}%
\sphinxstopmulticolumn
\\
\hline\sphinxstartmulticolumn{3}%
\begin{varwidth}[t]{\sphinxcolwidth{3}{3}}
\sphinxAtStartPar
properties
\par
\vskip-\baselineskip\vbox{\hbox{\strut}}\end{varwidth}%
\sphinxstopmulticolumn
\\
\hline\sphinxmultirow{2}{5}{%
\begin{varwidth}[t]{\sphinxcolwidth{1}{3}}
\begin{itemize}
\item {} 
\sphinxAtStartPar
\sphinxstylestrong{name}

\end{itemize}
\par
\vskip-\baselineskip\vbox{\hbox{\strut}}\end{varwidth}%
}%
&\sphinxstartmulticolumn{2}%
\begin{varwidth}[t]{\sphinxcolwidth{2}{3}}
\sphinxAtStartPar
Name of the image layer
\par
\vskip-\baselineskip\vbox{\hbox{\strut}}\end{varwidth}%
\sphinxstopmulticolumn
\\
\cline{2-3}\sphinxtablestrut{5}&
\sphinxAtStartPar
type
&
\sphinxAtStartPar
\sphinxstyleemphasis{string}
\\
\hline\sphinxmultirow{2}{9}{%
\begin{varwidth}[t]{\sphinxcolwidth{1}{3}}
\begin{itemize}
\item {} 
\sphinxAtStartPar
\sphinxstylestrong{tileSource}

\end{itemize}
\par
\vskip-\baselineskip\vbox{\hbox{\strut}}\end{varwidth}%
}%
&\sphinxstartmulticolumn{2}%
\begin{varwidth}[t]{\sphinxcolwidth{2}{3}}
\sphinxAtStartPar
Relative path to an image file in a supported format. See also the {\hyperref[\detokenize{docs/starting/images:images}]{\sphinxcrossref{\DUrole{std,std-ref}{Images section}}}}.
\par
\vskip-\baselineskip\vbox{\hbox{\strut}}\end{varwidth}%
\sphinxstopmulticolumn
\\
\cline{2-3}\sphinxtablestrut{9}&
\sphinxAtStartPar
type
&
\sphinxAtStartPar
\sphinxstyleemphasis{string}
\\
\hline
\end{tabular}
\par
\sphinxattableend\end{savenotes}


\subsubsection{LayerFilter}
\label{\detokenize{docs/advanced/tmap:layerfilter}}\label{\detokenize{docs/advanced/tmap:layerfilter}}

\begin{savenotes}\sphinxattablestart
\centering
\begin{tabular}[t]{|*{4}{\X{1}{4}|}}
\hline
\sphinxstartmulticolumn{4}%
\begin{varwidth}[t]{\sphinxcolwidth{4}{4}}
\sphinxAtStartPar
Description of an image filter to be applied to the pixels in an image layer. Required properties are shown in \sphinxstylestrong{bold} text
\par
\vskip-\baselineskip\vbox{\hbox{\strut}}\end{varwidth}%
\sphinxstopmulticolumn
\\
\hline
\sphinxAtStartPar
type
&\sphinxstartmulticolumn{3}%
\begin{varwidth}[t]{\sphinxcolwidth{3}{4}}
\sphinxAtStartPar
\sphinxstyleemphasis{array}
\par
\vskip-\baselineskip\vbox{\hbox{\strut}}\end{varwidth}%
\sphinxstopmulticolumn
\\
\hline\sphinxstartmulticolumn{4}%
\begin{varwidth}[t]{\sphinxcolwidth{4}{4}}
\sphinxAtStartPar
items
\par
\vskip-\baselineskip\vbox{\hbox{\strut}}\end{varwidth}%
\sphinxstopmulticolumn
\\
\hline\sphinxmultirow{6}{5}{%
\begin{varwidth}[t]{\sphinxcolwidth{1}{4}}
\begin{itemize}
\item {} 
\end{itemize}
\par
\vskip-\baselineskip\vbox{\hbox{\strut}}\end{varwidth}%
}%
&
\sphinxAtStartPar
type
&\sphinxstartmulticolumn{2}%
\begin{varwidth}[t]{\sphinxcolwidth{2}{4}}
\sphinxAtStartPar
\sphinxstyleemphasis{object}
\par
\vskip-\baselineskip\vbox{\hbox{\strut}}\end{varwidth}%
\sphinxstopmulticolumn
\\
\cline{2-4}\sphinxtablestrut{5}&\sphinxstartmulticolumn{3}%
\begin{varwidth}[t]{\sphinxcolwidth{3}{4}}
\sphinxAtStartPar
properties
\par
\vskip-\baselineskip\vbox{\hbox{\strut}}\end{varwidth}%
\sphinxstopmulticolumn
\\
\cline{2-4}\sphinxtablestrut{5}&\sphinxmultirow{2}{9}{%
\begin{varwidth}[t]{\sphinxcolwidth{1}{4}}
\begin{itemize}
\item {} 
\sphinxAtStartPar
\sphinxstylestrong{name}

\end{itemize}
\par
\vskip-\baselineskip\vbox{\hbox{\strut}}\end{varwidth}%
}%
&\sphinxstartmulticolumn{2}%
\begin{varwidth}[t]{\sphinxcolwidth{2}{4}}
\sphinxAtStartPar
Filter name. See {\hyperref[\detokenize{docs/advanced/tmap:filter}]{\sphinxcrossref{\DUrole{std,std-ref}{Filter}}}} for more details.
\par
\vskip-\baselineskip\vbox{\hbox{\strut}}\end{varwidth}%
\sphinxstopmulticolumn
\\
\cline{3-4}\sphinxtablestrut{5}&\sphinxtablestrut{9}&
\sphinxAtStartPar
type
&
\sphinxAtStartPar
\sphinxstyleemphasis{string}
\\
\cline{2-4}\sphinxtablestrut{5}&\sphinxmultirow{2}{13}{%
\begin{varwidth}[t]{\sphinxcolwidth{1}{4}}
\begin{itemize}
\item {} 
\sphinxAtStartPar
\sphinxstylestrong{value}

\end{itemize}
\par
\vskip-\baselineskip\vbox{\hbox{\strut}}\end{varwidth}%
}%
&\sphinxstartmulticolumn{2}%
\begin{varwidth}[t]{\sphinxcolwidth{2}{4}}
\sphinxAtStartPar
Filter parameter. See {\hyperref[\detokenize{docs/advanced/tmap:filter}]{\sphinxcrossref{\DUrole{std,std-ref}{Filter}}}} for more details.
\par
\vskip-\baselineskip\vbox{\hbox{\strut}}\end{varwidth}%
\sphinxstopmulticolumn
\\
\cline{3-4}\sphinxtablestrut{5}&\sphinxtablestrut{13}&
\sphinxAtStartPar
type
&
\sphinxAtStartPar
\sphinxstyleemphasis{string}
\\
\hline
\end{tabular}
\par
\sphinxattableend\end{savenotes}


\subsubsection{Filter}
\label{\detokenize{docs/advanced/tmap:filter}}\label{\detokenize{docs/advanced/tmap:filter}}

\begin{savenotes}\sphinxattablestart
\centering
\begin{tabulary}{\linewidth}[t]{|T|T|}
\hline
\sphinxstartmulticolumn{2}%
\begin{varwidth}[t]{\sphinxcolwidth{2}{2}}
\sphinxAtStartPar
TissUUmaps supports most filters available in OpenSeadragon via the \sphinxurl{https://github.com/usnistgov/OpenSeadragonFiltering} plugin.
\par
\vskip-\baselineskip\vbox{\hbox{\strut}}\end{varwidth}%
\sphinxstopmulticolumn
\\
\hline
\sphinxAtStartPar
enum
&
\sphinxAtStartPar
Color, Brightness, Exposure, Hue, Contrast, Vibrance, Noise, Saturation, Gamma, Invert, Greyscale, Threshold, Erosion, Dilation
\\
\hline
\end{tabulary}
\par
\sphinxattableend\end{savenotes}


\subsubsection{ColorScale}
\label{\detokenize{docs/advanced/tmap:colorscale}}\label{\detokenize{docs/advanced/tmap:colorscale}}

\begin{savenotes}\sphinxattablestart
\centering
\begin{tabulary}{\linewidth}[t]{|T|T|}
\hline
\sphinxstartmulticolumn{2}%
\begin{varwidth}[t]{\sphinxcolwidth{2}{2}}
\sphinxAtStartPar
TissUUmaps supports most of the color scales available in the D3.js library. See \sphinxurl{https://github.com/d3/d3-scale-chromatic} for reference. Note: the colors for ‘interpolateRainbow’ are currently overridden by a custom Turbo\sphinxhyphen{}like color scale in version 3.0.x of TissUUmaps.
\par
\vskip-\baselineskip\vbox{\hbox{\strut}}\end{varwidth}%
\sphinxstopmulticolumn
\\
\hline
\sphinxAtStartPar
enum
&
\sphinxAtStartPar
interpolateCubehelixDefault, interpolateRainbow, interpolateWarm, interpolateCool, interpolateViridis, interpolateMagma, interpolateInferno, interpolatePlasma, interpolateBlues, interpolateBrBG, interpolateBuGn, interpolateBuPu, interpolateCividis, interpolateGnBu, interpolateGreens, interpolateGreys, interpolateOrRd, interpolateOranges, interpolatePRGn, interpolatePiYG, interpolatePuBu, interpolatePuBuGn, interpolatePuOr, interpolatePuRd, interpolatePurples, interpolateRdBu, interpolateRdGy, interpolateRdPu, interpolateRdYlBu, interpolateRdYlGn, interpolateReds, interpolateSinebow, interpolateSpectral, interpolateTurbo, interpolateYlGn, interpolateYlGnBu, interpolateYlOrBr, interpolateYlOrRd
\\
\hline
\end{tabulary}
\par
\sphinxattableend\end{savenotes}


\subsubsection{Shape}
\label{\detokenize{docs/advanced/tmap:shape}}\label{\detokenize{docs/advanced/tmap:shape}}

\begin{savenotes}\sphinxattablestart
\centering
\begin{tabulary}{\linewidth}[t]{|T|T|}
\hline
\sphinxstartmulticolumn{2}%
\begin{varwidth}[t]{\sphinxcolwidth{2}{2}}
\sphinxAtStartPar
TissUUmaps supports most of the marker shapes that are also used by the Napari software, \sphinxurl{https://napari.org}. In addition to the name strings listed below, shape can also be specified by a corresponding index in range 0\sphinxhyphen{}13.
\par
\vskip-\baselineskip\vbox{\hbox{\strut}}\end{varwidth}%
\sphinxstopmulticolumn
\\
\hline
\sphinxAtStartPar
enum
&
\sphinxAtStartPar
cross, diamond, square, triangle up, star, clobber, disc, hbar, vbar, tailed arrow, triangle down, ring, x, arrow
\\
\hline
\end{tabulary}
\par
\sphinxattableend\end{savenotes}


\subsubsection{MarkerFile}
\label{\detokenize{docs/advanced/tmap:markerfile}}\label{\detokenize{docs/advanced/tmap:markerfile}}

\begin{savenotes}\sphinxattablestart
\centering
\begin{tabular}[t]{|*{3}{\X{1}{3}|}}
\hline
\sphinxstartmulticolumn{3}%
\begin{varwidth}[t]{\sphinxcolwidth{3}{3}}
\sphinxAtStartPar
Description of settings and GUI objects for a marker dataset loaded from CSV file. Required properties are shown in \sphinxstylestrong{bold} text.
\par
\vskip-\baselineskip\vbox{\hbox{\strut}}\end{varwidth}%
\sphinxstopmulticolumn
\\
\hline
\sphinxAtStartPar
type
&\sphinxstartmulticolumn{2}%
\begin{varwidth}[t]{\sphinxcolwidth{2}{3}}
\sphinxAtStartPar
\sphinxstyleemphasis{object}
\par
\vskip-\baselineskip\vbox{\hbox{\strut}}\end{varwidth}%
\sphinxstopmulticolumn
\\
\hline\sphinxstartmulticolumn{3}%
\begin{varwidth}[t]{\sphinxcolwidth{3}{3}}
\sphinxAtStartPar
properties
\par
\vskip-\baselineskip\vbox{\hbox{\strut}}\end{varwidth}%
\sphinxstopmulticolumn
\\
\hline\sphinxmultirow{2}{5}{%
\begin{varwidth}[t]{\sphinxcolwidth{1}{3}}
\begin{itemize}
\item {} 
\sphinxAtStartPar
\sphinxstylestrong{title}

\end{itemize}
\par
\vskip-\baselineskip\vbox{\hbox{\strut}}\end{varwidth}%
}%
&\sphinxstartmulticolumn{2}%
\begin{varwidth}[t]{\sphinxcolwidth{2}{3}}
\sphinxAtStartPar
Name of marker button
\par
\vskip-\baselineskip\vbox{\hbox{\strut}}\end{varwidth}%
\sphinxstopmulticolumn
\\
\cline{2-3}\sphinxtablestrut{5}&
\sphinxAtStartPar
type
&
\sphinxAtStartPar
\sphinxstyleemphasis{string}
\\
\hline\sphinxmultirow{3}{9}{%
\begin{varwidth}[t]{\sphinxcolwidth{1}{3}}
\begin{itemize}
\item {} 
\sphinxAtStartPar
comment

\end{itemize}
\par
\vskip-\baselineskip\vbox{\hbox{\strut}}\end{varwidth}%
}%
&\sphinxstartmulticolumn{2}%
\begin{varwidth}[t]{\sphinxcolwidth{2}{3}}
\sphinxAtStartPar
Optional description text shown next to marker button
\par
\vskip-\baselineskip\vbox{\hbox{\strut}}\end{varwidth}%
\sphinxstopmulticolumn
\\
\cline{2-3}\sphinxtablestrut{9}&
\sphinxAtStartPar
type
&
\sphinxAtStartPar
\sphinxstyleemphasis{string}
\\
\cline{2-3}\sphinxtablestrut{9}&
\sphinxAtStartPar
default
&\\
\hline\sphinxmultirow{2}{15}{%
\begin{varwidth}[t]{\sphinxcolwidth{1}{3}}
\begin{itemize}
\item {} 
\sphinxAtStartPar
\sphinxstylestrong{name}

\end{itemize}
\par
\vskip-\baselineskip\vbox{\hbox{\strut}}\end{varwidth}%
}%
&\sphinxstartmulticolumn{2}%
\begin{varwidth}[t]{\sphinxcolwidth{2}{3}}
\sphinxAtStartPar
Name of marker tab
\par
\vskip-\baselineskip\vbox{\hbox{\strut}}\end{varwidth}%
\sphinxstopmulticolumn
\\
\cline{2-3}\sphinxtablestrut{15}&
\sphinxAtStartPar
type
&
\sphinxAtStartPar
\sphinxstyleemphasis{string}
\\
\hline\sphinxmultirow{3}{19}{%
\begin{varwidth}[t]{\sphinxcolwidth{1}{3}}
\begin{itemize}
\item {} 
\sphinxAtStartPar
autoLoad

\end{itemize}
\par
\vskip-\baselineskip\vbox{\hbox{\strut}}\end{varwidth}%
}%
&\sphinxstartmulticolumn{2}%
\begin{varwidth}[t]{\sphinxcolwidth{2}{3}}
\sphinxAtStartPar
If the CSV file for the marker dataset should be automatically loaded when the TMAP project is opened. If this is false, the user instead has to click on the marker button in the GUI to load the dataset.
\par
\vskip-\baselineskip\vbox{\hbox{\strut}}\end{varwidth}%
\sphinxstopmulticolumn
\\
\cline{2-3}\sphinxtablestrut{19}&
\sphinxAtStartPar
type
&
\sphinxAtStartPar
\sphinxstyleemphasis{boolean}
\\
\cline{2-3}\sphinxtablestrut{19}&
\sphinxAtStartPar
default
&
\sphinxAtStartPar
false
\\
\hline\sphinxmultirow{3}{25}{%
\begin{varwidth}[t]{\sphinxcolwidth{1}{3}}
\begin{itemize}
\item {} 
\sphinxAtStartPar
hideSettings

\end{itemize}
\par
\vskip-\baselineskip\vbox{\hbox{\strut}}\end{varwidth}%
}%
&\sphinxstartmulticolumn{2}%
\begin{varwidth}[t]{\sphinxcolwidth{2}{3}}
\sphinxAtStartPar
Hide markers’ settings and add a toggle button instead.
\par
\vskip-\baselineskip\vbox{\hbox{\strut}}\end{varwidth}%
\sphinxstopmulticolumn
\\
\cline{2-3}\sphinxtablestrut{25}&
\sphinxAtStartPar
type
&
\sphinxAtStartPar
\sphinxstyleemphasis{boolean}
\\
\cline{2-3}\sphinxtablestrut{25}&
\sphinxAtStartPar
default
&
\sphinxAtStartPar
false
\\
\hline\sphinxmultirow{2}{31}{%
\begin{varwidth}[t]{\sphinxcolwidth{1}{3}}
\begin{itemize}
\item {} 
\sphinxAtStartPar
uid

\end{itemize}
\par
\vskip-\baselineskip\vbox{\hbox{\strut}}\end{varwidth}%
}%
&\sphinxstartmulticolumn{2}%
\begin{varwidth}[t]{\sphinxcolwidth{2}{3}}
\sphinxAtStartPar
A unique identifier used internally by TissUUmaps to reference the marker dataset
\par
\vskip-\baselineskip\vbox{\hbox{\strut}}\end{varwidth}%
\sphinxstopmulticolumn
\\
\cline{2-3}\sphinxtablestrut{31}&
\sphinxAtStartPar
type
&
\sphinxAtStartPar
\sphinxstyleemphasis{string}
\\
\hline\begin{itemize}
\item {} 
\sphinxAtStartPar
\sphinxstylestrong{expectedHeader}

\end{itemize}
&\sphinxstartmulticolumn{2}%
\begin{varwidth}[t]{\sphinxcolwidth{2}{3}}
\sphinxAtStartPar
{\hyperref[\detokenize{docs/advanced/tmap:expectedheader}]{\sphinxcrossref{\DUrole{std,std-ref}{ExpectedHeader}}}}
\par
\vskip-\baselineskip\vbox{\hbox{\strut}}\end{varwidth}%
\sphinxstopmulticolumn
\\
\hline\begin{itemize}
\item {} 
\sphinxAtStartPar
expectedRadios

\end{itemize}
&\sphinxstartmulticolumn{2}%
\begin{varwidth}[t]{\sphinxcolwidth{2}{3}}
\sphinxAtStartPar
{\hyperref[\detokenize{docs/advanced/tmap:expectedradios}]{\sphinxcrossref{\DUrole{std,std-ref}{ExpectedRadios}}}}
\par
\vskip-\baselineskip\vbox{\hbox{\strut}}\end{varwidth}%
\sphinxstopmulticolumn
\\
\hline\sphinxmultirow{2}{39}{%
\begin{varwidth}[t]{\sphinxcolwidth{1}{3}}
\begin{itemize}
\item {} 
\sphinxAtStartPar
\sphinxstylestrong{path}

\end{itemize}
\par
\vskip-\baselineskip\vbox{\hbox{\strut}}\end{varwidth}%
}%
&\sphinxstartmulticolumn{2}%
\begin{varwidth}[t]{\sphinxcolwidth{2}{3}}
\sphinxAtStartPar
Relative file path to CSV file in which marker data is stored. If array of string, then a dropdown is created instead of a button.
\par
\vskip-\baselineskip\vbox{\hbox{\strut}}\end{varwidth}%
\sphinxstopmulticolumn
\\
\cline{2-3}\sphinxtablestrut{39}&
\sphinxAtStartPar
type
&
\sphinxAtStartPar
\sphinxstyleemphasis{string} / \sphinxstyleemphasis{array}
\\
\hline\sphinxmultirow{4}{43}{%
\begin{varwidth}[t]{\sphinxcolwidth{1}{3}}
\begin{itemize}
\item {} 
\sphinxAtStartPar
settings

\end{itemize}
\par
\vskip-\baselineskip\vbox{\hbox{\strut}}\end{varwidth}%
}%
&
\sphinxAtStartPar
type
&
\sphinxAtStartPar
\sphinxstyleemphasis{array}
\\
\cline{2-3}\sphinxtablestrut{43}&
\sphinxAtStartPar
default
&
\sphinxAtStartPar
{[}{]}
\\
\cline{2-3}\sphinxtablestrut{43}&\sphinxstartmulticolumn{2}%
\begin{varwidth}[t]{\sphinxcolwidth{2}{3}}
\sphinxAtStartPar
items
\par
\vskip-\baselineskip\vbox{\hbox{\strut}}\end{varwidth}%
\sphinxstopmulticolumn
\\
\cline{2-3}\sphinxtablestrut{43}&\begin{itemize}
\item {} 
\end{itemize}
&
\sphinxAtStartPar
{\hyperref[\detokenize{docs/advanced/tmap:setting}]{\sphinxcrossref{Setting}}}
\\
\hline
\end{tabular}
\par
\sphinxattableend\end{savenotes}


\subsubsection{ExpectedHeader}
\label{\detokenize{docs/advanced/tmap:expectedheader}}\label{\detokenize{docs/advanced/tmap:expectedheader}}

\begin{savenotes}\sphinxatlongtablestart\begin{longtable}[c]{|*{3}{\X{1}{3}|}}
\hline

\endfirsthead

\multicolumn{3}{c}%
{\makebox[0pt]{\sphinxtablecontinued{\tablename\ \thetable{} \textendash{} continued from previous page}}}\\
\hline

\endhead

\hline
\multicolumn{3}{r}{\makebox[0pt][r]{\sphinxtablecontinued{continues on next page}}}\\
\endfoot

\endlastfoot
\sphinxstartmulticolumn{3}%
\begin{varwidth}[t]{\sphinxcolwidth{3}{3}}
\sphinxAtStartPar
Input field values for settings in a marker tab. Required properties are shown in \sphinxstylestrong{bold} text.
\par
\vskip-\baselineskip\vbox{\hbox{\strut}}\end{varwidth}%
\sphinxstopmulticolumn
\\
\hline
\sphinxAtStartPar
type
&\sphinxstartmulticolumn{2}%
\begin{varwidth}[t]{\sphinxcolwidth{2}{3}}
\sphinxAtStartPar
\sphinxstyleemphasis{object}
\par
\vskip-\baselineskip\vbox{\hbox{\strut}}\end{varwidth}%
\sphinxstopmulticolumn
\\
\hline\sphinxstartmulticolumn{3}%
\begin{varwidth}[t]{\sphinxcolwidth{3}{3}}
\sphinxAtStartPar
properties
\par
\vskip-\baselineskip\vbox{\hbox{\strut}}\end{varwidth}%
\sphinxstopmulticolumn
\\
\hline\sphinxmultirow{2}{5}{%
\begin{varwidth}[t]{\sphinxcolwidth{1}{3}}
\begin{itemize}
\item {} 
\sphinxAtStartPar
\sphinxstylestrong{X}

\end{itemize}
\par
\vskip-\baselineskip\vbox{\hbox{\strut}}\end{varwidth}%
}%
&\sphinxstartmulticolumn{2}%
\begin{varwidth}[t]{\sphinxcolwidth{2}{3}}
\sphinxAtStartPar
Name of CSV column to use as X\sphinxhyphen{}coordinate
\par
\vskip-\baselineskip\vbox{\hbox{\strut}}\end{varwidth}%
\sphinxstopmulticolumn
\\
\cline{2-3}\sphinxtablestrut{5}&
\sphinxAtStartPar
type
&
\sphinxAtStartPar
\sphinxstyleemphasis{string}
\\
\hline\sphinxmultirow{2}{9}{%
\begin{varwidth}[t]{\sphinxcolwidth{1}{3}}
\begin{itemize}
\item {} 
\sphinxAtStartPar
\sphinxstylestrong{Y}

\end{itemize}
\par
\vskip-\baselineskip\vbox{\hbox{\strut}}\end{varwidth}%
}%
&\sphinxstartmulticolumn{2}%
\begin{varwidth}[t]{\sphinxcolwidth{2}{3}}
\sphinxAtStartPar
Name of CSV column to use as Y\sphinxhyphen{}coordinate
\par
\vskip-\baselineskip\vbox{\hbox{\strut}}\end{varwidth}%
\sphinxstopmulticolumn
\\
\cline{2-3}\sphinxtablestrut{9}&
\sphinxAtStartPar
type
&
\sphinxAtStartPar
\sphinxstyleemphasis{string}
\\
\hline\sphinxmultirow{3}{13}{%
\begin{varwidth}[t]{\sphinxcolwidth{1}{3}}
\begin{itemize}
\item {} 
\sphinxAtStartPar
gb\_col

\end{itemize}
\par
\vskip-\baselineskip\vbox{\hbox{\strut}}\end{varwidth}%
}%
&\sphinxstartmulticolumn{2}%
\begin{varwidth}[t]{\sphinxcolwidth{2}{3}}
\sphinxAtStartPar
Name of CSV column to use as key to group markers by
\par
\vskip-\baselineskip\vbox{\hbox{\strut}}\end{varwidth}%
\sphinxstopmulticolumn
\\
\cline{2-3}\sphinxtablestrut{13}&
\sphinxAtStartPar
type
&
\sphinxAtStartPar
\sphinxstyleemphasis{string}
\\
\cline{2-3}\sphinxtablestrut{13}&
\sphinxAtStartPar
default
&
\sphinxAtStartPar
null
\\
\hline\sphinxmultirow{3}{19}{%
\begin{varwidth}[t]{\sphinxcolwidth{1}{3}}
\begin{itemize}
\item {} 
\sphinxAtStartPar
gb\_name

\end{itemize}
\par
\vskip-\baselineskip\vbox{\hbox{\strut}}\end{varwidth}%
}%
&\sphinxstartmulticolumn{2}%
\begin{varwidth}[t]{\sphinxcolwidth{2}{3}}
\sphinxAtStartPar
Name of CSV column to display for groups instead of group key value
\par
\vskip-\baselineskip\vbox{\hbox{\strut}}\end{varwidth}%
\sphinxstopmulticolumn
\\
\cline{2-3}\sphinxtablestrut{19}&
\sphinxAtStartPar
type
&
\sphinxAtStartPar
\sphinxstyleemphasis{string}
\\
\cline{2-3}\sphinxtablestrut{19}&
\sphinxAtStartPar
default
&
\sphinxAtStartPar
null
\\
\hline\sphinxmultirow{3}{25}{%
\begin{varwidth}[t]{\sphinxcolwidth{1}{3}}
\begin{itemize}
\item {} 
\sphinxAtStartPar
cb\_cmap

\end{itemize}
\par
\vskip-\baselineskip\vbox{\hbox{\strut}}\end{varwidth}%
}%
&\sphinxstartmulticolumn{2}%
\begin{varwidth}[t]{\sphinxcolwidth{2}{3}}
\sphinxAtStartPar
Name of D3 color scale to be used for color mapping. See {\hyperref[\detokenize{docs/advanced/tmap:colorscale}]{\sphinxcrossref{\DUrole{std,std-ref}{ColorScale}}}} for valid string values.
\par
\vskip-\baselineskip\vbox{\hbox{\strut}}\end{varwidth}%
\sphinxstopmulticolumn
\\
\cline{2-3}\sphinxtablestrut{25}&
\sphinxAtStartPar
type
&
\sphinxAtStartPar
\sphinxstyleemphasis{string}
\\
\cline{2-3}\sphinxtablestrut{25}&
\sphinxAtStartPar
default
&\\
\hline\sphinxmultirow{3}{31}{%
\begin{varwidth}[t]{\sphinxcolwidth{1}{3}}
\begin{itemize}
\item {} 
\sphinxAtStartPar
cb\_col

\end{itemize}
\par
\vskip-\baselineskip\vbox{\hbox{\strut}}\end{varwidth}%
}%
&\sphinxstartmulticolumn{2}%
\begin{varwidth}[t]{\sphinxcolwidth{2}{3}}
\sphinxAtStartPar
Name of CSV column containing scalar values for color mapping or hexadecimal RGB colors in format ‘\#ff0000’
\par
\vskip-\baselineskip\vbox{\hbox{\strut}}\end{varwidth}%
\sphinxstopmulticolumn
\\
\cline{2-3}\sphinxtablestrut{31}&
\sphinxAtStartPar
type
&
\sphinxAtStartPar
\sphinxstyleemphasis{string}
\\
\cline{2-3}\sphinxtablestrut{31}&
\sphinxAtStartPar
default
&
\sphinxAtStartPar
null
\\
\hline\sphinxmultirow{3}{37}{%
\begin{varwidth}[t]{\sphinxcolwidth{1}{3}}
\begin{itemize}
\item {} 
\sphinxAtStartPar
cb\_gr\_dict

\end{itemize}
\par
\vskip-\baselineskip\vbox{\hbox{\strut}}\end{varwidth}%
}%
&\sphinxstartmulticolumn{2}%
\begin{varwidth}[t]{\sphinxcolwidth{2}{3}}
\sphinxAtStartPar
JSON string specifying a custom dictionary for mapping group keys to group colors. Example: \sphinxcode{\sphinxupquote{\textquotesingle{}\{"key1": "\#ff0000", "key2": "\#00ff00", "key3": "\#0000ff"\}\textquotesingle{}}}
\par
\vskip-\baselineskip\vbox{\hbox{\strut}}\end{varwidth}%
\sphinxstopmulticolumn
\\
\cline{2-3}\sphinxtablestrut{37}&
\sphinxAtStartPar
type
&
\sphinxAtStartPar
\sphinxstyleemphasis{string}
\\
\cline{2-3}\sphinxtablestrut{37}&
\sphinxAtStartPar
default
&\\
\hline\sphinxmultirow{3}{43}{%
\begin{varwidth}[t]{\sphinxcolwidth{1}{3}}
\begin{itemize}
\item {} 
\sphinxAtStartPar
scale\_col

\end{itemize}
\par
\vskip-\baselineskip\vbox{\hbox{\strut}}\end{varwidth}%
}%
&\sphinxstartmulticolumn{2}%
\begin{varwidth}[t]{\sphinxcolwidth{2}{3}}
\sphinxAtStartPar
Name of CSV column containing scalar values for changing the size of markers
\par
\vskip-\baselineskip\vbox{\hbox{\strut}}\end{varwidth}%
\sphinxstopmulticolumn
\\
\cline{2-3}\sphinxtablestrut{43}&
\sphinxAtStartPar
type
&
\sphinxAtStartPar
\sphinxstyleemphasis{string}
\\
\cline{2-3}\sphinxtablestrut{43}&
\sphinxAtStartPar
default
&
\sphinxAtStartPar
null
\\
\hline\sphinxmultirow{3}{49}{%
\begin{varwidth}[t]{\sphinxcolwidth{1}{3}}
\begin{itemize}
\item {} 
\sphinxAtStartPar
scale\_factor

\end{itemize}
\par
\vskip-\baselineskip\vbox{\hbox{\strut}}\end{varwidth}%
}%
&\sphinxstartmulticolumn{2}%
\begin{varwidth}[t]{\sphinxcolwidth{2}{3}}
\sphinxAtStartPar
Numerical value for a fixed scale factor to be applied to markers
\par
\vskip-\baselineskip\vbox{\hbox{\strut}}\end{varwidth}%
\sphinxstopmulticolumn
\\
\cline{2-3}\sphinxtablestrut{49}&
\sphinxAtStartPar
type
&
\sphinxAtStartPar
\sphinxstyleemphasis{string}
\\
\cline{2-3}\sphinxtablestrut{49}&
\sphinxAtStartPar
default
&
\sphinxAtStartPar
1
\\
\hline\sphinxmultirow{3}{55}{%
\begin{varwidth}[t]{\sphinxcolwidth{1}{3}}
\begin{itemize}
\item {} 
\sphinxAtStartPar
pie\_col

\end{itemize}
\par
\vskip-\baselineskip\vbox{\hbox{\strut}}\end{varwidth}%
}%
&\sphinxstartmulticolumn{2}%
\begin{varwidth}[t]{\sphinxcolwidth{2}{3}}
\sphinxAtStartPar
Name of CSV column containing data for pie chart sectors. TissUUmaps expects labels and numerical values for sectors to be separated by ‘:’ characters in the CSV column data.
\par
\vskip-\baselineskip\vbox{\hbox{\strut}}\end{varwidth}%
\sphinxstopmulticolumn
\\
\cline{2-3}\sphinxtablestrut{55}&
\sphinxAtStartPar
type
&
\sphinxAtStartPar
\sphinxstyleemphasis{string}
\\
\cline{2-3}\sphinxtablestrut{55}&
\sphinxAtStartPar
default
&
\sphinxAtStartPar
null
\\
\hline\sphinxmultirow{3}{61}{%
\begin{varwidth}[t]{\sphinxcolwidth{1}{3}}
\begin{itemize}
\item {} 
\sphinxAtStartPar
pie\_dict

\end{itemize}
\par
\vskip-\baselineskip\vbox{\hbox{\strut}}\end{varwidth}%
}%
&\sphinxstartmulticolumn{2}%
\begin{varwidth}[t]{\sphinxcolwidth{2}{3}}
\sphinxAtStartPar
JSON string specifying a custom dictionary for mapping pie chart sector indices to colors. Example: \sphinxcode{\sphinxupquote{\textquotesingle{}\{0: "\#ff0000", 1: "\#00ff00", 2: "\#0000ff"\}\textquotesingle{}}}. If no dictionary is specified, TissUUmaps will use a default color palette instead.
\par
\vskip-\baselineskip\vbox{\hbox{\strut}}\end{varwidth}%
\sphinxstopmulticolumn
\\
\cline{2-3}\sphinxtablestrut{61}&
\sphinxAtStartPar
type
&
\sphinxAtStartPar
\sphinxstyleemphasis{string}
\\
\cline{2-3}\sphinxtablestrut{61}&
\sphinxAtStartPar
default
&\\
\hline\sphinxmultirow{3}{67}{%
\begin{varwidth}[t]{\sphinxcolwidth{1}{3}}
\begin{itemize}
\item {} 
\sphinxAtStartPar
shape\_col

\end{itemize}
\par
\vskip-\baselineskip\vbox{\hbox{\strut}}\end{varwidth}%
}%
&\sphinxstartmulticolumn{2}%
\begin{varwidth}[t]{\sphinxcolwidth{2}{3}}
\sphinxAtStartPar
Name of CSV column containing a name or an index for marker shape. See also {\hyperref[\detokenize{docs/advanced/tmap:shape}]{\sphinxcrossref{\DUrole{std,std-ref}{Shape}}}}.
\par
\vskip-\baselineskip\vbox{\hbox{\strut}}\end{varwidth}%
\sphinxstopmulticolumn
\\
\cline{2-3}\sphinxtablestrut{67}&
\sphinxAtStartPar
type
&
\sphinxAtStartPar
\sphinxstyleemphasis{string}
\\
\cline{2-3}\sphinxtablestrut{67}&
\sphinxAtStartPar
default
&
\sphinxAtStartPar
null
\\
\hline\sphinxmultirow{3}{73}{%
\begin{varwidth}[t]{\sphinxcolwidth{1}{3}}
\begin{itemize}
\item {} 
\sphinxAtStartPar
shape\_fixed

\end{itemize}
\par
\vskip-\baselineskip\vbox{\hbox{\strut}}\end{varwidth}%
}%
&\sphinxstartmulticolumn{2}%
\begin{varwidth}[t]{\sphinxcolwidth{2}{3}}
\sphinxAtStartPar
Name or index of a single fixed shape to be used for all markers. See {\hyperref[\detokenize{docs/advanced/tmap:shape}]{\sphinxcrossref{\DUrole{std,std-ref}{Shape}}}} for valid string values.
\par
\vskip-\baselineskip\vbox{\hbox{\strut}}\end{varwidth}%
\sphinxstopmulticolumn
\\
\cline{2-3}\sphinxtablestrut{73}&
\sphinxAtStartPar
type
&
\sphinxAtStartPar
\sphinxstyleemphasis{string}
\\
\cline{2-3}\sphinxtablestrut{73}&
\sphinxAtStartPar
default
&
\sphinxAtStartPar
cross
\\
\hline\sphinxmultirow{3}{79}{%
\begin{varwidth}[t]{\sphinxcolwidth{1}{3}}
\begin{itemize}
\item {} 
\sphinxAtStartPar
shape\_gr\_dict

\end{itemize}
\par
\vskip-\baselineskip\vbox{\hbox{\strut}}\end{varwidth}%
}%
&\sphinxstartmulticolumn{2}%
\begin{varwidth}[t]{\sphinxcolwidth{2}{3}}
\sphinxAtStartPar
JSON string specifying a custom dictionary for mapping group keys to group shapes. Example: \sphinxcode{\sphinxupquote{\textquotesingle{}\{"key1": "square", "key2": "diamond", "key3": "triangle up"\}\textquotesingle{}}}. See also {\hyperref[\detokenize{docs/advanced/tmap:shape}]{\sphinxcrossref{\DUrole{std,std-ref}{Shape}}}}.
\par
\vskip-\baselineskip\vbox{\hbox{\strut}}\end{varwidth}%
\sphinxstopmulticolumn
\\
\cline{2-3}\sphinxtablestrut{79}&
\sphinxAtStartPar
type
&
\sphinxAtStartPar
\sphinxstyleemphasis{string}
\\
\cline{2-3}\sphinxtablestrut{79}&
\sphinxAtStartPar
default
&\\
\hline\sphinxmultirow{3}{85}{%
\begin{varwidth}[t]{\sphinxcolwidth{1}{3}}
\begin{itemize}
\item {} 
\sphinxAtStartPar
opacity\_col

\end{itemize}
\par
\vskip-\baselineskip\vbox{\hbox{\strut}}\end{varwidth}%
}%
&\sphinxstartmulticolumn{2}%
\begin{varwidth}[t]{\sphinxcolwidth{2}{3}}
\sphinxAtStartPar
Name of CSV column containing scalar values for opacities
\par
\vskip-\baselineskip\vbox{\hbox{\strut}}\end{varwidth}%
\sphinxstopmulticolumn
\\
\cline{2-3}\sphinxtablestrut{85}&
\sphinxAtStartPar
type
&
\sphinxAtStartPar
\sphinxstyleemphasis{string}
\\
\cline{2-3}\sphinxtablestrut{85}&
\sphinxAtStartPar
default
&
\sphinxAtStartPar
null
\\
\hline\sphinxmultirow{3}{91}{%
\begin{varwidth}[t]{\sphinxcolwidth{1}{3}}
\begin{itemize}
\item {} 
\sphinxAtStartPar
opacity

\end{itemize}
\par
\vskip-\baselineskip\vbox{\hbox{\strut}}\end{varwidth}%
}%
&\sphinxstartmulticolumn{2}%
\begin{varwidth}[t]{\sphinxcolwidth{2}{3}}
\sphinxAtStartPar
Numerical value for a fixed opacity factor to be applied to markers
\par
\vskip-\baselineskip\vbox{\hbox{\strut}}\end{varwidth}%
\sphinxstopmulticolumn
\\
\cline{2-3}\sphinxtablestrut{91}&
\sphinxAtStartPar
type
&
\sphinxAtStartPar
\sphinxstyleemphasis{string}
\\
\cline{2-3}\sphinxtablestrut{91}&
\sphinxAtStartPar
default
&
\sphinxAtStartPar
1
\\
\hline\sphinxmultirow{3}{97}{%
\begin{varwidth}[t]{\sphinxcolwidth{1}{3}}
\begin{itemize}
\item {} 
\sphinxAtStartPar
tooltip\_fmt

\end{itemize}
\par
\vskip-\baselineskip\vbox{\hbox{\strut}}\end{varwidth}%
}%
&\sphinxstartmulticolumn{2}%
\begin{varwidth}[t]{\sphinxcolwidth{2}{3}}
\sphinxAtStartPar
Custom formatting string used for displaying metadata about a selected marker. See \sphinxurl{https://github.com/TissUUmaps/TissUUmaps/issues/2} for an overview of the grammer and keywords. If no string is specified, TissUUmaps will show default metadata depending on the context.
\par
\vskip-\baselineskip\vbox{\hbox{\strut}}\end{varwidth}%
\sphinxstopmulticolumn
\\
\cline{2-3}\sphinxtablestrut{97}&
\sphinxAtStartPar
type
&
\sphinxAtStartPar
\sphinxstyleemphasis{string}
\\
\cline{2-3}\sphinxtablestrut{97}&
\sphinxAtStartPar
default
&\\
\hline
\end{longtable}\sphinxatlongtableend\end{savenotes}


\subsubsection{ExpectedRadios}
\label{\detokenize{docs/advanced/tmap:expectedradios}}\label{\detokenize{docs/advanced/tmap:expectedradios}}

\begin{savenotes}\sphinxatlongtablestart\begin{longtable}[c]{|*{3}{\X{1}{3}|}}
\hline

\endfirsthead

\multicolumn{3}{c}%
{\makebox[0pt]{\sphinxtablecontinued{\tablename\ \thetable{} \textendash{} continued from previous page}}}\\
\hline

\endhead

\hline
\multicolumn{3}{r}{\makebox[0pt][r]{\sphinxtablecontinued{continues on next page}}}\\
\endfoot

\endlastfoot
\sphinxstartmulticolumn{3}%
\begin{varwidth}[t]{\sphinxcolwidth{3}{3}}
\sphinxAtStartPar
Radio button state and checkbox state for settings in a marker tab. Required properties are shown in \sphinxstylestrong{bold} text.
\par
\vskip-\baselineskip\vbox{\hbox{\strut}}\end{varwidth}%
\sphinxstopmulticolumn
\\
\hline
\sphinxAtStartPar
type
&\sphinxstartmulticolumn{2}%
\begin{varwidth}[t]{\sphinxcolwidth{2}{3}}
\sphinxAtStartPar
\sphinxstyleemphasis{object}
\par
\vskip-\baselineskip\vbox{\hbox{\strut}}\end{varwidth}%
\sphinxstopmulticolumn
\\
\hline\sphinxstartmulticolumn{3}%
\begin{varwidth}[t]{\sphinxcolwidth{3}{3}}
\sphinxAtStartPar
properties
\par
\vskip-\baselineskip\vbox{\hbox{\strut}}\end{varwidth}%
\sphinxstopmulticolumn
\\
\hline\sphinxmultirow{3}{5}{%
\begin{varwidth}[t]{\sphinxcolwidth{1}{3}}
\begin{itemize}
\item {} 
\sphinxAtStartPar
cb\_col

\end{itemize}
\par
\vskip-\baselineskip\vbox{\hbox{\strut}}\end{varwidth}%
}%
&\sphinxstartmulticolumn{2}%
\begin{varwidth}[t]{\sphinxcolwidth{2}{3}}
\sphinxAtStartPar
If markers should be colored by data in CSV column
\par
\vskip-\baselineskip\vbox{\hbox{\strut}}\end{varwidth}%
\sphinxstopmulticolumn
\\
\cline{2-3}\sphinxtablestrut{5}&
\sphinxAtStartPar
type
&
\sphinxAtStartPar
\sphinxstyleemphasis{boolean}
\\
\cline{2-3}\sphinxtablestrut{5}&
\sphinxAtStartPar
default
&
\sphinxAtStartPar
false
\\
\hline\sphinxmultirow{3}{11}{%
\begin{varwidth}[t]{\sphinxcolwidth{1}{3}}
\begin{itemize}
\item {} 
\sphinxAtStartPar
cb\_gr

\end{itemize}
\par
\vskip-\baselineskip\vbox{\hbox{\strut}}\end{varwidth}%
}%
&\sphinxstartmulticolumn{2}%
\begin{varwidth}[t]{\sphinxcolwidth{2}{3}}
\sphinxAtStartPar
If markers should be colored by group
\par
\vskip-\baselineskip\vbox{\hbox{\strut}}\end{varwidth}%
\sphinxstopmulticolumn
\\
\cline{2-3}\sphinxtablestrut{11}&
\sphinxAtStartPar
type
&
\sphinxAtStartPar
\sphinxstyleemphasis{boolean}
\\
\cline{2-3}\sphinxtablestrut{11}&
\sphinxAtStartPar
default
&
\sphinxAtStartPar
true
\\
\hline\sphinxmultirow{3}{17}{%
\begin{varwidth}[t]{\sphinxcolwidth{1}{3}}
\begin{itemize}
\item {} 
\sphinxAtStartPar
cb\_gr\_rand

\end{itemize}
\par
\vskip-\baselineskip\vbox{\hbox{\strut}}\end{varwidth}%
}%
&\sphinxstartmulticolumn{2}%
\begin{varwidth}[t]{\sphinxcolwidth{2}{3}}
\sphinxAtStartPar
If group color should be generated randomly
\par
\vskip-\baselineskip\vbox{\hbox{\strut}}\end{varwidth}%
\sphinxstopmulticolumn
\\
\cline{2-3}\sphinxtablestrut{17}&
\sphinxAtStartPar
type
&
\sphinxAtStartPar
\sphinxstyleemphasis{boolean}
\\
\cline{2-3}\sphinxtablestrut{17}&
\sphinxAtStartPar
default
&
\sphinxAtStartPar
false
\\
\hline\sphinxmultirow{3}{23}{%
\begin{varwidth}[t]{\sphinxcolwidth{1}{3}}
\begin{itemize}
\item {} 
\sphinxAtStartPar
cb\_gr\_dict

\end{itemize}
\par
\vskip-\baselineskip\vbox{\hbox{\strut}}\end{varwidth}%
}%
&\sphinxstartmulticolumn{2}%
\begin{varwidth}[t]{\sphinxcolwidth{2}{3}}
\sphinxAtStartPar
If group color should be read from custom dictionary
\par
\vskip-\baselineskip\vbox{\hbox{\strut}}\end{varwidth}%
\sphinxstopmulticolumn
\\
\cline{2-3}\sphinxtablestrut{23}&
\sphinxAtStartPar
type
&
\sphinxAtStartPar
\sphinxstyleemphasis{boolean}
\\
\cline{2-3}\sphinxtablestrut{23}&
\sphinxAtStartPar
default
&
\sphinxAtStartPar
false
\\
\hline\sphinxmultirow{3}{29}{%
\begin{varwidth}[t]{\sphinxcolwidth{1}{3}}
\begin{itemize}
\item {} 
\sphinxAtStartPar
cb\_gr\_key

\end{itemize}
\par
\vskip-\baselineskip\vbox{\hbox{\strut}}\end{varwidth}%
}%
&\sphinxstartmulticolumn{2}%
\begin{varwidth}[t]{\sphinxcolwidth{2}{3}}
\sphinxAtStartPar
If group color should be generated from group key
\par
\vskip-\baselineskip\vbox{\hbox{\strut}}\end{varwidth}%
\sphinxstopmulticolumn
\\
\cline{2-3}\sphinxtablestrut{29}&
\sphinxAtStartPar
type
&
\sphinxAtStartPar
\sphinxstyleemphasis{boolean}
\\
\cline{2-3}\sphinxtablestrut{29}&
\sphinxAtStartPar
default
&
\sphinxAtStartPar
true
\\
\hline\sphinxmultirow{3}{35}{%
\begin{varwidth}[t]{\sphinxcolwidth{1}{3}}
\begin{itemize}
\item {} 
\sphinxAtStartPar
pie\_check

\end{itemize}
\par
\vskip-\baselineskip\vbox{\hbox{\strut}}\end{varwidth}%
}%
&\sphinxstartmulticolumn{2}%
\begin{varwidth}[t]{\sphinxcolwidth{2}{3}}
\sphinxAtStartPar
If markers should be rendered as pie charts
\par
\vskip-\baselineskip\vbox{\hbox{\strut}}\end{varwidth}%
\sphinxstopmulticolumn
\\
\cline{2-3}\sphinxtablestrut{35}&
\sphinxAtStartPar
type
&
\sphinxAtStartPar
\sphinxstyleemphasis{boolean}
\\
\cline{2-3}\sphinxtablestrut{35}&
\sphinxAtStartPar
default
&
\sphinxAtStartPar
false
\\
\hline\sphinxmultirow{3}{41}{%
\begin{varwidth}[t]{\sphinxcolwidth{1}{3}}
\begin{itemize}
\item {} 
\sphinxAtStartPar
scale\_check

\end{itemize}
\par
\vskip-\baselineskip\vbox{\hbox{\strut}}\end{varwidth}%
}%
&\sphinxstartmulticolumn{2}%
\begin{varwidth}[t]{\sphinxcolwidth{2}{3}}
\sphinxAtStartPar
If markers should be scaled by data in CSV column
\par
\vskip-\baselineskip\vbox{\hbox{\strut}}\end{varwidth}%
\sphinxstopmulticolumn
\\
\cline{2-3}\sphinxtablestrut{41}&
\sphinxAtStartPar
type
&
\sphinxAtStartPar
\sphinxstyleemphasis{boolean}
\\
\cline{2-3}\sphinxtablestrut{41}&
\sphinxAtStartPar
default
&
\sphinxAtStartPar
false
\\
\hline\sphinxmultirow{3}{47}{%
\begin{varwidth}[t]{\sphinxcolwidth{1}{3}}
\begin{itemize}
\item {} 
\sphinxAtStartPar
shape\_col

\end{itemize}
\par
\vskip-\baselineskip\vbox{\hbox{\strut}}\end{varwidth}%
}%
&\sphinxstartmulticolumn{2}%
\begin{varwidth}[t]{\sphinxcolwidth{2}{3}}
\sphinxAtStartPar
If markers should get their shape from data in CSV column
\par
\vskip-\baselineskip\vbox{\hbox{\strut}}\end{varwidth}%
\sphinxstopmulticolumn
\\
\cline{2-3}\sphinxtablestrut{47}&
\sphinxAtStartPar
type
&
\sphinxAtStartPar
\sphinxstyleemphasis{boolean}
\\
\cline{2-3}\sphinxtablestrut{47}&
\sphinxAtStartPar
default
&
\sphinxAtStartPar
false
\\
\hline\sphinxmultirow{3}{53}{%
\begin{varwidth}[t]{\sphinxcolwidth{1}{3}}
\begin{itemize}
\item {} 
\sphinxAtStartPar
shape\_gr

\end{itemize}
\par
\vskip-\baselineskip\vbox{\hbox{\strut}}\end{varwidth}%
}%
&\sphinxstartmulticolumn{2}%
\begin{varwidth}[t]{\sphinxcolwidth{2}{3}}
\sphinxAtStartPar
If markers should get their shape from group
\par
\vskip-\baselineskip\vbox{\hbox{\strut}}\end{varwidth}%
\sphinxstopmulticolumn
\\
\cline{2-3}\sphinxtablestrut{53}&
\sphinxAtStartPar
type
&
\sphinxAtStartPar
\sphinxstyleemphasis{boolean}
\\
\cline{2-3}\sphinxtablestrut{53}&
\sphinxAtStartPar
default
&
\sphinxAtStartPar
true
\\
\hline\sphinxmultirow{3}{59}{%
\begin{varwidth}[t]{\sphinxcolwidth{1}{3}}
\begin{itemize}
\item {} 
\sphinxAtStartPar
shape\_gr\_rand

\end{itemize}
\par
\vskip-\baselineskip\vbox{\hbox{\strut}}\end{varwidth}%
}%
&\sphinxstartmulticolumn{2}%
\begin{varwidth}[t]{\sphinxcolwidth{2}{3}}
\sphinxAtStartPar
If group shape should be generated randomly
\par
\vskip-\baselineskip\vbox{\hbox{\strut}}\end{varwidth}%
\sphinxstopmulticolumn
\\
\cline{2-3}\sphinxtablestrut{59}&
\sphinxAtStartPar
type
&
\sphinxAtStartPar
\sphinxstyleemphasis{boolean}
\\
\cline{2-3}\sphinxtablestrut{59}&
\sphinxAtStartPar
default
&
\sphinxAtStartPar
true
\\
\hline\sphinxmultirow{3}{65}{%
\begin{varwidth}[t]{\sphinxcolwidth{1}{3}}
\begin{itemize}
\item {} 
\sphinxAtStartPar
shape\_gr\_dict

\end{itemize}
\par
\vskip-\baselineskip\vbox{\hbox{\strut}}\end{varwidth}%
}%
&\sphinxstartmulticolumn{2}%
\begin{varwidth}[t]{\sphinxcolwidth{2}{3}}
\sphinxAtStartPar
If group shape should be read from custom dictionary
\par
\vskip-\baselineskip\vbox{\hbox{\strut}}\end{varwidth}%
\sphinxstopmulticolumn
\\
\cline{2-3}\sphinxtablestrut{65}&
\sphinxAtStartPar
type
&
\sphinxAtStartPar
\sphinxstyleemphasis{boolean}
\\
\cline{2-3}\sphinxtablestrut{65}&
\sphinxAtStartPar
default
&
\sphinxAtStartPar
false
\\
\hline\sphinxmultirow{3}{71}{%
\begin{varwidth}[t]{\sphinxcolwidth{1}{3}}
\begin{itemize}
\item {} 
\sphinxAtStartPar
shape\_fixed

\end{itemize}
\par
\vskip-\baselineskip\vbox{\hbox{\strut}}\end{varwidth}%
}%
&\sphinxstartmulticolumn{2}%
\begin{varwidth}[t]{\sphinxcolwidth{2}{3}}
\sphinxAtStartPar
If a single fixed shape should be used for all markers
\par
\vskip-\baselineskip\vbox{\hbox{\strut}}\end{varwidth}%
\sphinxstopmulticolumn
\\
\cline{2-3}\sphinxtablestrut{71}&
\sphinxAtStartPar
type
&
\sphinxAtStartPar
\sphinxstyleemphasis{boolean}
\\
\cline{2-3}\sphinxtablestrut{71}&
\sphinxAtStartPar
default
&
\sphinxAtStartPar
false
\\
\hline\sphinxmultirow{3}{77}{%
\begin{varwidth}[t]{\sphinxcolwidth{1}{3}}
\begin{itemize}
\item {} 
\sphinxAtStartPar
opacity\_check

\end{itemize}
\par
\vskip-\baselineskip\vbox{\hbox{\strut}}\end{varwidth}%
}%
&\sphinxstartmulticolumn{2}%
\begin{varwidth}[t]{\sphinxcolwidth{2}{3}}
\sphinxAtStartPar
If markers should get their opacities from data in CSV column
\par
\vskip-\baselineskip\vbox{\hbox{\strut}}\end{varwidth}%
\sphinxstopmulticolumn
\\
\cline{2-3}\sphinxtablestrut{77}&
\sphinxAtStartPar
type
&
\sphinxAtStartPar
\sphinxstyleemphasis{boolean}
\\
\cline{2-3}\sphinxtablestrut{77}&
\sphinxAtStartPar
default
&
\sphinxAtStartPar
false
\\
\hline\sphinxmultirow{3}{83}{%
\begin{varwidth}[t]{\sphinxcolwidth{1}{3}}
\begin{itemize}
\item {} 
\sphinxAtStartPar
\_no\_outline

\end{itemize}
\par
\vskip-\baselineskip\vbox{\hbox{\strut}}\end{varwidth}%
}%
&\sphinxstartmulticolumn{2}%
\begin{varwidth}[t]{\sphinxcolwidth{2}{3}}
\sphinxAtStartPar
If marker shapes should be rendered without outline
\par
\vskip-\baselineskip\vbox{\hbox{\strut}}\end{varwidth}%
\sphinxstopmulticolumn
\\
\cline{2-3}\sphinxtablestrut{83}&
\sphinxAtStartPar
type
&
\sphinxAtStartPar
\sphinxstyleemphasis{boolean}
\\
\cline{2-3}\sphinxtablestrut{83}&
\sphinxAtStartPar
default
&
\sphinxAtStartPar
false
\\
\hline
\end{longtable}\sphinxatlongtableend\end{savenotes}


\subsubsection{RegionFile}
\label{\detokenize{docs/advanced/tmap:regionfile}}\label{\detokenize{docs/advanced/tmap:regionfile}}

\begin{savenotes}\sphinxattablestart
\centering
\begin{tabular}[t]{|*{3}{\X{1}{3}|}}
\hline
\sphinxstartmulticolumn{3}%
\begin{varwidth}[t]{\sphinxcolwidth{3}{3}}
\sphinxAtStartPar
Description of settings and regions loaded from GeoJSON file. Required properties are shown in \sphinxstylestrong{bold} text.
\par
\vskip-\baselineskip\vbox{\hbox{\strut}}\end{varwidth}%
\sphinxstopmulticolumn
\\
\hline
\sphinxAtStartPar
type
&\sphinxstartmulticolumn{2}%
\begin{varwidth}[t]{\sphinxcolwidth{2}{3}}
\sphinxAtStartPar
\sphinxstyleemphasis{object}
\par
\vskip-\baselineskip\vbox{\hbox{\strut}}\end{varwidth}%
\sphinxstopmulticolumn
\\
\hline\sphinxstartmulticolumn{3}%
\begin{varwidth}[t]{\sphinxcolwidth{3}{3}}
\sphinxAtStartPar
properties
\par
\vskip-\baselineskip\vbox{\hbox{\strut}}\end{varwidth}%
\sphinxstopmulticolumn
\\
\hline\sphinxmultirow{2}{5}{%
\begin{varwidth}[t]{\sphinxcolwidth{1}{3}}
\begin{itemize}
\item {} 
\sphinxAtStartPar
\sphinxstylestrong{title}

\end{itemize}
\par
\vskip-\baselineskip\vbox{\hbox{\strut}}\end{varwidth}%
}%
&\sphinxstartmulticolumn{2}%
\begin{varwidth}[t]{\sphinxcolwidth{2}{3}}
\sphinxAtStartPar
Name of region button
\par
\vskip-\baselineskip\vbox{\hbox{\strut}}\end{varwidth}%
\sphinxstopmulticolumn
\\
\cline{2-3}\sphinxtablestrut{5}&
\sphinxAtStartPar
type
&
\sphinxAtStartPar
\sphinxstyleemphasis{string}
\\
\hline\sphinxmultirow{3}{9}{%
\begin{varwidth}[t]{\sphinxcolwidth{1}{3}}
\begin{itemize}
\item {} 
\sphinxAtStartPar
comment

\end{itemize}
\par
\vskip-\baselineskip\vbox{\hbox{\strut}}\end{varwidth}%
}%
&\sphinxstartmulticolumn{2}%
\begin{varwidth}[t]{\sphinxcolwidth{2}{3}}
\sphinxAtStartPar
Optional description text shown next to region button
\par
\vskip-\baselineskip\vbox{\hbox{\strut}}\end{varwidth}%
\sphinxstopmulticolumn
\\
\cline{2-3}\sphinxtablestrut{9}&
\sphinxAtStartPar
type
&
\sphinxAtStartPar
\sphinxstyleemphasis{string}
\\
\cline{2-3}\sphinxtablestrut{9}&
\sphinxAtStartPar
default
&\\
\hline\sphinxmultirow{3}{15}{%
\begin{varwidth}[t]{\sphinxcolwidth{1}{3}}
\begin{itemize}
\item {} 
\sphinxAtStartPar
autoLoad

\end{itemize}
\par
\vskip-\baselineskip\vbox{\hbox{\strut}}\end{varwidth}%
}%
&\sphinxstartmulticolumn{2}%
\begin{varwidth}[t]{\sphinxcolwidth{2}{3}}
\sphinxAtStartPar
If the regions should be automatically loaded when the TMAP project is opened. If this is false, the user instead has to click on the region button in the GUI to load the regions.
\par
\vskip-\baselineskip\vbox{\hbox{\strut}}\end{varwidth}%
\sphinxstopmulticolumn
\\
\cline{2-3}\sphinxtablestrut{15}&
\sphinxAtStartPar
type
&
\sphinxAtStartPar
\sphinxstyleemphasis{boolean}
\\
\cline{2-3}\sphinxtablestrut{15}&
\sphinxAtStartPar
default
&
\sphinxAtStartPar
false
\\
\hline\sphinxmultirow{2}{21}{%
\begin{varwidth}[t]{\sphinxcolwidth{1}{3}}
\begin{itemize}
\item {} 
\sphinxAtStartPar
\sphinxstylestrong{path}

\end{itemize}
\par
\vskip-\baselineskip\vbox{\hbox{\strut}}\end{varwidth}%
}%
&\sphinxstartmulticolumn{2}%
\begin{varwidth}[t]{\sphinxcolwidth{2}{3}}
\sphinxAtStartPar
Relative file path to GeoJSON file in which marker data is stored. If array of string, then a dropdown is created instead of a button.
\par
\vskip-\baselineskip\vbox{\hbox{\strut}}\end{varwidth}%
\sphinxstopmulticolumn
\\
\cline{2-3}\sphinxtablestrut{21}&
\sphinxAtStartPar
type
&
\sphinxAtStartPar
\sphinxstyleemphasis{string} / \sphinxstyleemphasis{array}
\\
\hline\sphinxmultirow{4}{25}{%
\begin{varwidth}[t]{\sphinxcolwidth{1}{3}}
\begin{itemize}
\item {} 
\sphinxAtStartPar
settings

\end{itemize}
\par
\vskip-\baselineskip\vbox{\hbox{\strut}}\end{varwidth}%
}%
&
\sphinxAtStartPar
type
&
\sphinxAtStartPar
\sphinxstyleemphasis{array}
\\
\cline{2-3}\sphinxtablestrut{25}&
\sphinxAtStartPar
default
&
\sphinxAtStartPar
{[}{]}
\\
\cline{2-3}\sphinxtablestrut{25}&\sphinxstartmulticolumn{2}%
\begin{varwidth}[t]{\sphinxcolwidth{2}{3}}
\sphinxAtStartPar
items
\par
\vskip-\baselineskip\vbox{\hbox{\strut}}\end{varwidth}%
\sphinxstopmulticolumn
\\
\cline{2-3}\sphinxtablestrut{25}&\begin{itemize}
\item {} 
\end{itemize}
&
\sphinxAtStartPar
{\hyperref[\detokenize{docs/advanced/tmap:setting}]{\sphinxcrossref{Setting}}}
\\
\hline
\end{tabular}
\par
\sphinxattableend\end{savenotes}


\subsubsection{Setting}
\label{\detokenize{docs/advanced/tmap:setting}}\label{\detokenize{docs/advanced/tmap:setting}}

\begin{savenotes}\sphinxattablestart
\centering
\begin{tabular}[t]{|*{3}{\X{1}{3}|}}
\hline
\sphinxstartmulticolumn{3}%
\begin{varwidth}[t]{\sphinxcolwidth{3}{3}}
\sphinxAtStartPar
Advanced javascript setting, used to change the value of a property in a module, or to call a function in a module. If module.function is a function, then calls \sphinxcode{\sphinxupquote{module.function(value)}}, else affects \sphinxcode{\sphinxupquote{value}} to \sphinxcode{\sphinxupquote{module.function}}. Required properties are shown in \sphinxstylestrong{bold} text.
\par
\vskip-\baselineskip\vbox{\hbox{\strut}}\end{varwidth}%
\sphinxstopmulticolumn
\\
\hline
\sphinxAtStartPar
type
&\sphinxstartmulticolumn{2}%
\begin{varwidth}[t]{\sphinxcolwidth{2}{3}}
\sphinxAtStartPar
\sphinxstyleemphasis{object}
\par
\vskip-\baselineskip\vbox{\hbox{\strut}}\end{varwidth}%
\sphinxstopmulticolumn
\\
\hline\sphinxstartmulticolumn{3}%
\begin{varwidth}[t]{\sphinxcolwidth{3}{3}}
\sphinxAtStartPar
properties
\par
\vskip-\baselineskip\vbox{\hbox{\strut}}\end{varwidth}%
\sphinxstopmulticolumn
\\
\hline\sphinxmultirow{2}{5}{%
\begin{varwidth}[t]{\sphinxcolwidth{1}{3}}
\begin{itemize}
\item {} 
\sphinxAtStartPar
\sphinxstylestrong{module}

\end{itemize}
\par
\vskip-\baselineskip\vbox{\hbox{\strut}}\end{varwidth}%
}%
&\sphinxstartmulticolumn{2}%
\begin{varwidth}[t]{\sphinxcolwidth{2}{3}}
\sphinxAtStartPar
Module where the function or property lies.
\par
\vskip-\baselineskip\vbox{\hbox{\strut}}\end{varwidth}%
\sphinxstopmulticolumn
\\
\cline{2-3}\sphinxtablestrut{5}&
\sphinxAtStartPar
type
&
\sphinxAtStartPar
\sphinxstyleemphasis{string}
\\
\hline\sphinxmultirow{2}{9}{%
\begin{varwidth}[t]{\sphinxcolwidth{1}{3}}
\begin{itemize}
\item {} 
\sphinxAtStartPar
\sphinxstylestrong{function}

\end{itemize}
\par
\vskip-\baselineskip\vbox{\hbox{\strut}}\end{varwidth}%
}%
&\sphinxstartmulticolumn{2}%
\begin{varwidth}[t]{\sphinxcolwidth{2}{3}}
\sphinxAtStartPar
Function or property of the given module.
\par
\vskip-\baselineskip\vbox{\hbox{\strut}}\end{varwidth}%
\sphinxstopmulticolumn
\\
\cline{2-3}\sphinxtablestrut{9}&
\sphinxAtStartPar
type
&
\sphinxAtStartPar
\sphinxstyleemphasis{string}
\\
\hline\begin{itemize}
\item {} 
\sphinxAtStartPar
\sphinxstylestrong{value}

\end{itemize}
&
\sphinxAtStartPar
type
&
\sphinxAtStartPar
\sphinxstyleemphasis{number}
\\
\hline
\end{tabular}
\par
\sphinxattableend\end{savenotes}


\subsection{Example of a .tmap file}
\label{\detokenize{docs/advanced/tmap:example-of-a-tmap-file}}
\begin{sphinxVerbatim}[commandchars=\\\{\}]
\PYG{p}{\PYGZob{}}
\PYG{+w}{    }\PYG{n+nt}{\PYGZdq{}filename\PYGZdq{}}\PYG{p}{:}\PYG{+w}{ }\PYG{l+s+s2}{\PYGZdq{}TissUUmaps\PYGZus{}Example.tmap\PYGZdq{}}\PYG{p}{,}
\PYG{+w}{    }\PYG{n+nt}{\PYGZdq{}layers\PYGZdq{}}\PYG{p}{:}\PYG{+w}{ }\PYG{p}{[}
\PYG{+w}{        }\PYG{p}{\PYGZob{}}
\PYG{+w}{            }\PYG{n+nt}{\PYGZdq{}name\PYGZdq{}}\PYG{p}{:}\PYG{+w}{ }\PYG{l+s+s2}{\PYGZdq{}Round1\PYGZus{}A.tif\PYGZdq{}}\PYG{p}{,}
\PYG{+w}{            }\PYG{n+nt}{\PYGZdq{}tileSource\PYGZdq{}}\PYG{p}{:}\PYG{+w}{ }\PYG{l+s+s2}{\PYGZdq{}images/Round1\PYGZus{}A.tif.dzi\PYGZdq{}}
\PYG{+w}{        }\PYG{p}{\PYGZcb{},}
\PYG{+w}{        }\PYG{p}{\PYGZob{}}
\PYG{+w}{            }\PYG{n+nt}{\PYGZdq{}name\PYGZdq{}}\PYG{p}{:}\PYG{+w}{ }\PYG{l+s+s2}{\PYGZdq{}Round1\PYGZus{}C.tif\PYGZdq{}}\PYG{p}{,}
\PYG{+w}{            }\PYG{n+nt}{\PYGZdq{}tileSource\PYGZdq{}}\PYG{p}{:}\PYG{+w}{ }\PYG{l+s+s2}{\PYGZdq{}images/Round1\PYGZus{}C.tif.dzi\PYGZdq{}}
\PYG{+w}{        }\PYG{p}{\PYGZcb{}}
\PYG{+w}{    }\PYG{p}{],}
\PYG{+w}{    }\PYG{n+nt}{\PYGZdq{}layerOpacities\PYGZdq{}}\PYG{p}{:}\PYG{+w}{ }\PYG{p}{\PYGZob{}}
\PYG{+w}{        }\PYG{n+nt}{\PYGZdq{}0\PYGZdq{}}\PYG{p}{:}\PYG{+w}{ }\PYG{l+s+s2}{\PYGZdq{}1\PYGZdq{}}\PYG{p}{,}
\PYG{+w}{        }\PYG{n+nt}{\PYGZdq{}1\PYGZdq{}}\PYG{p}{:}\PYG{+w}{ }\PYG{l+s+s2}{\PYGZdq{}1\PYGZdq{}}
\PYG{+w}{    }\PYG{p}{\PYGZcb{},}
\PYG{+w}{    }\PYG{n+nt}{\PYGZdq{}layerVisibilities\PYGZdq{}}\PYG{p}{:}\PYG{+w}{ }\PYG{p}{\PYGZob{}}
\PYG{+w}{        }\PYG{n+nt}{\PYGZdq{}0\PYGZdq{}}\PYG{p}{:}\PYG{+w}{ }\PYG{k+kc}{true}\PYG{p}{,}
\PYG{+w}{        }\PYG{n+nt}{\PYGZdq{}1\PYGZdq{}}\PYG{p}{:}\PYG{+w}{ }\PYG{k+kc}{false}\PYG{p}{,}
\PYG{+w}{    }\PYG{p}{\PYGZcb{},}
\PYG{+w}{    }\PYG{n+nt}{\PYGZdq{}layerFilters\PYGZdq{}}\PYG{p}{:}\PYG{+w}{ }\PYG{p}{\PYGZob{}}
\PYG{+w}{        }\PYG{n+nt}{\PYGZdq{}0\PYGZdq{}}\PYG{p}{:}\PYG{+w}{ }\PYG{p}{[}
\PYG{+w}{            }\PYG{p}{\PYGZob{}}
\PYG{+w}{                }\PYG{n+nt}{\PYGZdq{}name\PYGZdq{}}\PYG{p}{:}\PYG{+w}{ }\PYG{l+s+s2}{\PYGZdq{}Color\PYGZdq{}}\PYG{p}{,}
\PYG{+w}{                }\PYG{n+nt}{\PYGZdq{}value\PYGZdq{}}\PYG{p}{:}\PYG{+w}{ }\PYG{l+s+s2}{\PYGZdq{}0,100,0\PYGZdq{}}
\PYG{+w}{            }\PYG{p}{\PYGZcb{}}
\PYG{+w}{        }\PYG{p}{],}
\PYG{+w}{        }\PYG{n+nt}{\PYGZdq{}1\PYGZdq{}}\PYG{p}{:}\PYG{+w}{ }\PYG{p}{[}
\PYG{+w}{            }\PYG{p}{\PYGZob{}}
\PYG{+w}{                }\PYG{n+nt}{\PYGZdq{}name\PYGZdq{}}\PYG{p}{:}\PYG{+w}{ }\PYG{l+s+s2}{\PYGZdq{}Color\PYGZdq{}}\PYG{p}{,}
\PYG{+w}{                }\PYG{n+nt}{\PYGZdq{}value\PYGZdq{}}\PYG{p}{:}\PYG{+w}{ }\PYG{l+s+s2}{\PYGZdq{}0,100,0\PYGZdq{}}
\PYG{+w}{            }\PYG{p}{\PYGZcb{}}
\PYG{+w}{        }\PYG{p}{]}
\PYG{+w}{    }\PYG{p}{\PYGZcb{},}
\PYG{+w}{    }\PYG{n+nt}{\PYGZdq{}filters\PYGZdq{}}\PYG{p}{:}\PYG{+w}{ }\PYG{p}{[}
\PYG{+w}{        }\PYG{l+s+s2}{\PYGZdq{}Color\PYGZdq{}}
\PYG{+w}{    }\PYG{p}{],}
\PYG{+w}{    }\PYG{n+nt}{\PYGZdq{}compositeMode\PYGZdq{}}\PYG{p}{:}\PYG{+w}{ }\PYG{l+s+s2}{\PYGZdq{}lighter\PYGZdq{}}\PYG{p}{,}
\PYG{+w}{    }\PYG{n+nt}{\PYGZdq{}markerFiles\PYGZdq{}}\PYG{p}{:}\PYG{+w}{ }\PYG{p}{[}
\PYG{+w}{        }\PYG{p}{\PYGZob{}}
\PYG{+w}{            }\PYG{n+nt}{\PYGZdq{}autoLoad\PYGZdq{}}\PYG{p}{:}\PYG{+w}{ }\PYG{k+kc}{false}\PYG{p}{,}
\PYG{+w}{            }\PYG{n+nt}{\PYGZdq{}comment\PYGZdq{}}\PYG{p}{:}\PYG{+w}{ }\PYG{l+s+s2}{\PYGZdq{}\PYGZdq{}}\PYG{p}{,}
\PYG{+w}{            }\PYG{n+nt}{\PYGZdq{}expectedHeader\PYGZdq{}}\PYG{p}{:}\PYG{+w}{ }\PYG{p}{\PYGZob{}}
\PYG{+w}{                }\PYG{n+nt}{\PYGZdq{}X\PYGZdq{}}\PYG{p}{:}\PYG{+w}{ }\PYG{l+s+s2}{\PYGZdq{}global\PYGZus{}x\PYGZdq{}}\PYG{p}{,}
\PYG{+w}{                }\PYG{n+nt}{\PYGZdq{}Y\PYGZdq{}}\PYG{p}{:}\PYG{+w}{ }\PYG{l+s+s2}{\PYGZdq{}global\PYGZus{}y\PYGZdq{}}\PYG{p}{,}
\PYG{+w}{                }\PYG{n+nt}{\PYGZdq{}cb\PYGZus{}cmap\PYGZdq{}}\PYG{p}{:}\PYG{+w}{ }\PYG{l+s+s2}{\PYGZdq{}\PYGZdq{}}\PYG{p}{,}
\PYG{+w}{                }\PYG{n+nt}{\PYGZdq{}cb\PYGZus{}col\PYGZdq{}}\PYG{p}{:}\PYG{+w}{ }\PYG{l+s+s2}{\PYGZdq{}null\PYGZdq{}}\PYG{p}{,}
\PYG{+w}{                }\PYG{n+nt}{\PYGZdq{}cb\PYGZus{}gr\PYGZus{}dict\PYGZdq{}}\PYG{p}{:}\PYG{+w}{ }\PYG{l+s+s2}{\PYGZdq{}\PYGZdq{}}\PYG{p}{,}
\PYG{+w}{                }\PYG{n+nt}{\PYGZdq{}gb\PYGZus{}col\PYGZdq{}}\PYG{p}{:}\PYG{+w}{ }\PYG{l+s+s2}{\PYGZdq{}Gene\PYGZdq{}}\PYG{p}{,}
\PYG{+w}{                }\PYG{n+nt}{\PYGZdq{}gb\PYGZus{}name\PYGZdq{}}\PYG{p}{:}\PYG{+w}{ }\PYG{l+s+s2}{\PYGZdq{}\PYGZdq{}}\PYG{p}{,}
\PYG{+w}{                }\PYG{n+nt}{\PYGZdq{}opacity\PYGZdq{}}\PYG{p}{:}\PYG{+w}{ }\PYG{l+s+s2}{\PYGZdq{}1\PYGZdq{}}\PYG{p}{,}
\PYG{+w}{                }\PYG{n+nt}{\PYGZdq{}opacity\PYGZus{}col\PYGZdq{}}\PYG{p}{:}\PYG{+w}{ }\PYG{l+s+s2}{\PYGZdq{}null\PYGZdq{}}\PYG{p}{,}
\PYG{+w}{                }\PYG{n+nt}{\PYGZdq{}pie\PYGZus{}col\PYGZdq{}}\PYG{p}{:}\PYG{+w}{ }\PYG{l+s+s2}{\PYGZdq{}null\PYGZdq{}}\PYG{p}{,}
\PYG{+w}{                }\PYG{n+nt}{\PYGZdq{}pie\PYGZus{}dict\PYGZdq{}}\PYG{p}{:}\PYG{+w}{ }\PYG{l+s+s2}{\PYGZdq{}\PYGZdq{}}\PYG{p}{,}
\PYG{+w}{                }\PYG{n+nt}{\PYGZdq{}scale\PYGZus{}col\PYGZdq{}}\PYG{p}{:}\PYG{+w}{ }\PYG{l+s+s2}{\PYGZdq{}null\PYGZdq{}}\PYG{p}{,}
\PYG{+w}{                }\PYG{n+nt}{\PYGZdq{}scale\PYGZus{}factor\PYGZdq{}}\PYG{p}{:}\PYG{+w}{ }\PYG{l+s+s2}{\PYGZdq{}0.5\PYGZdq{}}\PYG{p}{,}
\PYG{+w}{                }\PYG{n+nt}{\PYGZdq{}shape\PYGZus{}col\PYGZdq{}}\PYG{p}{:}\PYG{+w}{ }\PYG{l+s+s2}{\PYGZdq{}null\PYGZdq{}}\PYG{p}{,}
\PYG{+w}{                }\PYG{n+nt}{\PYGZdq{}shape\PYGZus{}fixed\PYGZdq{}}\PYG{p}{:}\PYG{+w}{ }\PYG{l+s+s2}{\PYGZdq{}cross\PYGZdq{}}\PYG{p}{,}
\PYG{+w}{                }\PYG{n+nt}{\PYGZdq{}shape\PYGZus{}gr\PYGZus{}dict\PYGZdq{}}\PYG{p}{:}\PYG{+w}{ }\PYG{l+s+s2}{\PYGZdq{}\PYGZdq{}}\PYG{p}{,}
\PYG{+w}{                }\PYG{n+nt}{\PYGZdq{}tooltip\PYGZus{}fmt\PYGZdq{}}\PYG{p}{:}\PYG{+w}{ }\PYG{l+s+s2}{\PYGZdq{}\PYGZdq{}}
\PYG{+w}{            }\PYG{p}{\PYGZcb{},}
\PYG{+w}{            }\PYG{n+nt}{\PYGZdq{}expectedRadios\PYGZdq{}}\PYG{p}{:}\PYG{+w}{ }\PYG{p}{\PYGZob{}}
\PYG{+w}{                }\PYG{n+nt}{\PYGZdq{}cb\PYGZus{}col\PYGZdq{}}\PYG{p}{:}\PYG{+w}{ }\PYG{k+kc}{false}\PYG{p}{,}
\PYG{+w}{                }\PYG{n+nt}{\PYGZdq{}cb\PYGZus{}gr\PYGZdq{}}\PYG{p}{:}\PYG{+w}{ }\PYG{k+kc}{true}\PYG{p}{,}
\PYG{+w}{                }\PYG{n+nt}{\PYGZdq{}cb\PYGZus{}gr\PYGZus{}dict\PYGZdq{}}\PYG{p}{:}\PYG{+w}{ }\PYG{k+kc}{false}\PYG{p}{,}
\PYG{+w}{                }\PYG{n+nt}{\PYGZdq{}cb\PYGZus{}gr\PYGZus{}key\PYGZdq{}}\PYG{p}{:}\PYG{+w}{ }\PYG{k+kc}{true}\PYG{p}{,}
\PYG{+w}{                }\PYG{n+nt}{\PYGZdq{}cb\PYGZus{}gr\PYGZus{}rand\PYGZdq{}}\PYG{p}{:}\PYG{+w}{ }\PYG{k+kc}{false}\PYG{p}{,}
\PYG{+w}{                }\PYG{n+nt}{\PYGZdq{}pie\PYGZus{}check\PYGZdq{}}\PYG{p}{:}\PYG{+w}{ }\PYG{k+kc}{false}\PYG{p}{,}
\PYG{+w}{                }\PYG{n+nt}{\PYGZdq{}scale\PYGZus{}check\PYGZdq{}}\PYG{p}{:}\PYG{+w}{ }\PYG{k+kc}{false}\PYG{p}{,}
\PYG{+w}{                }\PYG{n+nt}{\PYGZdq{}shape\PYGZus{}col\PYGZdq{}}\PYG{p}{:}\PYG{+w}{ }\PYG{k+kc}{false}\PYG{p}{,}
\PYG{+w}{                }\PYG{n+nt}{\PYGZdq{}shape\PYGZus{}fixed\PYGZdq{}}\PYG{p}{:}\PYG{+w}{ }\PYG{k+kc}{false}\PYG{p}{,}
\PYG{+w}{                }\PYG{n+nt}{\PYGZdq{}shape\PYGZus{}gr\PYGZdq{}}\PYG{p}{:}\PYG{+w}{ }\PYG{k+kc}{true}\PYG{p}{,}
\PYG{+w}{                }\PYG{n+nt}{\PYGZdq{}shape\PYGZus{}gr\PYGZus{}dict\PYGZdq{}}\PYG{p}{:}\PYG{+w}{ }\PYG{k+kc}{false}\PYG{p}{,}
\PYG{+w}{                }\PYG{n+nt}{\PYGZdq{}shape\PYGZus{}gr\PYGZus{}rand\PYGZdq{}}\PYG{p}{:}\PYG{+w}{ }\PYG{k+kc}{true}\PYG{p}{,}
\PYG{+w}{                }\PYG{n+nt}{\PYGZdq{}opacity\PYGZus{}check\PYGZdq{}}\PYG{p}{:}\PYG{+w}{ }\PYG{k+kc}{false}
\PYG{+w}{            }\PYG{p}{\PYGZcb{},}
\PYG{+w}{            }\PYG{n+nt}{\PYGZdq{}name\PYGZdq{}}\PYG{p}{:}\PYG{+w}{ }\PYG{l+s+s2}{\PYGZdq{} markers\PYGZdq{}}\PYG{p}{,}
\PYG{+w}{            }\PYG{n+nt}{\PYGZdq{}path\PYGZdq{}}\PYG{p}{:}\PYG{+w}{ }\PYG{l+s+s2}{\PYGZdq{}./istdeco\PYGZus{}codes\PYGZus{}n.csv\PYGZdq{}}\PYG{p}{,}
\PYG{+w}{            }\PYG{n+nt}{\PYGZdq{}title\PYGZdq{}}\PYG{p}{:}\PYG{+w}{ }\PYG{l+s+s2}{\PYGZdq{}Download markers\PYGZdq{}}\PYG{p}{,}
\PYG{+w}{            }\PYG{n+nt}{\PYGZdq{}uid\PYGZdq{}}\PYG{p}{:}\PYG{+w}{ }\PYG{l+s+s2}{\PYGZdq{}uniquetab\PYGZdq{}}
\PYG{+w}{        }\PYG{p}{\PYGZcb{}}
\PYG{+w}{    }\PYG{p}{],}
\PYG{+w}{    }\PYG{n+nt}{\PYGZdq{}regions\PYGZdq{}}\PYG{p}{:}\PYG{+w}{ }\PYG{p}{\PYGZob{}\PYGZcb{},}
\PYG{+w}{    }\PYG{n+nt}{\PYGZdq{}plugins\PYGZdq{}}\PYG{p}{:}\PYG{+w}{ }\PYG{p}{[}
\PYG{+w}{        }\PYG{l+s+s2}{\PYGZdq{}Spot\PYGZus{}Inspector\PYGZdq{}}
\PYG{+w}{    }\PYG{p}{],}
\PYG{+w}{    }\PYG{n+nt}{\PYGZdq{}hideTabs\PYGZdq{}}\PYG{p}{:}\PYG{+w}{ }\PYG{k+kc}{true}\PYG{p}{,}
\PYG{+w}{    }\PYG{n+nt}{\PYGZdq{}settings\PYGZdq{}}\PYG{p}{:}\PYG{+w}{ }\PYG{p}{[]}
\PYG{p}{\PYGZcb{}}
\end{sphinxVerbatim}

\sphinxstepscope


\chapter{Support}
\label{\detokenize{docs/support/index:support}}\label{\detokenize{docs/support/index::doc}}
\sphinxstepscope


\section{How to seek support and discussion}
\label{\detokenize{docs/support/support:how-to-seek-support-and-discussion}}\label{\detokenize{docs/support/support::doc}}
\sphinxAtStartPar
If you want to ask quentions about TissUUmaps please visit \sphinxhref{https://forum.image.sc/tag/tissuumaps}{forum.image.sc}.


\section{How to issue an error on GitHub}
\label{\detokenize{docs/support/support:how-to-issue-an-error-on-github}}
\sphinxAtStartPar
If you want to report a bug, please do so at \sphinxhref{https://github.com/TissUUmaps/TissUUmaps/issues}{issues on GitHub}.



\renewcommand{\indexname}{Index}
\printindex
\end{document}